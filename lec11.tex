\documentclass[a4paper,10pt]{article}
\usepackage{%
amsmath,%
amsfonts,%
amssymb,%
amsthm,%
hyperref,%
url,%
latexsym,%
epsfig,%
graphicx,%
psfrag,%
subfigure,%
color,%
tikz,%
pgf,%
pgfplots,%
pgfplotstable,%
pgfpages%
}
\usepgflibrary{shapes}
\usetikzlibrary{%
arrows,%
backgrounds,%
chains,%
decorations.pathmorphing,% /pgf/decoration/random steps | erste Graphik
decorations.text,%
matrix,%
positioning,% wg. " of "
fit,%
patterns,%
petri,%
plotmarks,%
scopes,%
shadows,%
shapes.misc,% wg. rounded rectangle
shapes.arrows,%
shapes.callouts,%
shapes%
}
\theoremstyle{plain}
\newtheorem{thm}{Theorem}[section]
\newtheorem{lem}[thm]{Lemma}
\newtheorem{prop}[thm]{Proposition}
\newtheorem{cor}[thm]{Corollary}
\theoremstyle{definition}
\newtheorem{defn}[thm]{Definition}
\newtheorem{conj}[thm]{Conjecture}
\newtheorem{exmp}[thm]{Example}
\theoremstyle{remark}
\newtheorem{rem}[thm]{Remark}
\newtheorem{note}[thm]{Note}

\usepackage{%
	amsfonts,%
	amsmath,%	
	amssymb,%
	amsthm,%
%	babel,%
	bbm,%
	%biblatex,%
	caption,%
	centernot,%
	color,%
	enumerate,%
	epsfig,%
	epstopdf,%
	etex,%
	geometry,%
	graphicx,%
	hyperref,%
	latexsym,%
	mathtools,%
	multicol,%
	pgf,%
	pgfplots,%
	pgfplotstable,%
	pgfpages,%
	proof,%
	psfrag,%
	subfigure,%	
	tikz,%
	ulem,%
	url%
}	

\usepackage[mathscr]{eucal}
\usepgflibrary{shapes}
\usetikzlibrary{%
  arrows,%
  backgrounds,%
  chains,%
  decorations.pathmorphing,% /pgf/decoration/random steps | erste Graphik
  decorations.text,%
  matrix,%
  positioning,% wg. " of "
  fit,%
  patterns,%
  petri,%
  plotmarks,%
  scopes,%
  shadows,%
  shapes.misc,% wg. rounded rectangle
  shapes.arrows,%
  shapes.callouts,%
  shapes%
}

\theoremstyle{plain}
\newtheorem{thm}{Theorem}[section]
\newtheorem{lem}[thm]{Lemma}
\newtheorem{prop}[thm]{Proposition}
\newtheorem{cor}[thm]{Corollary}

\theoremstyle{definition}
\newtheorem{defn}[thm]{Definition}
\newtheorem{conj}[thm]{Conjecture}
\newtheorem{exmp}[thm]{Example}
\newtheorem{assum}[thm]{Assumptions}
\newtheorem{axiom}[thm]{Axiom}

\theoremstyle{remark}
\newtheorem{rem}[thm]{Remark}
\newtheorem{note}[thm]{Note}

\newcommand{\norm}[1]{\left\lVert#1\right\rVert}
\newcommand{\indep}{\!\perp\!\!\!\perp}
\DeclarePairedDelimiter\abs{\lvert}{\rvert}%
%\DeclarePairedDelimiter\norm{\lVert}{\rVert}%
\newcommand{\tr}{\operatorname{tr}}
\newcommand{\R}{\mathbb{R}}
\newcommand{\Q}{\mathbb{Q}}
\newcommand{\N}{\mathbb{N}}
\newcommand{\E}{\mathbb{E}}
\newcommand{\Z}{\mathbb{Z}}
\newcommand{\B}{\mathscr{B}}
\newcommand{\C}{\mathcal{C}}
\newcommand{\T}{\mathscr{T}}
\newcommand{\F}{\mathcal{F}}
\newcommand{\G}{\mathcal{G}}
%\newcommand{\ba}{\begin{align*}}
%\newcommand{\ea}{\end{align*}}

% Debug
\newcommand{\todo}[1]{\begin{color}{blue}{{\bf~[TODO:~#1]}}\end{color}}


\makeatletter
\def\th@plain{%
  \thm@notefont{}% same as heading font
  \itshape % body font
}
\def\th@definition{%
  \thm@notefont{}% same as heading font
  \normalfont % body font
}
\makeatother
\date{}
\title{Lecture 11 : Discrete Time Markov Chains}
\author{Sireesha Madabhushi}
\begin{document}
\maketitle
\section{Discrete Time Markov Chains Contd}
\subsection{Theorem}
\textbf{Statement:}\\
An irreducible, aperiodic MC is of one of the following types:
\begin{itemize}
	\item All the states are either transient or null recurrent.\\
	$P_{ij}^n \longrightarrow 0$ as $n \longrightarrow \infty$ and there exists no stationary distribution
	\item All the states are positive recurrent. There exists a unique stationary distribution ${\pi_j, j = 0,1,2,3...}$.\\
    $\pi_j = \underset{n\rightarrow \infty}{\text{lim}} P^n_{ij} > 0$
\end{itemize}
\begin{proof}
We will initially prove the second case.
From renewal reward theorem, $\lim_{n \to \infty} \sum_{j=0}^{\infty} P_{ij}^{(n)} = 1/\mu_{jj}$ exists, which we take as $\pi_j$.\\
\begin{flalign*}
&\sum_{j=0}^{M} P_{ij}^n \leq \sum_{j=0}^{\infty} P_{ij}^n = 1, \forall M\\
&n \longrightarrow \infty \Rightarrow \sum_{j=0}^{M}\pi_j  \leq 1, \forall M\\	
&\Rightarrow \sum_{j=0}^{\infty}\pi_j  \leq 1\\
&P_{ij}^{n+1} = \sum_{k=0}^{\infty}P_{ik}^nP_{kj} \geq \sum_{k=0}^{M}P_{ik}^nP_{kj},\forall M\\
\end{flalign*}
Applying $n \to \infty$ on $P_{ij}^{n+1} \geq \sum_{k=0}^{M}P_{ik}^nP_{kj}$, we get\\
\begin{flalign*}
&n \longrightarrow \infty \Rightarrow \pi_j \geq \sum_{k=0}^{M} \pi_kP_{kj},\forall M\\
&\Rightarrow \pi_j \geq \sum_{k=0}^{\infty} \pi_kP_{kj},j \geq 0\\
\end{flalign*}	
Assuming that the inequality is strict for some $j$ and adding the inequalities, we get
\begin{flalign*}
&\sum_{j=0}^{\infty} \pi_j > \sum_{j=0}^{\infty} \sum_{k=0}^{\infty} \pi_kP_{kj} = \sum_{k=0}^{\infty} \pi_k \sum_{j=0}^{\infty} P_{kj} = \sum_{k=0}^{\infty} \pi_k\\
&(summation \: can\: be\: interchanged \:since\: \pi_k\: and\: P_{kj}\: are\: non-negative.)\\
\end{flalign*}	
This is a contradiction. Therefore,\\
\begin{flalign*}
&\pi_j = \sum_{k=0}^{\infty} \pi_k P_{kj}, j = 0,1,2....\\
\end{flalign*}
Putting $P_j = \dfrac{\pi_j}{\sum_{0}^{\infty}\pi_k}$ (normalization), we see that $\{P_j,j=0,1,2...\}$ is a stationary distribution and so atleast one stationary distribution exists. Let $\{P_j,j=0,1,2...\}$ be any stationary distribution. Then if $\{P_j,j=0,1,2...\}$ is the probability distribution of $X_0$, then\\
\begin{flalign*}
&P_j = P\{X_n = j\}\\
&P_j = \sum_{i=0}^{\infty} P\{X_n = j|X_0 = i\} P\{X_0 = i\}\\
&P_j = \sum_{i=0}^{\infty} P_{ij}^nP_i\\
&\Rightarrow P_j \geq \sum_{i=0}^{M} P_{ij}^nP_i, \forall M\\
&Applying \: n \longrightarrow \infty and then M \longrightarrow \infty \Rightarrow P_j \geq \sum_{i=0}^{\infty} \pi_j P_i = \pi_j\\
&P_j \geq \pi_j\\
\end{flalign*}
To show $P_j \leq \pi_j$, we use the fact that $P_{ij}^n \leq 1$.\\
\begin{flalign*}
&P_j = \sum_{i=0}^{\infty} P_{ij}^nP_i\\
&P_j = \sum_{i=0}^{M}P_{ij}^nP_i + \sum_{i=M+1}^{\infty}P_{ij}^nP_i\\
&P_j \leq \sum_{i=0}^{M} P_{ij}^{n}P_i + \sum_{i=M+1}^{\infty} P_i, \forall M\\
&n \longrightarrow \infty \Rightarrow P_j \leq \sum_{i=0}^{M}\pi_jP_i + \sum_{i=M+1}^{\infty} P_i, \forall M\\
&\because \sum_{0}^{\infty}P_i = 1, M \longrightarrow \infty \Rightarrow P_j \leq \sum_{i=0}^{\infty} \pi_j P_i = \pi_j\\
&P_j \leq \pi_j\\
\end{flalign*}
If the states are transient or null recurrent and $\{P_j,j=0,1,2...\}$ is a stationary distribution, then $P_j = \sum_{i=0}^{\infty} P\{X_n = j|X_0 = i\} P\{X_0 = i\}$ holds.\\
Then, $n \to \infty$ in $P_j= \sum_{i=0}^{n} \pi_j P_{ij}^n$, gives $P_j=0$ for every $j$. This contradicts the assumption that $P_j$ is a probability distribution.\\
Thus for the first case, no stationary distribution exists and the proof is complete.
\end{proof}
\textbf{Corollary 1:} An irreducible, aperiodic DTMC defined on a finite state space will be positive recurrent.
\begin{proof}
	Suppose that the DTMC is not positive recurrent, then \\
	\begin{flalign*}
		&\underset{n\rightarrow \infty}{\text{lim}} p_{ij}^{(n)} = 0\\
    \end{flalign*}
		Interchanging limit and finite summation gives
    \begin{flalign*}
		&\underset{n\rightarrow \infty}{\text{lim}} (\sum_{j \in S}p_{ij}) = \underset{n\rightarrow \infty}{\text{lim}} (1) = 1\\
	\end{flalign*}
	This is a contradiction.
	Hence the DTMC mentioned above is positive recurrent.
\end{proof}

\textbf{Corollary 2:} For an irreducible and aperiodic DTMC, $E_i[T_i] = 1/\pi_i$, where $E_i[T_i]$ denotes the expected number of time steps to return to the state $i$ and $\pi_i$ is its stationary distribution.\\

\subsection{Ergodic theorem for DTMCs}

\textbf{Claim:} Let $\{X_n\}$ be an irreducible, aperiodic and positive recurrent DTMC on $N_0$ with stationary distribution $\Pi$.\\
Let $f : N \longrightarrow R$, such that $\sum_{i \in N_0} \arrowvert f(i) \arrowvert \Pi_i < \infty$ (f is integrable over $N_0$ w.r.t $\Pi$).\\
Then for any initial distribution of $X_0$,\\
$\underset{n\rightarrow \infty}{\text{lim}} \dfrac{1}{n} \sum_{i=1}^{n} f(X_i) = \sum_{i \in N_0} \Pi(i)f(i)$ almost surely.\\

\begin{proof}

Fix $0 \in N_0$	.\\
Let $\tau_1, \tau_2,....$ be successive instants at which '0' is visited ($\tau_0 = 0$).\\
Let $\forall p \geq 0, R_{p+1} = \sum_{n = \tau_p + 1}^{\tau_{(p + 1)}} f(X_n) = $net reward earned at the end of cycle $(p+1)$.\\
Each cycle forms a renewal.By the strong Markov property, cycles are independent. 
At stopping time, state is $0$.\\
Average reward earned during the first $k$ cycles is given as follows:
\begin{flalign*}	
&\dfrac{1}{n} \sum_{k=1}^{n} R_k = \dfrac{1}{n} \sum_{k=1}^{n} \sum_{\tau_{k-1} + 1}^{\tau_k} f(X_i)\\
&\dfrac{1}{n} \sum_{k=1}^{n} R_k = \dfrac{1}{n} \sum_{t=0}^{\tau_n} f(X_t)\\
&\stackrel {n \longrightarrow \infty}{\longrightarrow} \dfrac{E_0[cycle \:  renewal]}{E_0[cycle \: length]} \: (from \: Renewal \: Reward \: Theorem)\\
&E_0[cycle \: length] = \dfrac{1}{\pi_0}\: (from \: Corollary \: 2)\\
&E_0[cycle \: reward] = E_0[\sum_{n=0}^{\tau_1}f(X_n)]\\
&E_0[cycle \: reward] = E_0[\sum_{n=0}^{\tau_1} \sum_{i \in N_0} f(i) \textbf{1}[X_n = i]]\\
&E_0[cycle \: reward] = \sum_{i \in N_0} f(i) E_0[\sum_{n=0}^{\tau_i} \textbf{1}[X_n = i]]\: (assuming\: f(i) > 0)\\
\end{flalign*}
To calculate the reward given above, we need to find the expected number of visits to state $i$ between successive visits to state 0
Let us consider the average number of visits to state $i$ i.e.,\\
$\dfrac{\underset{n\rightarrow \infty}{\text{lim}} \sum_{k=1}^{n} \textbf{1}[X_t = i]}{n}$. Let us call it $\circledR$.\\
We can calculate the above quantity in two ways.\\
\begin{enumerate}
	\item DTMC undergoes a renewal each time it hits state $i$.\\
	So, from elementary renewal theorem, \\
	$\circledR = 1/(inter\:arrival\: times\: of\: hitting\: i) = 1/(1/\pi_i) = \pi_i$
	\item Without loss of generality, renewal occurs whenever DTMC hits state $0$.\\
	By renewal reward theorem, $\circledR = \dfrac{E_0[\sum_{n=0}^{\tau_i} \textbf{1}[X_n = i]]}{1/\pi_0}$
\end{enumerate}
From (1) and (2), $E_0[\sum_{n=0}^{\tau_i} \textbf{1}[X_n = i]] = \dfrac{\pi_i}{\pi_0}$\\
$\dfrac{1}{\tau_n} \sum_{t=0}^{\tau_n} f(X_t) \stackrel {n \longrightarrow \infty}{\longrightarrow} \dfrac{\sum_{i \in N_0} f(i) \pi_i/\pi_0}{1/\pi_0}$

\end{proof}

\end{document}