% !TEX spellcheck = en_US
% !TEX spellcheck = LaTeX
\documentclass[a4paper,10pt,english]{article}
\usepackage{%
	amsfonts,%
	amsmath,%	
	amssymb,%
	amsthm,%
%	babel,%
	bbm,%
	%biblatex,%
	caption,%
	centernot,%
	color,%
	enumerate,%
	epsfig,%
	epstopdf,%
	etex,%
	geometry,%
	graphicx,%
	hyperref,%
	latexsym,%
	mathtools,%
	multicol,%
	pgf,%
	pgfplots,%
	pgfplotstable,%
	pgfpages,%
	proof,%
	psfrag,%
	subfigure,%	
	tikz,%
	ulem,%
	url%
}	

\usepackage[mathscr]{eucal}
\usepgflibrary{shapes}
\usetikzlibrary{%
  arrows,%
  backgrounds,%
  chains,%
  decorations.pathmorphing,% /pgf/decoration/random steps | erste Graphik
  decorations.text,%
  matrix,%
  positioning,% wg. " of "
  fit,%
  patterns,%
  petri,%
  plotmarks,%
  scopes,%
  shadows,%
  shapes.misc,% wg. rounded rectangle
  shapes.arrows,%
  shapes.callouts,%
  shapes%
}

\theoremstyle{plain}
\newtheorem{thm}{Theorem}[section]
\newtheorem{lem}[thm]{Lemma}
\newtheorem{prop}[thm]{Proposition}
\newtheorem{cor}[thm]{Corollary}

\theoremstyle{definition}
\newtheorem{defn}[thm]{Definition}
\newtheorem{conj}[thm]{Conjecture}
\newtheorem{exmp}[thm]{Example}
\newtheorem{assum}[thm]{Assumptions}
\newtheorem{axiom}[thm]{Axiom}

\theoremstyle{remark}
\newtheorem{rem}[thm]{Remark}
\newtheorem{note}[thm]{Note}

\newcommand{\norm}[1]{\left\lVert#1\right\rVert}
\newcommand{\indep}{\!\perp\!\!\!\perp}
\DeclarePairedDelimiter\abs{\lvert}{\rvert}%
%\DeclarePairedDelimiter\norm{\lVert}{\rVert}%
\newcommand{\tr}{\operatorname{tr}}
\newcommand{\R}{\mathbb{R}}
\newcommand{\Q}{\mathbb{Q}}
\newcommand{\N}{\mathbb{N}}
\newcommand{\E}{\mathbb{E}}
\newcommand{\Z}{\mathbb{Z}}
\newcommand{\B}{\mathscr{B}}
\newcommand{\C}{\mathcal{C}}
\newcommand{\T}{\mathscr{T}}
\newcommand{\F}{\mathcal{F}}
\newcommand{\G}{\mathcal{G}}
%\newcommand{\ba}{\begin{align*}}
%\newcommand{\ea}{\end{align*}}

% Debug
\newcommand{\todo}[1]{\begin{color}{blue}{{\bf~[TODO:~#1]}}\end{color}}


\makeatletter
\def\th@plain{%
  \thm@notefont{}% same as heading font
  \itshape % body font
}
\def\th@definition{%
  \thm@notefont{}% same as heading font
  \normalfont % body font
}
\makeatother
\date{}
<<<<<<< HEAD
\title{Lecture 10: Applications of Renewal Processes}
=======
\title{Lecture 10: Renewal Reward Processes}
>>>>>>> ce2648980fafcf4ce99432fc7fcce34e6b13ca61
\author{}

\begin{document}
\maketitle
%\section{Renewal theory Contd. -- Delayed Renewal processes }
\section{Renewal reward process}
Consider a renewal process $\{N(t), t \geq 0\}$ with \textit{iid} inter renewal times $\{X_n: n \in \N\}$ having common distribution $F$.  
The reward sequence $\{R_n: n \in \N\}$ consists of reward $R_n$ at the end of $n$th renewal interval $X_n$ for each $n \in \N$. 
Let $(X_n,R_n)$ be \textit{iid}. 
Then the \textbf{reward process} $\{R(t), t \geqslant 0\}$ consists of accumulated reward earned by time $t$ as 
\eq{
R(t) &=\sum_{i=1}^{N(t)}R_i.
} 
\begin{thm}
	\label{theorem}
Let $\E[|R|]$ and $\E[|X|]$ be finite.
\begin{enumerate}[(a)]
\item\label{item:Basic} $\lim_{t \to \infty} \frac{R(t)}{t} = \frac{\E[R]}{\E[X]} ~a.s.$
\item\label{item:Elem}  $\lim_{t \to \infty} \frac{\E[R(t)]}{t} = \frac{\E[R]}{\E[X]}$.
\end{enumerate}
\end{thm}
\begin{proof}
We can write 
\begin{align*}
\frac{R(t)}{t}&=\frac{\sum_{i=1}^{N(t)}R_i}{t} =\left(\frac{\sum_{i=1}^{N(t)}R_i}{N(t)} \right) \left(\frac{N(t)}{t}\right).
\end{align*}
\begin{enumerate}[(a)]
\item By the strong law of large numbers (almost sure convergence law) we obtain that, 
\begin{align*}
	\lim_{t \to \infty} \frac{\sum_{i=1}^{N(t)}R_i}{N(t)} = \E[R],
\end{align*}
and by the basic renewal theorem (almost sure convergence law) we obtain that, 
\begin{align*}
	\lim_{t \to \infty} \frac{N(t)}{t} = \frac{1}{\E[X]}.
\end{align*} 
%Thus $ (1) $ is proven.
\item 
Notice that $ N(t)+1 $ is a stopping time for the sequence $\{R_1,R_2,\dots\}$. 
This is true since
\begin{align*}
	\{N(t)+1 = n\} &= \{X_1+X_2+\cdots+X_{n-1} \leq t , X_n > t \}\\
	&= \{R_1+R_2+\cdots+R_{n-1} = R(t) , R_n \neq 0 \}.
\end{align*}
Moreover $ N(t)+1 $ is a stopping time for the sequence $\{ X_1,X_2,\dots \}$. 
So by algebra and Wald's lemma,
\begin{align*}
\E[R(t)] &= \E\left[\sum_{i=1}^{N(t)}R_i\right] \\
&= \E\left[\sum_{i=1}^{N(t)+1}R_i\right]-\E[R_{N(t)+1}]\\ 
&= (m(t)+1)\E[R_1]-\E[R_{N(t)+1}].
\end{align*}
Let $g(t)=\E[R_{N(t)+1}].$ So 
\begin{equation}
 \frac{\E[R(t)]}{t} = \frac{(m(t)+1)}{t}\E[R_1]-\frac{g(t)}{t}, \nonumber
\end{equation}
and the result will follow from the elementary renewal theorem if we can show that $ \frac{g(t)}{t} \to 0 $ as $ t \to \infty. $ So,
\begin{align*}
g(t) &= \E[R_{N(t)+1}1\{S_{N(t)}=0\}]+\E[R_{N(t)+1}1\{S_{N(t)}>0\}]\\
&=\E[R_{N(t)+1}|S_{N(t)}=0]P(X_1>t)+\int_{0}^{t}\E[R_{N(t)+1}|S_{N(t)}=u]F^c(t-u)dm(u),
\end{align*}
where the second equality follows from the fact that the interarrival times $ X_n, n \in \N $, are \textit{iid} with distribution $ F $. \\ 
However,
\begin{align*}
	\E[R_{N(t)+1}|S_{N(t)}=0] &= \E[R_1|X_1>t], \\
	\E[R_{N(t)+1}|S_{N(t)}=u] &= \E[R_n|X_1>t-u], 
\end{align*}
and so
\begin{align*}
g(t) &= \E[R_1|X_1>t]F^c(t)+\int_{0}^{t}\E[R_n|X_1>t-u]F^c(t-u)dm(u)\\
&= \E[R_1|X_1>t]F^c(t)+\int_{0}^{t}\E[R_1|X_1>t-u]F^c(t-u)dm(u),
\end{align*}
where the second equality follows from the fact that $ R_n, n \in \N $, are \textit{iid}. \\
Now, let
\begin{equation}
h(t)=\E[R_1|X_1>t]F^c(t) = \int_{x=t}^{\infty} \E[R_1|X_1=x]dF(x), \nonumber
\end{equation}
and note that since
\begin{equation}
\E[|R_1|] = \int_{x=0}^{\infty} \E[|R_1||X_1=x]dF(x) < \infty, \nonumber
\end{equation}
it follows that $h(t) \to 0$ as $t \to \infty.$ Hence, choosing $T$ such that $|h(u)| \leq \epsilon$ whenever $ u \geq T$, we have for all $t \geq T$ that
\begin{align*}
\frac{|g(t)|}{t} &\leq \frac{|h(t)|}{t} +\int_{0}^{t-T}\frac{|h(t-s)|}{t}dm(s)+\int_{t-T}^{t}\frac{|h(t-s)|}{t}dm(s)\\
&\leq \frac{\epsilon}{t}+ \frac{\epsilon m(t-T)}{t}+ \E[|R_1|]\frac{(m(t)-m(t-T))}{t} .
\end{align*}
Hence $\lim_{t \to \infty}\frac{g(t)}{t}= \frac{\epsilon}{\E[X]}$ by the
elementary renewal theorem, and the result follows since $\epsilon >
0$ is arbitrary.
\end{enumerate}
 \end{proof}

\begin{rem}
	 	$ (1) ~$ $R_{N(t)+1}$ has different distribution than $R_1$.\\
	 	\textit{Analysis:} Notice that $R_{N(t)+1}$ is related to $X_{N(t)+1}$ which is the length of the renewal interval containing the point $t$. Since larger renewal intervals have a greater chance of containing $t$, it follows that $X_{N(t)+1}$ tends to be larger than a ordinary renewal interval. Formally,
	 	\begin{align*}
	 		\Pr\{X_{N(t)+1} > x\}&= \sum_{n \in \N_0} \left( \left[ \int_{0}^t\Pr\{X_{N(t)+1} > x | S_{N(t)} = y, N(t)=n\}F^c(t-y)dm(y) \right] \Pr\{N(t)=n\} \right).
	 	\end{align*}
	 	Now we have,
	 	\begin{align*}
	 		\Pr\{X_{N(t)+1}>x | S_{N(t)}=y, N(t)=n\} & = \Pr\{X_{N(t)+1}>x | X_1+\cdots+X_n=y, X_{n+1}>t-y\} \\
	 		& = \Pr\{X_{n+1}>x | X_{n+1}>t-y\} \\
	 		& = \frac{\Pr\{X_{n+1}>\text{max}(x,t-y)\}}{\Pr\{X_{n+1}>t-y\}} \\
	 		& \geq F^c(x). 
	 	\end{align*}
	 	So we get that,
	 	\begin{align*}
	 		\Pr\{X_{N(t)+1}>x\}\geq \Pr\{X_1>x\}.
	 	\end{align*}
	 	Thus the remark follows.\\
	 	$ (2) ~$ $R(t)$ is the gradual reward during a cycle, 
	 	\begin{align*}
	 		\frac{\sum_{n=1}^{N(t)}R_n}{t} \leq  \frac{R(t)}{t} \leq \frac{\sum_{n=1}^{N(t)+1}R_n}{t}.
	 	\end{align*}
	 	\textit{Analysis:} The part 1 of the theorem \ref{theorem} under this regime follows since
	 	\begin{align*}
	 		\lim_{t \to \infty} \frac{\sum_{n=1}^{N(t)}R_n}{t} = \frac{\E\left[R\right]}{\E\left[X\right]},\\
	 		\lim_{t \to \infty} \frac{\sum_{n=1}^{N(t)+1}R_n}{t} = \frac{\E\left[R\right]}{\E\left[X\right]},
	 	\end{align*}
	 	by the similar arguments given in the proof of the theorem \ref{theorem}.\\
	 	The part 2 of the theorem \ref{theorem} under this regime follows since
	 	\begin{align*}
	 		\lim_{t \to \infty} \frac{\E\left[R_{N(t)+1}\right]}{t} = 0,
	 	\end{align*}
	 	by the similar arguments given in the proof of the theorem \ref{theorem}. Thus the remark follows. For more insights refer Chapter 3 in \textit{Stochastic Processes} by \textit{Sheldon M. Ross}.
\end{rem}

\subsubsection{Applications of the Renewal Reward Theorem}
To determine the average value of the age of a renewal process, consider the following reward system: Assume we are paid money at a rate equal to the age of the process. That is, at time $t$, we are paid at rate $A(t)$ and so the total earning by time $s$ is $\int_0^s A(t)dt.$ From the renewal reward theorem, we have 
\begin{align*}
\lim_{s\to \infty}\frac{\int_0^s A(t)dt}{s} = \frac{\E[\text{Reward per cycle}]}{\E[\text{Time of a cycle}]}.
\end{align*}
Now, since reward per cycle is $\int_0^X t dt = \frac{X^2}{2}$, we have that
\begin{align*}
\lim_{s\to \infty}\frac{\int_0^s A(t)dt}{s} = \frac{\E[X^2]}{2\E[X]}.
\end{align*}
A similar analysis to calculate the average excess time, where the reward per cycle is $\int_0^X (X-t)dt$ gives 
\begin{align*}
\lim_{s\to \infty}\frac{\int_0^s Y(t)dt}{s} = \frac{\E[X^2]}{2\E[X]}.
\end{align*}
Now since $X_{N(t)+1}=A(t)+Y(t)$, we see that its average value is given by
\begin{align*}
\lim_{s\to \infty}\frac{\int_0^s X_{N(t)+1}dt}{s} = \frac{\E[X^2]}{\E[X]}.
\end{align*}
It can be shown, under certain regularity conditions, that 
\begin{align*}
\lim_{t\to \infty} \E[R_{N(t)+1}] = \frac{\E[R_1 X_1]}{\E[X_1]}.
\end{align*}
Defining a cycle reward to equal the cycle length, we have 
\begin{align*}
\lim_{t\to \infty} \E[X_{N(t)+1}] = \frac{\E[X^2]}{\E[X]}.
\end{align*}
We see that this limit is always greater than $\E[X]$, except when $X$ is constant. Such a result was to be expected in view of the inspection paradox.

 
 \subsubsection{Example:} Suppose for an alternating renewal process, we earn at a rate of one per unit time  when the system is on and the reward for a cycle is the the time system is ON during that cycle.\\
 $ \lim_{t \to \infty} \frac{\text{Amount of ON time in } [0,t]}{t} = \lim_{t \to \infty} \frac{R(t)}{t}=\frac{\E[X]}{\E[X]+\E[Y]} = \lim_{t \to \infty}P(\text{ON at time t})$. 
\end{document}