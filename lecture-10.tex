% !TEX spellcheck = en_US
% !TEX spellcheck = LaTeX
\documentclass[a4paper,10pt,english]{article}
\usepackage{%
	amsfonts,%
	amsmath,%	
	amssymb,%
	amsthm,%
%	babel,%
	bbm,%
	%biblatex,%
	caption,%
	centernot,%
	color,%
	enumerate,%
	epsfig,%
	epstopdf,%
	etex,%
	geometry,%
	graphicx,%
	hyperref,%
	latexsym,%
	mathtools,%
	multicol,%
	pgf,%
	pgfplots,%
	pgfplotstable,%
	pgfpages,%
	proof,%
	psfrag,%
	subfigure,%	
	tikz,%
	ulem,%
	url%
}	

\usepackage[mathscr]{eucal}
\usepgflibrary{shapes}
\usetikzlibrary{%
  arrows,%
  backgrounds,%
  chains,%
  decorations.pathmorphing,% /pgf/decoration/random steps | erste Graphik
  decorations.text,%
  matrix,%
  positioning,% wg. " of "
  fit,%
  patterns,%
  petri,%
  plotmarks,%
  scopes,%
  shadows,%
  shapes.misc,% wg. rounded rectangle
  shapes.arrows,%
  shapes.callouts,%
  shapes%
}

\theoremstyle{plain}
\newtheorem{thm}{Theorem}[section]
\newtheorem{lem}[thm]{Lemma}
\newtheorem{prop}[thm]{Proposition}
\newtheorem{cor}[thm]{Corollary}

\theoremstyle{definition}
\newtheorem{defn}[thm]{Definition}
\newtheorem{conj}[thm]{Conjecture}
\newtheorem{exmp}[thm]{Example}
\newtheorem{assum}[thm]{Assumptions}
\newtheorem{axiom}[thm]{Axiom}

\theoremstyle{remark}
\newtheorem{rem}[thm]{Remark}
\newtheorem{note}[thm]{Note}

\newcommand{\norm}[1]{\left\lVert#1\right\rVert}
\newcommand{\indep}{\!\perp\!\!\!\perp}
\DeclarePairedDelimiter\abs{\lvert}{\rvert}%
%\DeclarePairedDelimiter\norm{\lVert}{\rVert}%
\newcommand{\tr}{\operatorname{tr}}
\newcommand{\R}{\mathbb{R}}
\newcommand{\Q}{\mathbb{Q}}
\newcommand{\N}{\mathbb{N}}
\newcommand{\E}{\mathbb{E}}
\newcommand{\Z}{\mathbb{Z}}
\newcommand{\B}{\mathscr{B}}
\newcommand{\C}{\mathcal{C}}
\newcommand{\T}{\mathscr{T}}
\newcommand{\F}{\mathcal{F}}
\newcommand{\G}{\mathcal{G}}
%\newcommand{\ba}{\begin{align*}}
%\newcommand{\ea}{\end{align*}}

% Debug
\newcommand{\todo}[1]{\begin{color}{blue}{{\bf~[TODO:~#1]}}\end{color}}


\makeatletter
\def\th@plain{%
  \thm@notefont{}% same as heading font
  \itshape % body font
}
\def\th@definition{%
  \thm@notefont{}% same as heading font
  \normalfont % body font
}
\makeatother
\date{}
\title{Lecture 10: Renewal Reward Processes}
\author{}

\begin{document}
\maketitle
%\section{Renewal theory Contd. -- Delayed Renewal processes }
\section{Renewal reward process}
Consider a renewal process $\{N(t), t \geqslant 0\}$ with \textit{iid} inter renewal times $\{X_n: n \in \N\}$ having common distribution $F$.  
The reward sequence $\{R_n: n \in \N\}$ consists of reward $R_n$ at the end of $n$th renewal interval $X_n$ for each $n \in \N$. 
Let $(X_n,R_n)$ be \textit{iid} with the reward $R_n$ earned in $n$th renewal possibly dependent on the duration $X_n$. 
Then the \textbf{reward process} $\{R(t), t \geqslant 0\}$ consists of accumulated reward earned by time $t$ as 
\eq{
R(t) &=\sum_{i=1}^{N(t)}R_i.
} 
\begin{thm}
	\label{theorem}
Let $\E[|R|]$ and $\E[|X|]$ be finite.
\begin{enumerate}[(a)]
\item\label{item:Basic} $\lim_{t \to \infty} \frac{R(t)}{t} = \frac{\E[R]}{\E[X]} ~a.s.$
\item\label{item:Elem}  $\lim_{t \to \infty} \frac{\E[R(t)]}{t} = \frac{\E[R]}{\E[X]}$.
\end{enumerate}
\end{thm}
\begin{proof}
We can write 
\begin{align*}
\frac{R(t)}{t}&=\frac{\sum_{i=1}^{N(t)}R_i}{t} =\left(\frac{\sum_{i=1}^{N(t)}R_i}{N(t)} \right) \left(\frac{N(t)}{t}\right).
\end{align*}
\begin{enumerate}[(a)]
\item By the strong law of large numbers (almost sure convergence law) we obtain that, 
\begin{align*}
	\lim_{t \to \infty} \frac{\sum_{i=1}^{N(t)}R_i}{N(t)} = \E[R],
\end{align*}
and by the basic renewal theorem (almost sure convergence law) we obtain that, 
\begin{align*}
	\lim_{t \to \infty} \frac{N(t)}{t} = \frac{1}{\E[X]}.
\end{align*} 
%Thus $ (1) $ is proven.
\item 
Since $N(t)+1 $ is a stopping time for the sequence $\{(X_1,R_1),(X_2,R_2),\dots\}$, 
%This is true since
%\begin{align*}
%	\{N(t)+1 = n\} &= \{X_1+X_2+\cdots+X_{n-1} \leq t , X_n > t \}\\
%	&= \{R_1+R_2+\cdots+R_{n-1} = R(t) , R_n \neq 0 \}.
%\end{align*}
%Moreover $ N(t)+1 $ is a stopping time for the inter-renewal duration sequence $\{ X_1,X_2,\dots \}$. 
by Wald's lemma, %we have 
\begin{align*}
\E[R(t)] &= \E\left[\sum_{i=1}^{N(t)}R_i\right] = \E\left[\sum_{i=1}^{N(t)+1}R_i\right]-\E R_{N(t)+1} = (m(t)+1)\E R_1-\E R_{N(t)+1}.
\end{align*}
Defining $g(t) \triangleq \E[R_{N(t)+1}]$, using elementary renewal theorem, it suffices to show that 
\eq{
\lim_{t \to\infty}\frac{g(t)}{t} = 0.
}
% So 
%\begin{equation}
% \frac{\E[R(t)]}{t} = \frac{(m(t)+1)}{t}\E[R_1]-\frac{g(t)}{t}, \nonumber
%\end{equation}
%and the result will follow from the elementary renewal theorem if we can show that $ \frac{g(t)}{t} \to 0 $ as $ t \to \infty. $ 
Observe that $R_{N(t)+1}$ is a regenerative process with the regenerative sequence being the renewal instants. 
We can write the kernel function as
\eq{
K(t) &\triangleq \E[R_{N(t)+1}1\{X_1 > t\}] = \int_{t}^{\infty}\E[R_1|X_1=u]dF(u) \leq  \int_{t}^{\infty}\E[|R_1||X_1=u]dF(u).
}
Using the solution to renewal function, we can write $g(t)$ in terms of renewal function $m(t)$ as 
\eq{
g(t) &= K(t) + (m \ast K)(t).
}
From finiteness of $\E|R|$, 
%\eq{
%\E[|R_1|] = \int_{0}^{\infty} \E[|R_1||X_1=u]dF(u) < \infty,
%}
it follows that $\lim_{t\to\infty}K(t) = 0$, and  we can choose $T$ such that $|K(u)| \leq \epsilon$ for all $ u \geq T$. 
Hence, for all $t \geq T$, we have
\begin{align*}
\frac{|g(t)|}{t} &\leq \frac{|K(t)|}{t} +\int_{0}^{t-T}\frac{|K(t-u)|}{t}dm(u)+\int_{t-T}^{t}\frac{|K(t-u)|}{t}dm(u)\\
&\leq \frac{\epsilon}{t}+ \frac{\epsilon m(t-T)}{t}+ \E[|R_1|]\frac{(m(t)-m(t-T))}{t}.
\end{align*}
Taking limits and applying elementary renewal and Blackwell's theorem, we get
\eq{
\lim\sup_{t\to\infty}\frac{|g(t)|}{t} &\leq \frac{\epsilon}{\E X}.
}
%Hence $\lim\sup_{t \to \infty}\frac{g(t)}{t}= \frac{\epsilon}{\E[X]}$ by the elementary renewal theorem, and 
The result follows since $\epsilon > 0$ was arbitrary.
%So,
%\begin{align*}
%g(t) &= \E[R_{N(t)+1}1\{S_{N(t)}=0\}]+\E[R_{N(t)+1}1\{S_{N(t)}>0\}]\\
%&=\E[R_{N(t)+1}|S_{N(t)}=0]P(X_1>t)+\int_{0}^{t}\E[R_{N(t)+1}|S_{N(t)}=u]F^c(t-u)dm(u),
%\end{align*}
%where the second equality follows from the fact that the interarrival times $ X_n, n \in \N $, are \textit{iid} with distribution $ F $. \\ 
%However,
%\begin{align*}
%	\E[R_{N(t)+1}|S_{N(t)}=0] &= \E[R_1|X_1>t], \\
%	\E[R_{N(t)+1}|S_{N(t)}=u] &= \E[R_n|X_1>t-u], 
%\end{align*}
%and so
%\begin{align*}
%g(t) &= \E[R_1|X_1>t]F^c(t)+\int_{0}^{t}\E[R_n|X_1>t-u]F^c(t-u)dm(u)\\
%&= \E[R_1|X_1>t]F^c(t)+\int_{0}^{t}\E[R_1|X_1>t-u]F^c(t-u)dm(u),
%\end{align*}
%where the second equality follows from the fact that $ R_n, n \in \N $, are \textit{iid}. \\
%
%Now, let
%\begin{equation}
%h(t)=\E[R_1|X_1>t]F^c(t) = \int_{x=t}^{\infty} \E[R_1|X_1=x]dF(x), \nonumber
%\end{equation}
%and note that since
%\begin{equation}
%\E[|R_1|] = \int_{x=0}^{\infty} \E[|R_1||X_1=x]dF(x) < \infty, \nonumber
%\end{equation}
%it follows that $h(t) \to 0$ as $t \to \infty.$ Hence, choosing $T$ such that $|h(u)| \leq \epsilon$ whenever $ u \geq T$, we have for all $t \geq T$ that
%\begin{align*}
%\frac{|g(t)|}{t} &\leq \frac{|h(t)|}{t} +\int_{0}^{t-T}\frac{|h(t-s)|}{t}dm(s)+\int_{t-T}^{t}\frac{|h(t-s)|}{t}dm(s)\\
%&\leq \frac{\epsilon}{t}+ \frac{\epsilon m(t-T)}{t}+ \E[|R_1|]\frac{(m(t)-m(t-T))}{t} .
%\end{align*}
%Hence $\lim_{t \to \infty}\frac{g(t)}{t}= \frac{\epsilon}{\E[X]}$ by the
%elementary renewal theorem, and the result follows since $\epsilon >
%0$ is arbitrary.
\end{enumerate}
 \end{proof}

\begin{lem} 
Reward $R_{N(t)+1}$ at the next renewal has different distribution than $R_1$.
\end{lem}
\begin{proof}
Notice that $R_{N(t)+1}$ is related to $X_{N(t)+1}$ which is the length of the renewal interval containing the point $t$. 
We have seen that larger renewal intervals have a greater chance of containing $t$. 
That is, $X_{N(t)+1}$ tends to be larger than a ordinary renewal interval. 
Formally,
\end{proof}
%	 	\begin{align*}
%	 		\Pr\{X_{N(t)+1} > x\}&= \sum_{n \in \N_0} \left( \left[ \int_{0}^t\Pr\{X_{N(t)+1} > x | S_{N(t)} = y, N(t)=n\}F^c(t-y)dm(y) \right] \Pr\{N(t)=n\} \right).
%	 	\end{align*}
%	 	Now we have,
%	 	\begin{align*}
%	 		\Pr\{X_{N(t)+1}>x | S_{N(t)}=y, N(t)=n\} & = \Pr\{X_{N(t)+1}>x | X_1+\cdots+X_n=y, X_{n+1}>t-y\} \\
%	 		& = \Pr\{X_{n+1}>x | X_{n+1}>t-y\} \\
%	 		& = \frac{\Pr\{X_{n+1}>\text{max}(x,t-y)\}}{\Pr\{X_{n+1}>t-y\}} \\
%	 		& \geq F^c(x). 
%	 	\end{align*}
%	 	So we get that,
%	 	\begin{align*}
%	 		\Pr\{X_{N(t)+1}>x\}\geq \Pr\{X_1>x\}.
%	 	\end{align*}
%Thus the remark follows.\\
\begin{lem}
Renewal reward theorem applies to a reward process $R(t)$ that accrues reward continuously over a renewal duration. 
The total reward in a renewal duration $X_n$ remains $R_n$ as before, with the sequence$\{(X_n,R_n): n\in \N\}$ being \textit{iid}. 
\end{lem}
\begin{proof}
Let the process $R(t)$ denote the accumulated reward till time $t$, when the reward accrual is continuous in time. 
Then, it follows that %$ (2) ~$ $R(t)$ is the gradual reward during a cycle, 
\begin{align*}
\frac{\sum_{n=1}^{N(t)}R_n}{t} &\leq  \frac{R(t)}{t} \leq \frac{\sum_{n=1}^{N(t)+1}R_n}{t}.
\end{align*}
Result follows from application of strong law of large numbers. 
%	 	\textit{Analysis:} The part 1 of the theorem \ref{theorem} under this regime follows since
%	 	\begin{align*}
%	 		\lim_{t \to \infty} \frac{\sum_{n=1}^{N(t)}R_n}{t} = \frac{\E\left[R\right]}{\E\left[X\right]},\\
%	 		\lim_{t \to \infty} \frac{\sum_{n=1}^{N(t)+1}R_n}{t} = \frac{\E\left[R\right]}{\E\left[X\right]},
%	 	\end{align*}
%	 	by the similar arguments given in the proof of the theorem \ref{theorem}.\\
%	 	The part 2 of the theorem \ref{theorem} under this regime follows since
%	 	\begin{align*}
%	 		\lim_{t \to \infty} \frac{\E\left[R_{N(t)+1}\right]}{t} = 0,
%	 	\end{align*}
%	 	by the similar arguments given in the proof of the theorem \ref{theorem}. Thus the remark follows. For more insights refer Chapter 3 in \textit{Stochastic Processes} by \textit{Sheldon M. Ross}.
\end{proof} 

\subsection{Limiting empirical average of age and excess times}
To determine the average value of the age of a renewal process, consider the following gradual reward process. 
We assume the reward rate to be equal to the age of the process at any time $t$, 
and 
\eq{
R(t) &= \int_{0}^{t}A(u)du.
}
Observe that age is a linear increasing function of time in any renewal duration. 
In $n$th renewal duration, it increases from $0$ to $X_n$, and the total reward $R_n = X_n^2/2$. 
%That is, at time $t$, we are paid at rate $A(t)$ and so the total earning by time $s$ is $\int_0^s A(t)dt.$ 
Hence, we obtain from the renewal reward theorem 
\begin{align*}
\lim_{t \to \infty}\frac{1}{t}\int_0^t A(u)du = \frac{\E R_n}{\E X_n} = \frac{\E X^2}{2\E X}.
\end{align*}
%Now, since reward per cycle is $\int_0^X t dt = \frac{X^2}{2}$, we have that
%\begin{align*}
%\lim_{s\to \infty}\frac{\int_0^s A(t)dt}{s} = \frac{\E[X^2]}{2\E[X]}.
%\end{align*}
A similar analysis to calculate the average excess time, where the reward per cycle is $\int_0^X (X-t)dt$ gives 
\begin{align*}
\lim_{t\to \infty}\frac{1}{t}\int_0^t Y(u)du = \frac{\E[X^2]}{2\E[X]}.
\end{align*}
Since $X_{N(t)+1}=A(t)+Y(t)$, we see that its average value is given by
\begin{align*}
\lim_{t\to \infty}\frac{1}{t}\int_0^t X_{N(u)+1}du = \frac{\E[X^2]}{\E[X]}.
\end{align*}
It can be shown, under certain regularity conditions, that 
\begin{align*}
\lim_{t\to \infty} \E[R_{N(t)+1}] = \frac{\E[R_1 X_1]}{\E[X_1]}.
\end{align*}
Defining a cycle reward to equal the cycle length, we have 
\begin{align*}
\lim_{t\to \infty} \E[X_{N(t)+1}] = \frac{\E[X^2]}{\E[X]}.
\end{align*}
We see that this limit is always greater than $\E[X]$, except when $X$ is constant. Such a result was to be expected in view of the inspection paradox. 
\end{document}
\subsection{Stationary probability and empirical average}
\begin{thm}
For an alternative renewal process $W = \{W(t) \in \{0,1\}: t \geqslant 0\}$ the stationary probability of being on is same as the limiting average time spent in the on duration if the renewal duration has finite mean. 
That is,
\eq{
\lim_{t\to\infty}P\{W(t) = 1\} &= \lim_{t\to\infty}\frac{1}{t}\int_{0}^{t}W(u)du.
}
\end{thm}
\begin{proof}
%\begin{shaded*}
Suppose for an alternating renewal process, we earn at a unit rate in on state.  
The aggregate reward in one renewal duration $X_n$ is the on time $Z_n$ in that duration. 
\eq{
 \lim_{t \to \infty} \frac{\text{Amount of on time in } [0,t]}{t} = \lim_{t \to \infty} \frac{R(t)}{t}=\frac{\E[Z_n]}{\E[X_n]} = \lim_{t \to \infty}P(\text{on at time t}).
 } 
 \end{proof}
 %\end{shaded*}
