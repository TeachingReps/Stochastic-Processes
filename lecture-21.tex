% !TEX spellcheck = en_US
% !TEX spellcheck = LaTeX
\documentclass[a4paper,10pt,english]{article}
\usepackage{%
	amsfonts,%
	amsmath,%	
	amssymb,%
	amsthm,%
%	babel,%
	bbm,%
	%biblatex,%
	caption,%
	centernot,%
	color,%
	enumerate,%
	epsfig,%
	epstopdf,%
	etex,%
	geometry,%
	graphicx,%
	hyperref,%
	latexsym,%
	mathtools,%
	multicol,%
	pgf,%
	pgfplots,%
	pgfplotstable,%
	pgfpages,%
	proof,%
	psfrag,%
	subfigure,%	
	tikz,%
	ulem,%
	url%
}	

\usepackage[mathscr]{eucal}
\usepgflibrary{shapes}
\usetikzlibrary{%
  arrows,%
  backgrounds,%
  chains,%
  decorations.pathmorphing,% /pgf/decoration/random steps | erste Graphik
  decorations.text,%
  matrix,%
  positioning,% wg. " of "
  fit,%
  patterns,%
  petri,%
  plotmarks,%
  scopes,%
  shadows,%
  shapes.misc,% wg. rounded rectangle
  shapes.arrows,%
  shapes.callouts,%
  shapes%
}

\theoremstyle{plain}
\newtheorem{thm}{Theorem}[section]
\newtheorem{lem}[thm]{Lemma}
\newtheorem{prop}[thm]{Proposition}
\newtheorem{cor}[thm]{Corollary}

\theoremstyle{definition}
\newtheorem{defn}[thm]{Definition}
\newtheorem{conj}[thm]{Conjecture}
\newtheorem{exmp}[thm]{Example}
\newtheorem{assum}[thm]{Assumptions}
\newtheorem{axiom}[thm]{Axiom}

\theoremstyle{remark}
\newtheorem{rem}[thm]{Remark}
\newtheorem{note}[thm]{Note}

\newcommand{\norm}[1]{\left\lVert#1\right\rVert}
\newcommand{\indep}{\!\perp\!\!\!\perp}
\DeclarePairedDelimiter\abs{\lvert}{\rvert}%
%\DeclarePairedDelimiter\norm{\lVert}{\rVert}%
\newcommand{\tr}{\operatorname{tr}}
\newcommand{\R}{\mathbb{R}}
\newcommand{\Q}{\mathbb{Q}}
\newcommand{\N}{\mathbb{N}}
\newcommand{\E}{\mathbb{E}}
\newcommand{\Z}{\mathbb{Z}}
\newcommand{\B}{\mathscr{B}}
\newcommand{\C}{\mathcal{C}}
\newcommand{\T}{\mathscr{T}}
\newcommand{\F}{\mathcal{F}}
\newcommand{\G}{\mathcal{G}}
%\newcommand{\ba}{\begin{align*}}
%\newcommand{\ea}{\end{align*}}

% Debug
\newcommand{\todo}[1]{\begin{color}{blue}{{\bf~[TODO:~#1]}}\end{color}}


\makeatletter
\def\th@plain{%
  \thm@notefont{}% same as heading font
  \itshape % body font
}
\def\th@definition{%
  \thm@notefont{}% same as heading font
  \normalfont % body font
}
\makeatother
\date{}
\title{Lecture 19 : Martingales}
\author{}
\begin{document}
\maketitle
\section{Martingales}
%A martingale is a type of stochastic process whose definition formalizes the concept of a fair game.
\begin{defn}
A stochastic process $\{Z_n,~n \in \N \}$ is said to be a \textbf{martingale} if 
\begin{enumerate}
\item $\E[|Z_n|]< \infty$, ~ \text{for all}~ n.
\item $\E[Z_{n+1}|Z_1,Z_2, \hdots Z_n]=Z_n$.
\end{enumerate}
If the equality in second condition is replaced by $\leq$ or $\geq$, then the process is called \textbf{supermartingale} or \textbf{submartingale}, respectively.
\end{defn}
\begin{rem} Taking expectation on both sides of part 2 of the above definition, we get $\E[Z_{n+1}]=\E[Z_n]$, and hence $\E[Z_{n+1}]=\E[Z_1]$, for all $n$.
\end{rem}
\begin{exmp}[Simple random walk]
Let $\{X_i\}$ be a sequence of independent random variables with mean $0$. Let $Z_n=\sum_{i=1}^n X_i$. Then, $\{Z_n,~n \in \N \}$ is a martingale. This is so because, $\E[Z_n]=0$ and 
\begin{align*}
\E[Z_{n+1}|Z_1,Z_2 \hdots Z_n] =\E[Z_{n}+X_{n+1}|Z_1,Z_2 \hdots Z_n]
%&=\E[Z_{n}|Z_1,Z_2 \hdots Z_n]+\E[X_{n+1}|Z_1,Z_2 \hdots Z_n]\\
&=Z_n.
\end{align*} 
\end{exmp}
\begin{exmp}
Let $\{X_i\}$ be a sequence of independent random variables with mean $1$. Let $Z_n=\Pi_{i=1}^n X_i$. Then, $\{Z_n,~n \in \N \}$ is a martingale. This is so because, $\E[Z_n]=1$ and 
\begin{align*}
\E[Z_{n+1}|Z_1,Z_2 \hdots Z_n] =\E[Z_{n}X_{n+1}|Z_1,Z_2 \hdots Z_n]
%&=Z_n\E[X_{n+1}|Z_1,Z_2 \hdots Z_n]\\
%&=Z_n\E[X_{n+1}]\\
&=Z_n.
\end{align*} 
\end{exmp}
\begin{exmp}[Branching Process] 
Let $\{X_n\}$ be a branching process. Let $X_0=1$. Then,
\begin{align*}
X_n = \sum_{i=1}^{X_{n-1}}Z_i,
\end{align*}
where $Z_i$ represents the number of offspring of the $i^{\text{th}}$ individual of the $(n-1)^{\text{st}}$ generation. conditioning on $X_{n-1}$ yields, $\E[X_n]= \mu^n$ where $\mu$ is the mean number of offspring per individual. Then $\{Y_n = X_n / \mu^n: n \in \N\}$ is a martingale because $\E[Y_n]= 1$ and 
\begin{align*}
\E[Y_{n+1}|Y_1, \hdots Y_n] %&= \frac{1}{\mu^{n+1}}\E[X_{n+1}|Y_1, \hdots Y_n]\\
&= \frac{1}{\mu^{n+1}}\E[\sum_{i=1}^{X_{n}}Z_i|Y_1, \hdots Y_n]
%&=  \frac{1}{\mu^{n+1}}X_{n}\E[Z_i]
= \frac{X_n}{\mu^n}=Y_n.
\end{align*}
\end{exmp}
\begin{exmp} [Doob's Martingale]
Let $X,Y_1,Y_2 \hdots$ be arbitrary random variables such that $\E[|X|]< \infty$. Then
\begin{align*}
Z_n =\E[X|Y_1,Y_2, \hdots Y_n]
\end{align*}
is a martingale. The integrability condition can be directly verified, and
\begin{align*}
\E[Z_{n+1}|Y_1,Y_2, \hdots Y_n]&= \E[\E[X|Y_1,\hdots Y_{n+1}]|Y_1,\hdots Y_{n}] = \E[X|Y_1,\hdots Y_{n}]]=Z_n.
\end{align*} 
%Thus the result follows. The above martingale is called the Doob type martingale.
\end{exmp}
\begin{exmp}
For any sequence of random variables $X_1,X_2 \hdots $, the random variables $X_i-\E[X_i|X_1 \hdots X_{i-1}]$ have zero mean. Define
\begin{align*}
Z_n =\sum_{i=1}^n X_i -\E[X_i|X_1,X_2, \hdots X_{i-1}] 
\end{align*}
 is  a martingale provided $\E[|Z_n|]< \infty$.  To verify the same, 
 \begin{align*}
\E[Z_{n+1}|Z_1 \hdots Z_n]&= \E[Z_n+X_n-\E[X_n|X_1 \hdots X_{n-1}]]\\
&= Z_n+\E[X_n-\E[X_n|X_1 \hdots X_{n-1}]]=Z_n.\\
\end{align*}
\end{exmp}
\subsection{Stopping Times}
\begin{defn}The positive integer values, possibly infinite, random variable $N$ is said to be a \textbf{random time} for the process $\{Z_n\}$ if the event $\{N=n\}$ is determined by the random variables $Z_1 \hdots Z_n$. If $\Pr\{N < \infty\}=1$, then the random time $N$ is said to be a \textbf{stopping time}. 
\end{defn}
\begin{defn}A \textbf{predictable sequence} $\{H_n:n\in \N\}$ for process $\{X_n\}$ is the one where $H_n$ is completely determined by $X_1,X_2,...,X_{n-1}$
\begin{align*}
    \sum_{m=1}^{n}H_m(X_m-X_{m-1})=(H\cdot X)_n
\end{align*}
\end{defn}
\begin{thm}
Let $\{X_n,n\geq 0\}$ be a super martingale if $\{H_n \geq 0: n \in \N\}$ is predictable and each $H_n$ is bounded then $(H \cdot X)_n$ is  super martingale
\end{thm}
\begin{proof}
\begin{align*}
\E[(H\cdot X)_{n+1}|(H\cdot X)_1 ... (H \cdot X)_n]=&\E[H_{n+1}(X_{n+1}-X_n)+(H\cdot X)_n|(H\cdot X)_1 ... (H \cdot X)_n]\\
=&H_{n+1}(\E[X_{n+1}|(H\cdot X)_1 ... (H \cdot X)_n]-X_n)+(H\cdot X)_n\\
\leq&(H\cdot X)_n
\end{align*}
\end{proof}
\begin{defn}
Let $T$ be a random time for the process $\{X_n:n\in \N\}$, then \textbf{stopping  process} $\{X_{T\wedge n}\}$ is defined as 
\begin{align*}
X_{T\wedge n} = X_n1_{\{n \leq T\}} + X_T1_{\{n > T\}}.
\end{align*}
\end{defn}
\begin{prop}
If $T$ is a random time for the martingale $\{X_n:n\in \N\}$, then the stopping process $\{X_{T\wedge n}\}$ is a martingale.
\end{prop}
\begin{proof}
\begin{align*}
    X_{T\wedge n}&= (H\cdot X)_n\hspace{2 mm} when\hspace{2 mm} H_n=1_{\{n\leq T\}}\\
    X_{T \wedge n}&= X_{T\wedge n-1}+1_{\{n \leq T\}}(X_n-X_{n-1})\\
\end{align*}
 $n\leq T$:\hspace{2 mm}  $X_{T\wedge n}= X_n$\\
 $n>T$: Since $n>T$ gives $n-1\geq T$ therefore $T\wedge n-1\geq T$ which implies $X_{T\wedge n}=X_{T}$ \\
 \begin{align*}
     X_{T\wedge n}= X_0 +  \sum_{m=1}^{n} 1_{\{m\leq T\}}(X_m-X_{m-1})
 \end{align*}
     It is suffice to show $\{1_{n\leq T}\}$ is a predictable sequence which is true since 
  \begin{align*}
  \{n\leq T\}=\{T>n-1\}=\{T<n-1\}^c
  \end{align*}
  Therefore from the previous theorem we have 
  \begin{align*}
      \E[X_{T\wedge n}]=\E[X_{T\wedge 1}]=\E[X_{1}]
  \end{align*}


%We claim that 
%\begin{align*}
%\bar{Z}_n= \bar{Z}_{n-1}+1_{\{n \leq N\}}(Z_n-Z_{n-1})
%\end{align*}
%The above equation can be directly verified by considering the two cases separately by
%\begin{align*}
%\bar{Z}_n &= 
%\begin{cases}
%Z_n & n \leq N,\\
%\bar{Z}_{n-1}=Z_N, & n > N.
%\end{cases}
%\end{align*}
%Further, since $N$ is a random time, we see that 
%\begin{align*}
%\E[\bar{Z}_{n+1}|Z_1 \hdots \bar{Z}_n] %&=\E[\bar{Z}_{n}+1_{\{n \leq N\}}(Z_n-Z_{n-1})|Z_1 \hdots \bar{Z}_n]\\
%&=\bar{Z}_{n}+1_{\{n \leq N\}} \E[(Z_n-Z_{n-1})|Z_1 \hdots \bar{Z}_n] =\bar{Z}_{n}.
%\end{align*}
\end{proof}

\begin{rem}
For any martingale $\{X_n: n \in \N\}$, we have $\E[X_{T\wedge n}]=\E[X_1]$, for all $n$.  Now assume that $T$ is a stopping time. It is immediate that 
\begin{align*}
\Pr\left\{\lim_{n\in \N}{X}_{T\wedge n} = X_T\right\} = 1.
\end{align*}
\end{rem}
But  is it true that
\begin{align*}
\lim_{n \in \N}\E[X_{T\wedge n}] = \E[X_T]?
\end{align*}
It so turns out that the above is true under some additional regularity constraints only. %We state the following theorem without proof.
\begin{thm}[Martingale Stopping Theorem]
\label{MartStopThm}
If $T$ is a stopping time for a martingale $\{X_n: n \in \N\}$ such that either of the following conditions is true:
\begin{enumerate}[(i)]
\item $T$ is bounded, 
\item $X_{T\wedge n}$ is uniformly bounded,
\item $\E[T] < \infty$, and for some real positive $K$, we have $\sup_{n \in \N}\E[|X_{n+1}-X_n||X_1 \hdots X_n] < K$,
\end{enumerate}
then $X_T$ is integrable and $\lim_{n \in \N}\E[X_{T\wedge n}] = \E[X_T]=\E[X_1]$.
\end{thm}
\begin{proof} We show this is true for all three cases.
\begin{enumerate}[(i)] 
\item Let $K$ be the bound on $T$ then for all $n \geq K$, we have $X_{T\wedge n} = X_T$, and hence it follows that
\begin{align*}
\E X_1 = \E X_{T\wedge n} &= \E X_T, ~\forall n \geq K.
\end{align*}
\item Dominated convergence theorem implies the result. 
\item Since $T$ is integrable and  
\begin{align*} 
X_{T\wedge n} \leq |X_1| + K T,
\end{align*}
we observe that $X_{T\wedge n}$ is bounded by an integrable random variable, and hence result follows from dominated convergence theorem.
\end{enumerate}
\end{proof}
\begin{cor}[Wald's Equation] If $T$ is a stopping time for $\{X_i,~ i \in \N\}$ \textit{iid} with $\E[|X|]< \infty$ and $\E[T]< \infty$, then
\begin{align*}
\E[\sum_{i=1}^{T}X_i]=\E[T]\E[X].
\end{align*}
\end{cor}
\begin{proof}
Let $\mu=\E[X]$. Then $\{Z_n = \sum_{i=1}^{n}(X_i-\mu): n \in \N\}$ is a martingale and hence from the Martingale stopping theorem, we have $\E[Z_T]=\E[Z_1]=0$. But 
\begin{align*}
\E[Z_T] %&= \E[\sum_{i=1}^T (X_i-\mu)]\\
%&=\E[\sum_{i=1}^N (X_i)-N\mu)]\\
&=\E\sum_{i=1}^N X_i- \mu\E N.
\end{align*}
Observe that condition $3$  for Martingale stopping theorem to hold can be directly verified. Hence the result follows. 
\end{proof}
%\section{ Submartingales, supermartingales and the Martingale Convergence Theorem}
%\begin{defn}
%A stochastic process $\{Z_n,~  n \geq 1\}$ having $\E[|Z_n|]< \infty$ for all $n$ is said to be a submartingale if
%\begin{equation}
%\label{Submartingale}
%\E[Z_{n+1}|Z_1 \hdots Z_n] \geq Z_n
%\end{equation}
%and is said to be a supermartingale if
%\begin{equation}
%\label{supermartingale}
%\E[Z_{n+1}|Z_1 \hdots Z_n] \leq Z_n
%\end{equation}
%\end{defn}
%From \ref{Submartingale}, for a submartingale
%\begin{align*}
%\E[Z_{n+1}] \geq \E[Z_n]
%\end{align*}
%where the inequality is reversed for a supermartingale. 
%\begin{thm}
%\label{Stoppingtime_theorem}
%If $N$ is a stopping time for $\{Z_n,~ n\geq 1\}$ such that any one of the following sufficient conditions is satisfied:
%\begin{enumerate}
%\item $\bar{Z}_n$ is uniformly bounded, or;
%\item $N$ is bounded, or;
%\item $\E[N]< \infty$, and there is an $M < \infty$ such that
%\begin{align*}
%\E[|Z_{n+1}-Z_n| |Z_1, \hdots Z_n]<M,
%\end{align*}
%then,
%\begin{align*}
%\E[Z_N] \geq \E[Z_1] ~ \text {for a submartingale}\\
%\E[Z_N] \leq \E[Z_1] ~ \text {for a supermartingale}.
%\end{align*}
%\end{enumerate}
%\end{thm}
%\begin{proof}
%We claim that 
%\begin{align*}
%\bar{Z}_n= \bar{Z}_{n-1}+1_{N \geq n}(Z_n-Z_{n-1})
%\end{align*}
%The above equation can be directly verified by considering the two cases separately viz. 
%\begin{enumerate}
%\item $N \geq n$: $\bar{Z}_n=Z_n$.
%\item $N < n:$ $\bar{Z}_{n-1}=\bar{Z}_{n}=Z_N$
%\end{enumerate}
%\begin{align*}
%\E[\bar{Z}_{n+1}|Z_1 \hdots \bar{Z}_n]&=\E[\bar{Z}_{n}+1_{n \leq N}(Z_n-Z_{n-1})|Z_1 \hdots \bar{Z}_n]\\
%&\stackrel{(a)}{=}\bar{Z}_{n}+1_{n \leq N} \E[(Z_n-Z_{n-1})|Z_1 \hdots \bar{Z}_n]\\
%& \geq \bar{Z}_{n},
%\end{align*}
%where in $(a)$ we have used the fact that $N$ is a random time. Also, we have $\E[\bar{Z}_{n}]=\E[Z_1]$, for all $n$.  Now assume that $N$ is a stopping time. It is immediate that
%\begin{align*}
% \bar{Z}_n \rightarrow Z_N ~ \text{w.p}~ 1.
%\end{align*}
%But  is it true that
%\begin{align*}
% \E[\bar{Z}_n] \rightarrow \E[Z_N] ~ \text{as n}~ \rightarrow \infty.
%\end{align*}
%which gives that 
%\begin{align*}
%\E[Z_N ] \geq \E[Z_1].
%\end{align*}
%\end{proof}

Before we state and prove martingale convergence theorem, we state some results which will be used in the proof of the theorem.
\begin{lem}
\label{StoppingTimeBound}
If $\{Z_i, i \in \N \}$ is  a submartingale and $T$ is a stopping time such that $\Pr\{T \leq n\}=1$ then
\begin{align*}
 \E Z_1 \leq \E Z_T \leq \E Z_n.
\end{align*}
\end{lem}
\begin{proof}
It follows from Theorem \ref{MartStopThm} that since $T$ is bounded, $\E[Z_T] \geq \E[Z_1]$. Now, since $T$ is a stopping time, we see that for $\{T = k\}$
\begin{align*}
\E[Z_n|Z_1, \hdots ,Z_T,T=k]&=\E[Z_n|Z_1 \hdots Z_k,T=k] = \E[Z_n|Z_1 \hdots Z_k] \geq Z_k = Z_T.
\end{align*}
%where $(a)$ follows from the fact that $N$ is  a stopping time. 
Result follows by taking expectation on both sides.
\end{proof}
\begin{lem}
\label{ConvexFuncSubmart}
If $\{Z_n,n \in \N\}$ is a martingale and $f$ is a convex function, then $\{f(Z_n),n \in \N\}$ is a submartigale.
\end{lem}
\begin{proof}
The result is a direct consequence of Jensen's inequality.
\begin{align*}
\E[f(Z_{n+1})|Z_1, \hdots Z_n] &\geq f(\E[Z_{n+1}|Z_1, \hdots Z_n])=f(Z_n).
\end{align*}
\end{proof}
\begin{con}
	Let ${X_n,n\geq0}$ be a sub martingale. Let $a< b$ and $N_0  = -1$.
	   \begin{align*}
		N_{2k-1}=&\inf\{m>N_{2k-2}:X_m\leq a \}.\\
		N_{2k}=&\inf\{m>N_{2k-1}:X_m\geq b \}.\\
		\end{align*}
		The above quantities $N_{2k-1}$,$N_{2k}$ are stopping times and the set containing values of $m$ in the transition from $a$ to $b$ can be defined as
		\begin{align*}
		\{N_{2k-1}<m\leq N_{2k} \}=&\{N_{2k-1}<m-1\} \cap \{m>N_{2k}\}^{c}\\
		=&\{m-1\geq N_{2k-1}\}\cap\{m-1\geq N_{2k}\}^{c}\\		
		\end{align*}
		Since the above set depends on $\{m-1\}$ values instead of $\{m\}$ values, So
		\begin{align*}
			H_m=&1_{\{N_{2k-1}<m\leq  N_{2k}\}}\\
			U_n=&\sup\{k:N_{2k}\leq n\}
		\end{align*}
		 $H_m$ defines a predictable sequence and $U_n$ is the number of up crossings completed in time $n$.
\end{con}
\begin{lem}[Upcrossing inequality]
	If \{$X_m: m \in \N_0$\} is a sub martingale,then
	\begin{align*}
	(b-a)\E U_n\leq& \E Y_n-\E Y_0\\
	\text{where }Y_n&:=a+ (X_n-a)^{+}
    \end{align*}
    \end{lem}
\begin{proof}
Since $X_n$ is a submartingale so is $Y_n$, as it is  a convex function of $X_n$.Since each up crossing has a gain slightly  more than $b-a$ the following inequality exists \\
\begin{align*}
(b-a)U_n\leq &(H\cdot Y)_n\\
=&\sum_{m=1}^{n}1_{\{N_{2k-1}<m\leq N_{2k}\}}(Y_{m+1}-Y{m})\\
=&\sum_{k=1}^{U_n}(Y_{N_{2k+1}}-Y_{N_{2k+1}})\\ 
\end{align*}
   Now let $K_m=1-H_m$ then clearly,
   \begin{align*}
   	Y_n-Y_0=(H\cdot Y)_n+(K\cdot Y)_n\\
   \end{align*} 
   we have by the submartingale property of Y\\
    \begin{align*}
    	\E[(K\cdot Y)_n] \geq \E[(K\cdot Y)_0]=0\\   	
    \end{align*} 
    Therefore,
    \begin{align*}
    \E[(H\cdot Y)_n]+\E[(K\cdot Y)_n]=& \E[Y_n-Y_0]\\
    \E[(H\cdot Y)_n]\leq& \E[Y_n-Y_0]\\
    (b-a)\E U_n\leq& \E(Y_n-Y_0).
    \end{align*}
    \end{proof}
\begin{thm}[Martingale Convergence Theorem]
\label{MartingaleConvergenceTheorem}
If $X_n$ is a submartingale with ${\sup}_{a\in A} \E[X_n^{+}]\leq \infty$ then ${\lim}_{n\in \N} X_n=X$ a.s with $\E[X]<\infty$.
\end{thm}
\begin{proof}    		
\begin{align*}
(X-a)^{+}\leq& X^{+}+|a|,\\
\E[U_n] \leq& \frac{\E[X_n^{+}] +|a|}{b-a}.
\end{align*}
    			$\lim_{n \in \N} U_n=U$ since, $\E [X_n^{+}] < \infty$ gives $U<\infty$ a.s. This conclusion leads to  
    			\begin{align*}
    			\Pr\{_{a,b \in \mathbb{Q}}{\cup}\{{\lim \inf}_{n\in \N}  X_n<a<b<{\lim \sup}_{n\in \N}  X_n \}\} = 0.
    			\end{align*}
    		  From the above probability we have a. s.
    		  \begin{align*}
    		  	{\lim \sup}_{n\in \N}  X_n &={\lim \inf}_{n\in \N}  X_n.
    		  	\end{align*}
    		  	Now the Fatou's lemma in measure theory guarantees
    		  	\begin {align*}
    		  	\E[X^{+}]\leq {\lim \inf}_{n\in \N}\E[X_n^{+}]<\infty,
    		  \end{align*}
    		  	which implies $X<\infty$  almost sure. To see $X>-\infty$, we observe that
    		  	\begin{align*}
    		  		\E[X_n^{-}]&=\E[X_n^{+}]-\E[X_n]\\
    		  		&\leq \E[X_n^{+}]-\E[X_0].
    		  		\end{align*} 
    		  		The above inequality comes from the submartingale property of $X_n$. Now from another application of Fatou's lemma gives,
    		  		\begin{align*}
    		  		\E[X^{-}]\leq {\lim \inf} _{n\in \N}\E[X_n^{-}] \leq {\sup}_{n\in \N}\E[X_n^{+}]-\E[X_0]<\infty.
    		  		\end{align*}
    		\end{proof}

%\begin{thm}[Martingale Convergence Theorem]
%\label{MartingaleConvergenceTheorem}
%If $\{Z_n,~n \geq 1\}$ is a martingale such that for some $M< \infty$
%\begin{align*}
%\E[|Z_n|] \leq M, ~ \text{for all}~ n
%\end{align*}
%then, with probability 1, $\lim_{n \rightarrow \infty}Z_n$ exists and is finite.
%\end{thm}
%\begin{proof}
%Assume $\E[Z_n^2]< \infty$ which is stronger than $\E[|Z_n|]< \infty$ (as a consequence of %Jensen's inequality). Observe that $\{Z_n^2\}$ is a submartingale (from Lemma %\ref{ConvexFuncSubmart}). Thus $\E[Z_n^2]<\infty$ and is non-decreasing in $n$. Thus, as $n %\rightarrow \infty$, $\E[Z_n^2]$ converges and let $\mu<\infty$ be given by $\mu=\lim_{n %\rightarrow \infty}\E[Z_n^2]$.
%\begin{equation}
%\label{KolmoBound}
%\Pr(\cup_{k \leq n} \{|Z_{m+k}-Z_m|> \epsilon\} )
%\end{equation}  
%\begin{align*}
%&\stackrel{(a)}{\leq }\E[(Z_{m+n}-Z_m)^2]/\epsilon^2
%&=\E[Z_{m+n}^2-2Z_mZ_{m+n}+Z_m^2]/\epsilon^2.
%\end{align*}
%Note that 
%\begin{align*}
%\E[Z_{m+n}Z_m]&=\E[\E[Z_mZ_{m+n}|Z_m]]\\
%&=\E[Z_m\E[Z_{m+n}|Z_m]]\\
%&=\E[Z_m^2].
%\end{align*}
%From \ref{KolmoBound}, 
%\begin{align*}
%\Pr(\cup_{k \leq n} \{|Z_{m+k}-Z_m|> \epsilon\}) \leq \frac{\E[Z_{m+n}^2]-\E[Z_m^2]}{\epsilon^2}.
%\end{align*}
%Letting $n \rightarrow \infty$
%\begin{align*}
%\Pr(\cup_{k \leq 1} \{|Z_{m+k}-Z_m|> \epsilon\}) \leq \frac{\mu-\E[Z_m^2]}{\epsilon^2}.
%\end{align*}
%Hence,
%\begin{align*}
%\Pr(\cup_{k \leq n} \{|Z_{m+k}-Z_m|> \epsilon\}) \rightarrow 0 ~\text{as}~ m \rightarrow \infty.
%\end{align*}
%Thus with probability 1, $\{Z_n\}$ will be  a Cauchy sequence, and thus $\lim_{n \rightarrow %\infty}Z_n$ will exist and be finite.`
%\end{proof}
%\begin{cor}
%If $\{Z_n,~m \geq 0\}$ is a non-negative martingale, then, with probability 1, $\lim_{n %\rightarrow \infty}Z_n$ exists and is finite.
%\end{cor}
%\begin{proof}
%Since $Z_n$ is non-negative,
%\begin{align*}
%\E[|Z_n|]=\E[Z_n]=\E[Z_1].
%\end{align*}
%\end{proof}

\end{document}