\documentclass[a4paper,english,10pt]{article}
\usepackage{%
	amsfonts,%
	amsmath,%	
	amssymb,%
	amsthm,%
%	babel,%
	bbm,%
	%biblatex,%
	caption,%
	centernot,%
	color,%
	enumerate,%
	epsfig,%
	epstopdf,%
	etex,%
	geometry,%
	graphicx,%
	hyperref,%
	latexsym,%
	mathtools,%
	multicol,%
	pgf,%
	pgfplots,%
	pgfplotstable,%
	pgfpages,%
	proof,%
	psfrag,%
	subfigure,%	
	tikz,%
	ulem,%
	url%
}	

\usepackage[mathscr]{eucal}
\usepgflibrary{shapes}
\usetikzlibrary{%
  arrows,%
  backgrounds,%
  chains,%
  decorations.pathmorphing,% /pgf/decoration/random steps | erste Graphik
  decorations.text,%
  matrix,%
  positioning,% wg. " of "
  fit,%
  patterns,%
  petri,%
  plotmarks,%
  scopes,%
  shadows,%
  shapes.misc,% wg. rounded rectangle
  shapes.arrows,%
  shapes.callouts,%
  shapes%
}

\theoremstyle{plain}
\newtheorem{thm}{Theorem}[section]
\newtheorem{lem}[thm]{Lemma}
\newtheorem{prop}[thm]{Proposition}
\newtheorem{cor}[thm]{Corollary}

\theoremstyle{definition}
\newtheorem{defn}[thm]{Definition}
\newtheorem{conj}[thm]{Conjecture}
\newtheorem{exmp}[thm]{Example}
\newtheorem{assum}[thm]{Assumptions}
\newtheorem{axiom}[thm]{Axiom}

\theoremstyle{remark}
\newtheorem{rem}[thm]{Remark}
\newtheorem{note}[thm]{Note}

\newcommand{\norm}[1]{\left\lVert#1\right\rVert}
\newcommand{\indep}{\!\perp\!\!\!\perp}
\DeclarePairedDelimiter\abs{\lvert}{\rvert}%
%\DeclarePairedDelimiter\norm{\lVert}{\rVert}%
\newcommand{\tr}{\operatorname{tr}}
\newcommand{\R}{\mathbb{R}}
\newcommand{\Q}{\mathbb{Q}}
\newcommand{\N}{\mathbb{N}}
\newcommand{\E}{\mathbb{E}}
\newcommand{\Z}{\mathbb{Z}}
\newcommand{\B}{\mathscr{B}}
\newcommand{\C}{\mathcal{C}}
\newcommand{\T}{\mathscr{T}}
\newcommand{\F}{\mathcal{F}}
\newcommand{\G}{\mathcal{G}}
%\newcommand{\ba}{\begin{align*}}
%\newcommand{\ea}{\end{align*}}

% Debug
\newcommand{\todo}[1]{\begin{color}{blue}{{\bf~[TODO:~#1]}}\end{color}}


\makeatletter
\def\th@plain{%
  \thm@notefont{}% same as heading font
  \itshape % body font
}
\def\th@definition{%
  \thm@notefont{}% same as heading font
  \normalfont % body font
}
\makeatother
\date{}

%opening
\title{Lecture 23: Polya's Urn Scheme}
\author{}

\begin{document}
\maketitle
\section{Optional Sampling theorem}
The gambling interpretation of the stochastic integral suggests that it is natural to let the amount bet at time $n$ depend on the outcomes of the first $n-1$ flips but not on the flip we are betting on, or on later flips. The next result shows that we cannot make money by gamblingon a fair game.

\begin{thm}Let $X_n$ be a martingale. If $H_n$ is predictable and each $H_n$ is bounded, then $(H\cdot X)_n$ is a martingale.
\end{thm}
\begin{proof} It is easy to check that $(H\cdot X)_n \in \mathcal{F}_n$. Thhe boundedness of the $H_n$ implies $E|(H\cdot X)_n|<\infty$ for each $n$. With this established, we can compute conditional expectations to conclude
\begin{align*}
E((H\cdot X)_{n+1}| \mathcal{F}_n)&=(H\cdot X)_n+E(H_{n+1}(X_{n+1}-X_n)| \mathcal{F}_n)\\
       &=(HX)_n+H_{n+1}E(X_{n+1}-X_n| \mathcal{F}_n)\\
       &=(H\cdot X)_n.\qedhere
\end{align*}
since $H_{n+1}\in \mathcal{F}_n$ and $E(X_{n+1}-X_n| \mathcal{F}_n)=0$
\end{proof}
The last theorem can be interpreted as: you can't make money by gambling on a fair game. This conclusion does not hold if we only assume that $H_n$ is optional, that is $H_n\in \mathcal(F)_n$, since then we can base our bet on the outcome of the coin we are betting on.

\begin{thm} Suppose $M_0,M_1,\dots $ is a martingale with respect to$\{\mathcal{F}_n\}$ and suppose $T$ is a stopping time. Suppose that $T$ is bounded, $T\leq K$. Then
\begin{equation*}
\mathbb{E}(M_T|\mathcal{F}_0)=M_0.
\end{equation*}
In particular, $\mathbb{E}(M_T)=\mathbb{E}(M_0)$.
 \end{thm}
To prove this fact, we first note that the event $\{T>n\}$ ismeasurable with respect to $\mathcal{F}_n$ (since we need only the information up through time $n$ to determine if we have stopped by time $n$). Since $M_T$ is the random variable which equals $M_j$ if $T=j$ we can write
\begin{equation*}
M_T = \sum_{j=0}^K M_j I\{T=j\}.
\end{equation*}
Let us take the conditional expectation with respect to $\mathcal{F}_{K-1}$,
\begin{equation*}
\mathbb{E}(M_T|\mathcal(F)_{K-1})=\mathbb{E}(M_KI\{T=K\}|\mathcal{F}_{K-1})+\sum_{j=0}^{K-1} \mathbb{E}(M_jI\{T=j\}|\mathcal{F}_{K-1}).
\end{equation*}
For $j\leq K-1, M_jI\{T=j\}$ is $\mathcal{F}_{K-1}$- measurable; hence
\begin{equation*}
\mathbb{E}(M_jI\{T=j\}|\mathcal{F}_{K-1})=M_jI\{T=j\}.
\end{equation*}
Since $T$ is known to be no more than $K$, the event $\{T=K\}$ is the same as the event $\{T>K-1\}$. The latter event is measurable with respect to $\mathcal{F}_{K-1}$. Hence using equality
\begin{equation*}
\mathbb{E}(YZ|\mathcal{F}_n)= Z\mathbb{E}(Y|\mathcal{F}_n).
\end{equation*}
Where $Y$ is any random variable and $Z$ is a random variable that is measurable with respect to finite number of random variables $X_1,X_2,\dots,X_n$.

\begin{align*}
\mathbb{E}(M_KI\{T=K\}|\mathcal{F}_{K-1}) &=\mathbb{E}(M_KI\{T>K-1\}|\mathcal{F}_{K-1})\\
								   &=I\{T>K-1\}\mathbb{E}(M_K|\mathcal{F}_{K-1})\\
								   &=I\{T>K-1\}M_{K-1}.
\end{align*}

The last equality follows from the fact the $M_n$ is a martingale. Therefore,
\begin{align*}
\mathbb{E}(M_T|\mathcal{F}_{K-1}) &=I\{T>K-1\}M_{K-1}+\sum_{j=0}^{K-1}M_jI\{T=j\}\\
								   &=I\{T>K-2\}M_{K-1}+\sum_{j=0}^{K-2}M_jI\{T=j\}.
\end{align*}
If we work through this argument again, this time conditioning with respect to $\mathcal{F}_{K-2}$, we gat
\begin{align*}
\mathbb{E}(M_T|\mathcal{F}_{K-2}) &=\mathbb{E}(\mathbb{E}(M_T|\mathcal{F}_{K-1})|\mathcal{F}_{K-2})\\
								   &=I\{T>K-3\}M_{K-2}+\sum_{j=0}^{K-3}M_jI\{T=j\}.
\end{align*}
We can continue this process untill we get
\begin{equation*}
\mathbb{E}(M_T|\mathcal{F}_0)=M_0.
\end{equation*}

There are many examples of interest where the stopping time $T$ is not bounded. Suppose $T$ is a stopping time with $\mathbb{P}\{T<\infty\}=1$, i.e., a rule that guarantees that one stops eventually. (Note that the time associated to the martingale betting strategy satisfies this condition.) When can we conclude that $\mathbb{E}(M_T)=\mathbb{E}(M_0)?$ To investigate this consider the stopping times $T_n=min\{T,n\}$. Note that
\begin{equation*}
M_T=M_{T_{n}}+M_TI\{T>n\}-M_nI\{T>n\}.
\end{equation*}
Hence,
\begin{equation*}
\mathbb{E}(M_T)=\mathbb{E}(M_{T_{n}})+\mathbb{E}(M_TI\{T>n\})-\mathbb{E}(M_nI\{T>n\}).
\end{equation*}

Since $T_n$ is a bounded stopping time, it follows from the above that $\mathbb{E}(M_{T_{n}})=\mathbb{E}(M_0)$. We would like to be able to say that the other termsdo not contribute as $n\rightarrow \infty$. The second term is not much of a problem.  Since the probability of the event $\{T>n\}$ goes to 0 as $n\rightarrow \infty$, we are taking the expectation of the random variable $M_T$ restricted to a smaller and smaller set. One can show that if $\mathbb{E}(|M_T|)<\infty$ then $\mathbb{E}(|M_T|I\{T>n\}) \rightarrow 0$.

The third term, if $M_n$ and $T$ are given satisfying
\begin{equation*}
lim_{n\rightarrow \infty}\mathbb{E}(|M_n|I\{T>n\})=0,
\end{equation*}
then we will be able to conclude that $\mathbb{E}(M_T)=\mathbb{E}(M_0)$. We summarize this as follows.\newline
\textbf{Optional Sampling Theorem.} Suppose $M_0,M_1,\dots$ is a martingale with respect to $\{\mathcal{F}_n\}$ and $T$ is a stopping time satisying $\mathbb{P}\{T<\infty\}=1$,
\begin{equation*}
\mathbb{E}(|M_T|)<\infty,
\end{equation*}
and
\begin{equation*}
\lim_{n\rightarrow \infty}\mathbb{E}(|M_n|I\{T>n\})=0.
\end{equation*}
Then, $\mathbb{E}(M_T)=\mathbb{E}(M_0)$.

\section{Polya's Urn Scheme}
Suppose an urn initially contains $b_0$ black balls and $w_0$ white balls. Suppose balls are sampled from the urn one at a time, but after each draw $1$ balls of the same color are returned to the urn. If first draw is a black, then replace $b_0$ with $b_0+1$ balls in the urn and $w_0$ with $w_0+1$ for white balls. The number of black balls in the first $n$ draws would then have a $Bin(n,\frac{b_0}{b_0+w_0})$.
Let $B_n$ be the number of black balls in urn after $n$ draws and $B_0 = b_0$. Now probability of getting a black ball in first draw is
 \begin{equation*}
\mathbb{P}(B_1=b_0+1)= \frac{b_0}{b_0+w_0},
\end{equation*}
and probability of a getting white ball in first draw is
 \begin{equation*}
\mathbb{P}(B_1=b_0)= \frac{w_0}{b_0+w_0}.
\end{equation*}
Similarly after two draws,
 \begin{align*}
\mathbb{P}(B_2=b_0)= \frac{w_0}{b_0+w_0} \cdot \frac{w_0+1}{b_0+w_0+1}\\
\mathbb{P}(B_2=b_0+1)= \frac{b_0}{b_0+w_0} \cdot \frac{w_0}{b_0+w_0+1}+\frac{w_0}{b_0+w_0} \cdot \frac{b_0}{b_0+w_0+1}\\
\mathbb{P}(B_2=b_0)= \frac{b_0}{b_0+w_0} \cdot \frac{b_0+1}{b_0+w_0+1}.
\end{align*}
For first three draws,
 \begin{align*}
\mathbb{P}(\text{first 3 draws are } bwb)= \frac{b_0}{b_0+w_0} \cdot \frac{w_0}{b_0+w_0+1} \cdot \frac{b_0+1}{b_0+w_0+2}\\
\mathbb{P}(\text{first 3 draws are } bbw)= \frac{b_0}{b_0+w_0} \cdot \frac{b_0+1}{b_0+w_0+1} \cdot \frac{w_0}{b_0+w_0+2}\\
\mathbb{P}(\text{first 3 draws are } wbb)= \frac{w_0}{b_0+w_0} \cdot \frac{b_0}{b_0+w_0+1} \cdot \frac{b_0+1}{b_0+w_0+2},
\end{align*}
where $b$ stands for black ball and $w$ stands for white ball. We observe above 3 equations, all of them are same. like wise
 \begin{align*}
\mathbb{P}(bbwww)= \frac{b_0(b_0+1)w_0(w_0+1)(w_0+2)}{\prod_{i=0}^4 (b_0+w_0+i)}\\
\mathbb{P}(bwwwb)= \frac{b_0w_0(w_0+1)(w_0+2)(b_0+1)}{\prod_{i=0}^4 (b_0+w_0+i)}.
\end{align*}
Again above two probabilities are equal.
\begin{defn}
An infinite sequence $\{X_i\}_{i=1}^{\infty}$ of random variables is exchangeable if $\forall$ $n=1,2,\dots$
 \begin{equation*}
X_1,\dots,X_n=X_{\pi(1)},\dots ,X_{\pi(n)}, \forall \pi \in S(n),
\end{equation*}
where $S(n)$ is the symmetric group, the group of permutations.
\end{defn}
Polya's Urn Model is one of the examples for exchangeability. An Example would be following
 \begin{equation*}
\mathbb{P}(bbwww)=\mathbb{P}(bwwwb)
\end{equation*}
If $\xi_1,\xi_2,\dots,\xi_n$ denote the sigma algebra for the color of the drawn ball i.e., $\xi_i$ represents the color of the $i^{th}$ ball, from the definition of excangeability
\begin{equation*}
(\xi_1,\xi_2,\xi_3,\xi_4,\xi_5)=(\xi_2,\xi_1,\xi_5,\xi_4,\xi_3).
\end{equation*}
\begin{note}
Polya's Urn scheme generate exchangeable sequences.
\end{note}
Let 
\begin{equation*}
X_n=\frac{B_n}{B_n+W_n}=\frac{B_n}{b_0+w_0+n},\ 0\leq X_n \leq1,
\end{equation*}
represents the proportion of black balls after $n$ draws, then given the past $\xi_1,\xi_2,\dots,\xi_n$
\begin{equation*}
B_{n+1} =
\left\{
	\begin{array}{ll}
		B_n  & w.p \ (1-\frac{B_n}{b_0+w_0+n}) \\
		B_{n+1} & if \ \xi_{n+1} \ w.p \ \frac{B_n}{b_0+w_0+n}.
	\end{array}
\right.
\end{equation*}
Now
\begin{align*}
\mathbb{E}[X_{n+1}|\xi_1,\xi_2,\dots\xi_n]&=\frac{1}{b_0+w_0+n+1}\mathbb{E}[B_{n+1}|\xi_1,\xi_2,\dots\xi_n]\\
								&=\frac{1}{b_0+w_0+n+1}\mathbb{E}[B_n(1-X_n)+(B_n+1)X_n]\\
								&=\frac{B_n+X_n}{b_0+w_0+n+1}\\
								&=X_n.
\end{align*}
That means it is a martingale.
\begin{note}
$X_n$ is a martingale.
\end{note}
\subsection{Analysis of the Polya urn model}
\begin{thm}(De Finetti 1931)
A binary sequence $\{X_n\}_{i=1}^{\infty}$ is exchageable iff there exixtes a distribution function $F(p)$ on $[0,1]$ such that for any $n\geq1$,
\begin{equation*}
\mathbb{P}(X_1=x_1,\dots,X_n=x_n)=\int_0^1 p^{S_n}(1-p)^{n-S_n}dF(p)
\end{equation*}
where $S_n=\sum_i x_i$.
\end{thm}
The distribution $F$ is a function of the limiting frequency
\begin{equation*}
Y=\bar{X}_{\infty}=\lim_{n \to \infty}\frac{\sum_i X_i}{n},  \mathbb{P}(Y\leq p)=F(p),
\end{equation*}
and conditioning on $Y=p$ results in iid Bernoulli draws
\begin{equation*}
\mathbb{P}(X_1=x_1,\dots,X_N=x_n|Y=p)= p^{S_n}(1-p)^{n-S_n},
\end{equation*}
and for the Polya urn model
\begin{equation*}
\lim_{n\to \infty} \bar{X}_n=Y~Beta\left(\frac{B_0}{B_0+W_0},\frac{W_0}{B_0+W_0} \right)
\end{equation*}
The result can be interpreted from a statistical, probabilistic and function analytic perspective.

We will use De Finetti's theorem to compute the limiting distribution for the Polya urn model 
\begin{equation*}
\lim_{n\to \infty} \bar{X}_n=Y~Beta\left(\frac{B_0}{B_0+W_0},\frac{W_0}{B_0+W_0} \right).
\end{equation*}
We first define the Beta and Gamma functions
\begin{equation*}
\beta(x,y)=\int_0^1 p^{x-1}(1-p)^{y-1}dp, \Gamma(x+1)=x\Gamma(x), \beta(a,b)=\frac{\Gamma(a)\Gamma(b)}{\Gamma(a+b)}.
\end{equation*}
The probability of observing $k$ black balls given $n$ draws
\begin{align}
\mathbb{P}(k \text{ black balls given } n \text{ draws})&=\dbinom{n}{k}\frac{B_0(B_0+1)\dots(B_0+k-1)W_0(W_0+1)\dots(W_0+n-k-1)}{(B_0+W_0)(B_0+W_0+1)\dots(B_0+W_0+n-1)}\\
								&=\dbinom{n}{k}\frac{\beta(B_0+k,B_0+n-k)}{\beta(B_0,W_0)}.
\end{align}
Note that the proportion of black balls at any stage $n$ of the process is
\begin{equation*}
\rho_n=\frac{B_n}{B_n+W_n}, \rho_{\infty}=\lim_{n\to \infty}\frac{B_n}{B_n+W_n}.
\end{equation*}
We know that 
\begin{equation*}
\mathbb{P}(k \text{ black balls given } n \text{ draws}|\rho_{\infty}=p)=\dbinom{n}{k}p^k(1-p)^{n-k},
\end{equation*}
and if $\rho_{\infty}~F(p)$ then,
\begin{equation*}
\mathbb{P}(k \text{ black balls given } n \text{ draws})=\int_0^1\mathbb{P}(k \text{ black balls given } n \text{ draws}|\rho_{\infty}=p)dF(p),
\end{equation*}
\begin{equation}
\mathbb{P}(k \text{ black balls given } n \text{ draws})=\dbinom{n}{k} \int_0^1p^k(1-p)^{n-k}dF(p).
\end{equation}
By equating (2) and (3) we obtain,
\begin{align*}
\int_0^1 p^k(1-p)^{n-k}dF(p) &= \frac{\beta(B_0+k,B_0+n-k)}{\beta(B_0,W_0)}\\
					   &=\frac{1}{\beta(B_0,W_0)}\int_0^1p^{B_0+k-1}(1-p)^{B_0+n-k-1}dp\\
					   &=\int_0^1 p^k (1-p)^{n-k}\frac{p^{B_0-1}(1-p)^{W_0-1}}{\beta(B_0,W_0)}dp,
\end{align*}
which gives the limiting distribution for Polya's urn scheme as
\begin{equation*}
f(p)=\frac{1}{\beta(B_0,W_0)}p^{B_0-1}(1-p)^{W_0-1}.
\end{equation*}

\end{document}