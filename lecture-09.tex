% !TEX spellcheck = en_US
% !TEX spellcheck = LaTeX
\documentclass[a4paper,10pt,english]{article}
\usepackage{%
	amsfonts,%
	amsmath,%	
	amssymb,%
	amsthm,%
%	babel,%
	bbm,%
	%biblatex,%
	caption,%
	centernot,%
	color,%
	enumerate,%
	epsfig,%
	epstopdf,%
	etex,%
	geometry,%
	graphicx,%
	hyperref,%
	latexsym,%
	mathtools,%
	multicol,%
	pgf,%
	pgfplots,%
	pgfplotstable,%
	pgfpages,%
	proof,%
	psfrag,%
	subfigure,%	
	tikz,%
	ulem,%
	url%
}	

\usepackage[mathscr]{eucal}
\usepgflibrary{shapes}
\usetikzlibrary{%
  arrows,%
  backgrounds,%
  chains,%
  decorations.pathmorphing,% /pgf/decoration/random steps | erste Graphik
  decorations.text,%
  matrix,%
  positioning,% wg. " of "
  fit,%
  patterns,%
  petri,%
  plotmarks,%
  scopes,%
  shadows,%
  shapes.misc,% wg. rounded rectangle
  shapes.arrows,%
  shapes.callouts,%
  shapes%
}

\theoremstyle{plain}
\newtheorem{thm}{Theorem}[section]
\newtheorem{lem}[thm]{Lemma}
\newtheorem{prop}[thm]{Proposition}
\newtheorem{cor}[thm]{Corollary}

\theoremstyle{definition}
\newtheorem{defn}[thm]{Definition}
\newtheorem{conj}[thm]{Conjecture}
\newtheorem{exmp}[thm]{Example}
\newtheorem{assum}[thm]{Assumptions}
\newtheorem{axiom}[thm]{Axiom}

\theoremstyle{remark}
\newtheorem{rem}[thm]{Remark}
\newtheorem{note}[thm]{Note}

\newcommand{\norm}[1]{\left\lVert#1\right\rVert}
\newcommand{\indep}{\!\perp\!\!\!\perp}
\DeclarePairedDelimiter\abs{\lvert}{\rvert}%
%\DeclarePairedDelimiter\norm{\lVert}{\rVert}%
\newcommand{\tr}{\operatorname{tr}}
\newcommand{\R}{\mathbb{R}}
\newcommand{\Q}{\mathbb{Q}}
\newcommand{\N}{\mathbb{N}}
\newcommand{\E}{\mathbb{E}}
\newcommand{\Z}{\mathbb{Z}}
\newcommand{\B}{\mathscr{B}}
\newcommand{\C}{\mathcal{C}}
\newcommand{\T}{\mathscr{T}}
\newcommand{\F}{\mathcal{F}}
\newcommand{\G}{\mathcal{G}}
%\newcommand{\ba}{\begin{align*}}
%\newcommand{\ea}{\end{align*}}

% Debug
\newcommand{\todo}[1]{\begin{color}{blue}{{\bf~[TODO:~#1]}}\end{color}}


\makeatletter
\def\th@plain{%
  \thm@notefont{}% same as heading font
  \itshape % body font
}
\def\th@definition{%
  \thm@notefont{}% same as heading font
  \normalfont % body font
}
\makeatother
\date{}
\title{Lecture 09: Age-Dependent Branching and Delayed Renewal}
\author{}

\begin{document}
\maketitle

\section{Age-dependent branching process }

Suppose a population where each organism lives for an \textit{iid} random time period of $X$ units with common distribution function $F$. 
Just before dying, each organism produces a number of offsprings $N$, an \textit{iid} discrete random variable with common distribution $P$.  
Let $X(t)$ denote the number of organisms alive at time $t$. 
The stochastic process $\{X(t),~ t \geqslant 0\}$ is called an age-dependent branching process. 
Let $X(t)$ be the number of individuals alive at time $t$. 
We are interested in computing $M(t)=\E X(t)$ when $m=\E[N] =\sum_{j \in \N}{j P_j}$. 
This is a popular model in biology for population growth of various organisms. 
%Another application of studying such a process is ``growth of a tree'', which is used by biologists to see how population grows. 

\begin{thm}
If $X(0)=1$, $m>1$ and $F$ is non lattice, then
\eq{
\lim_{t \to \infty} e^{-\alpha t}M(t)= \frac{m-1}{m^2 \alpha \int_{0}^{\infty}xe^{-\alpha x dF(x)}},
}
where $\alpha > 0$ is the unique solution to the equation $\int_{0}^{\infty}e^{-\alpha x } dF(x) = \frac{1}{m}$.
\end{thm}
\begin{proof}
Let $T_1$ and $N_1$ denote the life time and offsprings of the first organism. 
If $T_1 > t$, then $X(t)$ is still equal to $X(0) = 1$. 
If $T_1 \leq t$, then $X(T_1) = N_1$ and each of the offsprings start the population growth, 
independent of the past, and stochastically identical to the population growth of the original organism starting at time $T_1$. 
Hence, we can write $X(t) = \sum_{i=0}^{N_1}X_i(t-T_1)$ for this case.  
Therefore, 
%Then, conditioned on $T_1$, we can write %the life time of first organism is
\begin{align*}
M(t) &= \E[X(t)1\{T_1 > t\}] + \E[X(t)1\{T_1 \leq t\}] %\int_{0}^{\infty}\E[X(t)|T_1=y]dF(y)\\
%On splitting, \\
%&=\int_{0}^{t}\E[X(t)|T_1=y]dF(y) + \int_{t}^{\infty}\E[X(t)|T_1=y]dF(y)\\
%&\stackrel{(a)}{=}\int_{y=0}^{t} m M(t-y) dF(y) + \int_{y=t}^{\infty} 1 dF(y).
= \E[\int_{t}^{\infty}dF(u) + \int_{0}^{t}dF(u)\sum_{i=0}^{N_1}X_i(t-u)] = \bar{F}(t) + m\int_{0}^{t}M(t-u)dF(u).
\end{align*}
This looks almost like a renewal function. 
Multiplying both sides of the above equation by $e^{-\alpha t}$, we get 
\eq{
M(t)e^{-\alpha t} &= e^{-\alpha t}\bar{F}(t) + m \int_0^te^{-\alpha (t-u)}e^{-\alpha u}dF(u).
}
If $dG(t) \triangleq me^{-\alpha t}dF(t)$ was a density function on $\R_+$, then the above equation would exactly be a renewal equation, with the solution
\eq{
M(t)e^{-\alpha t} &= e^{-\alpha t}\bar{F}(t)+ \int_{0}^{t}e^{-\alpha(t - u)}\bar{F}(t-u)dm_G(u).
}
Here, $m_G(t) = \sum_{n \in \N}G_n(t)$ is the renewal function associated with inter-renewal distribution $G$. 
Clearly, the $\alpha > 0$ that ensures $G$ is a distribution function is the unique solution to the equation
\eq{
1 = G(1) = m\int_{0}^{\infty}e^{-\alpha t}dF(t).
}
Since $e^{-\alpha t}\bar{F}(t)$ is non-negative, monotone non-increasing and integrable, it directly Riemann integrable. 
Hence, we can apply key renewal theorem to the limiting value of solution to renewal equation to obtain 
\eq{
\lim_{t \to \infty}M(t)e^{-\alpha t} &= \frac{1}{\mu_G}\int_{0}^{\infty}e^{-\alpha t}\bar{F}(t)dt =  \frac{\int_{0}^{\infty}e^{-\alpha t}\bar{F}(t)dt}{m\int_{0}^{\infty}xe^{-\alpha t}dF(t)}.
}
%\begin{flalign}
%\label{renew}
%M(t)= F^c(t)+m\int_{0}^{t}M(t-y)dF(y)
%\end{flalign}

%We got a differential equation in M(t) and when we solve it, we get M(t). \\
%Let $\alpha$ denote the unique positive number such that $\int_{0}^{\infty}xe^{-\alpha x } dF(x) = \frac{1}{m}$ and $G(y)=m\int_{0}^{y}e^{-\alpha y} dF(y)$. \\
%Upon multiplying both sides of equation (\ref{renew}) by  $e^{-\alpha t}$ \\
%Let us define,
%\begin{align*}
%f(t) := e^{-\alpha t}M(t) \\
%h(t) :=e^{-\alpha t}F^{c}(t)\\ 
%G(y) :=\int_{0}^{y} m e^{- \alpha y} dF(y) 
%\end{align*}
%Thus considering equation (\ref{renew}) and multiplying both sides by $e^{-\alpha t}$, we get 
%\begin{align*}
%	e^{- \alpha t} M(t)=e^{- \alpha t} F^c(t)+m\int_{0}^{t} e^{- \alpha t} M(t-y)dF(y)\\
%	f(t) = h(t) + m\int_{0}^{t} e^{- \alpha (t-y)} e^{- \alpha y} M(t-y)dF(y)\\
%	f(t) = h(t) + (f*G)t
%\end{align*}
%
%\begin{flalign*}
%f&=h+f*G\\
%Recursively ,\\
%&=h+G*(h+f*G)\\
%\vdots
%&=h+h*\sum_{i=1}^{\infty}G_i\\
%&=h+h*m_G.
%\end{flalign*}
%Or, $f(t)=h(t)+\int_{0}^{t}h(t-s)dm_G(s)$. It can be shown that $h(t)$ is dRi and hence by Key Renewal thmrem, 
%\begin{flalign*}
%f(t) \to \frac{\int_{o}^{\infty}e^{-\alpha t}F^c(t) dt }{\int_{0}^{\infty}xdG(x) }.
%\end{flalign*}
Result follows from integration by parts, and showing that
\begin{align*}
\int_{0}^{\infty}e^{-\alpha t}\bar{F}(t) dt &= \frac{1}{\alpha} -\frac{1}{\alpha} \int_{0}^{\infty}e^{-\alpha t}dF(t) = \frac{1}{\alpha}\left(1- \frac{1}{m}\right).
%\int_{0}^{\infty} e^{-\alpha t}\int_{t}^{\infty}dF(x)dt \\
%&=  \int_{0}^{\infty} \int_{0}^{x} e^{-\alpha t} dt  dF(x)\\
%&= \int_{0}^{\infty} (1- e^{-\alpha x}) dF(x) \\
%&= \frac{1}{\alpha}(1-\frac{1}{m}) ~~ (\text{by the definition of} ~\alpha).\\
%&= \frac{m-1}{\alpha m}
\end{align*}
%Also $\int_{0}^{\infty}xdG(x) = m \int_{0}^{\infty}xe^{-\alpha x}dF(x)$. Therefore,
%\begin{align*}
%	\lim_{t \to \infty} m(t) = \frac{(m-1)}{\alpha m^2 \int_{0}^{\infty}x e^{-\alpha x} dF(x)}
%\end{align*}
\end{proof} 

\section{Delayed renewal process}
Many times in practice, we have a ``delayed start to a renewal process''. 
That is, the arrival process has \textit{iid} inter-arrival times $T_i$ for $i \geq 2$ with common distribution function $F$. 
Whereas, the first inter-arrival times $T_1$ is independent and has a different distribution $G$.  
%We come across such a process when initial arrival time has a different distribution and the next arrival times have a different distribution i.e., "delayed start to a renewal process".\\
%Let $\{X_n: n \in \N\}$ be independent but $X_1 \sim G$ and $X_i \sim F,~ i \geq 2$ then 
The associated counting process is called %general renewal process or 
a \textbf{delayed renewal process} and denoted by $\{N_D(t): t \geqslant 0\}$.  
Let $S_0=0$ and $n$th arrival instant $S_n =\sum_{i=1}^{n}T_i$. 
Then, then following inverse relationship holds between counting and arrival process, 
\eq{
N_D(t) &= \sup \{n \in \N: S_n \leq t\}.
}
\subsection{Distribution functions}
%We denote the distribution functions of counting process $N_D(t)$ and $n$th arrival instant $S_n$ by $P_n(t) \triangleq P\{N_D(t) = n\}$ and $F_n(t) \triangleq P\{S_n \leq t\}$ respectively. 
\begin{lem} 
The distribution function of $n$th arrival instant $S_n$ is
\eq{
P\{S_n \leq t\} &= (G \ast F_{n-1})(t).
}
\end{lem}
\begin{lem} 
The distribution function of counting process $N_D(t)$ is 
\eq{
P\{N_D(t) = n\} &= P(S_n \leq t)-P(S_{n+1} \leq t) = (G \ast F_{n-1})(t) - (G \ast F_{n})(t).
}
\end{lem} 
\begin{lem}
Let $m(t)$ be the renewal function associated with a renewal process with inter-arrival distribution $F$. 
Then, the renewal function $m_D(t) = \E N_D(t)$ associated with the delayed renewal process $N_D(t)$ is 
\eq{
m_D(t) &=\E N_D(t) = \sum_{n \in \N} (G \ast F_{n-1})(t) = G(t) + (G \ast m)(t).
}
\end{lem}
%\begin{flalign*}
%,\\
%P(N_D(t)=n) &= P(S_n \leq t)-P(S_{n+1} \leq t)\\
%&=G*F^{n-1}(t)-G*F^n(t),\\
%m_D(t)=\E[N_D(t)]= \sum_{n \in \N} G*F^{n-1}(t).
%\end{flalign*}
\begin{lem}
We denote the moment-generating function of a random variable with distribution $m_D(t), G(t), F(t)$ by $\tilde{m}_D(s), \tilde{G}, \tilde{F}$ respectively. 
Then, 
\eq{
\tilde{m}_D(s) = \frac{\tilde{G}(s)}{1-\tilde{F}(s)}
}
\end{lem}
\subsection{Limit theorems}
The limit theorems for delayed renewal process are identical to those for renewal processes. 
\begin{lem}[Basic renewal theorem] 
$\lim_{t \to \infty} \frac{N_D(t)}{t} = \frac{1}{\mu_F}$. 
\end{lem}
\begin{lem}[Elementary renewal theorem] 
$\lim_{t \to \infty} \frac{m_D(t)}{t} = \frac{1}{\mu_F}$.
\end{lem}
\begin{lem}[Blackwell's theorem] 
If the inter-renewal distribution $F$ is non-lattice, then 
\eq{
\lim_{t \to \infty} m_D(t+a)-m_D(t)=\frac{a}{\mu_F}.
}
If the distributions $F$ and $G$ are lattice with period $d$, then 
\eq{
\lim_{n \in \N}\E[\text{Number of renewals at } nd ] &=\frac{d}{\mu_F}.
}
\end{lem}
\begin{lem}[Key renewal theorem]
If $F$ is non-lattice, $\mu_F < \infty$ and $h \in \D$, then
\eq{
\lim_{t \to \infty} \int_{0}^{t}h(t-x)dm_D(x) = \frac{1}{\mu_F}\int_{0}^{\infty}h(t) dt.
}
\end{lem}
%\begin{prop}
%The following holds:
%\begin{enumerate}
%\item $\lim_{t \to \infty} \frac{N_D(t)}{t} = \frac{1}{\mu}$. 
%\item $\lim_{t \to \infty} \frac{m_D(t)}{t} = \frac{1}{\mu}$.
%\item  If $F$ is non-lattice, $\lim_{t \to \infty} m_D(t+a)-m_D(t)=\frac{a}{\mu_F}$.
%\item If $F$ and $G$ are lattice with period $d$, $\E$[ $ Number $of renewals at $nd$ ] =$\frac{d}{\mu_F}.$
%\item If $F$ is non-lattice, $\mu < \infty$ and $h$ is dRi, then 
%\begin{flalign*}
%\lim_{t \to \infty} \int_{0}^{t}h(t-x)dm_D(x) = \frac{\int_{0}^{\infty}h(t) dt}{\mu}.
%\end{flalign*}
%\end{enumerate}
%\end{prop}

\subsection{Distribution of the last renewal time}
%In the same manner as we derived the key lemma, refer Theorem 1.9 in lecture 6, for the last renewal time distribution of a standard renewal process, we can show for a delayed renewal process:
Using the regenerative process theory for $s \leq t$, we can write
\eq{
P\{S_{N(t)} \leq s\} %&=G^c(t) P(S_{N(t)} \leq s|S_{N(t)} = 0) + \int_{0}^{t}P(S_{N(t)} \leq s|S_{N(t)}=u)F^c(t-u)dm(u)\\
&= \bar{G}(t) +\int_{0}^{s}\bar{F}(t-u)dm(u).
}
 
\section{Equilibrium renewal process}
For $x \geq 0$, we can define the \textbf{equilibrium distribution} of $F$ as 
\eq{
F_e(x)=\frac{1}{\mu}\int_{0}^{x}\bar{F}(y)dy.
} 
\begin{lem} The moment generating function of $F_e(x)$ is 
\eq{
\tilde{F}_e(s) &= \frac{1-\tilde{F}(s)}{s\mu}.
}
\end{lem}
\begin{proof}
By definition, $\tilde{F}_e(s) = \E\left[\mathrm{e}^{-sX}\right]$, where $ X $ is a random variable with distribution function $ F_e(x) $. 
We use integration by parts, to write 
\begin{align*}
 \tilde{F}_e(s) &= \int_{0}^{\infty}\mathrm{e}^{-sx}dF_e(x) = \frac{1}{s\mu}-\frac{1}{s \mu}\int_{0}^{\infty}e^{-sx}dF(x)
% &= \frac{1}{\mu}\int_{0}^{\infty}\mathrm{e}^{-sx}F^c(x)dx\\
% &= \frac{1}{s\mu} - \frac{1}{\mu}\int_{0}^{\infty}\mathrm{e}^{-sx}F(x)dx\\
% &= \frac{1}{s\mu} - \frac{1}{s\mu}\int_{0}^{\infty}\mathrm{e}^{-sx}dF(x)\\
 = \frac{1}{s\mu}(1 - \tilde{F}(s)).
\end{align*}
%where the third and fourth equalities follows from the basic integration techniques.
 \end{proof}
A delayed renewal process with the initial arrival distribution $G=F_e$ is called the \textbf{equilibrium renewal process}. 
Observe that $F_e$ is the limiting distribution of the age and the excess time for the renewal process with common inter-renewal distribution $F$. 
Hence, if we start observing a renewal process at some arbitrarily large time $t$, 
then the observed renewal process is the equilibrium renewal process. 
This delayed renewal process exhibits stationary properties. 
That is, the limiting behaviors are exhibited for all times. 
\begin{thm}[renewal function]
The renewal function $m_e(t)$ for the equilibrium renewal process is linear for all times. 
That is, $m_e(t) =\frac{t}{\mu}$.
\end{thm}
\begin{proof}
We know that the Laplace transform of renewal function $m_e(t)$ is given by
\eq{
\tilde{m}_e(s) &= \frac{\tilde{G}(s)}{1-\tilde{F}(s)} = \frac{\tilde{F_e}(s)}{1-\tilde{F}(s)} = \frac{1}{s\mu}.
}
Further, we know that the Laplace transform of function $t/\mu$ is given by 
\eq{
\L_{t/\mu}(s) &= \frac{1}{\mu}\int_{0}^{\infty}e^{-sx}dx = \frac{1}{s \mu}.
}
%To prove $ (1) $, observe that $\tilde{m}_e(s)$. Hence, if $m_e(t) = \frac{t}{\mu}$ then, 
%\begin{flalign*}
%\tilde{m}_e(s) &= \int_{0}^{\infty}\mathrm{e}^{-st}dm_e(t)\\ 
%&= \frac{1}{\mu} \int_{0}^{\infty}\mathrm{e}^{-st}dt\\
%&= \frac{1}{s\mu}.
%\end{flalign*}	
Since moment generating function is a one-to-one map, $m_e(t) = \frac{t}{\mu}$ is the unique renewal function. 
\end{proof}
\begin{thm}[excess time]
The distribution of excess time $Y_e(t)$ for the equilibrium renewal process is stationary. 
That is, 
\eq{
P(Y_e(t) \leq x) = F_e(x),~t \geq 0.
}
\end{thm}
\begin{proof}
Since the excess time $Y_e(t)$ is regenerative process and $dm_e(t) = 1/\mu$, we can write 
\begin{flalign*}
P\{Y_e(t) >x\} %&= P(Y_e(t) >x|S_{N_e(t)}=0)P(S_{N_e(t)}=0) +P(Y_e(t) >x|S_{N_e(t)}=s)F^c(t-s)\frac{ds}{\mu}\\
 %&= P(X_1 >t+x,X_1>t) +P(X_2 >t+x-s|X_2 >t-s)F^c(t-s)\frac{ds}{\mu}\\
&= \bar{F}_e(t+x)+ \frac{1}{\mu}\int_{0}^{t}\bar{F}(t+x-u)du = \bar{F}_e(t+x) + \frac{1}{\mu}\int_{x}^{t+x}\bar{F}(y)dy
%&= 1 - \frac{1}{\mu} \int_{0}^{t+x}F^c(y)dy - \frac{1}{\mu} \int_{t+x}^{x}F^c(y)dy\\
%&= 1 - \frac{1}{\mu} \int_{0}^{x}F^c(y)dy\\
= \bar{F}_e(x).
\end{flalign*}
\end{proof}
\begin{thm}[stationary increments]
The counting process $\{N_e(t), t \geqslant 0\}$ for the equilibrium renewal process has stationary increments.
\end{thm}
\begin{proof}
When we start observing the process at time $s$, 
the observed renewal process is delayed renewal process with initial distribution being the original distribution. 
Hence, the number of renewals $N_e(t+s)-N_e(s) = N_e(t)$ in time interval of duration $t$ is shift invariant.
%\eq{
%N_e(t+s)-N_e(s) &= \lim_{s \to \infty} N(s+t) - N(s).
%} . 
\end{proof}
\subsection{Exponential renewal intervals}
Consider the case, when inter-renewal time distribution $F$ for a delay renewal process is exponential with rate $\lambda$. 
Here, one would expect the equilibrium distribution $F_e = F$, since Poisson process has stationary and independent increments. 
%\emph{Question}: What can you say about the equilibrium renewal process when $ F $ is distributed exponentially with the parameter $ \lambda $?
%\\
%\emph{Answer}: Let's look at the distribution of the first inter-arrival distribution, $ F_e $. So,
We observe that 
\begin{align*}
F_e(x) &= \frac{1}{\mu} \int_{0}^{x}\bar{F}(y)dy = \lambda \int_{0}^{x} \mathrm{e}^{-\lambda y} dy = 1 - e^{- \lambda x} = F(x).
\end{align*}	
%where the first equality follows from the definition of $ F_e $ for equilibrium renewal process, the second equality follows from the fact that the mean of exponential distribution is inverse of the parameter $ \lambda $.\\
We see that $F_e$ is also distributed exponentially with rate $\lambda$. 
Indeed, this is a Poisson process with rate $\lambda$. 
%So with all the properties of equilibrium renewal process, $ F_e $ and $ F $ being distributed exponentially with the same parameter $ \lambda $, says that this is a poisson process (not a delayed renewal process).


\end{document}

\subsubsection{Example:}
Let $\{X_n: n \in \N\}$ be iid discrete observed. A pattern $x_1,x_2 \hdots x_k$ is said to occur at time $n$ if $X_n=x_k,~X_{n-1}=x_{k-1}, \hdots X_{n-k+1}=x_1. $. If we have iid tosses and consider $N(n)$ as the number of times pattern $0,1,0,1$ appear in $n$ tosses, with $P(H)=p=1-q,$~the process is a delayed renewal processes. To find the mean number of tosses for the first time the pattern $0,1,0,1$ appear, \\
\begin{align*}
\lim_{n \to \infty}P[Renewal at toss n] &= 1.P[Renewal at toss n] + 0.P[No Renewal at toss n]\\
&= \frac{1}{p^2 q^2}.
\end{align*}
\begin{flalign*}
\E[\text{first time pattern}~ 0,1,0,1 ~ \text{appears}]&= \E[\text{first time pattern}~ 0,1 ~ \text{appears}]  \\
&+\E[\text{time between patterns}~ 0,1,0,1 ]\\
&=  p^{-1}q^{-1}+ p^{-2}q^{-2}.
\end{flalign*}

Similarly we can show that $\E[\text{first time}~ k \text{heads} ] = \sum_{i=1}^{n} p^{-i}$.

\subsection{Example:}

\begin{color}{blue} {\bf (Optional -- not covered in class)} \end{color}

Consider two coins and suppose  that each time is coin flipped, it lands tail with some unknown probability $p_i,~i=1,2.$ We are interested in coming up with a strategy that ensures that long term proportion of tails is $\min\{p_1,~p_2\}.$ One strategy is as follows: Set $n = 1$. In the $n^\text{th}$ round of coin flipping, flip the first coin till $n$ consecutive tails are obtained. Then flip the second coin till $n$ consecutive tails are obtained. Increment $n$ and repeat. \\

{\bf Claim.} $\lim_{m \to \infty} \frac{\# \mbox{tails in the first
    $m$ tosses}}{m} = \min\{p_1, p_2\}$ with probability $1$.\\

The proof is as follows. Let $p=\max\{p_1,p_2\}$ and $\alpha p
=\min\{p_1,p_2\}$. There is nothing to prove if $\alpha = 1$, so let
$\alpha < 1$. Call the coin with $P(T)=p$, the bad coin and the other,
the good coin. Let $B_n$ denote the number of flips in the
$n^\text{th}$ round of tossing the bad coin, and $G_n$ the number of
flips in the $n^\text{th}$ round of tossing the good coin. We first
prove the following lemma.
 \begin{lem}
   For any $\epsilon > 0$ with $\epsilon^{-1} \in \N$,
 $P(B_n \geq \epsilon G_n ~\text{for infinitely many rounds}~ n)=0$.
 \end{lem} 
 \begin{proof}
   For any $n \in \N$,
 \begin{flalign*}
   P\left(G_n \leq  \frac{B_n}{\epsilon}\right)&=\E[P(G_n \leq \frac{B_n}{\epsilon}|B_n)]\\
   &=\E[ \sum_{i=1}^{\frac{B_n}{\epsilon}}P(G_n = i |B_n)]\\
   &\leq\E[ \sum_{i=1}^{\frac{B_n}{\epsilon}} (\alpha p)^n]\\
   &= \E[{\frac{B_n}{\epsilon}}](\alpha p)^n \\
   &=\epsilon^{-1} \left( \sum_{i=1}^n \frac{1}{p^{i}}\right) (\alpha
   p)^n = \epsilon^{-1} \frac{p^{-n} - 1}{1 - p} (\alpha p)^n,
 \end{flalign*}
 where the inequality follows from the fact that $\{G_m = i\}$ implies
 that $i \geq m$ and that in cycle $m$, the coin flips numbered
 $i-m+1$ to $i$ are all tails. Hence,
 \[ \sum_{n=1}^\infty P\left(G_n \leq \frac{B_n}{\epsilon}\right) \leq
 \epsilon^{-1} \sum_{n=1}^\infty \frac{\alpha^{n}}{1 - p} < \infty.\]
 By the Borel-Cantelli lemma, it
 follows that $P(B_n \geq \epsilon G_n \text{for infinitely many
   $n$}) = 0$.
 \end{proof}
 With probability $1$, all but a finite number of rounds have at most
 an $\epsilon$ fraction of bad coin tosses, implying that $\lim_{m \to
   \infty} \frac{\# \mbox{bad coin tosses in the first $m$ tosses}}{m}
 \leq \epsilon$. Now taking a decreasing sequence $\epsilon_k = 1/k$,
 $k = 1, 2, 3, \ldots$, and using the continuity of probability, we
 get that with probability $1$, $\lim_{m \to \infty} \frac{\#
   \mbox{bad coin tosses in the first $m$ tosses}}{m} = 0$. This
 proves the claim using the strong law of large numbers for tosses of
 the good coin.

