% !TEX spellcheck = en_US
% !TEX spellcheck = LaTeX
\documentclass[a4paper,10pt,english]{article}
\usepackage{%
	amsfonts,%
	amsmath,%	
	amssymb,%
	amsthm,%
%	babel,%
	bbm,%
	%biblatex,%
	caption,%
	centernot,%
	color,%
	enumerate,%
	epsfig,%
	epstopdf,%
	etex,%
	geometry,%
	graphicx,%
	hyperref,%
	latexsym,%
	mathtools,%
	multicol,%
	pgf,%
	pgfplots,%
	pgfplotstable,%
	pgfpages,%
	proof,%
	psfrag,%
	subfigure,%	
	tikz,%
	ulem,%
	url%
}	

\usepackage[mathscr]{eucal}
\usepgflibrary{shapes}
\usetikzlibrary{%
  arrows,%
  backgrounds,%
  chains,%
  decorations.pathmorphing,% /pgf/decoration/random steps | erste Graphik
  decorations.text,%
  matrix,%
  positioning,% wg. " of "
  fit,%
  patterns,%
  petri,%
  plotmarks,%
  scopes,%
  shadows,%
  shapes.misc,% wg. rounded rectangle
  shapes.arrows,%
  shapes.callouts,%
  shapes%
}

\theoremstyle{plain}
\newtheorem{thm}{Theorem}[section]
\newtheorem{lem}[thm]{Lemma}
\newtheorem{prop}[thm]{Proposition}
\newtheorem{cor}[thm]{Corollary}

\theoremstyle{definition}
\newtheorem{defn}[thm]{Definition}
\newtheorem{conj}[thm]{Conjecture}
\newtheorem{exmp}[thm]{Example}
\newtheorem{assum}[thm]{Assumptions}
\newtheorem{axiom}[thm]{Axiom}

\theoremstyle{remark}
\newtheorem{rem}[thm]{Remark}
\newtheorem{note}[thm]{Note}

\newcommand{\norm}[1]{\left\lVert#1\right\rVert}
\newcommand{\indep}{\!\perp\!\!\!\perp}
\DeclarePairedDelimiter\abs{\lvert}{\rvert}%
%\DeclarePairedDelimiter\norm{\lVert}{\rVert}%
\newcommand{\tr}{\operatorname{tr}}
\newcommand{\R}{\mathbb{R}}
\newcommand{\Q}{\mathbb{Q}}
\newcommand{\N}{\mathbb{N}}
\newcommand{\E}{\mathbb{E}}
\newcommand{\Z}{\mathbb{Z}}
\newcommand{\B}{\mathscr{B}}
\newcommand{\C}{\mathcal{C}}
\newcommand{\T}{\mathscr{T}}
\newcommand{\F}{\mathcal{F}}
\newcommand{\G}{\mathcal{G}}
%\newcommand{\ba}{\begin{align*}}
%\newcommand{\ea}{\end{align*}}

% Debug
\newcommand{\todo}[1]{\begin{color}{blue}{{\bf~[TODO:~#1]}}\end{color}}


\makeatletter
\def\th@plain{%
  \thm@notefont{}% same as heading font
  \itshape % body font
}
\def\th@definition{%
  \thm@notefont{}% same as heading font
  \normalfont % body font
}
\makeatother
\date{}
\title{Lecture 07: Regenerative Processes}
\author{}
\begin{document}
\maketitle

\section{Regenerative processes}
A stochastic process $Z = \{Z_t, t \geqslant 0 \}$ with the state space $E$ is said to be \textbf{regenerative} if there exists a sequence $S = \{S_n: n \in \N\}$ of stopping times such that 
\begin{enumerate}[(a)]
\item \textbf{regeneration times:} $S$ is a renewal process, 
\item \textbf{regenerative property:} for any $n,m \in \N$ and any bounded function $f : E^n \to \R$, we have
\begin{align*}
\E[f(Z_{S_m+t_1}, \dots, Z_{S_m+t_n})|Z_u, u \leqslant S_m] &= \E f(Z_{t_1},\dots,Z_{t_n}).
\end{align*}
\end{enumerate}
In particular, if the stochastic process $Z$ is bounded, then for $f(x) = x$ and $n=m=1$, we have 
\eq{
\E[Z_{S_1 + t}|Z_u, u \leqslant S_1] &= \E Z_t.
}
The definition says that probability law is independent of the past and shift invariant at renewal times. 
That is after each renewal instant, the process becomes an independent probabilistic replica of the process starting from zero. 
Let $F$ be the distribution of inter-renewal times, then for an open subset $A \subseteq E$ and $t \geqslant 0$, we define 
\begin{xalignat*}{3}
&K(t)= P\{S_1 > t, Z_t \in A\}, && f(t) = P\{Z_t \in A\}.
\end{xalignat*} 
By the regeneration property applied at renewal instant $S_1$, we have
\begin{align*}
P\{Z_t \in A|S_1\} &= P\{Z_{t-S_1}\} = f(t-S_1) \text{ on } \{S_1 \leqslant t\}.
\end{align*}
Hence, we have the \textbf{renewal equation}
\begin{align*}
f(t) &= K(t) + \int_{0}^{t}dF(s)f(t-s) = K + F \ast f.
\end{align*}
We assume that the distribution function $F$ and the kernel $K$ are known, and we wish to find $f$, and characterize its asymptotic behavior. 

\begin{thm} 
The renewal equation has a unique solution $f = (1 + m) \ast K$, where $m(t) = \sum_{n \in \N}F_n(t)$ is the renewal function associated with the inter-renewal time distribution $F$. 
\end{thm}
\begin{proof}
It follows from the renewal equation that 
\begin{align*}
F\ast((1+m)\ast K) &= \sum_{n \in \N}F_n \ast K = m \ast K.
\end{align*}
Hence, it is clear that $m \ast K$ is a solution to the renewal equation. 
For uniqueness, let $f$ be another solution, then $h = f - K - m \ast K$ satisfies $h = F \ast h$, and hence $h = F_n \ast h$ for all $n \in \N$. 
From finiteness of $m(t)$, it follows that $F_n(t) \to 0$ as $n$ grows. 
Hence, $\lim_{n\in\N}(F_n\ast h)(t) = 0$ for each $t$. 
\end{proof}

\begin{prop} 
Let $Z$ be a regenerative process with state space $E$. 
Then for any $t \geq 0$ and open $A \subseteq E$, 
\begin{align*}
P\{Z_t \in A\} &= K(t) + \int_{0}^tdm(s)K(t-s),
\end{align*}
where $m$ is the renewal function and $K$ is the kernel function. 
\end{prop}

\begin{thm}[Key Lemma] Let $N(t)$ be a renewal process, with mean $m(t)$, \textit{iid} inter-renewal times $\{X_n\}$ with distribution function $F$, and $n^{\mathrm{th}}$ renewal instant $S_n$. Then,
\begin{equation*}
\Pr\{S_{N(t)}\leq s\}=\bar{F}(t)+\int_{0}^{s}\bar{F}(t-y)dm(y),\quad\quad t\geq s \geq 0.
\end{equation*}
\end{thm} 
\begin{proof} We can see that event of time of last renewal prior to $t$ being smaller than another time $s$ can be partitioned into disjoint events corresponding to number of renewals till time $t$. Each of these disjoint events is equivalent to occurrence of $n^{\mathrm{th}}$ renewal before time $s$ and $(n+1)^{\mathrm{st}}$ renewal past time $t$. That is,
\begin{equation*}
	\{S_{N(t)} \leq s\} = \bigcup_{n \in \N_0}\{ S_{N(t)} \leq s, N(t)=n\} = \bigcup_{n \in \N_0}\{ S_n \leq s, S_{n+1} > t\} .
\end{equation*}
Recognizing that $S_0 = 0$, $S_1 = X_1$, and that $S_{n+1} = S_n + X_{n+1}$, we can write
\begin{equation*}
	\Pr\{S_{N(t)} \leq s\} = \Pr\{X_1 > t\} + \sum_{n \in \N}\Pr\{ X_{n+1} + S_n > t, S_n \leq s\} .
\end{equation*}
We recall $F_n$, $n$-fold convolution of $F$, is the distribution function of $S_n$. Conditioning on $\{S_n = y\}$, we can write
\begin{align*}
	\Pr\{S_{N(t)} \leq s\} &= \bar{F}(t) + \sum_{n \in \N}\int_{y=0}^{s}\Pr\{ X_{n+1} > t - S_n, S_n \leq s| S_n = y\}dF_n(y),\\
	&= \bar{F}(t) + \sum_{n \in \N}\int_{y=0}^{s}\bar{F}(t-y)dF_n(y).
\end{align*}
Using monotone convergence theorem to interchange integral and summation, and noticing that $m(y) = \sum_{n\in\N}F_n(y)$, the result follows.
%\begin{flalign*}
%\Pr\{S_{N(t)} \leq s\} = \Pr\{X_1 > t\} +\sum_{n \in \N}\int_{y=0}^{s}\Pr\{ S_{n+1} > t|S_n=y\}dF_n(y)\\
%&= \Pr\{X_1 > t\} +\sum_{n \in \N}\int_{y=0}^{s}\Pr\{ X_{n+1} > t-y\}dF_n(y)\\
%&= \Pr\{X_1 > t\} +\sum_{n \in \N}\int_{y=0}^{s}\bar{F}(t-y)dF_n(y)\\
%&\stackrel{(a)}{=} \Pr\{X_1 >t) +\int_{y=0}^{s}\bar{F}(t-y)\sum_{n \in \N}dF_n(y),\\
%\end{flalign*}
%where the order of summation is exchanged by virtue of the fact that the summands are non-negative and the sum is finite (Fubini's theorem).
\end{proof}
\begin{shaded*}
%\begin{rem}\label{Remark:DensityLastRenewal} 
Key lemma tells us that distribution of $S_{N(t)}$ has probability mass at $0$ and density between $(0,t]$, that is,
\begin{align*}
\Pr\{S_{N(t)}=0\}&=\bar{F}(t),& dF_{S_{N(t)}}(y)&=\bar{F}(t-y)dm(y)~\quad 0 < y \leq t.
\end{align*}
%\end{rem}
%\begin{rem} 
Density of $S_{N(t)}$ has interpretation of renewal taking place in the infinitesimal neighborhood of $y$, and next inter-arrival after time $t-y$. To see this, we notice 
\begin{equation*}
dm(y)=\sum_{n \in \N}dF_n(y) =\sum_{n \in \N}\Pr\{n^{\text{th}} \text{renewal occurs in} (y,y+dy)\}.
\end{equation*}
Combining interpretation of density of inter-arrival time $dF(t)$, we get
\begin{equation*}
dF_{S_{N(t)}}(y)=\Pr\{\text{renewal occurs in }(y,y+dy) \text{ and next arrival after}~ t-y\}.
\end{equation*}
%\end{rem}
\end{shaded*}

\begin{lem} 
Let $F$ be the inter-renewal distribution such that $\inf\{x: F(x) = 1\} = \infty$, then for any $b > 0$ 
\begin{align*}
\sup_{t} \{m(t)-m(t-b) \}< \infty.
\end{align*} 
\end{lem}
\begin{proof}
We know that $m \ast F = m - 1$, so $m \ast (1- F) = 1$. 
Since the function $1-F$ is monotone, 
\begin{align*}
1 &= \int_0^tdm(s)[1-F(t-s)] \geq [m(t) - m(t-b)](1-F(b)),
\end{align*}
where $b$ is chosen so that $F(b) < 1$. Hence, the result follows. 
%\begin{align*}
%\sup_{t}[m(t) - m(t-b)] = \beta_b < \infty.
%\end{align*}
\end{proof}

%\begin{defn}[Lattice Random Variable] 
A non-negative random variable $X$ is said to be \textbf{lattice} if there exists $d \geq 0$ such that 
\begin{equation*}
\sum_{n\in\N}\Pr\{X = nd\} = 1.
\end{equation*}
For a lattice $X$, its period is defined as 
\begin{equation*}
d = \sup\{ d \in\R^+: \sum_{n\in\N}\Pr\{X = nd\} = 1 \}.
\end{equation*}
If $X$ is a lattice random variable, its distribution function $F$ is also called lattice.
%\end{defn}

%\subsection{Blackwell's Theorem}
\begin{thm}[Blackwell's Theorem] 
Let the inter-renewal times have distribution $F$, mean $\mu$, and the associated renewal function $m(t)$, such that $\inf\{x: F(x) = 1\} = \infty$. 
%Let $m(t)$ be the renewal function and the associated inter-arrival times with distribution $F$ and mean $\mu$. 
If $F$ is not lattice, then for all $a \geq 0$
\begin{equation*}
\lim_{t \to \infty} m(t+a) -m(t) = \frac{a}{\mu}.
\end{equation*}
If $F$ is lattice with period $d$, then 
\begin{equation*}
%\lim_{n \to \infty} \E[\text{number of renewals at } nd] = \frac{d}{\mu}.
\lim_{n \to \infty} m(t+d) - m(t) = \frac{d}{\mu}.
\end{equation*}
\end{thm}
\begin{proof} 
%From key renewal theorem, we have $\lim_{t \to \infty}m(t)/t = \frac{1}{\mu}$. For $\epsilon > 0$, pick $T$ such that for all $t > T$, we have
%\begin{align*}
%|m(t) - \frac{t}{\mu}| &\leq \epsilon/2,  |m(t+a) - \frac{t+a}{\mu}| &\leq \epsilon/2.
%\end{align*}
%Taking difference, we obtain for all $t \geq T$
%\begin{align*}
%|m(t+a) - m(t) - \frac{a}{\mu}| &\leq \epsilon.
%\end{align*}
We will not prove that 
\begin{equation}
\label{eq:LimitBlackwell}
g(a) = \lim_{t \to \infty} [m(t+a) - m(t)]
\end{equation}exists for non-lattice $F$. However, we show that if this limit does exist, it is equal to $a/\mu$ as a consequence of elementary renewal theorem. To this end, note that
\begin{equation*}
m(t+a+b) - m(t) = m(t+a+b) - m(t+a) + m(t+a) - m(t).
\end{equation*}
Taking limits on both sides of the above equation, we conclude that $g(a+b) = g(a) +  g(b)$. The only increasing solution of such a $g$ is
\begin{equation*}
g(a) = ca, \forall a > 0,
\end{equation*}
for some positive constant $c$. To show $c = \frac{1}{\mu}$, define a sequence $\{x_n, n \in \N\}$ in terms of $m(t)$ as 
\begin{equation*}
x_n = m(n) - m(n-1),~n \in \N.
\end{equation*}
Note that $\sum_{i=1}^nx_i = m(n)$ and $\lim_{n \in \N}x_n = g(1) = c$, hence we have
\begin{equation*}
%\lim_{n \to \infty}x_n = c \Rightarrow 
\lim_{n \in \N}\frac{\sum_{i=1}^nx_i}{n} = \lim_{n \in \N}\frac{m(n)}{n} \stackrel{(a)}{=} c,
\end{equation*}
where (a) follows from the fact that if a sequence $\{x_i\}$ converges to $c$, then the running average sequence $a_n = \frac{1}{n}\sum_{i=1}^n x_i$ also converges to $c$, as $n \to \infty$. 
Therefore, we can conclude $c = 1/\mu$ by elementary renewal theorem.

When $F$ is lattice with period $d$, the limit in~\eqref{eq:LimitBlackwell} doesn't exist. (See the following example). %\ref{eg:lattice limit}). 
However, the theorem is true for lattice trivially by elementary renewal theorem. Indeed, since $\frac{m(t)}{t} \to \frac{1}{\mu}$, we have that $m(nd)-m((n-1)d) \to \frac{1}{\mu}[nd-(n-1)d] = \frac{d}{\mu}.$
\end{proof}

\begin{shaded*}
%\label{eg:lattice limit}
For a trivial lattice example where the $\lim_{t\to \infty} m(t+a)-m(t)$ does not exist, consider a renewal process with $\Pr\{X_n = 1\} = 1$, that is, there is a renewal at every positive integer time instant with probability 1. Then $F$ is lattice with $d=1.$ Now, for $a=0.5$, and $t_n = n+(-1)^n 0.5$, we see that $\lim_{t_n \to \infty} m(t_n+a)-m(t_n)$ does not exist, and hence $\lim_{t \to \infty} m(t+a)-m(t)$ does not exist.
\end{shaded*}

%\begin{rem}
In the lattice case, if the inter arrivals are strictly positive, that is, there can be no more than one renewal at each $nd$, then we have that 
\begin{align}
\lim_{n\to \infty} P[{\text{renewal at nd}}] = \frac{d}{\mu}.
\end{align}
%\end{rem}

\end{document}