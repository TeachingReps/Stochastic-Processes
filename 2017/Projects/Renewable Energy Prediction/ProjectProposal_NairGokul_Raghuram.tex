\documentclass[a4paper,english,12pt]{article}
\usepackage{%
	amsfonts,%
	amsmath,%	
	amssymb,%
	amsthm,%
	algorithm,%
	babel,%
	bbm,%
	etex,%
	caption,%
	centernot,%
	color,%
	dsfont,%
	enumerate,%
	epsfig,%
	geometry,%
	graphicx,%
	hyperref,%
	latexsym,%
	mathtools,%
	multicol,%
	pgf,%
	pgfplots,%
	pgfplotstable,%
	pgfpages,%
	proof,%
	psfrag,%
	subfigure,%	
	tikz,%
	ulem,%
	url%
}	
\usepackage[noend]{algpseudocode}
\usepackage[mathscr]{eucal}
\usepgflibrary{shapes}
\usetikzlibrary{%
  arrows,%
  backgrounds,%
  chains,%
  decorations.pathmorphing,% /pgf/decoration/random steps | erste Graphik
  decorations.text,%
  fit,%
  matrix,%
  patterns,%
  petri,%
  positioning,% wg. " of "
  plotmarks,%
  scopes,%
  shadows,%
  shapes,%
  shapes.arrows,%
  shapes.callouts,%
  shapes.misc% wg. rounded rectangle
}

\theoremstyle{plain}
\newtheorem{thm}{Theorem}[section]
\newtheorem{lem}[thm]{Lemma}
\newtheorem{prop}[thm]{Proposition}
\newtheorem{cor}[thm]{Corollary}

\theoremstyle{definition}
\newtheorem{defn}[thm]{Definition}
\newtheorem{conj}[thm]{Conjecture}
\newtheorem{exmp}[thm]{Example}
\newtheorem{assum}[thm]{Assumptions}
\newtheorem{axiom}[thm]{Axiom}

\theoremstyle{remark}
\newtheorem{rem}{Remark}
\newtheorem{note}{Note}
\newtheorem{fact}{Fact}

\definecolor{lightgray}{gray}{0.9}

%\DeclarePairedDelimiter{\ceil}{\left\lceil}{\right\rceil}%
\newcommand{\eq}[1]{\begin{align*}#1\end{align*}}
\newcommand{\ceil}[1]{\left\lceil#1\right\rceil}%
\newcommand{\norm}[1]{\left\lVert#1\right\rVert}%
\newcommand{\indep}{\!\perp\!\!\!\perp}%
\DeclarePairedDelimiter\abs{\lvert}{\rvert}%
\newcommand\numberthis{\addtocounter{equation}{1}\tag{\theequation}}
\newcommand{\tr}{\operatorname{tr}}
\newcommand{\R}{\mathbb{R}}
\newcommand{\N}{\mathbb{N}}
\newcommand{\E}{\mathbb{E}}
\newcommand{\Z}{\mathbb{Z}}
\newcommand{\B}{\mathscr{B}}
\newcommand{\C}{\mathcal{C}}
\newcommand{\T}{\mathscr{T}}
\newcommand{\F}{\mathcal{F}}
\newcommand{\G}{\mathcal{G}}
\newcommand{\X}{\mathcal{X}}
%\newcommand{\ba}{\begin{align*}}
%\newcommand{\ea}{\end{align*}}
\DeclareMathOperator*{\argmax}{arg\,max}
\renewcommand{\qedsymbol}{$\blacksquare$}
\makeatletter
\def\BState{\State\hskip-\ALG@thistlm}
\makeatother

\makeatletter
\def\th@plain{%
  \thm@notefont{}% same as heading font
  \itshape % body font
}
\def\th@definition{%
  \thm@notefont{}% same as heading font
  \normalfont % body font
}
\makeatother
\date{}

%opening
\title{Stochastic Models for Renewable Energy Prediction}
\author{Team : Gokul Nair and Raghuram Bharadwaj}

\begin{document}
\maketitle

\section{Motivation}
Smart Grid has become a big talking point in recent times. With the possible extinction of non-renewable energy sources and our quest to make environment greener, efficient generation and usage of energy is very important. In this regard, a power grid model is developed, which integrates all the energy sources and takes optimal control decision on how much energy needs to be utilized from renewable sources and amount of energy that needs to be generated at the production site. We can try to solve this problem efficiently, provided we can predict the renewable energy that will be generated in the future time periods. In our project, we plan to solve this problem, where we construct probability models for predicting renewable (Solar / Wind ) energy.  
 \section{System Model}
We plan to begin with simple Poisson process and then build complex models that fit the data better. We would like to study the properties of the generation of energy. For example, Is it Markovian ? Does it have stationary increments etc and develop the model accordingly.   
\section{Problem Statement}
Constructing a probability model that fits the generation of renewable energy and predicts the future generation of energy. 
\subsection{Justification}
To the best of our knowledge, only simple poisson models have been considered in the literature. We want to specifically work on our Indian scenario and understand the properties of generation. We believe that, we can come up with new models that fit the data and give better prediction results.

This also gives us opportunity to implement the concepts we learn in class. 

\section{Plan}

\begin{itemize}
	\item Week 1-2 :  Literature Survey - To understand the power generation and current models of prediction. 
	\item Week 3   :  Extracting the necessary features and properties of generation. 
	\item Week 4-5 :  Developing new probability models.
	\item Week 6   :  Identifying the best probability model and summarizing results.  
\end{itemize}


\end{document}