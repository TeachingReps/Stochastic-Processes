\documentclass[a4paper,english,12pt]{article}
\usepackage{%
	amsfonts,%
	amsmath,%	
	amssymb,%
	amsthm,%
%	babel,%
	bbm,%
	%biblatex,%
	caption,%
	centernot,%
	color,%
	enumerate,%
	epsfig,%
	epstopdf,%
	etex,%
	geometry,%
	graphicx,%
	hyperref,%
	latexsym,%
	mathtools,%
	multicol,%
	pgf,%
	pgfplots,%
	pgfplotstable,%
	pgfpages,%
	proof,%
	psfrag,%
	subfigure,%	
	tikz,%
	ulem,%
	url%
}	

\usepackage[mathscr]{eucal}
\usepgflibrary{shapes}
\usetikzlibrary{%
  arrows,%
  backgrounds,%
  chains,%
  decorations.pathmorphing,% /pgf/decoration/random steps | erste Graphik
  decorations.text,%
  matrix,%
  positioning,% wg. " of "
  fit,%
  patterns,%
  petri,%
  plotmarks,%
  scopes,%
  shadows,%
  shapes.misc,% wg. rounded rectangle
  shapes.arrows,%
  shapes.callouts,%
  shapes%
}

\theoremstyle{plain}
\newtheorem{thm}{Theorem}[section]
\newtheorem{lem}[thm]{Lemma}
\newtheorem{prop}[thm]{Proposition}
\newtheorem{cor}[thm]{Corollary}

\theoremstyle{definition}
\newtheorem{defn}[thm]{Definition}
\newtheorem{conj}[thm]{Conjecture}
\newtheorem{exmp}[thm]{Example}
\newtheorem{assum}[thm]{Assumptions}
\newtheorem{axiom}[thm]{Axiom}

\theoremstyle{remark}
\newtheorem{rem}[thm]{Remark}
\newtheorem{note}[thm]{Note}

\newcommand{\norm}[1]{\left\lVert#1\right\rVert}
\newcommand{\indep}{\!\perp\!\!\!\perp}
\DeclarePairedDelimiter\abs{\lvert}{\rvert}%
%\DeclarePairedDelimiter\norm{\lVert}{\rVert}%
\newcommand{\tr}{\operatorname{tr}}
\newcommand{\R}{\mathbb{R}}
\newcommand{\Q}{\mathbb{Q}}
\newcommand{\N}{\mathbb{N}}
\newcommand{\E}{\mathbb{E}}
\newcommand{\Z}{\mathbb{Z}}
\newcommand{\B}{\mathscr{B}}
\newcommand{\C}{\mathcal{C}}
\newcommand{\T}{\mathscr{T}}
\newcommand{\F}{\mathcal{F}}
\newcommand{\G}{\mathcal{G}}
%\newcommand{\ba}{\begin{align*}}
%\newcommand{\ea}{\end{align*}}

% Debug
\newcommand{\todo}[1]{\begin{color}{blue}{{\bf~[TODO:~#1]}}\end{color}}


\makeatletter
\def\th@plain{%
  \thm@notefont{}% same as heading font
  \itshape % body font
}
\def\th@definition{%
  \thm@notefont{}% same as heading font
  \normalfont % body font
}
\makeatother
\date{}
\definecolor{lightgray}{gray}{0.9}
\title{Optimal Estimation of wireless sensor networks through mobile transceivers}
\author{Prathamesh Mayekar, Devyani Gupta }
\usepgfplotslibrary{units}
\usepackage[round]{natbib}
\usepackage{tikz}
\usetikzlibrary{intersections}
\usetikzlibrary{arrows,decorations.pathmorphing}

\DeclarePairedDelimiter{\ceeil}{\lceil}{\rceil}
\begin{document}
\maketitle 
\begin{figure}[!b]
	\centering
	\begin{tikzpicture}
	[photon/.style={decorate,decoration={snake}}]
	
	
	
	\draw[->]	(2,6) node [above]{$X_1(t)$} to (2,5.5);
	\draw (2,5) node {$s_1$} circle (.5cm);
	\draw [->] (2,4.5) to (2.65,4);
	
	\draw[->]	(5,6) node [above]{$X_2(t)$} to (5,5.5);
	\draw (5,5) node {$s_2$} circle (.5cm);
	
	
	\draw[->]	(8,6) node [above]{$X_3(t)$}  to (8,5.5);
	\draw (8,5) node {$s_3$} circle (.5cm);
	\draw [->] (8,4.5) to (7.65,4);
	
	
	
\draw [thick, dashed] (10.75,5) to (11.25,5);
	
	
	
	\draw[->]	(14,6) node [above]{$X_n(t)$} to (14,5.5);
	\draw (14,5) node {$s_n$} circle (.5cm);
	\draw [->] (14,4.5) to (14.65,4);
	
	
	
	\draw (2.25,3.25) rectangle (3,4) node[pos=.5]{$m_1$};
	\draw [->,photon] (2.5,3.25) to (2.5,2.25) node[xshift=30,yshift=10]{$(1,X_1(t))$};
	\node[cloud, cloud puffs=15.7, cloud ignores aspect, minimum width=1cm, minimum height=1.5cm, align=center, draw] (cloud) at (2.5cm, 1.25cm) {Ideal \\ Channel};
	
	
	\draw [->,photon] (2.5,.3) to (2.5,-.5);
	
	
	
	\draw (7.25,3.25) rectangle (8,4) node[pos=.5]{$m_2$};
	\draw [->,photon] (7.5,3.25) to (7.5,2.25) node[xshift=30,yshift=10]{$(3,X_3(t))$};
	\node[cloud, cloud puffs=15.7, cloud ignores aspect, minimum width=1cm, minimum height=1.5cm, align=center, draw] (cloud) at (7.5cm, 1.25cm) {Ideal \\ Channel};
	
	
	\draw [->,photon] (7.5,.3) to (7.5,-.5);
	
	
\draw [thick, dashed] (10.75,3.6) to (11.25,3.6);
	
	
	\draw (14.25,3.25) rectangle (15,4) node[pos=.5]{$m_k$};
	\draw [->,photon] (14.5,3.25) to (14.5,2.25) node[xshift=30,yshift=10]{$(n,X_n(t))$};
	\node[cloud, cloud puffs=15.7, cloud ignores aspect, minimum width=1cm, minimum height=1.5cm, align=center, draw] (cloud) at (14.5cm, 1.25cm) {Ideal \\ Channel};
	\draw [->,photon] (14.5,.3) to (14.5,-.5);
	
	
	\draw (1.5,-.5) rectangle (15.5,-1.5) node[pos=.5]{Central Estimator};
	
	\draw [->,] (8.5,-1.5) to (8.5,-2) node[below] {$(\hat{X_1},\hat{X_2},\dots,\hat{X_n})$};
	
	
	
	
	\end{tikzpicture}
	
	
	\caption{Schematic representation of transmission via mobile transceiver in wireless sensor networks}
	
	
\end{figure}

\section{Introduction}



In this report we formulate a  problem in which $n$ sensors make independent measurements and communicate with $k$ mobile transceivers, $k<n$, which send values observed by them  to a central estimator, who will then estimate all the measurements, over a noiseless channel. We wish to determine the allocation of these $k$  mobile transceivers  at each time instant $t \in [T]$, so as to minimize the sum of the joint distortion of all sensor nodes (mean square error, hamming distance) over the finite length horizon of time $T$. We also want the distortion at individual sensors to be below certain thresholds, which can depend on the importance of the data arriving at the sensor nodes. This problem can be classified as a joint scheduling and estimation problem.
%\section{Motivation}

Problems of remote estimation which include a scheduling component have been studied before. In \cite{imer2005optimal}, \cite{imer2010optimal} authors study the optimal estimation  and scheduling policy when the number of transmissions permitted for measurement are limited for single sensor. The motivation for such problems is the limited battery availability to the sensors.
Two sensors making independent measurements and communicating with a remote estimator over a collision channel was studied in
\cite{vasconcelos2013estimation}.

Motivated by these models, we propose a wireless sensor network model with mobile transceivers. 
One natural advantage of using mobile transceivers is that the sensors avoid  transmissions  which is one of the prime reasons of loss of energy at these sensors. Also collisions could be avoided, by an appropriate scheduling policy for the mobile transceivers.



















\section{Mathematical Model}\label{model}
Consider a set of sensors $\mathcal{S}=\{s_i:i \in [n]\}$ and a corresponding set of processes  $\{X_i(t), t \in \mathbb{N} : i \in [n]\}$. A sensor $s_i$ observes the process $X_i(t)$. 
 
We consider the processes $\{X_i(t): t \in \mathbb{N}\}$  to be $\textit{iid}$. Also the processes $X_i(t)$ are mutually independent of one another. Each process $X_i(t)$ has a mean  $\mu_i$ and variance  $\sigma_i^2$.

We also consider a set of mobile transceivers $\mathcal{M}=\{m_j: j \in [k]\}$. At any time instant $t$, the mobile receiver $m_j \in \mathcal{M}$ is present at one of the sensor nodes $s_i \in \mathcal{S}$ thus observes the value $X_i(t)$. We assume that if at start of time period $t$ the mobile transceiver $m_j$ is at sensor $s_i$ then at the start of time instant $t+1$ it can be at any sensor node location in $\mathcal{S}$.   The location of mobile transceiver $m_j$ at time $t$ is given by $u_j(t)$, such that \eq{u_j(t) \in [n] \ \forall t \in [T].}

These mobile transceivers transmit the values observed by them along with the sensor index they observed those values at to the central estimator for every time instant $t$. For example a mobile transceiver $m_j \in \mathcal{M}$ with a location $u_j(t)$ at time $t$ will transmit the output $o_j(t)$,  such that \eq{o_j(t)=(u_j(t), X_{u_j(t)}(t)) .} % At time $t$, the location of the mobile transceivers is given by the set $\{u_j(t): j \in [k]\}$. 
 We assume this transmissions happen over an ideal channel. Thus the central estimator will receive the set $\{o_j(t): j \in [k]   \}$.

The estimator uses a  estimation policy $\pi$, from the set of all admissible estimation policies $\Pi$ % $\pi \in \Pi$ 
which takes input $\{o_j(t): j \in [k]   \}$ from the mobile transceiver and outputs the estimates $\{\hat{X_i(t)}: i \in [n] \}$.


Our goal for the described mathematical model and a distortion metric $d$,  \eq{d:\mathbb{R} \times \mathbb{R}\rightarrow \mathbb{R^+},} is to minimize the expected sum of distortion between the actual values observed by the sensor nodes  and the estimated values over a finite horizon $T$. This optimization has to be carried out under upper bounds $\{\epsilon_i: i \in [n]\}$ on the distortions at individual sensors $\{s_i:i \in [n]\}$.

 Our handle for minimization is the policy $\theta\in\Theta $ for the scheduling of the mobile receivers at every time instant given by the set $\{u_j(t): j \in [k]\}$ and the estimation policy $\pi \in \Pi$ at the transceiver. The structure of all admissible scheduling policies $\Theta$ and all admissible estimation policies $\Pi$ will be described in  Sections \ref{APP} and \ref{AEP}.  
 Our problem can be mathematically stated as the following Stochastic Optimization Problem:

\begin{align}
& \ \min \sum_{i \in [n], t \in [T]} \mathbb{E}[ d(X_i(t),\hat{X_i(t)})]\\
& s.t. \nonumber\\
 &\ \sum_{t \in [T]} \mathbb{E}[d(X_i(t),\hat{X_i(t)})] \leq \epsilon_i \quad \forall i \in [n] \nonumber\\
& \theta \in \Theta , \pi \in \Pi\nonumber\\
\nonumber
\end{align}

For the sake of simplicity we restrict ourselves to the case of two sensors $n=2$, and one mobile transceiver $k=1$. Thus the sensors $s_1$, $s_2$ observe the processes $X_1(t)$ and $X_2(t)$. Also since there is only one mobile transceiver we drop the subscript for the mobile transceiver variables. Thus the mobile transceiver is labeled $m$, with its location at time $t$ given by $u(t)$ and its transmitted output by $o(t)$. Also we restrict ourselves to the   mean square error  distortion metric given as follows,
\eq{d(x,y)=(x-y)^2.}

Thus the problem is reduced to the following problem:
\begin{align}\label{Opt}
& \ \min \sum_{i \in [2], t \in [T]} \mathbb{E}[ (X_i(t)-\hat{X_i(t)})^2]\\
& s.t. \nonumber\\
 &\ \sum_{t \in [T]} \mathbb{E}[(X_i(t)-\hat{X_i(t)}^2)] \leq \epsilon_i \quad \forall i \in [2] \nonumber\\
& \theta \in \Theta , \pi \in \Pi\nonumber\\
\nonumber
\end{align}

\section{Analysis}\label{analysis}
In this section we  will first look at the information structure at the mobile transceiver and the central estimator and the set of admissible scheduling policies and estimation policies given by $\Theta$ and $\Pi$.

Also we will determine for the Stochastic Optimization problem described in Equation \refeq{Opt}, the optimal estimation policy  at every time instant. We will show that the optimal estimation policy at a particular time instant $t$ is independent of  previous locations of the mobile transceiver, and it only depends on its location at time $t$. We plug-in this optimal estimation policy in the Optimization problem given by equation \refeq{Opt} to determine the optimal movement of the transceiver's till time $t$.
\subsection{Information Structure }
Let $\F_t$ denote the information received by mobile transceiver $m$ and the central estimator till time $t$. Recall that $u(t) \in \{1,2\}$ denotes the location of mobile transceiver at time $t$. Also at this time instant the mobile transceiver receives the information $X_{u(t)}(t)$ and transmits the tuple $o(t)=(u(t), X_{u(t)})$. Thus $\F_t$  is given by \eq{\F_t=\sigma(o(1),o(2),\dots  ,o(t)).} Thus after time $t$ the mobile transceiver and central estimator have knowledge of some event $ A \in \F_t$.
\subsection{Admissible Scheduling Policies }\label{APP}
Scheduling policy is considered admissible if the decision of location of mobile transceiver at time $t+1$ is based on the information it has till time $t$. This information is denoted by  some singleton event  $A \in \F_t$. $A$ is one of the many sample paths the mobile transceiver can see at time $t$.
  We allow  mixed (randomized ) scheduling policies. Thus given an event $A \in \F_t$ and policy $\theta \in \Theta$ the location of mobile transceiver at time $t+1$ is given by, \eq{u(t+1)_{|A}=\theta(A)=\begin{cases}
      1, & w.p  \  p_A\\
      2, & w.p \ 1-p_A.
    \end{cases}}
\subsection{Admissible Estimation Policies}\label{AEP}
Estimation policy is considered admissible if the estimates at time $t$ of processes $\{X_i(t): i \in \{1,2\}\}$ are given by all the information received by estimator till time $t$. This information is denoted by  some singleton event $A \in \F_t$. $A$ is one of the many sample paths the mobile transceiver can see at time $t$. Thus an admissible policy $\pi$ would map some event $A \in \F_t$ to the estimates of the processes $\{\hat{X_i(t)}: i \in \{1,2\}\}$ at time $t$.
\subsection{Optimal Estimation Policy}
\begin{thm}\label{theta}
For a given location $u(t)$ of the mobile transceiver and its input \\ $o(t)=(u(t), X_{u(t)}(t))$ to the estimator at time $t$ , the optimal estimation policy $\pi^*$ for a mean square distortion metric is independent  of the mobile transceiver's locations till time $t-1$  and is given by,
\eq{&\hat{X_i(t)}=\begin{cases}
      X_i(t), & \text{if}\ i=u(t) \\
      \E[X_i(t)], & \text{otherwise}.
    \end{cases}}

\end{thm}
\begin{proof}


Recall that  $\F_t=\sigma(o(1),o(2),... o(t))$ denotes the information received by the estimator till time $t$. Consider the events   $\{A_i \in \F_t : \forall i \in \{1,2\}\}$  consisting of any outcomes till time $t$ such that  the  mobile transceiver is at location $i$ at time $t$.
 We will prove that for any such event $A_i \in \F_t$ , the  policy $\pi^*$ minimizes the conditional expectation \eq{\sum_{i \in [2]}\E[(X_i(t)-\hat{X_i(t)})^2|A_i].} This in-turn implies that the sum of distortion at time $t$  given by, \eq{\sum_{i \in [2]}\E[(X_i(t)-\hat{X_i(t)})^2]} is minimized.

Thus event $A_i \in \F_t$, will have mobile transceiver location $u(t)=i$. Also for such an event $A_i$, let $j$ be the index of sensor location at which the transceiver is not present.

 Thus the estimate, 
 \begin{equation}
 \hat{X_i(t)}= X_i(t) 
 \end{equation}
  will make the distortion equal to zero, i.e \begin{align}\label{mse_0}
  E[(X_i(t)-X_i(t))^2|A_i]=0.
  \end{align}

Also at this time instant  and given an event $A_i \in \F_t$, the information of the other sensor node $s_j$ is not transmitted. Thus we want to chose an estimator $\hat{X_{j}(t)}$ such that the conditional expected mean square error is minimized, i.e
\eq{\hat{X_{j}(t)}=arg \min_{y} \E[(X_{j}(t)-y)^2|A_i].}  It is well known that the estimate for minimizing the mean square distortion is the conditional expectation of $X_j(t)$ given $A_i$, i.e 
\begin{align}\label{mse}\hat{X_{j}(t)}&=\E[X_{j}(t)|A_i] \nonumber\\
&=\E[X_{j}(t)]=\mu_j
\end{align}
Here the last equality follows from the \textit{iid} nature of the process and the independence among both the processes.
\end{proof}

Note that from Equation \refeq{mse} we have that the mean square error  for the sensor at which the mobile transceiver is not present is given as follows,
\begin{align}\label{mse_2}
\E[(X_{j}(t)-\hat{X_{j}(t)})^2|A_i]=\sigma_j^2
\end{align}



\subsection{Optimal Scheduling Policy}
\begin{thm}\label{Placement  Policy}
A feasible solution exists for the Optimization problem in \refeq{Opt} if
\eq{1-\frac{\epsilon_1}{T \sigma_1^2} \leq \frac{\epsilon_2}{T \sigma_2^2}.}
The optimal scheduling policy $\theta^*$ for the mobile transceiver determines its location $u(t)$ as follows, 
\eq{&\hat{u(t)}=\begin{cases}
      1, & \text{w.p}\ p* \\
     2, & \text{w.p}\ 1-p*
    \end{cases} \forall t \in [T], \forall A \in \F_{t-1}.}
    Here $p^*$ is given as follows,
    \begin{align}\label{placement}
    & p^* =\begin{cases}
      \max \{1-\frac{\epsilon_1}{T \sigma_1^2}, 0\} & \sigma_1 ^2\leq \sigma_2^2\\
     \min\{\frac{\epsilon_2}{T \sigma_2^2},1\}, &  \sigma_1^2 > \sigma_2^2.
    \end{cases}.
    \end{align}
\end{thm}

\begin{proof}
Consider an event $E_{t-1,j}\in \F_{t-1} $ to consist of union of all sample paths till time $t-1$ for which the mobile transceivers location $(u(1)\dots u(t-1)) \in \{1,2\}^{t-1}$ is same. A mobile transceiver can have any one of the $2$ locations for a given time period, thus total number of different tuples $(u(1)\dots u(t-1)) \in \{1,2\}^{t-1}$are $2^{t-1}$, which in turn implies that there are $2^{t-1}$ events like $E_{t-1,j}$. Also note that  $\mathbb{P}(\cup_{j \in [2^{t-1}]}E_{t-1,j})=1$.

Let $\lambda(E_{t-1,j}) \in   [t-1]$ , denote the number of times the mobile transceiver is at location $1$ for any sample path in $E_{t-1,j}$.

 We use the optimal estimation policy $\pi^*$ given by Theorem \ref{theta} to minimize the distortion . Thus it follows from Equations \refeq{mse_0}, \refeq{mse_2} that the mean square error for any sample path in $E_{t-1,j}$ till time $t-1$  is the same for both the nodes and is given by \eq{\sum_{v \in [t-1]}(X_1(v)-\hat{X_1(v)})^2=(t-1-\lambda(E_{t-1,j}))\sigma_1^2\\
\sum_{v \in [t-1]}(X_2(v)-\hat{X_2(v)})^2=\lambda(E_{t-1,j})\sigma_2^2.} The constant distortion till time $t-1$ for any sample path in $E_{t-1,j}$, means that we can consider only those scheduling policies $\theta$ which will map any sample path in $E_{t-1,j}$ to the same mixed scheduling strategy  at time $t$ . The constant distortion for any sample path in $E_{t-1,j}$ ensures that the optimal policy $\theta^*$ lies in such restricted class of policies. We denote the mixed strategy for such a policy $\theta$ as follows,
\eq{u(t+1)_{|E_{t-1,j}}=\theta(E_{t-1,j})=\begin{cases}
      1, & w.p  \  p_{E_{t-1,j}}\\
      2, & w.p \ 1-p_{E_{t-1,j}}.
    \end{cases}}
    
    Thus the distortion at sensor nodes $1$ and $2$ at time $t$ for such a policy $\theta$  is given as follows,
    \eq{\E[(X_i(t)-\hat{X_i(t)})^2|E_{t-1,j}]=\begin{cases}
      \sigma_1^2 (1-p_{E_{t-1,j}}), &   \  i=1\\
      \sigma_2^2 (p_{E_{t-1,j}}) , &  \  i=2.
    \end{cases}}
    The above equation follows directly from the equations  \refeq{mse_0}, \refeq{mse_2}.
    
    Thus total distortion at time $t$ is given as,
    \eq{
   \sum_{i \in [2]}\E[(X_i(t)-\hat{X_i(t)})^2]= \sum_{i \in [2], j \in [2^{t-1}]}\E[(X_i(t)-\hat{X_i(t)})^2|E_{t-1,j}] \mathbb{P}(E_{t-1,j})\\
    \sum_{i \in [2]}\E[(X_i(t)-\hat{X_i(t)})^2]=\sigma_1^2 \sum_{ j \in [2^{t-1}]} (1-p_{E_{t-1,j}})\mathbb{P}(E_{t-1,j})+\sigma_2^2\sum_{ j \in [2^{t-1}]} (p_{E_{t-1,j}})\mathbb{P}(E_{t-1,j}) )
   .}
Let $p_t$ be given as follows,
\begin{align}\label{soe1}
p_t=\sum_{ j \in [2^{t-1}]} (p_{E_{t-1,j}})\mathbb{P}(E_{t-1,j}).
\end{align} Note that $\mathbb{P}(E_{t-1,j}$ is function of $\{p_i: i \in [t-1]\}$. We try to optimize over these parameters $\{p_t: \forall t \in [T]\}$ to minimize the distortion, then substituting the values of $p_t$ we find the optimal mixed strategies at all time periods and for all sample paths.

The optimization problem in \ref{Opt} is thus reduced to the following optimization problem,
\begin{align}
& \ \min \sum_{ t \in [T]} \sigma_1^2(1-p_t)+\sigma_2^2(p_t)\\
& s.t. \nonumber\\
 &\ \sum_{t \in [T]} \sigma_1^2(1-p_t) \leq \epsilon_1  \nonumber\\
&\ \sum_{t \in [T]} \sigma_2^2(p_t) \leq \epsilon_2  \nonumber\\
&\ p_t \in [0,1] \quad \forall t \nonumber\\
\nonumber
\end{align}
 
 We further reduce this optimization problem by substituting $p$ as follows,
 \begin{align}\label{soe2}
 p=\frac{\sum_{ t \in [T]}p_t}{T}.
 \end{align} 
 We will then back-substitute to get values of $p_t$, and then the mixed strategies by solving the system of equations \refeq{soe2}and \refeq{soe1}.
 Once again this substitution  will reduce the optimization problem to,
 \begin{align}\label{opt_3}
& \ \min T(\sigma_1^2(1-p)+\sigma_2^2(p))\\
& s.t. \nonumber\\
 &\ T( \sigma_1^2(1-p) )\leq \epsilon_1  \nonumber\\
&\ T( \sigma_2^2(p)) \leq \epsilon_2  \nonumber\\
&\ p \in [0,1] \nonumber\\
\nonumber
\end{align}
 We can see from the optimization problem \ref{opt_3} that if
  \eq{1-\frac{\epsilon_1}{T \sigma_1^2} \leq \frac{\epsilon_2}{T \sigma_2^2} ,} only then a feasible solution will exist for this linear program. This proves the first statement of the theorem.
  
  Also the solution to the Optimization problem \refeq{opt_3}, when feasible is given by,
  \eq{& p^* =\begin{cases}
      \max \{1-\frac{\epsilon_1}{T \sigma_1^2}, 0\} & \sigma_1^2 \leq \sigma_2^2\\
     \min\{\frac{\epsilon_2}{T \sigma_2^2},1\}, &  \sigma_1^2 > \sigma_2^2.
    \end{cases}.}
    
  But $p_t=p^*$ satisfies the system of equations \refeq{soe2}. Also for this value of $p_t, \forall t \in [T]$,  \eq{p_{E_{t-1,j}}=p^* \quad \forall t-1 \in [T], j \in [2^{t-1}],} satisfy the system of equations \refeq{soe1}. 
    This proves the second statement of the solution.
  
  
  \end{proof}


The Theorem \ref{Placement  Policy} implies that the probability of visiting a sensor node increases as its variance increase. A direct consequence of Theorem \ref{Placement Policy} is the following corollary

\begin{cor}
Suppose $\epsilon_1$ and $\epsilon_2$ are such that, \eq{\epsilon_1 \geq T \sigma_1^2\\
\epsilon_2 \geq T \sigma_2^2.}
Then the optimal policy $\theta^*$ is to stay at the sensor node which has more variance,

\eq{&\hat{u(t)}=\begin{cases}
      2, & \sigma_1^2 \leq \sigma_2^2  \\
     1, &  \sigma_1^2 > \sigma_2^2
    \end{cases} \quad \forall t \in [T].}

\end{cor}
\begin{proof}
Substituting the values of $\epsilon_1$ and $\epsilon_2$ in Equation \refeq{placement} we have \begin{align}
    & p^* =\begin{cases}
      0 & \sigma_1^2 \leq \sigma_2^2\\
     1, &  \sigma_1^2 > \sigma_2^2.
    \end{cases}.
    \end{align}
    

\end{proof}

\section{Illustration}
We illustrate our model through the following example. Consider two Bernoulli Processes $X_1(t)$ and $X_2(t)$ with parameters $q_1$ and $q_2$ respectively. These processes are measured by sensor nodes $s_1$ and $s_2$. There is one mobile  transceiver $m$ to collect the data.

For this problem we wish to find the optimal estimation and scheduling policy for a Finite Time Horizon $T=10$. The maximum permissible distortion at node's $s_1$ and $s_2$ is $1$ (i.e $\epsilon_1=\epsilon_2=1$). Probability $q_2$ for Bernoulli Process $X_2(t)$ is $0.2$. We study the optimal policies for such a setup when $q_1$
 varies from $0.1$ to $0.9$.
 
 The optimal estimation policy follows from Theorem \ref{theta}.
 \eq{&\hat{X_i(t)}=\begin{cases}
 		X_i(t), & \text{if}\ i=u(t) \\
 		q_i, & \text{otherwise}.
 \end{cases} \ \forall i \in \{1,2\}, \forall t \in [T].}

From Theorem \ref{Placement  Policy} the optimal scheduling policy is given by mixed stationary policy which at every time instant goes to sensor node $s_1$ with probability $p^*$ and goes to sensor node $s_2$ with probability $1-p^*$.

We illustrate the variation in these value $p^*$ for the given example by varying
 $q_1$ in Figure \ref{fig2}.
\begin{figure}[h]
	\centering
\begin{tikzpicture}
\begin{axis}[
title={},
xlabel={$q_1$},
ylabel={$p^*$},
xmin=0, xmax=1,
ymin=0, ymax=1,
xtick={0,0.2, 0.4, 0.6 , 0.8, 1},
ytick={0,0.2, 0.4, 0.6 , 0.8, 1},
legend pos=north west,
ymajorgrids=true,
xmajorgrids=true,
grid style=dashed,
  mark size=2.0pt,
]

\addplot[
color=blue,
mark=*,
]
coordinates {
	(0.1, 0)(0.2, .375)(0.3, 0.625)(0.4, 0.625)(0.5, 0.625)(0.6, 0.625)(0.7, 0.625)(0.8, 0.375)(0.9, 0) 
};


\end{axis}
\end{tikzpicture}
\caption{Variation of $p^*$ with respect to $q_1$}
\label{fig2}
\end{figure}

As $q_1$ increase from $0.1$ to $0.5$, $\sigma_1^2$ increases. Also when  $q_1$ increase from $0.5$ to $0.9$, $\sigma_1^2$ decreases. 
When $q_1 \in[0, 0.2]$ ,$\sigma_1^2 \leq \sigma_2^2$.  When $q_1 \in[0.2, 0.8]$ , $\sigma_1^2 \geq \sigma_2^2$.
Also when $q_1 \in [0.8,1]$ , $\sigma_1^2 \leq \sigma_2^2$. Thus the transceiver has a greater probability $p^*$ of going to sensor node $s_1$ when $q \in [0.2, 0.8]$.
\section{Conclusion}

We proposed a model for collection of data in wireless sensor networks through mobile transceivers.This  models tackles the problem of limited battery availability at sensor nodes.

For the two sensors and one transceiver model we found the optimal estimation and scheduling policy. We proved that the optimal estimation policy depends only on the current location of mobile transceiver. We also showed that the optimal scheduling of mobile transceiver is a mixed policy which remains same for all time periods. In effect we showed that the existence of stationary optimal polices for the problem proposed.


\bibliographystyle{plainnat}
\bibliography{concentration}




\end{document}