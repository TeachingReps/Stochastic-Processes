\documentclass[a4paper,english,12pt]{article}
\usepackage{%
	amsfonts,%
	amsmath,%	
	amssymb,%
	amsthm,%
	algorithm,%
	babel,%
	bbm,%
	etex,%
	caption,%
	centernot,%
	color,%
	dsfont,%
	enumerate,%
	epsfig,%
	geometry,%
	graphicx,%
	hyperref,%
	latexsym,%
	mathtools,%
	multicol,%
	pgf,%
	pgfplots,%
	pgfplotstable,%
	pgfpages,%
	proof,%
	psfrag,%
	subfigure,%	
	tikz,%
	ulem,%
	url%
}	
\usepackage[noend]{algpseudocode}
\usepackage[mathscr]{eucal}
\usepgflibrary{shapes}
\usetikzlibrary{%
  arrows,%
  backgrounds,%
  chains,%
  decorations.pathmorphing,% /pgf/decoration/random steps | erste Graphik
  decorations.text,%
  fit,%
  matrix,%
  patterns,%
  petri,%
  positioning,% wg. " of "
  plotmarks,%
  scopes,%
  shadows,%
  shapes,%
  shapes.arrows,%
  shapes.callouts,%
  shapes.misc% wg. rounded rectangle
}

\theoremstyle{plain}
\newtheorem{thm}{Theorem}[section]
\newtheorem{lem}[thm]{Lemma}
\newtheorem{prop}[thm]{Proposition}
\newtheorem{cor}[thm]{Corollary}

\theoremstyle{definition}
\newtheorem{defn}[thm]{Definition}
\newtheorem{conj}[thm]{Conjecture}
\newtheorem{exmp}[thm]{Example}
\newtheorem{assum}[thm]{Assumptions}
\newtheorem{axiom}[thm]{Axiom}

\theoremstyle{remark}
\newtheorem{rem}{Remark}
\newtheorem{note}{Note}
\newtheorem{fact}{Fact}

\definecolor{lightgray}{gray}{0.9}

%\DeclarePairedDelimiter{\ceil}{\left\lceil}{\right\rceil}%
\newcommand{\eq}[1]{\begin{align*}#1\end{align*}}
\newcommand{\ceil}[1]{\left\lceil#1\right\rceil}%
\newcommand{\norm}[1]{\left\lVert#1\right\rVert}%
\newcommand{\indep}{\!\perp\!\!\!\perp}%
\DeclarePairedDelimiter\abs{\lvert}{\rvert}%
\newcommand\numberthis{\addtocounter{equation}{1}\tag{\theequation}}
\newcommand{\tr}{\operatorname{tr}}
\newcommand{\R}{\mathbb{R}}
\newcommand{\N}{\mathbb{N}}
\newcommand{\E}{\mathbb{E}}
\newcommand{\Z}{\mathbb{Z}}
\newcommand{\B}{\mathscr{B}}
\newcommand{\C}{\mathcal{C}}
\newcommand{\T}{\mathscr{T}}
\newcommand{\F}{\mathcal{F}}
\newcommand{\G}{\mathcal{G}}
\newcommand{\X}{\mathcal{X}}
%\newcommand{\ba}{\begin{align*}}
%\newcommand{\ea}{\end{align*}}
\DeclareMathOperator*{\argmax}{arg\,max}
\renewcommand{\qedsymbol}{$\blacksquare$}
\makeatletter
\def\BState{\State\hskip-\ALG@thistlm}
\makeatother

\makeatletter
\def\th@plain{%
  \thm@notefont{}% same as heading font
  \itshape % body font
}
\def\th@definition{%
  \thm@notefont{}% same as heading font
  \normalfont % body font
}
\makeatother
\date{}
 
%opening
\title{\textbf{Proposal for SPQT Project\\Characterization of Earthquake as a Stochastic Process}}
\author{Nazeeh K M}

\begin{document}
\maketitle

\section{Motivation}
Prediction of earthquake occurrence and its magnitude is of great interest, as earthquakes create catastrophe in the affected area. If we assume earthquake to be a random process and we can use probability models for the prediction. Different available random process models which can be used to obtain the arrival of an earthquake have to be studied in detail, so that all the characteristics can be captured.
\section{System Model}
The temporal occurrence of earthquake is most commonly described as a Poisson process. It provides a simple framework for evaluation of probability of occurrence of an earthquake, but it does not take into account prior seismic activity. Other models such as Non-homogeneous Poisson models, renewal models, etc. can also be used to account for other factors such as variation with time, use of arrival time distributions other than exponential, etc. Markov and semi-Markov models can also be used for the same.

Poisson model is most widely used for for seismic hazard analysis because of its simplicity, ease of use and lack of sufficient data to support more sophisticated models. But if more data becomes available other models can also be used, as it will become possible to estimate the parameters required for the sophisticated models.
\section{Problem Statement}
Use the available models to characterize earthquakes and an attempt to predict earthquakes under given conditions.
\subsection{Justification}
For a given site and source, prior seismic activity will reduce the probability of occurrence of another earthquake, which cannot be captured by Poisson model. So the use of other models have to be included in the probabilistic seismic analysis, to obtain realistic results.
\section{Plan}
\begin{itemize}
\item Week 1: Literature survey - Use of Poisson process and its limitations
\item Week 2: Literature survey - Use of other models and their limitations
\item Week 3: Site specific study - Using Poisson model
\item Week 4: Site specific study - Using other models
\item Week 5: Check the applicability to other sites
\item Week 6: Summarize
\end{itemize}

\end{document}