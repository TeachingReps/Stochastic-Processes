% !TEX spellcheck = en_US
% !TEX spellcheck = LaTeX
\documentclass[a4paper,10pt,english]{article}
\usepackage{%
	amsfonts,%
	amsmath,%	
	amssymb,%
	amsthm,%
%	babel,%
	bbm,%
	%biblatex,%
	caption,%
	centernot,%
	color,%
	enumerate,%
	epsfig,%
	epstopdf,%
	etex,%
	geometry,%
	graphicx,%
	hyperref,%
	latexsym,%
	mathtools,%
	multicol,%
	pgf,%
	pgfplots,%
	pgfplotstable,%
	pgfpages,%
	proof,%
	psfrag,%
	subfigure,%	
	tikz,%
	ulem,%
	url%
}	

\usepackage[mathscr]{eucal}
\usepgflibrary{shapes}
\usetikzlibrary{%
  arrows,%
  backgrounds,%
  chains,%
  decorations.pathmorphing,% /pgf/decoration/random steps | erste Graphik
  decorations.text,%
  matrix,%
  positioning,% wg. " of "
  fit,%
  patterns,%
  petri,%
  plotmarks,%
  scopes,%
  shadows,%
  shapes.misc,% wg. rounded rectangle
  shapes.arrows,%
  shapes.callouts,%
  shapes%
}

\theoremstyle{plain}
\newtheorem{thm}{Theorem}[section]
\newtheorem{lem}[thm]{Lemma}
\newtheorem{prop}[thm]{Proposition}
\newtheorem{cor}[thm]{Corollary}

\theoremstyle{definition}
\newtheorem{defn}[thm]{Definition}
\newtheorem{conj}[thm]{Conjecture}
\newtheorem{exmp}[thm]{Example}
\newtheorem{assum}[thm]{Assumptions}
\newtheorem{axiom}[thm]{Axiom}

\theoremstyle{remark}
\newtheorem{rem}[thm]{Remark}
\newtheorem{note}[thm]{Note}

\newcommand{\norm}[1]{\left\lVert#1\right\rVert}
\newcommand{\indep}{\!\perp\!\!\!\perp}
\DeclarePairedDelimiter\abs{\lvert}{\rvert}%
%\DeclarePairedDelimiter\norm{\lVert}{\rVert}%
\newcommand{\tr}{\operatorname{tr}}
\newcommand{\R}{\mathbb{R}}
\newcommand{\Q}{\mathbb{Q}}
\newcommand{\N}{\mathbb{N}}
\newcommand{\E}{\mathbb{E}}
\newcommand{\Z}{\mathbb{Z}}
\newcommand{\B}{\mathscr{B}}
\newcommand{\C}{\mathcal{C}}
\newcommand{\T}{\mathscr{T}}
\newcommand{\F}{\mathcal{F}}
\newcommand{\G}{\mathcal{G}}
%\newcommand{\ba}{\begin{align*}}
%\newcommand{\ea}{\end{align*}}

% Debug
\newcommand{\todo}[1]{\begin{color}{blue}{{\bf~[TODO:~#1]}}\end{color}}


\makeatletter
\def\th@plain{%
  \thm@notefont{}% same as heading font
  \itshape % body font
}
\def\th@definition{%
  \thm@notefont{}% same as heading font
  \normalfont % body font
}
\makeatother
\date{}
\title{Lecture 27: Brownian Motion}
\author{}

\begin{document}
\maketitle
\section{Introduction}
\begin{defn} A continuous time stochastic process $\{X(t), t \geqslant 0\}$ is called \textbf{standard Brownian motion process} if the following hold.
\begin{enumerate}[i)]
\item $X(0) = 0$.
\item Process has stationary and independent increments.
\item For every $t > 0$, process is normally distributed with mean $0$ and variance $t$.
\end{enumerate}
\end{defn}

\begin{defn}
Let $S_n$ be a simple symmetric random walk with step sizes $\{X_i: i \in \N\}$ being iid Rademacher random variables. That is,
\begin{align*}
\Pr\{X_i = 1\} &= \Pr\{X_i = -1\} = \frac{1}{2}.
\end{align*}
Let $\Delta t > 0$, and we choose $\Delta x = \sqrt{\Delta t}, n(t) = \frac{t}{\Delta t}$ to define \textbf{limiting scaled symmetric random walk}
\begin{align*}
X(t) \triangleq \lim_{\Delta t \to 0}\Delta xS_{n(t)}.
\end{align*}
\end{defn}
\begin{lem} For scaled symmetric random walk $\Delta xS_{n(t)}$, we have 
\begin{xalignat*}{3}
&\E \Delta xS_{n(t)} = 0,&&\Var \Delta xS_{n(t)} = \frac{(\Delta x)^2}{\Delta t} t.
\end{xalignat*}
\end{lem}
\begin{thm} Limiting scaled symmetric random walk $X(t)$ is a standard Brownian motion.
\end{thm}
\begin{proof} We will show that the limiting process satisfies all three criterion in the definition of standard Brownian motion. 
\begin{enumerate}[i)]
\item First condition holds trivially since $X(0) = S_0$. 
\item Since step sizes are independent random variables, it follows that the random walk $\Delta xS_{n(t)}$ has independent increments. 
Further, since step sizes have identical distribution, distribution of $S_{n(t)} - S_{n(s)}$ is same as distribution of $S_{n(t-s)}$ for $s,t$ multiples of $\Delta t$. 
By continuity of probability, we conclude that $X(t)$ as stationary independent increments. 
\item From previous lemma, it is clear that mean and variance of $X(t)$ are zero and $t$ respectively. 
Further, it follows from central limit theorem that limit of $\frac{S_{n(t)}}{\sqrt{n(t)}}= \frac{X(t)}{\sqrt{t}}$ is standard normal random variable, as $n(t) \to \infty$.
\end{enumerate}
\end{proof}
\begin{thm} Standard Brownian motion $X(t)$ is a.s. path-wise continuous and nowhere differentiable function.
\end{thm}
\begin{rem} Since standard Brownian motion satisfies independent increment property, it is a Markov process.
\end{rem}
\begin{rem} Due to stationary increments property, distribution of $X(t) - X(s)$ is same as distribution of $X(t-s)$.
\end{rem}
\begin{cor} Finite dimensional density of a standard Brownian motion at times $t_1 < t_2 < \ldots < t_n$ is given by 
\begin{align*}
f(x_1, x_2, \ldots, x_n) = \Pr\{X(t_i) = x_i, i \in [n]\} = \prod_{i \in [n]}f_{t_i - t_{i-1}}(x_i - x_{i-1}),
\end{align*}
where $t_0 = x_0 = 0$ and 
\begin{align*}
f_t(x) = \frac{1}{\sqrt{2 \pi t}}\exp\left(-\frac{x^2}{2t}\right).
\end{align*}
\end{cor}
\begin{defn} A stochastic process $\{X(t), t\geqslant 0\}$ is called a \textbf{Gaussian process} if all finite dimensional distributions are multivariate normal.
\end{defn}
\begin{cor} Standard Brownian motion process is a Gaussian process.
\end{cor}
\begin{prop} Conditional density of standard Brownian motion $X(s)$ given $X(t) = B$ for $s < t$ is given by
\begin{align*}
f_{s|t}(x|B) &= \frac{1}{\sqrt{2\pi s(t-s)/t}} \exp\left(-\frac{(x-Bs/t)^2}{2s(t-s)/t}\right), 
\end{align*}
%for some positive normalization constant $K$.
\end{prop}
\begin{proof} Result follows from Bayes rule for densities applied to standard Brownian motion to get
\begin{align*}
f_{s|t}(x|B) &= \frac{f_s(x)f_{t-s}(B-x)}{f_t(B)} = \frac{1}{\sqrt{2\pi s(t-s)/t}}\exp\left(-\frac{x^2}{2s}-\frac{(B-x)^2}{2(t-s)} + \frac{B^2}{2t}\right)\\
&=  \frac{1}{\sqrt{2\pi s(t-s)/t}}\exp\left(-\frac{t(x-sB/t)^2}{2s(t-s)}\right).
\end{align*}
\end{proof}
\begin{cor} Conditional mean and variance of standard Brownian motion $X(s)$ given $X(t) = B$ for $s < t$ is given by 
\begin{xalignat*}{3}
&\E[X(s)|X(t) = B] = B \frac{s}{t}, && \Var[X(s)|X(t) = B] = s\left(1-\frac{s}{t}\right).
\end{xalignat*}
\end{cor}
\begin{cor} Let $s/t = \alpha < 1$. Then, the conditional distribution of $X(s)$ given $X(t)$ is normal with mean $\alpha X(t)$ and variance $t \alpha(1-\alpha)$.
\end{cor}
\begin{thm} A continuous time stochastic process $\{X(t), t \geqslant 0\}$ is a standard Brownian motion process iff it is a zero mean Gaussian process with 
\begin{align*}
\Cov(X(t), X(s)) &= s \wedge t.
\end{align*}
\end{thm}
\begin{proof} Let $X(t)$ be a standard Brownian motion, then it is a zero mean Gaussian process. 
Therefore, by linearity of covariance for independent random variables, we have
\begin{align*}
\Cov(X(t),X(s))  = \Cov(X(t)-X(s),X(s)) + \Var X(s).
\end{align*}
Let $s \leq t$, then result follows from that $X(t) - X(s)$ is independent of $X(s)$ by independent increments property and that $X(s)$ is zero mean. 
\end{proof}
\begin{defn} Consider a standard Brownian motion process $\{X(t), t \geqslant 0\}$ in duration $[0,1]$. Conditional stochastic process $\{X(t), t \in [0,1]| X(1) = 0\}$ is called a \textbf{Brownian bridge}.
\end{defn}
\begin{prop} Brownian bridge is a zero-mean Gaussian process with 
\begin{align*}
\Cov(X(s), X(t)|X(1) = 0) &= s(1- t),\text{ for } s \in [t, 1].
\end{align*}
\end{prop}
\begin{proof} We use tower property of conditional expectations, and mean and variance of conditional Brownian motion to obtain
\begin{align*}
\Cov(X(s), X(t)|X(1) = 0) & = \E \left[X(t) \E[X(s) | X(t)]| X(1) = 0\right] = \frac{s}{t} \E \left[X(t)^2| X(1) = 0\right] = s(1-t).
\end{align*}
\end{proof}
\begin{rem} Brownian bridge is equivalent to a zero mean Gaussian process with covariance function $s(1-t)$ for $s \leq t$.
\end{rem}
\begin{prop} Let $\{X(t): t \geqslant 0\}$ is a Brownian motion, then $\{Z(t) : t \in [0, 1]\}$ is a Brownian bridge process when $Z(t) = X(t) - tX(1)$.
\end{prop}
\begin{proof} It is clear that $Z(t)$ is a Gaussian process. 
It suffices to show that it is zero mean and $\Cov(Z(s), Z(t)) = s(1-t)$ for $s \leq t$. 
It is clear that $\E Z(t) = 0$ for all $t \in [0,1]$. 
Further, for $s \leq t$, we have 
\begin{align*}
\Cov(Z(s),Z(t)) &= \Cov(X(s) - sX(1), X(t) - tX(1)) = s - st - st +st = s(1 - t).
\end{align*}
\end{proof}
\begin{defn} 
Let $\{X_i: i \in \N\}$ be a sequence of iid random variables, then we can define number of first $n$ random variables exceeding $s$ as 
\begin{align*}
N_n(s) &= \sum_{i \in [n]}1_{\{X_i \leq s \}},
\end{align*}
and \textbf{empirical distribution function}
\begin{align*}
F_n(s) &= \frac{1}{n}\sum_{i \in [n]}1_{\{X_i \leq s \}}.
\end{align*}
For each $n \in \N$, we can define a process $\{\alpha_n(s): s \in \R\}$, where $\alpha_n(s) = \sqrt{n}(F_n(s) - s )$. \textbf{Limiting process} is defined as $\alpha(s) = \lim_{n \in \N}\alpha_n(s)$ for all $s$.
\end{defn}
\begin{lem} $N_n(t) - N_n(s)$ given $N_n(s)$ is binomial $(n - N_n(s), (t-s)/(1-s))$.
\end{lem}
\begin{proof} Clearly, $N_n(t) - N_n(s)$ given $N_n(s)$ is sum of $n-N_n(s)$ Bernoulli random variables each with mean
\begin{align*}
\E[1_{\{X_i \in [s, t)\}}| X_i > s] = \frac{t-s}{1-s}.
\end{align*}
\end{proof}
\begin{prop} Let $\{X_i : i \in \N\}$ be a sequence of iid uniform $(0,1)$ random variables. Then the following are true.
\begin{enumerate}[i)]
\item For a fixed $s \in (0,1)$, we have  $\lim_{n \in \N} F_n(s) = s$, a. s.
\item Glivenko-Cantelli Theorem: 
\begin{align*}
\lim_{n \in \N}\sup_{s \in (0,1)}|F_n(s) - s| = 0,\text{ a. s.}
\end{align*}
\item For any $s \in (0,1)$, limiting random variable $\alpha(s) = \lim_{n \in \N}\alpha_n(s)$ is zero mean Gaussian with variance $s(1-s)$.
%\begin{align*}
%\Pr\left\{ F_n(s) > s + \frac{t}{\sqrt{n}} \right\} &= \frac{1}{\sqrt{2\pi s(1-s)}} \int_{x > t}\exp\left(-\frac{x^2}{2s(1-s)}\right).
%\end{align*}
That is, \item Point-wise limiting process $\alpha = \lim_{n \in \N}\alpha_n$ is a Brownian bridge process. 
%zero mean Gaussian process with covariance 
%\begin{align*}
%\Cov(\alpha(s), \alpha(t)) &= s(1-t).
%\end{align*}
\end{enumerate}
\end{prop}
\begin{proof} Recall that $1_{\{X_i \leq s\}}$ is a Bernoulli random variable with mean $s$. 
Hence, $N_n(s)$ is binomial $(n,s)$.
\begin{enumerate}[i)]
\item Follows from strong law of large numbers.
\item Omitted.
\item Follows from central limit theorem.
\item It suffices to show $\E \alpha_n(s) = 0$ and $\Cov(\alpha_n(s), \alpha_n(t)) = s(1-t)$ for $s \in [t,1]$ for all $n \in \N$. 
To this end, we observe that
\begin{align*}
\Cov(\alpha_n(s), \alpha_n(t)) &= \frac{1}{n} \E N_n(s) N_n(t) - nst.
\end{align*}
Further, we see for $s \in [t,1]$, 
\begin{align*}
\E[N_n(s) N_n(t)] &= \E[\E[N_n(s)N_n(t)|N_n(s)]] = \E\left[ N_n(s)\left(N_n(s) + (n-N_n(s))\frac{t-s}{1-s}\right)\right]\\
& = \E N_n(s)^2\frac{1-t}{1-s} + n\frac{t-s}{1-s}\E N_n(s) = ns(1-t) + n^2st.
\end{align*}
Hence, the result follows.
\end{enumerate}
\end{proof}
\begin{rem} This can be generalized to sequence of iid random variables $\{X_i :  i \in \N\}$ with a common continuous distribution function $F$, where $\{Y_i = F(X_i): i \in \N\}$ is an iid uniform $(0,1)$ random variable. %In this case, 
%\begin{xalignat*}{5}
%&N_n(s) = \sum_{i=1}^n1_{\{F(X_i) < s\}}, && F_n(s) = \frac{N_n(s)}{n},&& \alpha_n(s) = \sqrt{n}(F_n(s) - s).
%\end{xalignat*}
\end{rem}
\begin{prop} Let $X$ be a sequence of iid random variables with common continuous distribution function $F$, then 
\begin{align*}
\lim_{n \in \N}\Pr\left\{\sqrt{n}\sup_{x}|F_n(x) - F(x)| < a\right\} & = \Pr\left\{\max_{t \in [0,1]}|Z(t)| < a\right\},
\end{align*}
where $\{Z(t) : t \geqslant 0\}$ is a Brownian bridge process.
\end{prop}
\begin{proof}
Let $F(y_s) = s$, and define
\begin{align*}
\alpha_n(s) = \sqrt{n}(F_n(y_s)-F(y_s)) = \sqrt{n}(F_n(F^{-1}s) - s) = \sqrt{n}\left(\frac{1}{n}\sum_{i \in [n]}1_{\{F(X_i) < s\}} - s\right).
\end{align*}
Then, process $\{\alpha_n(s) : s \in [0,1]\}$ converges to Brownian bridge $\{Z(t) : t \in [0,1]\}$. 
Hence, $\sup_x \sqrt{n}|F_n(x)-F(x)|$ converges to $\sup_{t \in [0,1]}|Z(t)|$ in distribution.
\end{proof}
\section{Hitting Times}
\begin{defn} The first time a Brownian motion $\{X(t): t \geqslant 0\}$ hits $a$ is denoted by $T_a$.
\end{defn}
\begin{thm} Distribution of hitting time of a standard Brownian motion $X(t)$ is given by
\begin{align*}
\Pr\{T_a \leq t\} &= \frac{2}{\sqrt{2 \pi}}\int_{y > |a|/\sqrt{t}}\exp\left(-\frac{y^2}{2}\right)dy.
\end{align*}
\end{thm}
\begin{proof} First, we observe that by symmetry we have
\begin{align*}
\Pr\{X(t) \geq a\} &= \Pr\{X(t) \geq a | T_a \leq t\}\Pr\{T_a \leq t\} = \frac{1}{2}\Pr\{T_a \leq t\}.%+  \Pr\{X(t) \geq a | T_a> t\}\Pr\{T_a > t\} 
\end{align*}
Since we know distribution of $X(t)$ is zero mean Gaussian with variance $t$, for $a > 0$ and substituting $y = x/\sqrt{t}$ we get
\begin{align*}
\Pr\{X(t) \geq a\} &= \frac{1}{\sqrt{2\pi}}\int_{y > a/\sqrt{t}}\exp(-\frac{y^2}{2})dy.
\end{align*}
By symmetry, $T_a$ for $a < 0$ has same distribution as $T_{-a}$ and hence the result follows.
\end{proof}
\begin{cor} Hitting time $T_a$ is finite a.s. but has infinite mean.
\end{cor}
\begin{proof} It follows from continuity of probability that 
\begin{align*}
\Pr\{T_a < \infty \} &= \lim_{t \to \infty}\Pr\{T_a \leq t\} = 1.
\end{align*}
Further, using definition of expectation, exchanging integrals, and lower bounding the integrand by its minimum, we get
\begin{align*}
\E T_a &= \int_{0}^{\infty}\Pr\{T_a > t \} dt %= \frac{2}{\sqrt{2\pi}}\int_{0}^{\infty}\int_{0}^{|a|/\sqrt{t}}e^{-y^2/2}dy dt 
= \frac{2}{\sqrt{2\pi}}\int_{0}^{\infty}\int_{0}^{a^2/y^2} e^{-y^2/2} dt dy \geq \frac{2a^2e^{-\frac{1}{2}}}{\sqrt{2\pi}}\int_{0}^{1}\frac{1}{y^2}dy.
\end{align*}
\end{proof}
\begin{rem} Hitting time $T_n$ for a symmetric random walk is finite a.s. and has infinite mean.
\end{rem}

\begin{rem} For a standard Brownian motion $X(t)$, 
\begin{align*}
\Pr\left\{\max_{s \in [0,1]}X(s) \geq a \right\} &= \Pr\{T_a \leq t\}.
\end{align*}
\end{rem}

\section{Arcsine laws}
\begin{thm} For a standard Brownian motion $X(t)$ and $0 < t_1 < t_2$, we have
\begin{align*}
\Pr\left\{  X(s) = 0, \text{ for some }s \in (t_1,t_2) \right\} &= 1 - \frac{2}{\pi}\arcsin\sqrt{x}.
\end{align*}
\end{thm}
\begin{proof} We can condition the event $\{ X(s) = 0, \text{ for some }s \in [t_1,t_2] \}$ on $\{X(t_1) = x\}$, and use continuity of Brownian motion sample path and symmetry to get
\begin{align*}
\Pr\left\{X(s) = 0, \text{ for some }s \in (t_1,t_2) | X(t_1) = x\right\} &= \Pr\{T_{|x|} \leq t_2 - t_1\}.
\end{align*}
Result follows from evaluating unconditional probability since we know distribution of hitting time.
\end{proof}

\begin{cor} For a standard Brownian motion $X(t)$ and $x \in (0,1)$
\begin{align*}
\Pr\left\{  X(s) \neq 0, s \in (xt,t) \right\} &= \frac{2}{\pi}\arcsin\sqrt{x}.
\end{align*}
\end{cor}
%\begin{prop} For $x \in (0,1)$ and a standard Brownian motion $X(t)$
%\begin{align*}
%\Pr\{X(t) \neq 0, t \in (xt, t)\} &= \frac{2}{\pi}\arcsin\sqrt{x}.
%\end{align*}
%\end{prop}


\begin{defn} For a simple symmetric random walk $\{S_n: n \in \N_0\}$ with iid step sizes $\{X_n: n \in \N\}$, we can define a regenerative process $\{Y_n: n \in \N\}$ as 
\begin{align*}
Y_n = 1_{\{S_n > 0\}} - 1_{\{S_n < 0\}},~n \in \N.
\end{align*}
Here the renewal times are 
\begin{align*}
T_n &= \inf\{ i > T_{n-1}: S_i = 0\}.
\end{align*}
\end{defn}
\begin{rem} Let $u_n \triangleq \Pr\{S_{2n} = 0\}$, then $u_n = \binom{2n}{n}\left(\frac{1}{2}\right)^{2n}$ and $u_n = \frac{2n-1}{2n}u_{n-1}$. By Stirling's approximation $n! \sim n^{n+0.5}e^{-n}\sqrt{2\pi}$, we get $u_n \approx 1/\sqrt{n\pi}$. That is, $\lim_{n \in \N}u_n = 0$
\end{rem}
\begin{lem}[Ballot Problem] For a simple symmetric random walk, the following holds for $n > m$
\begin{align*}
P_{n,m} &\triangleq \Pr\{T_1 > n+m | S_{n+m} = n-m\} = \frac{n-m}{n+m}.
\end{align*}
\end{lem}
\begin{proof}
We can write recursively,
\begin{align*}
P_{n,m} &= \Pr\{T_1 > n+m| X_{n+m} = 1, S_{n+m} = n-m\}\frac{n}{n+m} + \Pr\{T_1 > n+m| X_{n+m} = -1, S_{n+m} = n-m\}\frac{m}{n+m},\\
&= \Pr\{T_1 > n+m-1 | S_{n+m-1} = n-m-1\}\frac{n}{n+m} + \Pr\{T_1 > n+m-1 | S_{n+m-1} = n-m+1 \}\frac{m}{n+m},\\
&= P_{n-1,m}\frac{n}{n+m} +P_{n,m-1}\frac{m}{n+m}.
\end{align*}
Result follows by induction.
\end{proof}
\begin{lem} For a simple symmetric random walk $S_n$, with the associated renewal times are distributed as 
\begin{align*}
\Pr\{T_1 = 2n\} &= \frac{\binom{2n}{n}\left(\frac{1}{2}\right)^{2n}}{2n-1} = \frac{u_n}{2n-1}.
\end{align*}
\end{lem}
\begin{proof} We observe that 
\begin{align*}
\Pr\{T_1 = 2n\} &= \Pr\{T_1 = 2n | S_{2n} = 0 \}u_n.
\end{align*}
Further, conditioning on last realization of $X_i$, we get the result by symmetry
\begin{align*}
\Pr\{T_1 = 2n | S_{2n} = 0 \} &= \frac{1}{2} \Pr\{T_1 = 2n | X_{2n} = -1,S_{2n} = 0 \} +  \frac{1}{2} \Pr\{T_1 = 2n | X_{2n} = 1, S_{2n} = 0 \} \\
&=  \frac{1}{2} \Pr\{T_1 > 2n-1 | S_{2n-1} = 1\} +  \frac{1}{2} \Pr\{T_1 > 2n-1 | S_{2n-1} = -1\}  = P_{n,n-1}.
\end{align*}
\end{proof}
\begin{lem} For a simple symmetric random walk $S_n$, 
\begin{align*}
\Pr\{T_1 > 2n\} &= u_n.
\end{align*}
\end{lem}
\begin{proof} It is clear that 
\begin{align*}
\Pr\{T_1 > 2n\} &= 1 - \sum_{i=1}^{n}\Pr\{T_{1} = 2i \} = 1 - \sum_{i=1}^n\frac{u_k}{2k-1}.
\end{align*}
Result follows by induction since $u_{2n-1} = 2n u_n/(2n-1)$.
\end{proof}
\begin{cor} Simple symmetric random walk returns to $0$ with probability $1$.
\end{cor}
\begin{proof} Since $\Pr\{T_1 = \infty\}  = \lim_{n \in \N}u_n = 0$.
\end{proof}
\begin{lem} For $k \in [n]_0$, 
\begin{align*}
\Pr\{S_{2k} = 0, S_i \neq 0, i \in (2k, 2n]\} &= u_ku_{n-k}.
\end{align*}
\end{lem}
\begin{proof} It is straightforward using Markov property of random walks,
\begin{align*}
\Pr\{S_{2k} = 0, S_{2i} \neq 0, i \in [k+1, n]\} &= \Pr\{S_2k = 0\}\Pr\{S_{2i} \neq 0, i \in [k+1, n] | S_{2k} = 0\} = u_ku_{n-k}.
\end{align*}
\end{proof}
%\begin{thm}
%\end{thm}
%\begin{thm} For any $x \in (0,1)$, the proportion of time in $(0,2n)$ symmetric random walk is positive less than $x$ is given by 
%\begin{align*}
%\frac{1}{2n}\sum_{i\in [2n]}1_{\{S_i > 0\}} = \frac{2}{\pi}\arcsin\sqrt{x}.
%\end{align*}
%\end{thm}
%\begin{proof} By Stirling's approximation, we have 
%\begin{align*}
%u_ku_{n-k} \sim \frac{1}{\pi\sqrt{k(n-k)}}.
%\end{align*}
%\end{proof}
\begin{thm} For any $x \in (0,1)$ and large $n$
\begin{align*}
\Pr\{S_i \neq 0, i \in [2nx, 2n]\} \approx \frac{2}{\pi}\arcsin\sqrt{x}.
\end{align*}
\end{thm}
\begin{proof} 
Further, we can rewrite
\begin{align*}
\Pr\{S_i \neq 0, i \in [2nx, 2n]\} &= \sum_{i=0}^{nx-1}\Pr\{S_{2i} = 0, S_j \neq 0, j \in (2i,2n]\} = \sum_{k \in [0,nx-1]}u_ku_{n-k}.
\end{align*}
Result follows by Stirling's approximation, 
\begin{align*}
\sum_{k \in [0,nx-1]}u_ku_{n-k} \approx \frac{1}{\pi} \sum_{k/n \in [0,x-1/n]} \frac{1}{\sqrt{\frac{k}{n}(1-\frac{k}{n})}}\frac{1}{n} \approx \frac{1}{\pi} \int_{y \in [0,x]} \frac{1}{\sqrt{y(1-y)}}dy.
\end{align*}
\end{proof}
\end{document}
