% !TEX spellcheck = en_US
% !TEX spellcheck = LaTeX
\documentclass[a4paper,10pt, english]{article}
\usepackage{%
	amsfonts,%
	amsmath,%	
	amssymb,%
	amsthm,%
%	babel,%
	bbm,%
	%biblatex,%
	caption,%
	centernot,%
	color,%
	enumerate,%
	epsfig,%
	epstopdf,%
	etex,%
	geometry,%
	graphicx,%
	hyperref,%
	latexsym,%
	mathtools,%
	multicol,%
	pgf,%
	pgfplots,%
	pgfplotstable,%
	pgfpages,%
	proof,%
	psfrag,%
	subfigure,%	
	tikz,%
	ulem,%
	url%
}	

\usepackage[mathscr]{eucal}
\usepgflibrary{shapes}
\usetikzlibrary{%
  arrows,%
  backgrounds,%
  chains,%
  decorations.pathmorphing,% /pgf/decoration/random steps | erste Graphik
  decorations.text,%
  matrix,%
  positioning,% wg. " of "
  fit,%
  patterns,%
  petri,%
  plotmarks,%
  scopes,%
  shadows,%
  shapes.misc,% wg. rounded rectangle
  shapes.arrows,%
  shapes.callouts,%
  shapes%
}

\theoremstyle{plain}
\newtheorem{thm}{Theorem}[section]
\newtheorem{lem}[thm]{Lemma}
\newtheorem{prop}[thm]{Proposition}
\newtheorem{cor}[thm]{Corollary}

\theoremstyle{definition}
\newtheorem{defn}[thm]{Definition}
\newtheorem{conj}[thm]{Conjecture}
\newtheorem{exmp}[thm]{Example}
\newtheorem{assum}[thm]{Assumptions}
\newtheorem{axiom}[thm]{Axiom}

\theoremstyle{remark}
\newtheorem{rem}[thm]{Remark}
\newtheorem{note}[thm]{Note}

\newcommand{\norm}[1]{\left\lVert#1\right\rVert}
\newcommand{\indep}{\!\perp\!\!\!\perp}
\DeclarePairedDelimiter\abs{\lvert}{\rvert}%
%\DeclarePairedDelimiter\norm{\lVert}{\rVert}%
\newcommand{\tr}{\operatorname{tr}}
\newcommand{\R}{\mathbb{R}}
\newcommand{\Q}{\mathbb{Q}}
\newcommand{\N}{\mathbb{N}}
\newcommand{\E}{\mathbb{E}}
\newcommand{\Z}{\mathbb{Z}}
\newcommand{\B}{\mathscr{B}}
\newcommand{\C}{\mathcal{C}}
\newcommand{\T}{\mathscr{T}}
\newcommand{\F}{\mathcal{F}}
\newcommand{\G}{\mathcal{G}}
%\newcommand{\ba}{\begin{align*}}
%\newcommand{\ea}{\end{align*}}

% Debug
\newcommand{\todo}[1]{\begin{color}{blue}{{\bf~[TODO:~#1]}}\end{color}}


\makeatletter
\def\th@plain{%
  \thm@notefont{}% same as heading font
  \itshape % body font
}
\def\th@definition{%
  \thm@notefont{}% same as heading font
  \normalfont % body font
}
\makeatother
\date{}
\title{Lecture 06: Renewal Theory}
\author{}

\begin{document}
\maketitle

\section{Introduction}
One of the characterization for the Poisson process is of it being a counting process with \textit{iid} exponential inter-arrival times. 
Now we shall relax the ``exponential" part. 
As a result, we no longer have the nice properties such as independent and stationary increments that Poisson processes had. 
However, we can still get some great results which also apply to Poisson Processes. 

\subsection{Renewal instants}
%\begin{defn}[Inter-arrival Times] 
We will consider \textbf{inter-arrival times} $\{X_i: i \in \N\}$ to be a sequence of non-negative \textit{iid} random variables with a common distribution $F$, 
with finite mean $\mu$ and $F(0) < 1$. 
We interpret $X_n$ as the time between $(n - 1)^{\text{st}}$ and the $n^{\text{th}}$ renewal event. 
%	\begin{enumerate}
%		%\item Positive inter-arrival time,% i.e. $X_n \geq 0$,
%		\item finite mean $\mu$, %i.e. $(0 \leq \mu = \E[X_1] < \infty)$, and
%		\item $F(0) < 1$.%= \Pr\{X_n \leq 0\} = \Pr\{X_n = 0\} < 1$.
%	\end{enumerate}
%\end{defn}
%\begin{defn}[Renewal Instants] 
Let $S_n$ denote the time of $n^{\text{th}}$ \textbf{renewal instant} and assume $S_0 = 0$. 
Then, we have
\begin{align*} 
S_n = \sum_{i=1}^n X_i, \quad n\in \N. 
\end{align*}
%\end{defn}
Second condition on inter-arrival times implies non-degenerate renewal process. 
If $F(0)$ is equal to $1$ then it is a trivial process. 
%\begin{defn} 
A counting process $\{N(t),t \geq 0\}$ with \textit{iid} general inter-arrival times is called a \textbf{renewal process}, written as 
%\end{defn}
%\begin{defn}[Renewal process] Let $\{N(t), t \geq 0\}$ be the counting process that counts number of events by time $t$. Then,
	\begin{align*} 
	N(t) = \sup\{n \in \N_0 : S_n \leq t\} = \sum_{n \in \N}1_{\{S_n \leq t\}}.
	\end{align*} 
%This counting process $\{N(t), t \geq 0\}$ is called a renewal process.
%\end{defn}

\begin{lem}[Inverse Relationship]
	There is an inverse relationship between time of $n^{\text{th}}$ event $S_n$, and the counting process $N(t)$. That is
	\begin{align}
	\label{eq:InverseRelationship}
	\{S_n \leq t\} \iff \{N(t) \geq n\}.
	\end{align}
	%since $N(t) = \sum_{n \in \N}1_{\{S_n \leq t\}}$.
\end{lem}

\begin{lem}[Finiteness of $N(t)$]
For all $t > 0$, the number of renewals $N(t)$ in time $[0,t)$ is finite. 
\end{lem}
\begin{proof}
We are interested in knowing how many renewals occur per unit time. 
From strong law of large numbers, we know that the set
\begin{align*} 
\left\{\frac{S_n}{n} = \mu, n \in \N\right\},
\end{align*}
has probability measure unity. 
Further, since $\mu > 0$, we must have $S_n$ growing arbitrarily large as $n$ increases. 
Thus, $S_n$ can be finite for at most finitely many $n$. 
Indeed, the following set 
\begin{align*}
\{N(t) \geq n, n \in \N\} &= \{S_n \leq t, n \in \N\} = \left\{\frac{S_n}{n} \leq \frac{t}{n}, n \in \N \right\}.% \subseteq\{\mu \leq 0\} = \emptyset. 
\end{align*}
has measure zero for any finite $t$.  
Therefore, $N(t)$ must be finite, and
%\begin{align*} 
$N(t) = \max\{n \in \N_0 : S_n \leq t\}$.
%\end{align*} 
\end{proof}

\subsection{Distribution functions}
%\begin{defn} 
The distribution of renewal instant $S_n$ is denoted by $F_n(t) \triangleq \Pr\{S_n \leq t\}$ for all $t \in \R$. 
%\end{defn}
\begin{lem} The distribution function $F_n$ for renewal instant $S_n$ can be computed inductively 
\begin{xalignat*}{3}
&F_1 = F,&&F_n = F_{n-1}\ast F \triangleq \int_{0}^{t} F_{n-1}(t-y)dF(y),
\end{xalignat*}
where $\ast$ denotes convolution.
\end{lem}
\begin{proof} It follows from induction over sum of \textit{iid} random variables.
\end{proof}
%We need to know the distribution of $N(t)$. 
\begin{lem} Counting process $N(t)$ assumes non-negative integer values with distribution
	\begin{align*}
	\Pr\{N(t) = n\} = \Pr\{S_n \leq t\} - \Pr\{S_{n+1} \leq t\} = F_n(t) - F_{n+1}(t).
	\end{align*}
\end{lem}
\begin{proof} 
It follows from the inverse relationship between renewal instants and the renewal process~\eqref{eq:InverseRelationship}.
\end{proof}
%Denote $F_n = F^{*(n)}$ where $*$ denotes convolution. Essentially, $F^{*(n)}$ is the distribution of $S_n$.
%\begin{defn} 
Mean of the counting process $N(t)$ is called the \textbf{renewal function} denoted by $m(t) = \E[N(t)]$. 
%\end{defn}
%We are interested in the following quantity.
%\begin{xalignat*}{3}
%	&m(t) = \E[N(t)].
%	%, &&M_{N(t)}(\theta) = \E[e^{\theta N(t)}].
%\end{xalignat*} 
\begin{prop} Renewal function can be expressed in terms of distribution of renewal instants as
	\begin{align*} 
	m(t) = \sum_{n \in \N} F_n(t).
	\end{align*}
\end{prop}
\begin{proof} 
Using the inverse relationship between counting process and the arrival instants, we can write 
\begin{align*}
m(t) &= \E[N(t)] = \sum_{n \in \N} \Pr\{N(t) \geq n\} = \sum_{n \in \N} \Pr\{S_n \leq t\} = \sum_{n \in \N} F_n(t).
\end{align*}
%where the second equality follows from the fact that the expectation of a random variable being represented in terms of the \textit{ccdf} of the corresponding random variable. 
%The third equality follows from the inverse relationship as seen in \eqref{eq:InverseRelationship}. 
%	Alternatively,
%	\begin{align*}
%	m(t) &= \E[N(t)]. \\
%	&= \E \left[\sum_{n \in \N} \mathbb{I}_{\{S_n \leq t\}} \right] \\
%	&= \sum_{n \in \N} \E \left[ \mathbb{I}_{\{S_n \leq t\}} \right] \\
%	&= \sum_{n \in \N} \Pr\{S_n \leq t\} = \sum_{n \in \N} F_n(t).
%	\end{align*}
%	where the third equality follows from the Monotone Convergence Theorem. 
We can exchange integrals and summations since the integrand is positive using monotone convergence theorem.
\end{proof}

\begin{prop} Renewal function is bounded for all finite times.
	%\begin{align*} 
	%m(t) < \infty \quad \forall 0 \leq t < \infty
	%\end{align*}
\end{prop}
%\begin{proof}
%Since $F(0) < 1$ and $F$ is right continuous, there exists a $b > 0$ such that $F(b) < 1$. 
%Choose a $k \in \N$ such that $t \leq kb$. 
%Then,
%\begin{align*}
%\{S_k \leq t \} \subseteq \{S_k \leq kb\} \subseteq \Omega \setminus \bigcap_{j =1}^{k}\{X_j > b\}.
%\end{align*} 
%From independence of inter-arrival times $X_j$ it follows that 
%\begin{align*}
%\Pr\{S_k \leq t \} \leq  1 - \prod_{j = 1}^{k}\bar{F}(b) = 1 - \beta.
%\end{align*} 
%\end{proof}
\begin{proof}
Since we assumed that $\Pr\{X_n = 0\} < 1$, it follow from continuity of probabilities that there exists $\alpha > 0$, such that $\Pr\{X_n \geq \alpha\} = \beta >0$. Define
\begin{align*}
\bar{X}_n = \alpha 1_{\{X_n \geq \alpha\}}.
\end{align*}
Note that since $X_i$'s are \textit{iid}, so are $\bar{X}_i$'s, which are bivariate random variables taking values in $\{0, \alpha\}$ with probabilities $1-\beta$ and $\beta$ respectively. % (which will be evident from the proof of the distribution function of the number of arrivals till time t). 
Let $\bar{N}(t)$ denote the renewal process with inter-arrival times $\bar{X}_n$, with arrivals at integer multiples of $\alpha$. 
Since $\bar{X}_n \leq X_n$, we have $\bar{N}(t) \geq N(t)$ for all sample paths. 
Hence, it follows that $\E N(t) \leq \E \bar{N}(t)$, and we will show that $\E\bar{N}(t)$ is finite. 
%The number of arrivals at each arrival instant $k\alpha$ is \textit{iid} and we can write 
%%Moreover, $X_n \geq \bar{X}_n$, and we can write 	
%\begin{align*}
%\Pr\{\bar{N}(0) = n\} %&= \Pr\{\bar{X}_1=\bar{X}_2=\dots=\bar{X}_n=0,\bar{X}_{n+1}=\alpha\} 
%&= \Pr\{X_1 < \alpha,X_2 < \alpha,\ldots,X_n < \alpha,X_{n+1} \geq \alpha\} = (1-\beta)^n\beta.
%%&= \prod_{i=1}^{n} \Pr\{X_i < \alpha\} . \Pr\{X_{n+1} \geq \alpha \} \\
%%= \left(1- \Pr\{X_1 \geq \alpha \}\right) ^{n} \Pr\{X_1 \geq \alpha\}.
%\end{align*}
We can write the joint distribution of number of arrivals at each arrival instant $l\alpha$, as 
\begin{align*}
\Pr\{\bar{N}(0)=n_1, \bar{N}(\alpha) = n_2\} &= \Pr\{X_i \leq \alpha, i \leq n_1, X_{n_1+1} \geq \alpha, X_i < \alpha, n_1 +2 \leq i \leq n_2, X_{n_2+1} \geq \alpha \}\\
&=  (1-\beta)^{n_1}\beta(1-\beta)^{n_2-1}\beta
\end{align*}
%where the third equality follows from the fact that $ X_i, i \in \N$ are mutually independent, fourth equality follows from the fact that $ X_i, i \in \N$ are identical.
%\\
It follows that the number of arrivals is independent at each arrival instant $k\alpha$ and geometrically distributed with mean $1/\beta$ and $(1-\beta)/\beta$ for $k \geq 1$ and $k = 0$ respectively.  
%The number of arrivals till time $t$ therefore is Geometric with mean $\frac{1}{\Pr\{X_n \geq \alpha\}}$. 
Thus, for all $t \geq 0$, 
\begin{align*}
\E N(t) \leq \E[\bar{N}(t)] \leq \frac{\lceil\frac{t}{\alpha} \rceil}{\beta} \leq \frac{\frac{t}{\alpha} + 1}{\beta} < \infty.
\end{align*}
%Since $\E[N(t)] \leq \E[\bar{N}(t)]$ which follows from $N(t) \leq \bar{N}(t)$, we are done.
\end{proof} 

\subsection{Basic renewal theorem}

\begin{lem}
Let $N(\infty) \triangleq \lim_{t \to \infty} N(t)$. Then, $\Pr\{N(\infty) = \infty\} = 1$.	
\end{lem}

\begin{proof}
It suffices to show $\Pr\{N(\infty) < \infty\} = 0$. 
Since $\E[X_n] < \infty$, we have $\Pr\{X_n = \infty\} = 0$ and 
\begin{flalign*}
\Pr\{N(\infty) < \infty\} &= \Pr\bigcup_{n \in \N} \{N(\infty) < n\}= \Pr\bigcup_{n \in \N} \{S_n = \infty\} = \Pr\{\bigcup_{n \in \N} \{X_n = \infty\}\} \leq \sum_{n \in \N}\Pr\{X_n = \infty\} =0.
\end{flalign*}
%The last step follows from the fact that $\E[X_n] < \infty$.
\end{proof}
Notice that $N(t)$ increases to infinity with time. 
We are interested in rate of increase of $N(t)$ with $t$. 

\begin{thm}[Basic Renewal Theorem]
\begin{align*}
\lim_{t \to \infty} \frac{N(t)}{t} = \frac{1}{\mu} \quad \mbox{almost surely}.
\end{align*}
\end{thm}
\begin{proof}
Note that $S_{N(t)}$ represents the time of last renewal before $t$, and $S_{N(t)+1}$ represents the time of first renewal after time $t$.
	\begin{figure}[h!]
		\includegraphics[width=.9\linewidth]{Figures/lecture_5_fig_1.png}
		\caption{Time-line visualization}
	\end{figure}
	Consider $S_{N(t)}$. By definition, we have
	\[S_{N(t)} \leq t < S_{N(t)+1}\]
	Dividing by $N(t)$, we get 
	\[\frac{S_{N(t)}}{N(t)} \leq \frac{t}{N(t)} < \frac{S_{N(t)+1}}{N(t)}\]
	By Strong Law of Large Numbers (SLLN) and the previous result, we have
	\[\lim_{t \to \infty}\frac{S_{N(t)}}{N(t)} = \mu \quad \mbox{a.s.}\] 
	Also
	\[\lim_{t \to \infty} \frac{S_{N(t)+1}}{N(t)} = \lim_{t \to \infty} \frac{S_{N(t)+1}}{N(t)+1}.\frac{N(t)+1}{N(t)} \]
Hence by squeeze theorem, the result follows.
\end{proof}

\begin{shaded*}
Suppose, you are in a casino with infinitely many games. 
Every game has a probability of win $X$, \textit{iid} uniformly distributed between $(0,1)$. 
One can continue to play a game or switch to another one. We are interested in a strategy that maximizes the long-run proportion of wins.
Let $N(n)$ denote the number of losses in $n$ plays. 
Then the fraction of wins $P_W(n)$ is given by 
\begin{align*}
P_W(n) = \frac{n-N(n)}{n}.
\end{align*}
We pick a strategy where any game is selected to play, and continue to be played till the first loss. Note that, time till first loss is geometrically distributed with mean $\frac{1}{1-X}$. We shall show that this fraction approaches unity as $n \to \infty$. By the previous proposition, we have:
\begin{align*}
\lim_{n \to \infty} \frac{N(n)}{n} &= \frac{1}{\E[\mbox{Time till first loss}]} \\
&= \frac{1}{\E\left[\frac{1}{1-X}\right]} = \frac{1}{\infty} = 0
\end{align*}
Hence Renewal theorems can be used to compute these long term averages. We'll have many such theorems in the following sections.
\end{shaded*}

\subsection{Elementary renewal theorem}
Basic renewal theorem implies $N(t)/t$ converges to $1/\mu$ almost surely. Now, we are interested in convergence of $\E[N(t)]/t$. Note that this is not obvious, since almost sure convergence doesn't imply convergence in mean. 
\begin{shaded*}
Consider the following example. 
Let $X_n$ be a Bernoulli random variable with $\Pr\{X_n = 1\} = 1/n$. 
Let $Y_n = nX_n$. 
%Then, 
%	\begin{align*}
%	Y_n = \begin{cases}
%	n, & \mbox{ w.p.  } 1/n,\\
%	0, & \mbox{ w.p.  } 1- 1/n.
%	\end{cases}
%	\end{align*}
Then, $\Pr\{ Y_n = 0 \} = 1 - 1/n$. %Then $\sum \Pr\{Y_n > \epsilon\} < \infty$. Hence by Borel Cantelli Lemma, $Y_n \to 0$
That is $Y_n \to 0$ a.s. However, $\E[Y_n] = 1$ for all $n \in \N$. 
So $\E[Y_n] \to 1$.
\end{shaded*}
Even though, basic renewal theorem does \textbf{NOT} imply it, we still have $\E[N(t)]/t$ converging to $1/\mu$.

\begin{thm}[Elementary renewal theorem] Let $m(t)$ denote mean $\E[N(t)]$ of renewal process $N(t)$, then under the hypotheses of basic renewal theorem, we have 
	\begin{align*}
	\lim_{t \to \infty}\frac{m(t)}{t} = \frac{1}{\mu}.
	\end{align*}
\end{thm}
\begin{proof}
	Take $\mu < \infty$. We know that $S_{N(t)+1} > t$. Therefore, taking expectations on both sides and using Proposition~\ref{prop:WaldRenewal}, we have 
	\begin{align*}
	\mu (m(t) + 1) > t.
	\end{align*}
	Dividing both sides by $\mu t$ and taking $\liminf$ on both sides, we get
\begin{align*}
%\label{eq:LiminfMean}
\liminf_{t \to \infty} \frac{m(t)}{t} \geq \frac{1}{\mu}.
\end{align*}
	
	%Thus we now have to show 
	%\[\limsup_{t \to \infty} \frac{m(t)}{t} \leq \frac{1}{\mu}\]
We employ a truncated random variable argument to show the reverse inequality. We define truncated inter-arrival times $\{\bar{X}_n\}$ as 
\begin{align*}
\bar{X}_n = X_n 1_{\{X_n \leq M\}} + M1_{\{X_n > M\}}.
\end{align*}
We will call $\E[\bar{X}_n] = \mu_M$. Further, we can define arrival instants $\{\bar{S}_n\}$ and renewal process $\bar{N}(t)$ for this set of truncated inter-arrival times $\{\bar{X}_n\}$ as 
\begin{align*}
\bar{S}_n &= \sum_{k=1}^n \bar{X}_k, & \bar{N}(t) &= \sup\{n \in \N_0: \bar{S}_n \leq t\}.
\end{align*}
Note that since $S_n \geq \bar{S}_n$, the number of arrivals would be higher for renewal process $\bar{N}(t)$ with truncated random variables, i.e. 
\begin{align}
\label{eq:TruncRenewalInequality}
N(t) \leq \bar{N}(t).
\end{align}
Further, due to truncation of inter-arrival time, next renewal happens with-in $M$ units of time, i.e.
\begin{align*}
\bar{S}_{\bar{N}(t)+1} \leq t+M.
\end{align*}
Taking expectations on both sides in the above equation, using Wald's lemma for renewal processes, %using Proposition~\ref{prop:WaldRenewal}, 
dividing both sides by $t \mu_M$, and taking $\limsup$ on both sides, we obtain
\begin{align*}
\limsup_{t \to \infty}\frac{\bar{m}(t)}{t} \leq \frac{1}{\mu_M}.
\end{align*}
Taking expectations on both sides of~\eqref{eq:TruncRenewalInequality} and letting $M$ go arbitrary large on RHS, we get
\begin{align*}
%\label{eq:LimsupMean}
\limsup_{t \to \infty}\frac{m(t)}{t} \leq \frac{1}{\mu}.
\end{align*}
	%Now observe that LHS is independent of $M$. Take limits $M \to \infty$, noting that $\mu_M \to \mu$ (Why?) to get
	%\[\limsup_{t \to \infty}\frac{m(t)}{t} \leq \frac{1}{\mu}\]
	%Putting it all together,
	%\[\lim_{t \to \infty}\frac{m(t)}{t} = \frac{1}{\mu}\]
Result follows for finite $\mu$ from combining liminf and limsup of the $m(t)/t$
%~\eqref{eq:LiminfMean} and~\eqref{eq:LiminfMean}. 
When $\mu$ grows arbitrary large, results follow from liminf of $m(t)/t$, %~\eqref{eq:LiminfMean}, 
where RHS is zero. 
\end{proof}

\subsection{Central limit theorem for renewal processes}
\begin{thm}
Let $X_n$ be \textit{iid} random variables with $\mu = \E[X_n] < \infty$ and $\sigma^2 = Var(X_n) < \infty$. Then
\[\frac{N(t)-\frac{t}{\mu}}{\sigma \sqrt{\frac{t}{\mu^3}}} \to^d N(0,1) \]
\end{thm}
\begin{proof}
Take $u = \frac{t}{\mu} + y \sigma \sqrt{\frac{t}{\mu^3}}$. We shall treat $u$ as an integer and proceed, the proof for general $u$ is an exercise. Recall that $\{N(t) < u\} \iff \{S_u > t\}$. By equating probability measures on both sides, we get
\begin{align*}
\Pr\{N(t) < u\} = \Pr\left\{\frac{S_u - u\mu}{\sigma \sqrt{u}} > \frac{t - u\mu}{\sigma \sqrt{u}}\right\} = \Pr\left\{\frac{S_u - u\mu}{\sigma \sqrt{u}} > -y\left(1 + \frac{y\sigma}{\sqrt{tu}}\right)^2\right\}.
\end{align*}
By central limit theorem, $\frac{S_u - u\mu}{\sigma \sqrt{u}}$ converges to a normal random variable with zero mean and unit variance as $t$ grows. Also, note that 
\begin{align*}
\lim_{t \to \infty} -y\left(1 + \frac{y\sigma}{\sqrt{tu}}\right)^2 = -y.
\end{align*}
These results combine with the symmetry of normal random variable to give us the result.
\end{proof}

\appendix
\section{Wald's Lemma}
%Before we get into Wald's Lemma, let us first define what a stopping time is.
%\begin{defn}[Stopping Time]
%Let $\{X_n: n\in \N\}$ be independent random variables. 
An integer random variable $T$ is called a \textbf{stopping time} with respect to the \textit{independent} random sequence $\{X_n: n \in \N\}$ if the event $\{N=n\}$ depends only on $\{X_1,\cdots,X_n\}$ and is independent of $\{X_{n+1}, X_{n+2},\cdots\}$. 
%\end{defn}

Intuitively, if we observe the $X_n$'s in sequential order and $N$ denotes the number observed before stopping then. Then, we have stopped after observing, $\{X_1, \ldots, X_N\}$, and before observing $\{X_{N+1}, X_{N+2}, \ldots\}$. 
The intuition behind a stopping time is that it's value is determined by past and present events but NOT by future events. 
\begin{shaded*}
\begin{enumerate}
\item For instance, while traveling on the bus, the random variable measuring ``Time until bus crosses Majestic and after that one stop" is a stopping time as it's value is determined by events before it happens. On the other hand ``Time until bus stops before Majestic is reached'' would not be a stopping time  in the same context. This is because we have to cross this time, reach Majestic and then realize we have crossed that point. 
\item Consider $X_n \in \{0,1\}$ \textit{iid} Bernoulli$(1/2)$. Then $N = \min \{n \in \N:\quad \sum_{i=1}^n X_i = 1\}$ is a stopping time. For instance, $\Pr\{N=2\} = \Pr\{X_1=0,X_2=1\}$ and hence $N$ is a stopping time by definition.
 \item \textbf{Random Walk Stopping Time} Consider $X_n$ \textit{iid} bivariate random variables with 
	\begin{align*}
	\Pr\{X_n = 1\} = \Pr\{X_n = -1\} = \frac{1}{2}. 
	\end{align*}
	Then $N = min \{n \in \N:\quad \sum_{i=1}^n X_i = 1\}$ is a stopping time.
\end{enumerate}
\end{shaded*}
\subsection{Properties of stopping time}
Let $N_1,N_2$ be two stopping times with respect to independent random sequence $\{X_i : i \in \N \}$ then,
\begin{enumerate}[i\_]
\item $N_1+N_2$ is a stopping time.
\item $\min \{N_1,N_2\} $ is a stopping time.
\end{enumerate}
\begin{proof}
Let $\{X_i : i \in \N \} $ be an independent random sequence, and $N_1,N_2$ associated stopping times.  
\begin{enumerate}[i\_]
\item It suffices to show that the event $\{N_1+N_2=n\}$ depends only on random variables $\{X_1, \dots, X_n\}$ and independent of $\{X_{n+1}, \dots\}$. 
To this end, we observe that 
\begin{align*}
\{N_1+N_2 = n \} &= \bigcup_{k=0}^{n} \{N_1 = k,N_2 = n-k\}.
\end{align*}
Result follows since the events $\{N_1 = k\}$ and $\{N_2=n-k\}$ depend solely on $\{X_1, \dots, X_n\}$ for all $k \in \{0, \dots, n\}$. 
%Hence, $N_1+N_2$ is a stopping time.
\item It suffices to show that the event $\min \{N_1,N_2\} > n\}$ depends solely on $\{X_1, \dots, X_n\}$. 
\begin{align*}
\min \{N_1,N_2\} > n\} = \{N_1 > n\} \cap \{N_2 > n\}.
%	From ~De ~Morgan's ~Law ~we ~get ~\{\min \{N_1,N_2\} \leq n\} &= \{N_1 \leq n\} \cap \{N_2 \leq n\}
%	\\
%	& \indep \{X_{n+1},X_{n+2},\ldots\}.
\end{align*}
The result follows since the events $\{N_1 > n\}$ and $\{N_2 > n\}$ depend solely on $\{X_1, \dots, X_n\}$. 
\end{enumerate}
\end{proof}

\begin{lem}[Wald's Lemma]
Let $\{X_i:\quad i\in \N\}$ be \textit{iid} random variables with finite mean $\E[X_1]$ and let $N$ be a stopping time with respect to this set of variables, such that $\E[N] < \infty$. 
Then,
\begin{align*}
\E\left[\sum_{n=1}^N X_n\right] &= \E[X_1]\E[N].
\end{align*}
\end{lem}
\begin{proof} 
We first show that the event $\{N \geq n\}$ is independent of $X_k$, for any $k \geq n$. 
To this end, observe that 
\begin{align*}
\{N \geq k\} = \{N < k\}^c = \{N \leq k-1\}^c = \left(\bigcup_{i=1}^{k-1} \{N = i\}\right)^c. 
\end{align*}
Recall that $N$ is a stopping time and the event $\{N=i\}$ depends only on $\{X_1,\ldots, X_i\}$, by definition.  
Therefore, $\{N \geq k\}$ depends only on $\{X_1,\ldots, X_{k-1}\}$, and is independent of the future and present samples. 
Hence, we can write the $N$th renewal instant for a stopping time $N$ as 
\begin{align*}
\E\left[\sum_{n=1}^N X_n\right] &= \E\left[\sum_{n \in \N} X_n 1_{\{N \geq n\}}\right] = \sum_{n \in \N} \E X_n \E\left[1_{\{N \geq n\}}\right] = \E X_1\E\left[ \sum_{n \in \N} 1_{\{N \geq n\}}\right] = \E[X_1]\E[N].
\end{align*}
We exchanged limit and expectation in the above step, which is not always allowed. 
We were able to do it since the summand is positive and we apply monotone convergence theorem. 
%I'd like to point out here that in step (\ref{tricky}), you cannot always exchange infinite sums and expectations. 
%But here you can do so, because of the application of 
%Refer Ross/Wolff for more information. 
%Therefore, we can write
%	\begin{align*}
%	\sum_{n \in \N} \E\left[X_n 1_{\{N \geq n\}}\right] &= \sum_{n \in \N} \E\left[X_n\right]\E\left[ 1_{\{N \geq n\}}\right] \\
%	&= \E\left[X_1\right] \sum_{n \in \N} \Pr\{N \geq n\} \\
%	&= \E[X_1]\E[N].
%	\end{align*} 
%where the third equality follows from the fact that the expectation of a random variable being represented in terms of the \textit{ccdf} of the corresponding random variable.
\end{proof}

\begin{prop}[Wald's Lemma for Renewal Process] \label{prop:WaldRenewal}
	Let $\{X_n, n \in \N\}$ be \textit{iid} inter-arrival times of a renewal process $N(t)$ with $\E[X_1] < \infty$, and let $m(t) = \E[N(t)]$ be its renewal function. Then, $N(t)+1$ is a stopping time and 
	\begin{align*}
	\E\left[\sum_{i=1}^{N(t)+1}X_i\right] = \E[X_1][1+m(t)].
	\end{align*}
\end{prop}
\begin{proof} It is easy to see that $\{N(t)+1=n\}$ depends solely on $\{X_1,\ldots,X_n\}$ from the discussion below.
	\begin{align*}
	\left\{N(t) + 1 = n \right\} \iff \{S_{n-1} \leq t < S_n\} \iff \left\{\sum_{i=1}^{n-1} X_i \leq t < \sum_{i=1}^{n-1} X_i + X_n\right\}.
	\end{align*}
	Thus $N(t)+1$ is a stopping time, and the result follows from Wald's Lemma.
\end{proof}
\end{document}