\documentclass[a4paper,10pt]{article}
\usepackage{%
amsmath,%
amsfonts,%
amssymb,%
amsthm,%
hyperref,%
url,%
latexsym,%
epsfig,%
graphicx,%
psfrag,%
subfigure,%
color,%
tikz,%
pgf,%
pgfplots,%
pgfplotstable,%
pgfpages%
}
\usepgflibrary{shapes}
\usetikzlibrary{%
arrows,%
backgrounds,%
chains,%
decorations.pathmorphing,% /pgf/decoration/random steps | erste Graphik
decorations.text,%
matrix,%
positioning,% wg. " of "
fit,%
patterns,%
petri,%
plotmarks,%
scopes,%
shadows,%
shapes.misc,% wg. rounded rectangle
shapes.arrows,%
shapes.callouts,%
shapes%
}
\theoremstyle{plain}
\newtheorem{thm}{Theorem}[section]
\newtheorem{lem}[thm]{Lemma}
\newtheorem{prop}[thm]{Proposition}
\newtheorem{cor}[thm]{Corollary}
\theoremstyle{definition}
\newtheorem{defn}[thm]{Definition}
\newtheorem{conj}[thm]{Conjecture}
\newtheorem{exmp}[thm]{Example}
\theoremstyle{remark}
\newtheorem{rem}[thm]{Remark}
\newtheorem{note}[thm]{Note}

\usepackage{%
	amsfonts,%
	amsmath,%	
	amssymb,%
	amsthm,%
%	babel,%
	bbm,%
	%biblatex,%
	caption,%
	centernot,%
	color,%
	enumerate,%
	epsfig,%
	epstopdf,%
	etex,%
	geometry,%
	graphicx,%
	hyperref,%
	latexsym,%
	mathtools,%
	multicol,%
	pgf,%
	pgfplots,%
	pgfplotstable,%
	pgfpages,%
	proof,%
	psfrag,%
	subfigure,%	
	tikz,%
	ulem,%
	url%
}	

\usepackage[mathscr]{eucal}
\usepgflibrary{shapes}
\usetikzlibrary{%
  arrows,%
  backgrounds,%
  chains,%
  decorations.pathmorphing,% /pgf/decoration/random steps | erste Graphik
  decorations.text,%
  matrix,%
  positioning,% wg. " of "
  fit,%
  patterns,%
  petri,%
  plotmarks,%
  scopes,%
  shadows,%
  shapes.misc,% wg. rounded rectangle
  shapes.arrows,%
  shapes.callouts,%
  shapes%
}

\theoremstyle{plain}
\newtheorem{thm}{Theorem}[section]
\newtheorem{lem}[thm]{Lemma}
\newtheorem{prop}[thm]{Proposition}
\newtheorem{cor}[thm]{Corollary}

\theoremstyle{definition}
\newtheorem{defn}[thm]{Definition}
\newtheorem{conj}[thm]{Conjecture}
\newtheorem{exmp}[thm]{Example}
\newtheorem{assum}[thm]{Assumptions}
\newtheorem{axiom}[thm]{Axiom}

\theoremstyle{remark}
\newtheorem{rem}[thm]{Remark}
\newtheorem{note}[thm]{Note}

\newcommand{\norm}[1]{\left\lVert#1\right\rVert}
\newcommand{\indep}{\!\perp\!\!\!\perp}
\DeclarePairedDelimiter\abs{\lvert}{\rvert}%
%\DeclarePairedDelimiter\norm{\lVert}{\rVert}%
\newcommand{\tr}{\operatorname{tr}}
\newcommand{\R}{\mathbb{R}}
\newcommand{\Q}{\mathbb{Q}}
\newcommand{\N}{\mathbb{N}}
\newcommand{\E}{\mathbb{E}}
\newcommand{\Z}{\mathbb{Z}}
\newcommand{\B}{\mathscr{B}}
\newcommand{\C}{\mathcal{C}}
\newcommand{\T}{\mathscr{T}}
\newcommand{\F}{\mathcal{F}}
\newcommand{\G}{\mathcal{G}}
%\newcommand{\ba}{\begin{align*}}
%\newcommand{\ea}{\end{align*}}

% Debug
\newcommand{\todo}[1]{\begin{color}{blue}{{\bf~[TODO:~#1]}}\end{color}}


\makeatletter
\def\th@plain{%
  \thm@notefont{}% same as heading font
  \itshape % body font
}
\def\th@definition{%
  \thm@notefont{}% same as heading font
  \normalfont % body font
}
\makeatother
\date{}
\title{Lecture 12: Convergence of DTMCs and Coupling theorem}
\author{Sireesha Madabhushi}
\begin{document}
\maketitle
\section{Discrete Time Markov Chains Contd.}

\subsection{Total Variation Distance}

Given two probability distributions p and q defined on set of natural numbers \textbf{$N_0$}, their total variation distance is defined as\\
$d_{TV}(p,q) = \dfrac{1}{2} \lVert p - q \rVert_1$\\
\textbf{FACT:} $d_{TV}(p,q) = sup_{S \subseteq N_0} [p(S) - q(S)]$
\textbf{Definition: Convergence in total variation\\}\\
Let ${X_n}_{n \geq 0}$ be a $N_0$-valued stochastic process if $\exists$ a probability distribution $\Pi$ on $N_0$ such that \\
$lim_{n \rightarrow \infty} d_{TV}(P[X_{n} \in .], \Pi) = 0$ i.e.\\

$lim_{n \rightarrow \infty} \sum{i \in N_{0}} \lvert P[X_{n}=i] - \Pi(i) \rvert = 0$\\
Then, we say that $P[X_{n} \in .] \rightarrow \Pi$ in total variation distance as $n \rightarrow \infty$.\\
\textbf{NOTE:}\\
If $X_{n} \rightarrow \Pi$, then $\forall$ bounded functions, $f:N_{0} \rightarrow R$, $lim_{n \rightarrow \infty} E[f(X{n})] = \Sigma_{i \in N_0} \Pi(i)f(i)$\\
\textbf{Theorem: Convergence in variation of DTMC}\\
Let ${X_n}_{n \geq 0}$ be an ergodic DTMC on $N_{0}$ (irreducible, aperiodic and positive recurrent) with stationary distribution $(\Pi_{j})_{j \in N_0}$.\\
Then $\forall$ initial distribution of $X_0$, we have $X_n \rightarrow \Pi$.\\
In other words, if $\mu_j$ is the initial distribution, then \\

 $lim_{n \rightarrow \infty} \lVert \mu P^n - \Pi \rVert_1 = 0$\\
 where, $P_{ij} = P[X_{n+1} = j/X_n = i]$\\

 
\subsection{The Coupling method}
\textbf{Definition:}\\
Consider two stochastic processes $\{X_n'\}_n$ and $\{X_n''\}_n$ on $N_0$, defined on the same probability space.\\
$\{X_n'\}_n$ and $\{X_n''\}_n$ are said to couple if $\exists$ an a.s. finite random time $\tau$ such that $\forall n \geq \tau : X_n' = X_n''$ a.s.\\
Moreover, $\tau$ is called a coupling time of the process.\\

\textbf{Coupling Inequality:}\\
If $\tau$ is a coupling time, then $\forall n \geq 0$,
$d_{TV} (X_n',X_n'') = \Sigma_{i \in N_0} \lvert P[X_n' = i] - P[X_n'' = i] \rvert \leq P[\tau > n]$\\

Remark: Variation distance is bounded based on the coupling time.\\

\textbf{Proof:} Consider any $A \subseteq N_0$.\\
\begin{eqnarray*}
\begin{aligned}
& P[X_n' \in A] - P[X_n'' \in A] = P[X_n' \in A,\tau \leq n] + P[X_n' \in A,\tau > n] - P[X_n'' \in A, \tau \leq n] - P[X_n'' \in A,\tau > n] \\
& P[X_n' \in A] - P[X_n'' \in A] = P[X_n' \in A,\tau > n] - P[X_n'' \in A,\tau > n] (from \: the \: definition \: of \: coupling \: between \: X_n' \: and \: X_n'')\\
& P[X_n' \in A] - P[X_n'' \in A] \leq P[X_n' \in A,\tau > n] \leq P[\tau > n]\\
\end{aligned}
\end{eqnarray*}
 

\subsection{Theorem[DTMC Convergence in $d_{TV}$]}	
Let ${X_n}$ be a homogenous $N_0$ valued ergodic DTMC with transition probability $p_{ij}$ with stationary distribution $(\Pi_j)_{j \in N_0}$. Then for any initial distribution of $X_0$,\\
\begin{eqnarray*}
	\begin{aligned}
 lim_{n \longrightarrow \infty} \sum_{i \in N_0} \lvert P[X_n = i] - \Pi(i) \rvert = 0\\
\end{aligned}
\end{eqnarray*}
\textbf{Proof:}
We will prove using the coupling argument.\\
Let $\{X_n^{(1)}\}_{n \geq 0}$ and $\{X_n^{(2)}\}_{n \geq 0}$ be two independent ergodic DTMCs with transition matrix P and initial distributions $\mu$ and $\Pi$ respectively.\\
Construct the product DTMC $Z_n = (X_n^{(1)}, X_n^{(2)})_{n \in N_0}$\\
$\{Z_n\}_{n \geq 0}$ has transition probabilities\\
\begin{eqnarray*}
\begin{aligned}
	P[Z_n = (i,j)/Z_{n-1} = (k,l)] = p_{ki} p_{lj}
\end{aligned}
\end{eqnarray*}
\textbf{CLAIMS:}
\begin{itemize}
	\item $\{Z_n\}_{n \geq 0}$ is irreducible and aperiodic.
	\item $\{Z_n\}_{n \geq 0}$ is positive recurrent.
\end{itemize}
\textbf{Proof:}\\
$\Pi_z(i,j) = \Pi(i) \Pi(j)$ is a stationary distribution
$$
\tau = 
\begin{cases}
inf  \{n \geq 0: X_n^{(1)} = X_n^{(2)}\}\\
\infty , if \{X_n^{(1)}\}$ and $\{X_n^{(2)}\} $ never meet$\\
\end{cases}
$$
$\tau$ is a stopping time for the process $\{Z_n\}$, which is positive recurrent.\\
This implies $P[\tau < \infty] = 1$\\
Consider a process $X_n'$ defined as follows\\
$$
X_n' =
\begin{cases}
X_n^{(1)}, n \leq \tau\\
X_n^{(2)}, n > \tau\\
\end{cases}
$$

\textbf{CLAIM:}\\
${X_n'}_{n \geq 0}$ is a homogenous DTMC with transition matrix $P$ and initial distribution $\mu$. It inherits all the properties of $X_n^{(1)}$.\\
Also, $\tau$ is a coupling time for $\{X_n^{(1)}\}_{n \geq 0}$ and $\{X_n^{(2)}\}_{n \geq 0}$.\\
From the coupling inequality,\\
\begin{eqnarray*}
\begin{aligned}
 \sum_{i \in N_0} \lvert P(X_n' = i) - P(X_n^{(2)} = i) \rvert \leq P[\tau < n]\\
 n \longrightarrow \infty:  P(X_n^{(2)} = i) \longrightarrow \Pi(i),  P[\tau > n] \longrightarrow 0\\
 n \longrightarrow \infty \Rightarrow P(X_n' = i) = \Pi(i) 
\end{aligned}
\end{eqnarray*}
We can get bounds on the rate of convergence by bounding $P[\tau > n]$

\subsection{Example}
Let $X$ and $Y$ be two binomial distributions defined as follows:\\
$X \sim Bin(n,p)$\\
$Y \sim Bin(n,q)$ and $p > q$\\
What is the relation between $P_1[X > k]$ and $P_2[Y > k]$  $\forall$ k?\\
\textbf{Solution:}\\
Consider $n$ Bernoulli random variables, $Z_1,Z_2,....Z_n \sim Ber(p)$\\
Define $W_1,W_2....W_n$ such that $W_i \sim Ber(p*q/p)$ when the event $\{Z_i = 1\}$ which occurs with probability $p$ and $W_i = 0$ when the event $\{Z_i = 0\}$
occurs with probability $q$.\\
$\Rightarrow W_i \sim Ber(q)$ and $Z_i \sim Ber(p)$, with $p > q$ \\
From the definition, we have $\sum_i W_i \leq \sum_i Z_i $\\
$\Rightarrow P[\sum_i W_i > k] \leq  P[\sum_i Z_i > k]$\\
We know that $\sum_i Z_i = X$ and $\sum_i W_i = Y$\\
Therefore, $\forall$ k, $P_1[X > k] \geq P_2[Y>k]$\\

\section{Mean time spent in the transient states}
Consider a finite state space DTMC.'\\
Let the transient states be $\{1,2,...t\}$.\\
\begin{eqnarray*}
	\begin{aligned}
	&	Q_{t \times t} = P_{[t] \times [t]}\\
	&	Q_{t \times t} = \begin{bmatrix} p_{11}&-&-&-&p_{1t}\\ p_{21}&-&-&-&p_{2t} \\ -&-&-&-&-\\ p_{t1}&-&-&-&p_{tt} \end{bmatrix}
    \end{aligned}
\end{eqnarray*}
\textbf{NOTE:} All row sums of $Q$ cannot be equal to 1.\\
Atleast one row should not sum upto 1, else it contradicts the claim that $Q$ is a transition matrix for the set of transient states.\\
For $i,j \in [t]$, let\\
\begin{eqnarray*}
	\begin{aligned}
& m_{ij} = E_i[\sum_{n=0}^{\infty} \textbf{1} \{X_n = j\}]\\
& m_{ij} = \sum_{n=0}^{\infty} p_{ij}^{(n)} < \infty\\
& m_{ij} = \textbf{1}[i = j] + \sum_{k=1}^{[t]} p_{ik}m_{kj} \\
& M = I + QM\\
& M(I - Q) = I\\
& M = (I - Q)^{-1}\\
& M = I + Q + Q^{2} + Q^{3} + ......\\
    \end{aligned}
\end{eqnarray*}
$M$ is called the fundamental matrix.\\
Here, $(I - Q)$ is invertible because all row sums of $Q$ don't sum upto 1.\\
\textbf{Note:}\\
$\forall i,j \in [t]$, consider $f_{ij} = \mathbb{E}_i[\textbf{1} \{\exists n \geq 0: X_n = j \}]$\\
\begin{eqnarray*}
\begin{aligned}
& m_{ij} = \mathbb{E}_i[\# transitions\ to\ j \in \{0,1,2...\infty\}]\\
& m_{ij} = f_{ij} m_{jj}\\
&\Rightarrow i,j \in [t]; f_{ij} = \dfrac{m_{ij}}{m_{jj}}
\end{aligned}
\end{eqnarray*}
 
\end{document}