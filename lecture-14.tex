\documentclass[a4paper,10pt,english]{article}
\usepackage{%
	amsfonts,%
	amsmath,%	
	amssymb,%
	amsthm,%
%	babel,%
	bbm,%
	%biblatex,%
	caption,%
	centernot,%
	color,%
	enumerate,%
	epsfig,%
	epstopdf,%
	etex,%
	geometry,%
	graphicx,%
	hyperref,%
	latexsym,%
	mathtools,%
	multicol,%
	pgf,%
	pgfplots,%
	pgfplotstable,%
	pgfpages,%
	proof,%
	psfrag,%
	subfigure,%	
	tikz,%
	ulem,%
	url%
}	

\usepackage[mathscr]{eucal}
\usepgflibrary{shapes}
\usetikzlibrary{%
  arrows,%
  backgrounds,%
  chains,%
  decorations.pathmorphing,% /pgf/decoration/random steps | erste Graphik
  decorations.text,%
  matrix,%
  positioning,% wg. " of "
  fit,%
  patterns,%
  petri,%
  plotmarks,%
  scopes,%
  shadows,%
  shapes.misc,% wg. rounded rectangle
  shapes.arrows,%
  shapes.callouts,%
  shapes%
}

\theoremstyle{plain}
\newtheorem{thm}{Theorem}[section]
\newtheorem{lem}[thm]{Lemma}
\newtheorem{prop}[thm]{Proposition}
\newtheorem{cor}[thm]{Corollary}

\theoremstyle{definition}
\newtheorem{defn}[thm]{Definition}
\newtheorem{conj}[thm]{Conjecture}
\newtheorem{exmp}[thm]{Example}
\newtheorem{assum}[thm]{Assumptions}
\newtheorem{axiom}[thm]{Axiom}

\theoremstyle{remark}
\newtheorem{rem}[thm]{Remark}
\newtheorem{note}[thm]{Note}

\newcommand{\norm}[1]{\left\lVert#1\right\rVert}
\newcommand{\indep}{\!\perp\!\!\!\perp}
\DeclarePairedDelimiter\abs{\lvert}{\rvert}%
%\DeclarePairedDelimiter\norm{\lVert}{\rVert}%
\newcommand{\tr}{\operatorname{tr}}
\newcommand{\R}{\mathbb{R}}
\newcommand{\Q}{\mathbb{Q}}
\newcommand{\N}{\mathbb{N}}
\newcommand{\E}{\mathbb{E}}
\newcommand{\Z}{\mathbb{Z}}
\newcommand{\B}{\mathscr{B}}
\newcommand{\C}{\mathcal{C}}
\newcommand{\T}{\mathscr{T}}
\newcommand{\F}{\mathcal{F}}
\newcommand{\G}{\mathcal{G}}
%\newcommand{\ba}{\begin{align*}}
%\newcommand{\ea}{\end{align*}}

% Debug
\newcommand{\todo}[1]{\begin{color}{blue}{{\bf~[TODO:~#1]}}\end{color}}


\makeatletter
\def\th@plain{%
  \thm@notefont{}% same as heading font
  \itshape % body font
}
\def\th@definition{%
  \thm@notefont{}% same as heading font
  \normalfont % body font
}
\makeatother
\date{}
%opening
\title{Lecture 14: Foster's Stability Criterion}
\date{}%01 Mar 2016}
\author{}

\begin{document}
\maketitle
\section{Foster's theorem for Markov chains}
For many Markov chains, finding stationary distribution is impossible. 
At the least, we would like to know if they are positive recurrent. 

\begin{lem} 
An irreducible Markov chain is positive recurrent iff there exists a state $i \in E$ such that
\begin{align*}
\lim_{N \to \infty} \E\Big[ \sum_{n=1}^{N} \boldsymbol{1}_{\{X_n=i \}} \Big] > 0. % \Longleftrightarrow x \ is \ positive \ recurrent  
\end{align*}
\end{lem}
\begin{proof}
For an irreducible Markov chain, it suffices to show that there exists a positive recurrent state $i \in E$. 
By renewal theory, for any state $i \in E$ of a Markov chain $X$,
\begin{align*}
\lim_{N \to \infty} \E\Big[ \sum_{n=1}^{N} \boldsymbol{1}_{\{X_n=i\}} \Big] = \frac{1}{\mu _{ii}} 
\end{align*}
where $\mu_{ii}$ is the mean return time to state $i$, 
\eq{
\mu _{ii}=
\begin{cases}
	\infty, & i\text{ transient},\\
	\sum_{m \in \N}m f^{m}_{ii} &i \text{ recurrent}.
\end{cases}
}
This implies that a state $i \in E$ being positive recurrent is equivalent to the given hypothesis $1/\mu_{ii} > 0$. 
%\begin{align*}
%\lim_{N \to \infty} \E\Big[ \sum_{n=1}^{N} \boldsymbol{1}_{\{X_n=i \}} \Big] > 0. % \Longleftrightarrow x \ is \ positive \ recurrent  
%\end{align*}
\end{proof}

The following theorem gives a sufficient criterion for the positive recurrence of a Markov chain in terms of a Lyapunov function $L$. 
The value $L(i)$ for any state $i$ attained by Markov chain denotes ``energy'' or ``potential'' of that state. 
The idea is that if the mean energy decreases for all but finitely many states, the Markov chain keeps returning to a finite set of states. 
That is, the Markov chain is positive recurrent. 
For any positive function $g: E \to \R_+$ and threshold $k$, we can define level sets 
\eq{
F(g,k) &= \{ i\in E: g(i) \leq k\}.
}
\begin{thm}[F. G. Foster,1950]
Let $X = \{X_n: n \in \N_0\}$ be an irreducible Markov chain on countable state space $E$. 
If there exist a function $L : E \to \R_+$ with bounded mean $\E L(X_0) < \infty$, 
such that for some $K, k \geq 0$ and $\epsilon>0$ the following conditions hold
\begin{enumerate}[i\_]
\item \textbf{Unbounded growth of energy.} $F(L,k)$ is a finite set.
\item \textbf{Bounded energy on bounded set.} $\E[L(X_n)|X_{n-1}] < K$,when $X_{n-1}) \in F(L,k)$.
\item \textbf{Negative drift on unbounded set.} $\E[L(X_n)-L(X_{n-1})|X_{n-1}] < -\epsilon$ if $X_{n-1})\notin F(L,k)$
\end{enumerate}
Then the Markov chain $X$ is positive recurrent. 
%(L $\equiv$ "potential function" or energy or Lyapunov function).
\end{thm}
\begin{proof}
%Since the Markov chain is irreducible, iconsider
For the Markov chain $X$ and the Lyapunov function $L$, choose $K,k$ and $\epsilon$ such that the hypotheses hold.  
We refer to the finite set of states $F(L,k)$ as $F$. 
%\eq{
%F &= \{i \in E: L(i) \leq k\}.
%}
We can write the following using telescopic sum
\eq{
\E L(X_n) - \E L(X_0) %&=  \sum_{n=1}^{N}\E[L(X_n)-L(X_{n-1})] 
&=  \sum_{n=1}^{N}\E[L(X_n)-L(X_{n-1})1\{X_{n-1} \notin F \}] + \sum_{n=1}^{N}\E[L(X_n)-L(X_{n-1}) 1\{X_{n-1} \in F\}].
}
Using the second and third hypotheses, we can write 
\eq{
\E L(X_n) - \E L(X_0) &\leq -\sum_{n=1}^{N}\epsilon\E1\{X_{n-1} \notin F \} + \sum_{n=1}^{N}K \E1\{X_{n-1} \in F\}
= -\epsilon N + \sum_{n=1}^{N}(K+\epsilon) \E1\{X_{n-1} \in F\}.
}
%\begin{align*}
%& 0 \leq \E[L(X_n)] = \E[L(X_0)] +   \sum_{n=1}^{N}\E[L(X_n)-L(X_{n-1})]\\
%&= \E[L(X_0)] + \sum_{n=1}^{N}\E[L(X_n)-L(X_{n-1})]\boldsymbol{1}_{\{L(X_{n-1})>k\}} + \sum_{n=1}^{N}\E[L(X_n)-L(X_{n-1})]\boldsymbol{1}_{\{L(X_{n-1})\leq k\}}\\
%&\leq \E[L(X_0)] + \sum_{n=1}^{N}\E[-\epsilon \        \boldsymbol{1}_{\{L(X_{n-1})>k\}}] + \sum_{n=1}^{N}\E[K \ \boldsymbol{1}_{\{L(X_{n-1})\leq k\}}]\\
%&\to \Bigg[\E\sum_{n=1}^{N} \boldsymbol{1}_{\{L(X_{n-1})\leq k\}}\Bigg](K + \epsilon) \geq -\E[L(X_0)] + \epsilon N\\\\
%&\to \frac{1}{N} \Bigg[\E\sum_{n=1}^{N} \boldsymbol{1}_{\{L(X_{n-1})\leq k\}}\Bigg] \geq -\frac{\E[L(X_0)]}{N(K+\epsilon)} + \frac{\epsilon}{K+\epsilon}\\\\
%&\to \limsup_{N\to \infty}\frac{1}{N} \Bigg[\E\sum_{n=1}^{N} \boldsymbol{1}_{\{L(X_{n-1})\leq k\}}\Bigg] \geq \frac{\epsilon}{K+\epsilon}
%\end{align*}
Since $L$ is a positive function, we can re-arrange the above equation to write 
\eq{
\frac{1}{N}\E\sum_{n=1}^{N}1\{X_{n-1} \in F\} &= \sum_{i \in F}\left[\frac{1}{N}\E\sum_{n=1}^{N}1\{X_{n-1} = i\}\right] \geq \frac{\epsilon}{K + \epsilon} - \frac{\E L(X_0)}{N(K+\epsilon)}
}
Taking limit superior as $N \uparrow \infty$ on both sides, we obtain
\eq{
\sum_{i \in F}\left[\lim\sup_{N \to \infty}\frac{1}{N}\E\sum_{n=1}^{N}1\{X_{n-1} = i\}\right] & \geq \frac{\epsilon}{K + \epsilon}.
}
%Let $F = \{x \in \N_0: L(x) \leq k\}, \ |F| < \infty$. Now,
%\begin{align*}
%& \limsup_{N\to \infty}\frac{1}{N} \Bigg[\E\sum_{n=1}^{N} \boldsymbol{1}_{\{L(X_{n-1})\leq k\}}\Bigg]\\\\
%&=\limsup_{N\to \infty}\frac{1}{N} \Bigg[\E\sum_{n=1}^{N}\sum_{x \in F} \boldsymbol{1}_{\{X_{n-1}\leq k\}}\Bigg] \leq \sum_{x \in F} \Bigg[ \limsup_{N\to \infty}\frac{1}{N} \E\sum_{n=1}^{N} \boldsymbol{1}_{\{X_{n-1}\leq k\}}\Bigg]\\\\ 
%&=\sum_{x \in F} \Bigg[ \limsup_{N\to \infty}\frac{1}{N} \E\sum_{n=1}^{N} \boldsymbol{1}_{\{X_{n-1}\leq k\}}\Bigg] \geq \frac{\epsilon}{K+\epsilon}>0
%\end{align*}
Therefore there exist some $x \in F$ such that ,
\eq{
\limsup_{N\to \infty}\frac{1}{N} \E\Bigg[ \sum_{n=1}^{N} \boldsymbol{1}_{\{X_{n-1}\leq k\}}\Bigg] \geq \frac{\epsilon}{(K+\epsilon)|F|}>0.
}
%Therefore there exist some $x \in F$ such that ,
%\[\lim_{N\to \infty}\frac{1}{N} \E\Bigg[\sum_{n=1}^{N} \boldsymbol{1}_{\{X_{n-1}\leq k\}}\Bigg] >0\]
\end{proof}
\section{Applications of Foster's theorem: Queue scheduling/Max-weight scheduling}
Consider $N$ queue served by a single server in discrete time(Figure 1).
At time slot t = 1,2,3....\\
$A_i(t) \in \N_0$ packets arrive to each queue $i \in [N]$ independently.\\
\begin{figure}
\centering
\includegraphics[scale=0.6]{Figures/saa.png}
\caption{N queues single server, server chooses one queue and serve upto one packet}
\label{fig: example 1.1}
\end{figure}
\begin{enumerate}
\item $\E[A_i(t)] = \lambda _i$
\item $\mathbb{P}[A_i(t)=0] > 0$
\item $\E[A_i(t)^2] \leq C$
\end{enumerate}
Server picks one queue $Q(t) \in [N]$ for service. Let $R_i(t) =  \boldsymbol{1}\{Q(t) = i\}$.One packet is served from $Q(t)$ if it is not empty.Let $X_i(t)$ = number of packets in queue $i$ just before time slot $t$.\\
\[X_i(t+1) = (X_i(t) + A_i(t) - R_i(t))_+\]
Where \[a_+ = max(0,a)\]
aslo \[X_i(t+1) = X_i(t) + A_i(t) - R_i(t) + L_i(t)       \]Where 
\begin{equation*}
L_i(t) = 
\begin{cases}
  1 \ \ \ \,service \ attempted \ when \ i \ is \ empty\\
  0 \ \ \ \,otherwise 
\end{cases}
\end{equation*}
Mean rate of arrivals to system : $= \sum_{i=1}^{N} \lambda _i
$\\
Maximum rate of departure $= 1$, we will assume $ \sum_{i=1}^{N} \lambda _i < 1$.
\begin{thm}[Max-weight scheduling algorithm, 1992]
\[Q(t) = \arg \max _{i \in N} X_i(t)  \] 
that is serve  the longest queue.Under MAX-WT ,$X(t) = (X_i(t))_{i=1}^{N}$ is a DTMC which is irreducible and aperiodic on state space $\N_0^N$.As long as $\sum_{i=1}^{N} \lambda _i < 1$, $\{X_n\}$ is positive recurrent.
\end{thm}
\begin{proof}
By Foster's theorem, define the Lyapunov function:
\[L(x) = \frac{1}{2} \sum_{i=1}^{N}x_i^2\] consider
\begin{align*}
& L(X(t)) - L(X(t-1))\\
&= \frac{1}{2} \sum_{i=1}^{N}\Big[ (X_i(t))^2 - (X_i(t-1))^2 \Big]\\
&= \frac{1}{2} \sum_{i=1}^{N}\Big[ \{X_i(t-1) + A_i(t-1) - R_i(t-1) + L_i(t-1)\}^2 - (X_i(t-1))^2 \Big]\\
& \leq \frac{1}{2} \sum_{i=1}^{N}\Big[ \{X_i(t-1) + A_i(t-1) - R_i(t-1)\}^2 - (X_i(t-1))^2 \Big]
\end{align*}
Therefore 
\[\E\Big[ L(X(t)) - L(X(t-1)) | X(t-1) = x \Big]  \leq \frac{1}{2} \sum_{i=1}^{N} \E \Big[ (x_i + A_i(t-1) - R_i(t-1))^2 - x_i^2 | X(t-1) = x \Big]\]
\begin{align*}
&= \frac{1}{2} \sum_{i=1}^{N} \Bigg[ 2x_i \E \Big[ (A_i(t-1) - R_i(t-1)| X(t-1) = x \Big] + \E \Big[ (A_i(t-1) - R_i(t-1))^2 | X(t-1) = x \Big] \Bigg]\\
&= \sum_{i=1}^{N}x_i \lambda _i - \sum_{i=1}^{N}x_i R_i(t-1) + \frac{N}{2}(1+C)\\
&= \sum_{i=1}^{N}x_i \lambda _i + \frac{N}{2}(1+C) - \max _i x_i\\
& \leq \frac{N}{2}(1+C) + (\max _i x_i)(\sum_{i=1}^{N} \lambda _i - 1)\\
&= C_1 - \epsilon (\max _i x_i)
\end{align*}
Foster's theorem applies with $k = \max \{\frac{\|x\|_2 ^2}{2}\}$
\end{proof}
\end{document}

