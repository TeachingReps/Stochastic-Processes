\documentclass[a4paper,10pt]{article}
\usepackage{%
	amsfonts,%
	amsmath,%	
	amssymb,%
	amsthm,%
	algorithm,%
	babel,%
	bbm,%
	etex,%
	caption,%
	centernot,%
	color,%
	dsfont,%
	enumerate,%
	epsfig,%
	geometry,%
	graphicx,%
	hyperref,%
	latexsym,%
	mathtools,%
	multicol,%
	pgf,%
	pgfplots,%
	pgfplotstable,%
	pgfpages,%
	proof,%
	psfrag,%
	subfigure,%	
	tikz,%
	ulem,%
	url%
}	
\usepackage[noend]{algpseudocode}
\usepackage[mathscr]{eucal}
\usepgflibrary{shapes}
\usetikzlibrary{%
  arrows,%
  backgrounds,%
  chains,%
  decorations.pathmorphing,% /pgf/decoration/random steps | erste Graphik
  decorations.text,%
  fit,%
  matrix,%
  patterns,%
  petri,%
  positioning,% wg. " of "
  plotmarks,%
  scopes,%
  shadows,%
  shapes,%
  shapes.arrows,%
  shapes.callouts,%
  shapes.misc% wg. rounded rectangle
}

\theoremstyle{plain}
\newtheorem{thm}{Theorem}[section]
\newtheorem{lem}[thm]{Lemma}
\newtheorem{prop}[thm]{Proposition}
\newtheorem{cor}[thm]{Corollary}

\theoremstyle{definition}
\newtheorem{defn}[thm]{Definition}
\newtheorem{conj}[thm]{Conjecture}
\newtheorem{exmp}[thm]{Example}
\newtheorem{assum}[thm]{Assumptions}
\newtheorem{axiom}[thm]{Axiom}

\theoremstyle{remark}
\newtheorem{rem}{Remark}
\newtheorem{note}{Note}
\newtheorem{fact}{Fact}

\definecolor{lightgray}{gray}{0.9}

%\DeclarePairedDelimiter{\ceil}{\left\lceil}{\right\rceil}%
\newcommand{\eq}[1]{\begin{align*}#1\end{align*}}
\newcommand{\ceil}[1]{\left\lceil#1\right\rceil}%
\newcommand{\norm}[1]{\left\lVert#1\right\rVert}%
\newcommand{\indep}{\!\perp\!\!\!\perp}%
\DeclarePairedDelimiter\abs{\lvert}{\rvert}%
\newcommand\numberthis{\addtocounter{equation}{1}\tag{\theequation}}
\newcommand{\tr}{\operatorname{tr}}
\newcommand{\R}{\mathbb{R}}
\newcommand{\N}{\mathbb{N}}
\newcommand{\E}{\mathbb{E}}
\newcommand{\Z}{\mathbb{Z}}
\newcommand{\B}{\mathscr{B}}
\newcommand{\C}{\mathcal{C}}
\newcommand{\T}{\mathscr{T}}
\newcommand{\F}{\mathcal{F}}
\newcommand{\G}{\mathcal{G}}
\newcommand{\X}{\mathcal{X}}
%\newcommand{\ba}{\begin{align*}}
%\newcommand{\ea}{\end{align*}}
\DeclareMathOperator*{\argmax}{arg\,max}
\renewcommand{\qedsymbol}{$\blacksquare$}
\makeatletter
\def\BState{\State\hskip-\ALG@thistlm}
\makeatother

\makeatletter
\def\th@plain{%
  \thm@notefont{}% same as heading font
  \itshape % body font
}
\def\th@definition{%
  \thm@notefont{}% same as heading font
  \normalfont % body font
}
\makeatother
\date{}
\title{Lecture 6: Key Renewal Theorem and Applications}
\author{Parimal Parag}
\begin{document}
\maketitle
\section{Key Renewal Theorem and Applications}
\begin{defn}[Lattice Random Variable] A non-negative random variable $X$ is said to be \textbf{lattice} if there exists $d \geq 0$ such that 
\begin{equation*}
\sum_{n\in\mathbb{N}}\Pr\{X = nd\} = 1.
\end{equation*}
For a lattice $X$, its period is defined as 
\begin{equation*}
d = \min\{ d \in\mathbb{R}^+: \Pr\{X = d\} > 0 \}.
\end{equation*}
If $X$ is a lattice random variable, it distribution function $F$ is also called lattice.
\end{defn}
\subsection{Blackwell's Theorem}
\begin{thm}[Blackwell's Theorem] Let $N(t)$ be a renewal process with mean $m(t)$, and inter-arrival times with distribution $F$ and mean $\mu$. If $F$ is not lattice, then for all $a \geq 0$
\begin{equation*}
\lim_{t \to \infty} m(t+a) -m(t) = \frac{a}{\mu}.
\end{equation*}
If $F$ is lattice with period $d$, then 
\begin{equation*}
\lim_{n \to \infty} E[\text{number of renewals at } nd] = \frac{d}{\mu}.
\end{equation*}
\end{thm}
\begin{proof} 
%From key renewal theorem, we have $\lim_{t \to \infty}m(t)/t = \frac{1}{\mu}$. For $\epsilon > 0$, pick $T$ such that for all $t > T$, we have
%\begin{align*}
%|m(t) - \frac{t}{\mu}| &\leq \epsilon/2,  |m(t+a) - \frac{t+a}{\mu}| &\leq \epsilon/2.
%\end{align*}
%Taking difference, we obtain for all $t \geq T$
%\begin{align*}
%|m(t+a) - m(t) - \frac{a}{\mu}| &\leq \epsilon.
%\end{align*}
We will not prove that 
\begin{equation}
\label{eq:LimitBlackwell}
g(a) = \lim_{t \to \infty} [m(t+a) - m(t)]
\end{equation}exists for non-lattice $F$. However, we show that if this limit does exist, it is equal to $a/\mu$ as a consequence of elementary renewal theorem. To this end, note that
\begin{equation*}
m(t+a+b) - m(t) = m(t+a+b) - m(t+a) + m(t+a) - m(t).
\end{equation*}
Taking limits on both sides of the above equation, we conclude $g(a+b) = g(a) +  g(b)$. Only increasing solution of such a $g$ is
\begin{equation*}
g(a) = ca, \forall a > 0,
\end{equation*}
for some positive constant $c$. To show $c = \frac{1}{\mu}$, define a sequence $\{x_n, n \in \mathbb{N}\}$ in terms of $m(t)$ as 
\begin{equation*}
x_n = m(n) - m(n-1),~n \in \mathbb{N}.
\end{equation*}
Note that $\sum_{i=1}^nx_i = m(n)$ and $x_n = g(1)$, hence
\begin{equation*}
\lim_{n \to \infty}x_n = c \Rightarrow \lim_{n \to \infty}\frac{\sum_{i=1}^nx_i}{n} = \lim_{n \to \infty}\frac{m(n)}{n} = c.
\end{equation*}
Therefore, we can conclude $c = 1/\mu$ by elementary renewal theorem.

When $F$ is lattice with period $d$, limit in~\eqref{eq:LimitBlackwell} doesn't exist. However, the theorem is true for lattice trivially by elementary renewal theorem.
\end{proof}

\subsection{Directly Riemann Integrable}
\begin{defn}[Directly Riemann Integrable] A function $h: [0,\infty] \rightarrow \mathbb{R}$ is \textbf{directly Riemann integrable} if the partial sums obtained by summing the infimum and supremum of $h$, taken over intervals obtained by partitioning the positive axis, are finite and both converge to the same limit, for all finite positive interval lengths. That is,
\begin{equation*}
	\lim_{\delta \rightarrow 0} \delta \sum_{n \in \mathbb{N}}\sup_{ u \in [(n-1)\delta,n\delta]}h(u)=\lim_{\delta \rightarrow 0} \delta \sum_{n \in \mathbb{N}}\inf_{ u \in [(n-1)\delta,n\delta]}h(u)  
\end{equation*}   
 If both limits exist and are equal, the integral value is equal to the limit. 
\end{defn}
	Compare this definition with the definition of Riemann integrals. A function $g: [0, M] \rightarrow \mathbb{R} $ is Riemann integrable if 
   \begin{eqnarray*}
  \lim_{\delta \rightarrow 0} \delta \sum_{k=0}^{M/\delta}\sup_{ u \in [(n-1)\delta,n\delta]}g(u)=\lim_{\delta \rightarrow 0} \delta \sum_{k=0}^{M/\delta}\inf_{ u \in [(n-1)\delta,n\delta]}g(u)  
  \end{eqnarray*} 
   and in that case, limit is the value of the integral. For $h$ defined on $[0,\infty]$, $\int_{0}^{\infty}h(u)du = \lim_{M \rightarrow \infty}\int_{0}^{M}h(u)du$, if the limit exists. For many functions, this limit may not exist.
\begin{prop}[Sufficiency for Directly Riemann Integrable] Following are sufficient conditions for a function $h$ to be directly Riemann integrable.
  \begin{enumerate}
  \item If $h$ is bounded and continuous and $h$ is non increasing. 
  \item If $h$ is bounded above by a directly Riemann integrable function.
	\item If $h$ is non-negative, non-increasing, and with bounded integral.
  \end{enumerate}
\end{prop}
\begin{prop}[Tail Property] If $h$ is non-negative, directly Riemann integrable, and has bounded integral value, then 
\begin{equation*}
\lim_{t \rightarrow \infty} h(t)=0.
\end{equation*}
\end{prop}
\subsection{Key Renewal Theorem}
\begin{thm}[Key Renewal Theorem] Let $N(t)$ be a renewal process having mean $m(t)$, and \emph{iid} inter-arrival times with mean $\mu$ and distribution function $F$. If $F$ is non-lattice, and if a function $h(t)$ is directly Riemann integrable, then
\begin{equation*}
\lim_{t \rightarrow \infty} \int_{0}^{\infty}h(t-x)dm(x)=\frac{1}{\mu}\int_{0}^{\infty}h(t)dt,
\end{equation*}
where 
\begin{align*}
m(t)&=\sum_{n \in \mathbb{N}}F_n(t)
, & \mu&= \int_{0}^{\infty}F^c(t).
\end{align*}
\end{thm}
%\textbf{Remark:} and we can deduce one theorem from the other. 

\begin{prop}[Equivalence] Blackwell's theorem and key renewal theorem are equivalent.
\end{prop}
\begin{proof} Let's assume key renewal theorem is true. We select $h$ as a simple function with value unity on interval $[0, a]$ and zero elsewhere. That is,
\begin{equation*}
h(x) = 1_{\{ x \in [0,a]\}}.
\end{equation*}
It is easy to see that this function is directly Riemann integrable. With this selection of $h$, Blackwell's theorem follows.

To see how we can prove the key renewal theorem from Blackwell's theorem, observe from Blackwell's theorem that,
\begin{align*}
&\lim_{a \rightarrow 0}\lim_{t \rightarrow \infty} \frac{m(t
+a)-m(t)}{a}=\frac{1}{\mu}\\
&\stackrel{(a)}{\Rightarrow}\lim_{t \rightarrow \infty}\frac{dm(t)}{dt}=\frac{1}{\mu},
\end{align*}     
where in $(a)$ we can exchange the order of limits under certain regularity conditions.Reverse can be proved by approximating a directly Riemann integrable function with simple functions. We defer the formal proof for a later stage.
\end{proof}
\begin{rem} Key renewal theorem is very useful in computing limiting value of some function $g(t)$, probability or expectation of an event at arbitrary time $t$, for a renewal process. This value is computed by conditioning on the time of last renewal prior to time $t$.
\end{rem}
\begin{thm}[Key Lemma] Let $N(t)$ be a renewal process, with mean $m(t)$, \emph{iid} inter-renewal times $\{X_n\}$ with distribution function $F$, and $n^{\mathrm{th}}$ renewal instant $S_n$. Then,
\begin{equation*}
\Pr\{S_{N(t)}\leq s\}=F^c(t)+\int_{0}^{s}F^c(t-y)dm(y),\quad\quad t\geq s \geq 0.
\end{equation*}
\end{thm} 
\begin{proof} We can see that event of time of last renewal prior to $t$ being smaller than another time $s$ can be partitioned into disjoint events corresponding to number of renewals till time $t$. Each of these disjoint events is equivalent to occurrence of $n^{\mathrm{th}}$ renewal before time $s$ and $(n+1)^{\mathrm{st}}$ renewal past time $t$. That is,
\begin{equation*}
	\{S_{N(t)} \leq s\} = \bigcup_{n \in \mathbb{N}_0}\{ S_{N(t)} \leq s, N(t)=n\} = \bigcup_{n \in \mathbb{N}_0}\{ S_n \leq s, S_{n+1} > t\} .
\end{equation*}
Recognizing that $S_0 = 0$, $S_1 = X_1$, and that $S_{n+1} = S_n + X_{n+1}$, we can write
\begin{equation*}
	\Pr\{S_{N(t)} \leq s\} = \Pr\{X_1 > t\} + \sum_{n \in \mathbb{N}}\Pr\{ X_{n+1} + S_n > t, S_n \leq s\} .
\end{equation*}
We recall $F_n$, $n$-fold convolution of $F$, is the distribution function of $S_n$. Conditioning on $\{S_n = y\}$, we can write
\begin{align*}
	\Pr\{S_{N(t)} \leq s\} &= F^c(t) + \sum_{n \in \mathbb{N}}\int_{y=0}^{s}\Pr\{ X_{n+1} > t - S_n, S_n \leq s| S_n = y\}dF_n(y),\\
	&= F^c(t) + \sum_{n \in \mathbb{N}}\int_{y=0}^{s}F^c(t-y)dF_n(y).
\end{align*}
Using monotone convergence theorem to interchange integral and summation, and noticing that $m(y) = \sum_{n\in\mathbb{N}}F_n(y)$, the result follows.
%\begin{flalign*}
%\Pr\{S_{N(t)} \leq s\} = \Pr\{X_1 > t\} +\sum_{n \in \mathbb{N}}\int_{y=0}^{s}\Pr\{ S_{n+1} > t|S_n=y\}dF_n(y)\\
%&= \Pr\{X_1 > t\} +\sum_{n \in \mathbb{N}}\int_{y=0}^{s}\Pr\{ X_{n+1} > t-y\}dF_n(y)\\
%&= \Pr\{X_1 > t\} +\sum_{n \in \mathbb{N}}\int_{y=0}^{s}F^c(t-y)dF_n(y)\\
%&\stackrel{(a)}{=} \Pr\{X_1 >t) +\int_{y=0}^{s}F^c(t-y)\sum_{n \in \mathbb{N}}dF_n(y),\\
%\end{flalign*}
%where the order of summation is exchanged by virtue of the fact that the summands are non-negative and the sum is finite (Fubini's theorem).
\end{proof}
\begin{rem}\label{Remark:DensityLastRenewal} Key lemma tells us that distribution of $S_{N(t)}$ has probability mass at $0$ and density between $(0,t]$.
\begin{align*}
\Pr\{S_{N(t)}=0\}&=F^c(t),& dF_{S_{N(t)}}(y)&=F^c(t-y)dm(y)~\quad 0 < y \leq t.
\end{align*}
\end{rem}
\begin{rem} Density of $S_{N(t)}$ has interpretation of renewal taking place in the infinitesimal neighborhood of $y$, and next inter-arrival after time $t-y$. To see this, we notice 
\begin{equation*}
dm(y)=\sum_{n \in \mathbb{N}}dF_n(y) =\sum_{n \in \mathbb{N}}\Pr\{n^{\text{th}} \text{renewal occurs in} (y,y+dy)\}.
\end{equation*}
Combining interpretation of density of inter-arrival time $dF(t)$, we get
\begin{equation*}
dF_{S_{N(t)}}(y)=\Pr\{\text{renewal occurs in }(y,y+dy) \text{ and next arrival after}~ t-y\}.
\end{equation*}
\end{rem}
%%%%%%%%%%%%%%%%%%%%%%%%%%%%%%%%%%%%%%%%%%%%%%%%%%%%%%%%
\subsection{Alternating Renewal Processes}
%%%%%%%%%%%%%%%%%%%%%%%%%%%%%%%%%%%%%%%%%%%%%%%%%%%%%%%%
Alternating renewal processes form an important class of renewal processes, and model many interesting applications. We find one natural application of  key renewal theorem  in this section. 
\begin{defn}[Alternating Renewal Process] Let $\{(Z_n,Y_n),~n \in \mathbb{N}\}$ be an \emph{iid} random process, where $Y_n$ and $Z_n$ are not necessarily independent. A renewal process where each inter-arrival time $X_n$ consist of ON time $Z_n$followed by OFF time $Y_n$, is called \textbf{alternating renewal process}. We denote distribution for ON, OFF, renewal periods by $H, G, F$ respectively. Let 
\begin{equation*}
P(t)=\Pr\{\text{ON at time}~ t\}.
\end{equation*}
\end{defn}
%Let $Y_n \sim G$, $Z_n \sim H, F \sim Z_n+Y_n \triangleq X_n$. The random variable $Z_n$ denotes the ON time of a system and $Y_n$ denotes the OFF time of the system. Let $P(t)=P(\text{ON at time}~ t)$.
\begin{thm}[ON Probability] \label{Thm:OnProbability}
If $\mathbb{E}[Z_n+Y_n]< \infty $ and $F$ is non-lattice, then
\begin{equation*}
P(t) = H^c(t)+\int_{0}^{t}H^c(t-y)dm(y).
\end{equation*}
\end{thm} 
\begin{proof} To find time dependent probability $P(t)$, we can partition the event of system being ON at time $t$ on value of last renewal time $S_{N(t)}$. That is, we can write
\begin{equation*}
\{\text{ON at time}~ t\} =\bigcup_{y \in [0, t)}\{\text{ON at time } t, S_{N(t)} = y\}.% = \bigcup_{y \in [0, t)}\{Z_1 > t- y, S_{N(t)} = y\}.
\end{equation*}
Since any ON time is possibly only dependent on the corresponding OFF time and no past renewal times. Conditioned on $\{S_{N(t)} = y \}$ system stays ON at time $t$, \emph{iff} ON time is longer than $t-y$ conditioned on renewal time being larger than $t-y$. That is, 
\begin{equation*}
\{\text{ON at time } t| S_{N(t)} = y\} =\{Z_1 > t - y| Z_1 + Y_1 > t-y\}.
\end{equation*}
Result follows from these observations along with the density of $S_{N(t)}$ from Remark~\ref{Remark:DensityLastRenewal}.
\end{proof}

\begin{cor}[Limiting ON Probability]
If $\mathbb{E}[Z_n+Y_n]< \infty $ and $F$ is non-lattice, then
\begin{equation*}
\lim_{t \rightarrow \infty}P(t)=\frac{\mathbb{E}[Z_n]}{\mathbb{E}[Y_n]+\mathbb{E}[Z_n]}.
\end{equation*}
\end{cor}
\begin{proof} Since $H$ is distribution function, 
\begin{align*}
\lim_{t \rightarrow \infty}H^c(t) &= 0, & \int_0^\infty H^c(t)dt &= E[Z_n].
\end{align*}
Applying key renewal theorem to Theorem~\ref{Thm:OnProbability}, we get the result.
%\begin{align*}
%P(t)&= \Pr\{\text{ON at time t}, S_{N(t)}=0\}+\Pr\{\text{ON at time t}, S_{N(t)}>0\}\\
%&=\Pr\{\text{ON at time t}, S_{N(t)}=0\}+\int_{y=0}^{t}\Pr\{\text{ON at time t}| S_{N(t)}=y\}dF_{S_{N(t)}}(y)\\
%&=\Pr\{Z_1>t)+\int_{y=0}^{t}\Pr\{Z>t-y| Z+Y > t-y\}dF_{S_{N(t)}}(y)\\
%&\stackrel{(a)}{=}H^c(t)+\int_{y=0}^{t}\frac{H^c(t-y)}{F^c(t-y)}F^c(t-y)dm(y)\\
%&= H^c(t)+\int_{y=0}^{t}H^c(t-y))dm(y),\\
%\end{align*}
%where $(a)$ follows from the remark following Theorem 3.Now apply key renewal theorem to obtain the required result. Since $H^c(t) \rightarrow 0$ as $t \rightarrow \infty$, we get
%\begin{flalign}
%P(t) \rightarrow \frac{\int_{0}^{\infty}H^c(t)dt}{\mu}=\frac{\mathbb{E}[Z_n]}{\mathbb{E}[Y_n]+\mathbb{E}[Z_n]}.
%\end{flalign}
\end{proof}

Many processes of practical interest can be modeled by an alternate renewal process. 
\begin{exmp}[Age and Excess Time] Consider a renewal process and let $A(t)$ be the time from $t$ since the last renewal and $Y(t)$ be the time from $t$ till the next renewal. That is,
\begin{align*}
Y(t) &=S_{N(t)+1}-t,\\
A(t) &=t-S_{N(t)}.
\end{align*}   
Suppose we need to find $\lim_{t \rightarrow \infty}P(A(t) \leq x)$ for some fixed  x. Now, observe that $P(A(t) \leq x)=\mathbb{E}[1_{A(t) \geq x}]$ which is the mean time when the ``age at $t$" is less than $x$ which is equal to $\mathbb{E}[\min\{x,X\}]$. Hence, we get\\
\begin{equation*}
\lim_{t \rightarrow \infty} P(A(t) \leq x)= \frac{\int_{t=x}^{\infty}F^c(t)dt}{\mu}.
\end{equation*}  
\end{exmp}
It is to be mentioned that $P(Y(t)\leq x)$ also yield the same limit as $t \rightarrow \infty$. This can be observed by noting that if we consider the reversed processes (an identically distributed renewal process), $Y(t)$,  the ``excess life time" at $t$ is same as the age at $t$, $A(t)$ of the original process.
%%%%%%%%%%%%%%%%%%%%%%%%%%%%%%%%%%%%%%%%%%%%%%%%%%%%%%%%%%5  
\subsubsection{The Inspection Paradox}
%%%%%%%%%%%%%%%%%%%%%%%%%%%%%%%%%%%%%%%%%%%%%%%%%%%%%%%%%%
Define $X_{N(t)+1}=A(t)+Y(t)$ as the length of the renewal interval containing $t$, in other words, the length of current renewal interval. Inspection paradox says that $P(X_{N(t)+1} >x)\geq F^c(x)$. That is, for any $x$, the length of the current renewal interval to be greater than $x$ is always more likely than that for an ordinary renewal interval. Formally,
\begin{flalign*}
P(X_{N(t)+1}>x)&= P(X_{N(t)+1}>x, S_{N(t)+1}=0)+P(X_{N(t)+1}>x,  S_{N(t)+1}>0)\\
&\geq  P(X_{N(t)+1}>x, S_{N(t)+1}=0)\\
&= P(X_2 >x)\\
&= F^c(x).
\end{flalign*}
\end{document}