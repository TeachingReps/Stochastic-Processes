% !TEX spellcheck = en_US
% !TEX spellcheck = LaTeX
%\documentclass[a4paper,10pt,english]{article}
\documentclass[all-lectures.tex]{subfiles}
%\usepackage{%
	amsfonts,%
	amsmath,%	
	amssymb,%
	amsthm,%
%	babel,%
	bbm,%
	%biblatex,%
	caption,%
	centernot,%
	color,%
	enumerate,%
	epsfig,%
	epstopdf,%
	etex,%
	geometry,%
	graphicx,%
	hyperref,%
	latexsym,%
	mathtools,%
	multicol,%
	pgf,%
	pgfplots,%
	pgfplotstable,%
	pgfpages,%
	proof,%
	psfrag,%
	subfigure,%	
	tikz,%
	ulem,%
	url%
}	

\usepackage[mathscr]{eucal}
\usepgflibrary{shapes}
\usetikzlibrary{%
  arrows,%
  backgrounds,%
  chains,%
  decorations.pathmorphing,% /pgf/decoration/random steps | erste Graphik
  decorations.text,%
  matrix,%
  positioning,% wg. " of "
  fit,%
  patterns,%
  petri,%
  plotmarks,%
  scopes,%
  shadows,%
  shapes.misc,% wg. rounded rectangle
  shapes.arrows,%
  shapes.callouts,%
  shapes%
}

\theoremstyle{plain}
\newtheorem{thm}{Theorem}[section]
\newtheorem{lem}[thm]{Lemma}
\newtheorem{prop}[thm]{Proposition}
\newtheorem{cor}[thm]{Corollary}

\theoremstyle{definition}
\newtheorem{defn}[thm]{Definition}
\newtheorem{conj}[thm]{Conjecture}
\newtheorem{exmp}[thm]{Example}
\newtheorem{assum}[thm]{Assumptions}
\newtheorem{axiom}[thm]{Axiom}

\theoremstyle{remark}
\newtheorem{rem}[thm]{Remark}
\newtheorem{note}[thm]{Note}

\newcommand{\norm}[1]{\left\lVert#1\right\rVert}
\newcommand{\indep}{\!\perp\!\!\!\perp}
\DeclarePairedDelimiter\abs{\lvert}{\rvert}%
%\DeclarePairedDelimiter\norm{\lVert}{\rVert}%
\newcommand{\tr}{\operatorname{tr}}
\newcommand{\R}{\mathbb{R}}
\newcommand{\Q}{\mathbb{Q}}
\newcommand{\N}{\mathbb{N}}
\newcommand{\E}{\mathbb{E}}
\newcommand{\Z}{\mathbb{Z}}
\newcommand{\B}{\mathscr{B}}
\newcommand{\C}{\mathcal{C}}
\newcommand{\T}{\mathscr{T}}
\newcommand{\F}{\mathcal{F}}
\newcommand{\G}{\mathcal{G}}
%\newcommand{\ba}{\begin{align*}}
%\newcommand{\ea}{\end{align*}}

% Debug
\newcommand{\todo}[1]{\begin{color}{blue}{{\bf~[TODO:~#1]}}\end{color}}


\makeatletter
\def\th@plain{%
  \thm@notefont{}% same as heading font
  \itshape % body font
}
\def\th@definition{%
  \thm@notefont{}% same as heading font
  \normalfont % body font
}
\makeatother
\date{}
%\title{Lecture 01: Poisson Processes}
%\author{}
\setcounter{chapter}{4}

\begin{document}
%\maketitle

%\chapter{Continuous-Time Markov Chains}
\setcounter{section}{3}
\setcounter{subsection}{0}

\section*{\centering{Lecture 16 \linebreak }}

%\section{Irreducibility and Recurrence}
Let $\{X_t\}$ be a Markov chain, $\{Y_0, Y_1, \dots\}$ be its jump Markov chain, $\{T_0, T_1, \dots\}$ the sojourn times. Let $P$ be the transition matrix for $\{Y_k\}$.
$\hat{P} \to P$,
\begin{itemize}
\item $i \to j$ in $\{Y_n\}$ if $\exists$ $n_1$ s.t $P^{n_1}_{ij} > 0$,
\item $i \to j$ in $\{X_t\}$ if $\exists$ $t_1 >0$ s.t $P_{t_1}(i,j) > 0$.
\end{itemize}

Since $P_t(i,i)\to 1$ as $t \to 0$ for all $i$, $P_t(i,i)  > \epsilon$ for all $t  \in [0, \delta]$, for some $\epsilon > 0$ and $\delta > 0$.
Then $P_{nt}(i,i) \ge (P_{t}(i,i))^n$ implies that $P_t(i,i)  >0$ for all $t \in [0, \delta n]$ and hence $P_t(i,i) > 0$ for all $t$.
This also follows directly from $P_t(i,i)  > P[T_0 >t] > 0$ for $T_0 \sim exp(\lambda_{i})$.
%Consider a CTMC $\{X_t\}$  that is irreducible, ergodic and let us consider the limiting probabilities $\pi_i$. 
%Let $\{Y_k\}$ be the jump MC with transition matrix $\hat{P}$. 
%Let $\mu_i$ be its stationary probabilities.
\begin{prop}
The following statements are equivalent, under minimal construction
\begin{enumerate}
\item $i \to j$ in $\{Y_n\}$,
\item $i \to j$ in $\{X_t\}$,
\item $P_t(i,j)  > 0 $, $\forall$ $ t>0$.
\end{enumerate}
\end{prop}

\begin{proof}{$(1)	\Rightarrow (2), (3)$}\\
	
$\exists$ $n_1$ s.t $P_{n_1}(i,j) > 0$,  therefore $i \xrightarrow{T_0} i_1 \xrightarrow{T_1} i_2 \rightarrow \dots \rightarrow i_{n_1-1} \xrightarrow{T_n} j$ where $i, i_1, \dots, i_{n_1-1}$ are not absorbing states. Then,
 $\infty > \lambda(i),\lambda(i_1),\dots,\lambda(i_{n_1-1})>0$, $\lambda(j) < \infty$.\\\\
$T_0 \sim \exp(\lambda_{ii})$ where $\infty > \lambda_{ii} = \lambda_i q_{ii} > 0$.\\\\
$P_t(i,j) \ge P[\sum_{k=0}^{n-1} T_k \le \frac{t}{2} , T(j) > \frac{t}{2} > 0] >0,$ \quad $\forall$ $t>0$.\\\\
Hence $P_t(i,j) >0$.

{$(3)	\Rightarrow (2)$} is clear from the definition itself.

{$(2)	\Rightarrow (1)$}\\
	
%$\exists$ $t_1 >0$ s.t $P_{t_1}(i,j)  > 0$. Then, in the minimal construction,
$P(\omega(\Delta) > t_1) > P_{t_1}(i,j)  > 0$. Therefore, there is a finite path $i \to i_1 \to \dots \to i_n \to j$,
such that,
$P(i \to i_1 \to \dots \to i_{n_1} \to j) > 0$, $P_{ii_1}, \dots, P_{i_{n_1-1}i_{n_1}} , P_{i_ni} >0$.
This implies $i \to j$ in Markov chain $\{Y_n\}$.
\end{proof}
In particular this also implies that closed irreducible classes are same in $\{Y_k\}$ and $\{X_t\}$ and
$\{Y_n\}$ is irreducible $\Leftrightarrow$ $\{X_t\}$ is irreducible.

Let $\omega(t) = \inf{\{t >0, \enskip s.t. \enskip  X_t = i \enskip and \enskip \lim_{s \uparrow t}X_s \neq i\}}$.
This is the first time MC visits state $i$, after exiting from $i$ if $X_0 = i$.
\begin{defn}
A state $i$ is \textit{transient} if $P_i\{\omega(i)< \infty\} <1$.
It is \textit{recurrent} if $P_i\{\omega(i)< \infty\} = 1$. A recurrent chain is \textit{positive recurrent} if $\E_i[\omega(i)] < \infty$, otherwise \textit{null recurrent}.
\end{defn}
\begin{thm}
State $i$ is recurrent for $\{Y_k\}$ iff it is for $\{X_t\}$ in a minimal construction.
\end{thm}
\begin{proof}
Let $i$ be recurrent for $\{Y_k\}$. Let $\tau(i) = \inf{\{k>0 \enskip s.t.  \enskip Y_k = i\}}$.\\
Then $P[\tau(i) < \infty] = 1$ and $\omega(i) = \sum_{k=0}^{\tau(i)-1} T_k$, where $T_k$ is the sojourn time of $X_t$ in state $Y_k$. Then
\eq{
P_i[\omega(i) < \infty] &= P_i\left[\sum_{k=0}^{\tau(i)-1} T_k < \infty\right]\\
&= \sum_{n=1}^{\infty} P_i\left[\sum_{k=0}^{n-1} T_k < \infty | \tau(i) = n\right] P[\tau(i)=n].
}
But
\eq{
P_i\left[\sum_{k=0}^{n-1} T_k < \infty | \tau(i) = n\right] &= \sum_{i_1, i_2, i_3, \dots, i_{n-1}} P_i\left[\sum_{k=0}^{n-1} T_k < \infty | \tau(i) = n, Y_j = i_j, j= 1, 2, \dots, n-1\right] \\
& \enskip \enskip \enskip \enskip \enskip \enskip  \enskip \enskip \enskip \enskip \enskip \enskip \enskip \enskip \enskip \enskip \enskip  P[Y_j = i_j, j= 1, 2, \dots, n-1|\tau(i) = n]\\ &= 1.
}
Thus
\eq{
P_i[\omega(i) < \infty] = 1.
}

%Now assume that $i$ is recurrent for $\{X_t\}$. In a minimal construction, $\omega(i) < \omega(\Delta)$. Then, the number of jump till time $\omega(i)$ is finite.
Therefore, $P_i[\tau(i) < \infty] = P_i[\omega(i) < \infty] = 1$.
\end{proof}
The above theorem implies that $i$ is transient in $\{X_t\}$  iff it is in $\{Y_k\}$.

If  $\{X_t\}$ is irreducible then we have seen that $Y_k$ is irreducible.
If $i$ is recurrnet/transient in $\{X_t\}$ then so is it in $\{Y_k\}$. Then in $\{Y_k\}$ every state is recurrent/transient. Thus it is so in $\{X_t\}$. Therefore, in $\{X_t\}$ also recurrence/transience is a class property. However positive recurrence of $\{Y_k\}$ does not imply $\{X_t\}$ and vice versa.

Let $\{X_t\}$ be irreducible and $i$ be recurrent in $\{X_t\}$. We take visit times to $i$ as regenerative epochs with $\omega(i)$ as a regeneration length. Then from delayed regenerative process limit theorem for a bounded function $f$,
\eq{
\E_j[f(X_t)] \to \E[f(X_0],
}
where $\pi$ is a stationary measure for $X = \{X_t\}$. Taking $X_0 = i$,
\eq{
\E_{\pi}[f(X_0)] =  \frac{\E_i[\int_{0}^{\omega(i)} f(X_t) dt]}{\E[\omega(i)]}.
}
If $i$ is null recurrent then $\E[\omega(i)] = \infty$ and $\E_{\pi}[f(X_0)] = 0$. If $i$ is positive recurrent then $\pi$ can be normalized to get a stationary (unique) distribution for $X$. Also, then
\begin{align}
\label{eq17-1}
\pi(j) = \frac{\E_i[\int_{0}^{\omega(i)} 1_{\{X_t=j\}} dt]}{\E_i[\omega(i)]} >0, \enskip \forall j.
\end{align}
Furthermore, as $t \to \infty$,
\eq{
\P_t(i,j) \to \pi(j)
}
and also, $P_t(k,j) \to \pi(j)$ for all $k$ and $j$.
If $i$ is null recurrent, then for all $j$ , as $t \to \infty$
\eq{
P_t(i,j) \to 0, \quad P_t(k,j) \to 0 \quad \forall j,k.
}
\begin{prop}
If $i$ is transient then $P_t(j,i) \to 0$ as $t \to \infty$ for any $j \in S$.
\end{prop}
\begin{proof}
Now $P_i[\omega(i) < \infty] \triangleq p <1$. Let $N$ be number of times state $i$ is visited. Then $P[N=n] = p^n (1-p)$ and $P[N<\infty]=1$. Let $\tilde{\omega}(i) \stackrel{d}{=} [\omega(i) | \omega(i) < \infty]$ a when $X_0=i$, and $Z_k$ be $i.i.d.$ $\exp(\lambda_{i})$.

Let $\tilde{\omega}_k(i)$ $i.i.d.$ $\sim$ $\tilde{\omega}(i)$.
Then
\eq{
P_t(i,i) \le P_i[\sum_{k=1}^{N} (\tilde{\omega}_k(i) + Z_k) > t] \to 0 \quad \text{as} \quad t \to \infty.
}
Also, for $j \neq i$,
\eq{
P_t(j,i) \le P_j[\tilde{\omega}(i) + \sum_{k=1}^{N} (\tilde{\omega}_k(i) + Z_k) > t] \to 0,
}
where $\tilde{\omega}(i) \stackrel{\text{d}}{=} [\omega(i) \mid \omega(i) < \infty]$ when $X_0 = j$.
\end{proof}

Now assume $X$ is recurrent, irreducible, and $Y = \{Y_k\}$ is positive recurrent with the unique stationary distribution $\mu$. Let $N(j)$ be the number of visits to state $j$ between two visits of $i$. Let $\tau(i)$ be the intervisit time to state $i$ in $(Y_k)$. Then
\eq{
\mu(j) = \E[N(j)]/ \E_i[\tau(i)].
}
Also, from \ref{eq17-1}
\begin{align}
\label{eq17-2}
%\pi(j) &= \frac{\E_i[\int_{0}^{\omega(i)} 1_{\{X_t=j\}} dt]}{\E_i[\omega(i)]} \nonumber \\
&= \frac{\E_i[\sum_{k=1}^{N(j)} T_k(j)]}{\E_i[\sum_k \sum_{k=1}^{N(k)} T_k(j)]}
\end{align}
where $T_k(j)$ is the sojourn time in state $j$ on $k$th visit to state $j$. The RHS of \ref{eq17-2} is
\eq{
\frac{\E_j[N(j)]/\lambda_j}{\sum_{l} \E_i[N(l)]/\lambda_l} &= \frac{\mu(j) \E[\tau(i)]/ \lambda_j}{ \sum_l \mu(l) \E[\tau(i)]/\lambda_l}\\
&= \frac{\mu(j)/\lambda_j}{\sum_l \mu(l)/\lambda_l}.
}
Thus, if $\sum_l \mu(l)/\lambda_l < \infty$ then $\pi(j) > 0$ for all $j$ and $X$ is positive recurrent.
More directly, if $\E[\tau_i] < \infty$, then $\E_i[\omega(i)] = \E[\tau_i] \sum_l \mu(l)/\lambda_l$
implies that if $\sum_l \mu(l)/\lambda_l < \infty$ then $E_i[\omega(i)] < \infty $ and hence $i$ is positive recurrent. Thus, also $X$.

%If $S$ is finite, $\pi P_t = \pi$ and taking derivative, $\pi \frac{dP_t}{dt} = 0$.
Take derivative at $t=0$, we get $\pi Q =0$. More generally we have
\begin{prop}
An irreducible positive recurrent, nonexplosive MC $X$ is positive recurrent iff we can find a probability measure $\pi$ $s.t.$ $\pi Q = 0$. Then $\pi$ is the unique stationary distribution of $X$.
\end{prop}
More along the lines mentioned above, we also have
\begin{prop}
A sufficient condition for positive recurrence of an irreducible chain $X$ is that $\exists$ a distribution $\pi$ s.t. $\pi Q = 0$ and $\sum_i \pi(i) \lambda_i < \infty$. 
Then $\{Y_k\}$ is also positive recurrent with the unique stationary distribution $\mu(j) = \pi(j) \lambda_j.$
\end{prop}
\section{Time Reversibility}
Let $\{X_t\}$ be irreducible and positive recurrent and $\pi$ is its stationary distribution. 
We consider $\{X_t\}$ under stationarity. 
For fixed $T$ consider $Y_t = X_{T-t}$. It is a Markov chain with transition function, 
\eq{
\tilde{P}_t(ij) &= P[Y_t=j|Y_0=i] = P[X_{T-t}=j|X_T=i]\\
&=\frac{P[X_{T-t}=j,X_T=i]}{P[X_T=i]}\\
&=\frac{P[X_{T}=i|X_{T-t}=j] P[X_{T-t}=j]}{P[X_T=i]}\\
&=\frac{P_t(j,i)\pi(j)}{\pi(i)}.
}
The stationary distribution of the reverse process is same as for the forward process, because the fraction of the time spent by  the MC in state $i$ in both directions is same.
\begin{defn}
$\{X_t\}$ is \textit{time reversible} if 
\eq{
\tilde{P}_t(ij)= \frac{P_t(j,i)\pi(j)}{\pi(i)} = P_t(i,j).
}
\end{defn}
Taking derivative at $t=0$,
\eq{
Q_{ij} = \frac{\pi(j)}{\pi(i)}Q_{ji}, \quad \forall i, j.
}
The above equation is called \textit{detailed balance}. 
\end{document}