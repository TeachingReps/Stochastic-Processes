% !TEX spellcheck = en_US
% !TEX spellcheck = LaTeX
%\documentclass[a4paper,10pt,english]{article}
\documentclass[all-lectures.tex]{subfiles}
%\usepackage{%
	amsfonts,%
	amsmath,%	
	amssymb,%
	amsthm,%
%	babel,%
	bbm,%
	%biblatex,%
	caption,%
	centernot,%
	color,%
	enumerate,%
	epsfig,%
	epstopdf,%
	etex,%
	geometry,%
	graphicx,%
	hyperref,%
	latexsym,%
	mathtools,%
	multicol,%
	pgf,%
	pgfplots,%
	pgfplotstable,%
	pgfpages,%
	proof,%
	psfrag,%
	subfigure,%	
	tikz,%
	ulem,%
	url%
}	

\usepackage[mathscr]{eucal}
\usepgflibrary{shapes}
\usetikzlibrary{%
  arrows,%
  backgrounds,%
  chains,%
  decorations.pathmorphing,% /pgf/decoration/random steps | erste Graphik
  decorations.text,%
  matrix,%
  positioning,% wg. " of "
  fit,%
  patterns,%
  petri,%
  plotmarks,%
  scopes,%
  shadows,%
  shapes.misc,% wg. rounded rectangle
  shapes.arrows,%
  shapes.callouts,%
  shapes%
}

\theoremstyle{plain}
\newtheorem{thm}{Theorem}[section]
\newtheorem{lem}[thm]{Lemma}
\newtheorem{prop}[thm]{Proposition}
\newtheorem{cor}[thm]{Corollary}

\theoremstyle{definition}
\newtheorem{defn}[thm]{Definition}
\newtheorem{conj}[thm]{Conjecture}
\newtheorem{exmp}[thm]{Example}
\newtheorem{assum}[thm]{Assumptions}
\newtheorem{axiom}[thm]{Axiom}

\theoremstyle{remark}
\newtheorem{rem}[thm]{Remark}
\newtheorem{note}[thm]{Note}

\newcommand{\norm}[1]{\left\lVert#1\right\rVert}
\newcommand{\indep}{\!\perp\!\!\!\perp}
\DeclarePairedDelimiter\abs{\lvert}{\rvert}%
%\DeclarePairedDelimiter\norm{\lVert}{\rVert}%
\newcommand{\tr}{\operatorname{tr}}
\newcommand{\R}{\mathbb{R}}
\newcommand{\Q}{\mathbb{Q}}
\newcommand{\N}{\mathbb{N}}
\newcommand{\E}{\mathbb{E}}
\newcommand{\Z}{\mathbb{Z}}
\newcommand{\B}{\mathscr{B}}
\newcommand{\C}{\mathcal{C}}
\newcommand{\T}{\mathscr{T}}
\newcommand{\F}{\mathcal{F}}
\newcommand{\G}{\mathcal{G}}
%\newcommand{\ba}{\begin{align*}}
%\newcommand{\ea}{\end{align*}}

% Debug
\newcommand{\todo}[1]{\begin{color}{blue}{{\bf~[TODO:~#1]}}\end{color}}


\makeatletter
\def\th@plain{%
  \thm@notefont{}% same as heading font
  \itshape % body font
}
\def\th@definition{%
  \thm@notefont{}% same as heading font
  \normalfont % body font
}
\makeatother
\date{}
%\title{Lecture 01: Poisson Processes}
%\author{}
\setcounter{chapter}{4}

\begin{document}
%\maketitle

%\chapter{Martingales}
\setcounter{section}{4}
\setcounter{subsection}{1}

\section*{\centering{Lecture 20 \linebreak }}
[Martingale Convergence Theorem]
\begin{exmp}
Let $Z$ be a random variable with  $\E[|Z|] < \infty$, and  $\{Y_n\}$ a sequence of random variables. Define
$X_0=1$, $X_n = \E[Z|Y_0, Y_1, \dots, Y_n]$, Then $\E[X_{n+1}|Y_0, Y_1, \dots, Y_n] = \E[Z|Y_0, Y_1, \dots, Y_n] = X_n$. Also,
\eq{
\E[|X_n|] &= \E[|\E[Z|Y_0, Y_1, \dots, Y_n]|]\\
&\le \E[\E[|Z||Y_0, Y_1, \dots, Y_n]]\\
&= \E[|Z|]\\
&< \infty \quad \quad \text{for all} \quad n \ge 1.
}
Hence $\{X_n\}$ is a martingale $w.r.t.$ $\{Y_n\}$ (called \textit{Doob's Martingale}).
Also, $X_n \to X_\infty = \E[Z|Y_0, Y_1, \dots]$.
\end{exmp}

\begin{exmp}
Let $\{X_n\}$ be a MC with transition matrix $P$.
If $h$ is a function such that 
\eq{
h(i) = \sum_{j \in S} p_{ij}h(j) = \E[h(X_1)|X_0=i].
}

Define $Y_n = h(X_n)$. Then,
\eq{
\E[Y_{n+1} |X_0, X_1, \dots, X_n] &= \E[h(X_{n+1})|X_0, X_1, \dots, X_n]\\
&= \E[h(X_{n+1})|X_n]\\
&= h(X_n) = Y_n.
}
Therefore $\{Y_n\}$ is a martingale $w.r.t.$ $\{X_n\}$ .
If the equality is replaced with $\le$ then it is submartingale.
\end{exmp}

Suppose $h$ is bounded. Since $Y_n = h(X_n)$ is a submartingale, $Y_n \to Y_\infty$ $a.s.$ and $\E[Y_\infty] < \infty$.
Assume, $\{X_n\}$ is irreducible and recurrent.
Consider $i, j \in S$, $i\neq j$. 
State-$i$ occurs infinitely often $w.p.$1 and state-$j$ also occurs infinitely often $w.p.$1.
Thus, 
$Y_n = h(X_n) = h(i)$ and $h(j)$ infinitely often with $w.p.$1.
Therefore, for $Y_n$ to converge $a.s.$,
\eq{
h(i) = h(j), \quad \forall i,j \in S.
}

\section{Applications to Markov chain}
In this section we use martingale convergence theorems to get conditions for recurrence and transience of Markov chains.
\begin{thm}
Let $\{X_n\}$ be an irreducible, MC with state space $S$ and transition matrix $P$.
It is transient if and only iff $\exists$ a state-$i$ and $h:S \setminus \{i\} \to \R$, $h$ is bounded, non-zero and satisfies 
\eq{
h(j) = \sum_{k \neq i} p_{jk}(h(k)) \quad \quad \forall j \neq i.
}
\end{thm}
\begin{proof}
Suppose $\{X_n\}$ is transient. 
Fix a state $i$. 
Let $T(i)$ be the first time chain enters state $i$.
Define $h(j) = P_j[T(i) = \infty]$. 
It is  bounded.
Since  $\{X_n\}$ is transient, $P_j[T(i) = \infty] > 0$.
Also, 
\eq{
P_j[T(i) = \infty] = \sum_{k \neq i} P_{jk} P_k[T(i)=\infty].
}

Now we assume that such an $h$ exists and we show that MC is transient.\\
Define, $\tilde{h}$ on $S$ such that $\tilde{h}(j) = h(j)$ \quad $\forall$ $j \neq i$, $\tilde{h}(i)=0$.
Thus,
when $j \neq i$,
\eq{
\E[\tilde{h}(X_1)|X_0 = j] = \sum_k P_{jk} \tilde{h}(k) = \sum_{k \neq i} P_{jk} h(k) = h(j) = \tilde{h}(j) = \tilde{h}(X_0).
}
When $j=i$,
\eq{
\E[\tilde{h}(X_1)|X_0 = i] = \sum_{k \neq i}  P_{ik} h(k) \ge \tilde{h}(i).
}
Therefore,
\eq{
\E[\tilde{h}(X_1)|X_0] \ge \tilde{h}(X_0).
}
Thus from the previous example, $\tilde{h}(X_n) = Y_n$ is a submartingale and it is bounded.
Hence, it converges  to $Y_\infty$ $a.s.$.

If $\{X_n\}$ is recurrent then  as shown above, $\tilde{h}$ is a constant. But $\tilde{h}(i) = 0$ and $\tilde{h}$ is non-zero. Therefore it cannot be recurrent.
\end{proof}
\begin{thm}
Let  $\{X_n\}$ be irreducible. If $\exists$ $h:S \to \R$ such that $h(i) \to \infty$ as $i \to \infty$ and there is a finite set $E_0 \subset S$ such that $\E[h(X_i)|X_0=i] \le h(i)$, $\forall i \notin E_0$, then $\{X_n\}$ is recurrent.
\end{thm}
\begin{proof}
We can if needed add a constant to make $h \ge 0$. 
Let $T$ be entrance time to set $\E_0$.
$X_0=i$, \quad $i \notin E_0$. $Y_n = h(X_n) 1_{\{T>n\}}$, $\F_n = \{X_0, X_1, \dots, X_n\}$.

Then,
\eq{
\E[Y_{n+1}|\F_n] &= \E[h(X_{n+1}) 1_{\{T>n+1\}}|\F_n]\\
&\le  \E[h(X_{n+1}) 1_{\{T>n\}}|\F_n]\\
&= 1_{\{T>n\}} \E[h(X_{n+1})|\F_n]\\
&\le 1_{\{T>n\}} h(X_n)\\
&= Y_n.
}
Therefore, $Y_n$ is a nonnegative supermartingale and $Y_n \to Y_\infty$ $a.s.$ with $P[Y_\infty < \infty] = 1$.

Suppose $X_n$ is transient. Then  $X_n$ will be out of any finite set $\{i:h(i) \le a\}$ after some time. Thus,
\eq{
h(X_n) \to  \infty \quad a.s..
}
But $Y_\infty < \infty$  \quad $a.s.$.
Therefore,
\eq{
P_i[T < \infty] = 1, \quad \forall i \notin E_0.
}
Thus, finite set $E_0$ is being visited infinitely often $w.p. 1$.
Since $E_0$ is a finite set, at least one of the states $i \in E_0$ is being visited infinitely often $w.p.1$., That state is recurrent.
Then $\{X_n\}$ is recurrent. 
Hence a contradiction.
\end{proof}
Under slightly stronger conditions, we get positive recurrence of the MC.

\begin{thm}
\label{thm1}
Let  $\{X_n\}$  be irreducible. $h:S \to \R$ s.t. $h$ is lower bounded (make it $\ge$ 0 by adding a constant) and $E_0$ is finite such that $\E[h(X_1)|X_0=i] \le h(i) - \epsilon$ \quad $\forall i \notin E_0$,  for some $\epsilon >0$, and $\E[h(X_1)|X_0=i] < \infty$, \quad $\forall i \in E_0$. Then $\{X_n\}$ is positive recurrent.
\end{thm}
\begin{proof}
Let $T$ be the entrance time to set $\E_0$, $X_0 = i$, $i \notin E_0$, $Y_n = h(X_n) 1_{\{T>n\}}$. 
Then
\eq{
\E[Y_{n+1}|\F_n] &= \E[h(X_{n+1}) 1_{\{T>n+1\}}|\F_n]\\
&\le \E[h(X_{n+1}) 1_{\{T>n\}}|\F_n]\\
&= 1_{\{T>n\}} \E[h(X_{n+1})|\F_n]\\
&\le 1_{\{T>n\}} [h(X_{n})-\epsilon]\\
&= Y_n -\epsilon 1_{\{T>n\}}.
}
Take expectaions on both sides,
\eq{
\E[Y_{n+1}] &\le \E[Y_n] - \epsilon P[T>n]\\
&\le \E[Y_{n-1}] - \epsilon P[T>n-1] -\epsilon P[T>n] \\
\dots
&\le \E[Y_0] - \epsilon \sum_{k=0}^{n} P[T>k].
}
Take $n \to \infty$,
\eq{
0 \le \E[Y_0] - \epsilon \sum_{k=0}^{\infty} P[T>k] = \E[Y_0] - \epsilon \E_i[T].
}
Thus,
\eq{
\E_i[T] \le \frac{\E[Y_0]}{\epsilon} = \frac{h(i)}{\epsilon} < \infty, \quad \forall i \notin E_0.
}

For $i \in E_0$, 
\eq{
\E_i[T] &= \sum_{j\in E_0} p_{ij} + \sum_{j \notin E_0} p_{ij} \E_{j}[T+1] \\
&= 1 + \sum_{j \notin E_0} p_{ij} \E_{j}[T]\\
&\le 1+ \sum_{j \notin E_0}\frac{1}{\epsilon} p_{ij} h(j)\\
&\le 1+\frac{1}{\epsilon} \E_i[h(X_1)]\\
&< \infty.
}

Therefore, starting from any initial state, mean time to reach the finite set $E_0$ is finite.
We can show that this implies that $\{X_n\}$ is positive recurrent.
\end{proof}

\begin{thm}
$\{X_n\}$ is irreducible, $\exists$ a bounded function $h:S \to \R$ and a finite set $\E_0 \subset S$ $s.t.$  
\eq{
\E[h(X_1)|X_0=i] \ge h(i), \quad i \notin E_0
}
and $h(i) > h(j)$ for some $i \notin E_0$ and all $j \in E_0$. Then $\{X_n\}$ is transient.
\end{thm}
\begin{proof}
Take $Y_n = h(X_{n \wedge T})$, $T$ the entrance time to $E_0$, $X_0 = i \notin E_0$.We can show that $Y_n$ is a submartingale.\\
Since $h$ is bounded, $Y_n \to Y_\infty$ \quad $a.s.$. Also,
\eq{
\E[Y_\infty] \ge \E[Y_0] = h(i).
}
and the fact that $Y_\infty < h(i)$ on $\{T < \infty\}$  $\Rightarrow P_i[T=\infty] >0$ .
Therefore, $\{X_n\}$ is transient.
\end{proof}
\begin{exmp}
Consider a discrete time queue with $X_n$ the number of requests in the queue in the beginning of slot $n$. Then $X_{n+1} = (X_n + Y_n)^+$ where $Y_n = A_n - S_n$, $A_n$ is the number of arrivals in slot $n$, $S_n$ is number of requests we can serve in slot $n$. Then $Y_n$ is $i.i.d.$, integer valued.
Taking $h(i)=i$ we show that if $\E[Y_n] < 0$ then we have positive recurrence.

We have
\eq{
\E[X_1|X_0 =i] -i = \E[(i+Y_0)^+] - i.
}
As $i \to \infty$, $|\E[(i+Y_0)^+] - \E[i+Y_0]| \to \infty$. Therefore if $\E[Y_0] < -\epsilon$ for some $\epsilon > 0$ then $\E[X_1|X_0 = i]-i < \epsilon/2$ for all $i$ large enough. Then we get positive recurrence of $\{X_k\}$ from theorem \ref{thm1}.
\end{exmp}
\end{document}