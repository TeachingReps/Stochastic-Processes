% !TEX spellcheck = en_US
% !TEX spellcheck = LaTeX
%\documentclass[a4paper,10pt,english]{article}
\documentclass[all-lectures.tex]{subfiles}
%\usepackage{%
	amsfonts,%
	amsmath,%	
	amssymb,%
	amsthm,%
	algorithm,%
	babel,%
	bbm,%
	etex,%
	caption,%
	centernot,%
	color,%
	dsfont,%
	enumerate,%
	epsfig,%
	geometry,%
	graphicx,%
	hyperref,%
	latexsym,%
	mathtools,%
	multicol,%
	pgf,%
	pgfplots,%
	pgfplotstable,%
	pgfpages,%
	proof,%
	psfrag,%
	subfigure,%	
	tikz,%
	ulem,%
	url%
}	
\usepackage[noend]{algpseudocode}
\usepackage[mathscr]{eucal}
\usepgflibrary{shapes}
\usetikzlibrary{%
  arrows,%
  backgrounds,%
  chains,%
  decorations.pathmorphing,% /pgf/decoration/random steps | erste Graphik
  decorations.text,%
  fit,%
  matrix,%
  patterns,%
  petri,%
  positioning,% wg. " of "
  plotmarks,%
  scopes,%
  shadows,%
  shapes,%
  shapes.arrows,%
  shapes.callouts,%
  shapes.misc% wg. rounded rectangle
}

\theoremstyle{plain}
\newtheorem{thm}{Theorem}[section]
\newtheorem{lem}[thm]{Lemma}
\newtheorem{prop}[thm]{Proposition}
\newtheorem{cor}[thm]{Corollary}

\theoremstyle{definition}
\newtheorem{defn}[thm]{Definition}
\newtheorem{conj}[thm]{Conjecture}
\newtheorem{exmp}[thm]{Example}
\newtheorem{assum}[thm]{Assumptions}
\newtheorem{axiom}[thm]{Axiom}

\theoremstyle{remark}
\newtheorem{rem}{Remark}
\newtheorem{note}{Note}
\newtheorem{fact}{Fact}

\definecolor{lightgray}{gray}{0.9}

%\DeclarePairedDelimiter{\ceil}{\left\lceil}{\right\rceil}%
\newcommand{\eq}[1]{\begin{align*}#1\end{align*}}
\newcommand{\ceil}[1]{\left\lceil#1\right\rceil}%
\newcommand{\norm}[1]{\left\lVert#1\right\rVert}%
\newcommand{\indep}{\!\perp\!\!\!\perp}%
\DeclarePairedDelimiter\abs{\lvert}{\rvert}%
\newcommand\numberthis{\addtocounter{equation}{1}\tag{\theequation}}
\newcommand{\tr}{\operatorname{tr}}
\newcommand{\R}{\mathbb{R}}
\newcommand{\N}{\mathbb{N}}
\newcommand{\E}{\mathbb{E}}
\newcommand{\Z}{\mathbb{Z}}
\newcommand{\B}{\mathscr{B}}
\newcommand{\C}{\mathcal{C}}
\newcommand{\T}{\mathscr{T}}
\newcommand{\F}{\mathcal{F}}
\newcommand{\G}{\mathcal{G}}
\newcommand{\X}{\mathcal{X}}
%\newcommand{\ba}{\begin{align*}}
%\newcommand{\ea}{\end{align*}}
\DeclareMathOperator*{\argmax}{arg\,max}
\renewcommand{\qedsymbol}{$\blacksquare$}
\makeatletter
\def\BState{\State\hskip-\ALG@thistlm}
\makeatother

\makeatletter
\def\th@plain{%
  \thm@notefont{}% same as heading font
  \itshape % body font
}
\def\th@definition{%
  \thm@notefont{}% same as heading font
  \normalfont % body font
}
\makeatother
\date{}
%\title{Lecture 01: Poisson Processes}
%\author{}
\setcounter{chapter}{3}

\begin{document}
%\maketitle

\chapter{Continuous-Time Markov Chains}
\setcounter{section}{0}
\setcounter{subsection}{0}

\section*{\centering{Lecture 13 \linebreak }}
\section{Introduction}
Consider a continuous-time stochastic process $\{X(t),t\ge0\}$ taking values in a countable (can be finite) set $S$.
A process $\{X(t),t\ge0\}$ is a \textit{Continuous-Time Markov Chain} (CTMC) if for all $s,t \ge 0$, and $i,j,x(u) \in S, 0\le u\le s$,
\eq{
\P\{X(t+s)=j| X(s)=i, X(u)=x(u),0 \le u<s\} = \P\{X(t+s)=j|X(s)=i\}.
}
%In other words, a continuous-time Markov chain is a stochastic process having the Markovian property that the conditional distribution of the state at time $t+s$, given the present state $s$ and all past states depends only on the present state and is independent of the past. 
If in addition,
\eq{
P\{X(t+s)=j| X(s)=i\}  \triangleq P_t(i,j)
}
is independent of $s$, then the continuous-time Markov chain is said to have stationary transition probabilities. 
All the Markov chains we consider will be assumed to have stationary transition probabilities.

Further, we will restrict to pure jump processes: the sample paths of the process are piecewise constant, right continuous. We will see that such versions of the processes can usually be constructed.

By Markov property, for a pure jump process the sojourn time $T_i$ in state $i$ satisfies,
\eq{
P\{T_i>s+t | T_i>s\} = P\{T_i>t\}
}
for all $s,t \ge 0$.Hence, the random variable $T_i$ is memoryless and must thus be exponentially distributed, say with parameter $\lambda_i$.
If $\lambda_i = 0$ then
\eq{
P[T_i \ge t] = e^{-\lambda_i t} = 1
}
for all $t$ and the state $i$ is called \textit{absorbing}.
If $\lambda_i = \infty$ then
\eq{
	P[T_i \ge t] = 0
}
for all $t$ and the state $i$ is called \textit{instataneous}. We will assume $\lambda_i < \infty$ for all states $i$. For a pure jump process this will hold.

For a Markov jump process,
\begin{itemize}
\item the amount of time it spends in a state $i$ before making a transition into a different
state is exponentially distributed with mean, say, $\frac{1}{\lambda_i}$, and
\item when the process leaves state $i$, it next enters state $j$ with probability $P_{ij}$.
\\$P_{ij}$ satisfies, for $i$ not an absorbing state,
\eq{
P_{ii} = 0, \quad \sum_j P_{ij} = 1, \quad  \forall i,
}
\end{itemize}
and if $i$ is an absorbing state, $P_{ii}=1$.

In other words, a Continuous-Time Markov Chain (CTMC) is a stochastic process that
moves from state to state in accordance with a (discrete-time) Markov chain, but is such that the amount of time it spends in each state, before proceeding to the next state, is exponentially distributed independently of the next state visited. 


\section{Strong Markov property, Minimal construction}

%A random variable $\tau$ is a \textbf{stopping time} if for each $t \in \R_+$, 
%\eq{
%	\{\tau \leq t\} \in \F_t.
%}
A random variable $\tau$ is a stopping time with minimal construction if the event $\{\tau \leq t\}$ can be determined completely by the collection $\{X(u): u \leqslant t\}$. 
A stochastic process $X$ has \textbf{strong Markov property} if for any almost surely finite stopping time $\tau$, 
\eq{
	P\{X(\tau+s) = j|X(u), u \leq \tau, X(\tau) = i\} &= P\{X(\tau+s) = j|X(\tau) = i\} = P_s(i,j).
}
\begin{lem}
	\label{Lemma:StrongMarkovProperty}
	A continuous time jump Markov chain $X$ has the strong Markov property. 
\end{lem}
\begin{proof}

	Let $\tau$ be an almost surely finite stopping time with conditional distribution $F$ on the collection of events $\{X(u): u \leq s\}$. 
	Then, 
	\eq{
		\Pr\{X(\tau+s) =j |X(u), u \leq \tau, X_\tau = i\} &= \int_{0}^{\infty}dF(t)\Pr\{X(t+s) = j| X(u), u \leq t, \tau = t, X_\tau = i\}\\
		&= P_t(i,j)
	}
\end{proof}

We give a \textit{minimal construction} of a CTMC with given $\lambda_i, s$ and $P_{ij,s}$.
Construct a DTMC $Y_0, Y_1, \dots, $ with parameterse $P_{i,j}$ and construct exponential random variables $T_1, T_2, \dots$ independent of each other, where $T_n \sim exp(\lambda(Y_n))$.

Let $S_n = \sum_{k=1}^{n} T_k$ and $S_0 = 0$.

Define $X_t = Y_j$ if $S_j \le t < S_{j+1}$.
If $\omega(\Delta) \triangleq \sup_n S_n < \infty$, then $X_t = \Delta$ for $t \ge \omega(\Delta)$,
where $\Delta$ is an element which is not in the state space $S$.  
On the extended state space $S \cup \{\Delta\}$ we define $P_t(\Delta, \Delta) =1$ for all $t>0$ and $P(\Delta, \Delta) =1$.
There are other possibilies to define the MC after $\omega(\Delta)$ but the above construction makes $P_t(i,j)$, $i,j \in S$ minimal. When  $\omega(\Delta) < \infty$ , we say the MC has \textit{explosion}.

One can show that for any initial condition $i$,
\eq{
\R(\omega) =  \{\omega:\omega(\Delta) < \infty \} = \{ \omega:\sum_{k=0}^{\infty} \frac{1}{\lambda(Y_k(\omega))} < \infty\} \quad a.s
}
Easier to verify the conditions for non explosion of the MC are the following.

\begin{lem}
Any of the following conditions are sufficient for $P\{\omega:\omega(\Delta) < \infty \} = 0$:
\begin{enumerate}
	\item $\sup_i \lambda(i) < \infty$,
	\item $S$ is finite,
	\item $\{Y_n\}$ is recurrent.
\end{enumerate}
\end{lem}
\begin{proof}
	
\quad \quad  If $\omega(\Delta) < \infty$, then $\frac{1}{\lambda(Y_k(\omega))} \to 0$ as $k \to \infty$.

\quad \quad  If $(1)$ is true, $\lambda(Y_k(\omega)) \le \bar{\lambda}$.
therefore, $\lambda(Y_k(\omega)) \to \infty$ is not possible.

\quad \quad  If $(2)$ is true, since $\lambda(i) < \infty$ $\forall i$, \quad $\sup_i \lambda(i) < \infty$.

\quad \quad   If $(3)$ is true $Y_k(\omega) = i$ for an infinite number of $k$ $w.p. 1$.
%In $\lambda(Y_k(\omega))$, $\lambda(i)$ will occur infinitely often $w.p. 1$.
Therefore,
\eq{
\lambda(Y_k(\omega)) \centernot{\to} \infty, \quad a.s.,
}
\eq{
	P[\lambda(Y_k(\omega)) \to \infty] = 0.
}
\end{proof}


\end{document}

