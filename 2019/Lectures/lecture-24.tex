% !TEX spellcheck = en_US
% !TEX spellcheck = LaTeX
%\documentclass[a4paper,10pt,english]{article}
\documentclass[all-lectures.tex]{subfiles}
%\usepackage{%
	amsfonts,%
	amsmath,%	
	amssymb,%
	amsthm,%
%	babel,%
	bbm,%
	%biblatex,%
	caption,%
	centernot,%
	color,%
	enumerate,%
	epsfig,%
	epstopdf,%
	etex,%
	geometry,%
	graphicx,%
	hyperref,%
	latexsym,%
	mathtools,%
	multicol,%
	pgf,%
	pgfplots,%
	pgfplotstable,%
	pgfpages,%
	proof,%
	psfrag,%
	subfigure,%	
	tikz,%
	ulem,%
	url%
}	

\usepackage[mathscr]{eucal}
\usepgflibrary{shapes}
\usetikzlibrary{%
  arrows,%
  backgrounds,%
  chains,%
  decorations.pathmorphing,% /pgf/decoration/random steps | erste Graphik
  decorations.text,%
  matrix,%
  positioning,% wg. " of "
  fit,%
  patterns,%
  petri,%
  plotmarks,%
  scopes,%
  shadows,%
  shapes.misc,% wg. rounded rectangle
  shapes.arrows,%
  shapes.callouts,%
  shapes%
}

\theoremstyle{plain}
\newtheorem{thm}{Theorem}[section]
\newtheorem{lem}[thm]{Lemma}
\newtheorem{prop}[thm]{Proposition}
\newtheorem{cor}[thm]{Corollary}

\theoremstyle{definition}
\newtheorem{defn}[thm]{Definition}
\newtheorem{conj}[thm]{Conjecture}
\newtheorem{exmp}[thm]{Example}
\newtheorem{assum}[thm]{Assumptions}
\newtheorem{axiom}[thm]{Axiom}

\theoremstyle{remark}
\newtheorem{rem}[thm]{Remark}
\newtheorem{note}[thm]{Note}

\newcommand{\norm}[1]{\left\lVert#1\right\rVert}
\newcommand{\indep}{\!\perp\!\!\!\perp}
\DeclarePairedDelimiter\abs{\lvert}{\rvert}%
%\DeclarePairedDelimiter\norm{\lVert}{\rVert}%
\newcommand{\tr}{\operatorname{tr}}
\newcommand{\R}{\mathbb{R}}
\newcommand{\Q}{\mathbb{Q}}
\newcommand{\N}{\mathbb{N}}
\newcommand{\E}{\mathbb{E}}
\newcommand{\Z}{\mathbb{Z}}
\newcommand{\B}{\mathscr{B}}
\newcommand{\C}{\mathcal{C}}
\newcommand{\T}{\mathscr{T}}
\newcommand{\F}{\mathcal{F}}
\newcommand{\G}{\mathcal{G}}
%\newcommand{\ba}{\begin{align*}}
%\newcommand{\ea}{\end{align*}}

% Debug
\newcommand{\todo}[1]{\begin{color}{blue}{{\bf~[TODO:~#1]}}\end{color}}


\makeatletter
\def\th@plain{%
  \thm@notefont{}% same as heading font
  \itshape % body font
}
\def\th@definition{%
  \thm@notefont{}% same as heading font
  \normalfont % body font
}
\makeatother
\date{}
%\title{Lecture 01: Poisson Processes}
%\author{}
\setcounter{chapter}{6}

\begin{document}
%\maketitle

%\chapter{Renewal Theory}
\setcounter{section}{1}
\setcounter{subsection}{1}
\chapter{Queuing Theory}
\section*{\centering{Lecture 24 \linebreak }}
\subsection{General principles}
In this chapter we start with some general principles in queuing theory which are valid for many different queuing systems.\\ %{\color{red} Picture to be included.}\\

\begin{figure}
\centering
\includegraphics[scale=0.4]{Figures/queue_length_evolution.jpg}
\caption{Evolution of queue length}
\label{fig_queue_length_evolution}
\end{figure}
\textbf{GI/GI/1-FCFS queue:} 

Let $\{A_k\}$ i.i.d be the inter-arrival times and $\{s_k\}$ i.i.d be the service times. Let $q_t$ be the queue length at time $t$, $W_k$ be the waiting time of the $k^{th}$ customer. If $\mu = \E[X_1] = \E[s_1] - \E[A_1] < 0$, $q_t \to q_\infty$ in distribution as $t \to \infty$. Let $S_k$ be the sojourn time of the $k^{th}$ customer and $0$ be a regeneration epoch. Then, $W_k$ and the sojourn time $S_k = W_k + s_k$ also have stationary ditributions and have the same regeneration epochs. Then denoting by $\overline{\tau}$ and $\tau$ the regeneration lengths of $\{q_t\}$ and $\{s_k\}$, ( assuming $\E_\pi[q_\infty]$ and $\E_\pi[s_1]$ are finite, this requires $\E[s_1^2] < \infty$)
\begin{align*}
\E_{\pi}[q_\infty] &= \frac{\E\left[ \int_0^{\overline{\tau}} q_t dt\right]}{\E[\overline{\tau}]} \\
&= \frac{\E\left[ \sum_{k=0}^{\tau-1} S_k \right]}{\E[\tau]\E[A_1]} \\
&= \frac{ \E_{\pi}[S_1] }{\E[A_1]} \\
&= \lambda \E_{\pi}[S_1].
\end{align*}
where $\lambda$ is the arrival rate and $\E_{\pi}[ S_1]$ is the mean sojourn time under stationarity. Figure \ref{fig_queue_length_evolution} shows the evolution of queue length in one regeneration cycle. $\int_0^{\overline{\tau}} q_t dt$ is the area under the curve for $q(t)$. This is equal to  $\sum_{k=0}^{\tau-1} S_k $ as the break of the area in the Figure \ref{fig_queue_length_evolution} shows.\\
\indent The above is an example of a general result called \textit{Little's law}. 
$\E_{\pi}[Number\ in\ the\ system] = (Arrival\ rate) * \E_{\pi}[Sojourn \ time]$. This holds for a general queuing system with the same proof if $\int_0^\tau q_t dt = \sum_{k=0}^{\tau -1} S_k$ is valid in that system. We will see many examples of this in next few lectures.\\

\indent \textbf{$GI/GI/1$-Last Come First Serve (LCFS):} When a new custmorer arrives, the service of the current customer is stopped and servicing of the latest customer begins. After completion of a service, the server resumes service of the customer it was serving before to complete the remaining service. \\
\indent An example of this type of queueing is a stack in a computer system.

\textbf{Priority queues:} There are different classes of customers and each class is assigned a priority. The custmoer with the highest priority in the queue is served before others. 

All  the above schemes have an important property - work conservation:
\begin{enumerate}
\item The server is never idle when there is work in the system.
\item Workload will never be increased by policies and queuing schemes. 
\end{enumerate}
\indent Irrespective of the policy the queue becomes empty and gets an arrival to the empty queue, at the same time in all the work-conserving queues. Thus, the  regeneration epochs for $W_n$, $q_t$ or $V_t$ in the different queues remain same. Hence, $\E[\tau]< \infty$, $\E[\overline{\tau}] < \infty$ and has stationary distributions for all these processes for any of the work conserving policy if $\E[X_1] < 0$. But, the stationary distribution of $w$, $q$ etc. may be different for different queues. All these queuing systems satisfy Little's law.
%Let $\overline{\tau}_{LCFS}$ and $\overline{\tau}_{FCFS}$  be the regeneration length of workload process of LCFS and FCFS policies respectively. Then, $\overline{\tau}_{LCFS} = \overline{\tau}_{FCFS}$ a.s. Let $W_n^{LCFS}$ and $W_n^{FCFS}$ be the waiting time of LCFS and FCFS policies respectively. The regeneration lenght of $\{W_n^{LCFS}\}$ and $\{W_n^{FCFS}\}$ are equal a.s. But, $\E[W_n^{LCFS}]$ and $\E[W_n^{FCFS}]$ are not equal in general. 

%\subsection{Palm Theory goes to the next lecture}
%Consider a $GI/GI/1$ queue. Let $T_n$ be the $n^{th}$ regeneration epochs of $\{V_t\}$. Then $W_n = V_{T_n}$ a.s. But, $W_\infty \stackrel{d}{\neq} V_\infty$ in general. Also, in renewal process, we have seen inspection paradox where $X_{N(t)} \stackrel{d}{\neq} X_n$. This shows that distriubion of the process sampled at some points may be different from the distribution of the process.\\
%\indent Let  $\{X_t, -\infty<t<\infty\}$ be a stochastic process and $\tau = {\dots,\tau_{-1},\tau_{0}},\tau_{1},\dots$ be a point process. Let $N(t)$ be number of point of $\tau$ in the interval $(0,t]$. Let $Z = (X_t,\tau)$ and $\theta_t$ be the shift operator defined as $\theta_s Z = (\theta_t X,\theta \tau)$ where $(\theta_t X)_s = X_{t+s}$ and $(\theta_t \tau)_n = t+\tau_n$. $Z$ is a stationary process
\end{document}

