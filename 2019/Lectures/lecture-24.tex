% !TEX spellcheck = en_US
% !TEX spellcheck = LaTeX
%\documentclass[a4paper,10pt,english]{article}
\documentclass[all-lectures.tex]{subfiles}
%\usepackage{%
	amsfonts,%
	amsmath,%	
	amssymb,%
	amsthm,%
%	babel,%
	bbm,%
	%biblatex,%
	caption,%
	centernot,%
	color,%
	enumerate,%
	epsfig,%
	epstopdf,%
	etex,%
	geometry,%
	graphicx,%
	hyperref,%
	latexsym,%
	mathtools,%
	multicol,%
	pgf,%
	pgfplots,%
	pgfplotstable,%
	pgfpages,%
	proof,%
	psfrag,%
	subfigure,%	
	tikz,%
	ulem,%
	url%
}	

\usepackage[mathscr]{eucal}
\usepgflibrary{shapes}
\usetikzlibrary{%
  arrows,%
  backgrounds,%
  chains,%
  decorations.pathmorphing,% /pgf/decoration/random steps | erste Graphik
  decorations.text,%
  matrix,%
  positioning,% wg. " of "
  fit,%
  patterns,%
  petri,%
  plotmarks,%
  scopes,%
  shadows,%
  shapes.misc,% wg. rounded rectangle
  shapes.arrows,%
  shapes.callouts,%
  shapes%
}

\theoremstyle{plain}
\newtheorem{thm}{Theorem}[section]
\newtheorem{lem}[thm]{Lemma}
\newtheorem{prop}[thm]{Proposition}
\newtheorem{cor}[thm]{Corollary}

\theoremstyle{definition}
\newtheorem{defn}[thm]{Definition}
\newtheorem{conj}[thm]{Conjecture}
\newtheorem{exmp}[thm]{Example}
\newtheorem{assum}[thm]{Assumptions}
\newtheorem{axiom}[thm]{Axiom}

\theoremstyle{remark}
\newtheorem{rem}[thm]{Remark}
\newtheorem{note}[thm]{Note}

\newcommand{\norm}[1]{\left\lVert#1\right\rVert}
\newcommand{\indep}{\!\perp\!\!\!\perp}
\DeclarePairedDelimiter\abs{\lvert}{\rvert}%
%\DeclarePairedDelimiter\norm{\lVert}{\rVert}%
\newcommand{\tr}{\operatorname{tr}}
\newcommand{\R}{\mathbb{R}}
\newcommand{\Q}{\mathbb{Q}}
\newcommand{\N}{\mathbb{N}}
\newcommand{\E}{\mathbb{E}}
\newcommand{\Z}{\mathbb{Z}}
\newcommand{\B}{\mathscr{B}}
\newcommand{\C}{\mathcal{C}}
\newcommand{\T}{\mathscr{T}}
\newcommand{\F}{\mathcal{F}}
\newcommand{\G}{\mathcal{G}}
%\newcommand{\ba}{\begin{align*}}
%\newcommand{\ea}{\end{align*}}

% Debug
\newcommand{\todo}[1]{\begin{color}{blue}{{\bf~[TODO:~#1]}}\end{color}}


\makeatletter
\def\th@plain{%
  \thm@notefont{}% same as heading font
  \itshape % body font
}
\def\th@definition{%
  \thm@notefont{}% same as heading font
  \normalfont % body font
}
\makeatother
\date{}
%\title{Lecture 01: Poisson Processes}
%\author{}
\setcounter{chapter}{7}

\begin{document}
%\maketitle

%\chapter{Renewal Theory}
\setcounter{section}{1}
\setcounter{subsection}{1}
%\chapter{Queuing Theory}
\section*{\centering{Lecture 24 \linebreak }}
\section{Palm Theory, PASTA}
Consider a $GI/GI/1$ queue. Let $T_n$ be the $n^{th}$ arrival epoch, $\{V_t\}$ the workload process and $W_n$ the waiting time of the $n^{th}$ arrival. Then $W_n = V_{T_n^-}$ a.s. But, $W_\infty \stackrel{d}{\neq} V_\infty$ in general. Also, in renewal processes, we have seen inspection paradox where $X_{N(t)} \stackrel{d}{\neq} X_n$. This shows that the distribution of the process sampled at some random points may be different from the distribution of the process. In the following, we relate the two distributions.\\
\indent Let  $X = \{X_t, -\infty<t<\infty\}$ be a stochastic process and $T = \{\dots,T_{-1},T_{0},T_{1},\dots\}$ with $ \dots <T_{-1}<0\leq T_0 < T_1 < \dots $ be a point process. Let $N(t)$ be the number of points of $T$ in the interval $(0,t]$. Let $Z = (X,T)$ and $\theta_s$ be the shift operator defined as $\theta_s Z = (\theta_s X,\theta_s T)$ where $(\theta_s X)_t = X_{t+s}$ and $(\theta_s T)_n = T_{N(s)+ n} -s$. $Z$ is a stationary process if $\theta_s Z$ does not depend on $s$: $\P\{Z \in A\} = \P\{\theta_s Z \in A\}$ for all measurable $A$. \\
\indent Now, we ask the question if $Z$ is stationary, is $\{X_{T_k}\}$ stationary? If yes, when is the distribution of $\{X_{T_k}\}$ same as that of $\{X_t\}$? This is answered in the following theorem called \textit{Poisson Arrivals See Time Averages (PASTA)}. 
\begin{thm*}[PASTA]
If $X_t$ is right continuous, $\{X_s, s<t\}$ and $\{N_s - N_t, s\geq t\}$ are independent and $\{N_t\}$ is a Poisson process, $X_{T_k^-}$ is stationary and has the same distribution as $X_t$.
\end{thm*}
We will prove this theorem later in this lecture. Let us consider an application of PASTA to $\{V_t\}$ in $M/G/1$ queue. The arrival process is Poisson and the conditions of the theorem hold. Thus, $W_n$ and $V_t$ have the same distribution under stationarity. \\
\indent Consider the following quantity:
\begin{align*}
\lambda(t) = \frac{\E[N(t,t+h] ]}{h}.
\end{align*}
For a Poisson process $\lambda(t) = \lambda = $ rate of the Poisson process. 
\begin{prop}
If the process is stationary, $\lambda(t)$ does not depend on $t$ or $h$. 
\begin{proof}
Let $\phi(h) = \E[N(t,t+h] ]$. By stationarity, $\phi$ does not depend on $t$. We have 
\begin{align*}
\phi(h_1+h_2) &=  \E[N(t,t+h_1+h_2]] \\
&= \E[N(t,t+h_1]]  + \E[N(t+h_1,t+h_1+h_2]] \\
&= \phi(h_1) + \phi(h_2).
\end{align*}
This shows that $\phi$ is linear for a stationary process. So, $\phi(h) = h \phi(1)$. Thus, we have $\lambda(t) = h \E[N(t,t+1)]$.
\end{proof}
\end{prop}
With $h=1$, we can interpret $\lambda$ as the mean number of arrivals in unit time. Hence, $\lambda$ is called the \textit{intensity }of the process $N$.\\
\indent Define a probability measure $P_0$ as 
\begin{align*}
\P_0\{Z \in F\} = \frac{\E\left[\sum_{t <T_i \leq t+h} 1\{ \theta_{T_i} Z \in F\}\right]}{\lambda h}.
\end{align*}
The distribution $\P_0$  is called the \textit{Palm distribution} of $Z$. 
\begin{prop}
The process $\{Z\}$ under $\P_0$ is event stationary: $\P_0\{\theta_{T_1} Z \in F\} = \P_0\{Z \in F\}$. 
\begin{proof}
%We want to show $\P_0\{\theta_{T_1} Z \in F\} = \P_0\{Z \in F\}$. 
We have 
\begin{align*}
\P_0\{\theta_{T_1} Z \in F\} &= \frac{\E\left[\sum_{t <T_i \leq t+h} 1\{ \theta_{T_i} (\theta_{T_1} Z) \in F\}\right]}{\lambda h} \\
&= \frac{\E\left[\sum_{i=1}^{N(h)} 1\{ \theta_{T_{i+1}} Z \in F\}\right]}{\lambda h} \\
&\leq \frac{\E\left[\sum_{i=1}^{N(h)+1} 1\{ \theta_{T_i} Z \in F\}\right]}{\lambda h} \\
&\leq \frac{\E\left[\sum_{i=1}^{N(h)} 1\{ \theta_{T_i} Z \in F\}\right]}{\lambda h} +\frac{1}{\lambda h}\\
&= \P_0\{Z \in F\} + \frac{1}{\lambda h.}
\end{align*}
Letting $h \to \infty$, we get $\P_0\{\theta_{T_1} Z \in F\} \leq \P_0\{Z \in F\}$. Similarly, for $F^c$ we get, $\P_0\{\theta_{T_1} Z \in F^c\} \leq \P_0\{Z \in F^c\}$ which implies $\P_0\{\theta_{T_1} Z \in F\} \geq \P_0\{Z \in F\}$. Thus, we have the result. 
\end{proof}
\end{prop}
This proposition shows that under $\P_0$, the process $Z$ is event stationary. \\

\indent Next, define a probability measure $\P_1$ from $P_0$ as 
\begin{align*}
\P_1\{Z \in A\} = \frac{\E_0[\int_0^{T_k} 1\{\theta_t Z \in A\}]}{k E_0[T_1]},
\end{align*}
where $\E_0$ is the expectation with respect to $\P_0$. By event stationarity, this does not depend on $k$. If $\P_0$ is the Palm distribution of $Z$ under $\P$, then $\P_1 = \P$. This is called the \textit{Palm inversion formula}. This is generalization of the formula for regenerative processes. 
\begin{prop}
The process $Z$ is time stationary under $\P_1$.
\begin{proof}
We want to show $\P_1\{\theta_s Z\in A \} = \P_1\{Z \in A\}\ \forall s$.
\begin{align*}
\P_1\{\theta_s Z \in A \} &= \frac{\E_0[\int_0^{T_k} 1\{\theta_t (\theta_s Z) \in A\}]}{k \E_0[T_1]} \\
&= \frac{\E_0[\int_0^{T_k} 1\{\theta_{t+s} Z \in A\}]}{k \E_0[T_1]} \\
&= \frac{\E_0[\int_s^{T_k+s} 1\{\theta_t Z \in A\}]}{k \E_0[T_1]}  \\
&\leq \frac{\E_0[\int_0^{T_k+s} 1\{\theta_t Z \in A\}]}{k \E_0[T_1]}  \\
&\leq \frac{\E_0[\int_0^{T_k} 1\{\theta_t Z \in A\}]}{k \E_0[T_1]}  + \frac{s}{k \E_0[T_1]}. \\
\end{align*}
By taking $k\to \infty$, we get $\P_1\{\theta_s Z\in A \} \leq \P_1\{Z \in A\}$. Similarly for $A^c$, we get $\P_1\{\theta_s Z\in A^c \} \leq \P_1\{Z \in A^c\}$ which implies $\P_1\{\theta_s Z\in A \} \geq \P_1\{Z \in A\}$. This shows $\P_1\{\theta_s Z\in A \} = \P_1\{Z \in A\}$.
\end{proof}
\end{prop}
\begin{lem}
$\P_0\{T_0 =0\} = 1$.
\begin{proof} We have
\begin{align*}
\P_0\{T_0 =0\} &= \frac{\E[\sum_{0 < T_i \leq 1} 1\{(\theta_{T_i} T)_0 = 0\}]}{\lambda} \\
&= \frac{\E[N(1)] }{\lambda} && \text{ $(\theta_{T_i} T)_0 = 0$ always by definition of shift}\\
&= \frac{\lambda}{\lambda} = 1 && \qedhere 
\end{align*}
\end{proof}
\end{lem}
\begin{lem}
$\E_0[T_1] = \frac{1}{\lambda}$.
\begin{proof}
\begin{align*}
\lambda \E_0[T_1] &= \E\left[ \sum_{k=1}^{N(1)} (T_{k+1} - T_k)\right] \\
&= \E[ (T_{N(1)+1} - T_0) 1\{T_0 \leq 1\}] \\
&= 1 \qedhere
\end{align*}
\end{proof}
\end{lem}

\begin{lem}
\begin{align*}
\lim_{h \to 0} \frac{\P\{T_0 \leq h\}}{h} &= \lambda \\
\end{align*}
\begin{proof}
\begin{align*}
\vspace{-20mm}
\frac{\P\{T_0 \leq h\}}{h} &=  \frac{\E_0[ \int_{0}^{T_1} 1\{(\theta_t T)_0 \leq h\}]}{h \E_0[T_1]} \\
&= \lim_{h \downarrow 0} \frac{\lambda \E_0[ \min(T_1,h) ]}{h} \\
&= \lambda \qedhere
\end{align*}
\end{proof}
\end{lem}
We can also show that $\P\{T_1 \leq h\}/h \to 0$ as $h \downarrow 0$. Thus, 
\begin{align*}
\P_0\{Z \in F\} &= \lim_{h \downarrow 0} \frac{\E\left[ \sum_{0<T_i \leq h} 1\{\theta_{T_i} Z \in F\}\right]}{h \lambda} \\
&= \lim_{h \downarrow 0} \frac{\E\left[ 1\{\theta_{T_0} Z \in F\}; T_0 \leq h\right]}{\P\{T_0 \leq h\}} \\
&= \lim_{h \downarrow 0} \P\{ 1\{\theta_{T_0} Z \in F\}| T_0 \leq h\}
\end{align*}
This justifies the heuristic interpretation of $\P_0\{Z \in F\}$ as $\P\{Z \in F| T_0 = 0\}$.
\subsection{Rate conservation laws}
Let $\{X_t\}$ be a stochastic process whose sample paths have jumps at $\{\dots,T_{-2},T_{-1},T_0,T_1,T_2,\dots\}$ with intensity $\lambda$. The jump size at $T_k$ is $U_k$ ($U_k =0$ is allowed) and between the jumps, $\{X_t\}$ is differentiable with $dX_t/dt = Y_t$ (sample pathwise). The rate conservation law states that
\begin{prop} $\lambda \E_0\left[U_0 \right] + \E[Y_0] = 0$.
\begin{proof}
Sample pathwise 
\begin{align*}
X_1 - X_0 = \sum_{i: 0<T_i \leq 1} U_i + \int_0^1 Y_t dt.
\end{align*}
Taking expectation and by stationarity,
\begin{align*}
0 = \E[X_1]  - \E[X_0] = \lambda \E[U_0] + \E\left[\int_0^1 Y_t dt\right]
\end{align*}
Again, by stationarity of $Y_t$, $\E[\int_0^1 Y_t dt] = \E[Y_0]$.
\end{proof}
\end{prop}
\indent \textbf{Example: $GI/GI/1$ queue:} Let $q^A_n$ be the queue length seen by $n^{th}$ arrival and $q_n^D$ be the queue length left behind by $n^{th}$ departure. If $\E[A_1] > \E[s_1]$, then $\P\{q^A_\infty \leq k\} = \P\{q^D_\infty \leq  k\}$. This follows from rate conservation law as follows: Define $X_t = 1\{q_t \geq k\}$. We have $U_t = +1 \text{ or } -1$ and $Y_t = 0$ a.s. Therefore, $\lambda \E_0[U_0] + 0 = 0$. Thus, $\P_0\{U_0 = 1\} = \P_0\{U_0 = -1\}$, which implies $\P_0\{q^A_\infty \leq k \} = \P_0\{q^D_\infty \leq k\}$.

\subsection{PASTA}
\begin{thm}
Let $\{X_t\}$ be a stochastic process with $X_t \in \R^d$ and right continuous sample paths. Let $N_t$ be a Poisson process with rate $\lambda$ such that $\{X_s,s<t\}$ is independent of $\{N_s, s \geq t\}$ for all $s$ and $t$. Then, 
\begin{align*}
\text{(time stationary) }\P\{X_0 \in A\} = \P_0\{ X_{0^-} \in A\} \text{ (event stationary) }.
\end{align*}
\begin{proof}
\begin{align*}
\P_0\{ X_{0^-} \in A\}  &= \frac{\E\left[ \sum_{0 < T_i \leq 1} 1\{(\theta_{T_i} X)_{0^-} \in A\} \right]}{\lambda} \\
&= \frac{\E\left[ \int_0^1 1\{(\theta_t X)_{0^-} \in A; t \text{ is an event time}\} dt\right]}{\lambda} \\
&= \frac{\int_0^1 \P\{X_{t^-} \in A; t \text{ is an event time}\}dt}{\lambda} \\
&= \frac{\int_0^1 \P\{X_{t^-} \in A| t \text{ is an event time}\} \P\{ t \text{ is an event time}\}dt} {\lambda} \\
&= \frac{\int_0^1 \P\{X_{t^-} \in A\} \lambda dt} {\lambda} && \left(\parbox{3cm}{$X_{t^-}$ is independent \\ of $\{X_s, s\leq t\}$} \right)\\
&= \int_0^1 \P\{ X_{t} \in A\} \lambda dt && \left(\parbox{3cm}{right continuity and \\ stationarity of $\{X_t\}$} \right)  \\
&= \P\{X_{0} \in A\}. && \left(\text{stationarity of $\{X_t\}$}\right) \qedhere
\end{align*}
\end{proof}
\end{thm}
\end{document}
