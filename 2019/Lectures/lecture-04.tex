% !TEX spellcheck = en_US
% !TEX spellcheck = LaTeX
\documentclass[all-lectures.tex]{subfiles}
%\usepackage{%
	amsfonts,%
	amsmath,%	
	amssymb,%
	amsthm,%
%	babel,%
	bbm,%
	%biblatex,%
	caption,%
	centernot,%
	color,%
	enumerate,%
	epsfig,%
	epstopdf,%
	etex,%
	geometry,%
	graphicx,%
	hyperref,%
	latexsym,%
	mathtools,%
	multicol,%
	pgf,%
	pgfplots,%
	pgfplotstable,%
	pgfpages,%
	proof,%
	psfrag,%
	subfigure,%	
	tikz,%
	ulem,%
	url%
}	

\usepackage[mathscr]{eucal}
\usepgflibrary{shapes}
\usetikzlibrary{%
  arrows,%
  backgrounds,%
  chains,%
  decorations.pathmorphing,% /pgf/decoration/random steps | erste Graphik
  decorations.text,%
  matrix,%
  positioning,% wg. " of "
  fit,%
  patterns,%
  petri,%
  plotmarks,%
  scopes,%
  shadows,%
  shapes.misc,% wg. rounded rectangle
  shapes.arrows,%
  shapes.callouts,%
  shapes%
}

\theoremstyle{plain}
\newtheorem{thm}{Theorem}[section]
\newtheorem{lem}[thm]{Lemma}
\newtheorem{prop}[thm]{Proposition}
\newtheorem{cor}[thm]{Corollary}

\theoremstyle{definition}
\newtheorem{defn}[thm]{Definition}
\newtheorem{conj}[thm]{Conjecture}
\newtheorem{exmp}[thm]{Example}
\newtheorem{assum}[thm]{Assumptions}
\newtheorem{axiom}[thm]{Axiom}

\theoremstyle{remark}
\newtheorem{rem}[thm]{Remark}
\newtheorem{note}[thm]{Note}

\newcommand{\norm}[1]{\left\lVert#1\right\rVert}
\newcommand{\indep}{\!\perp\!\!\!\perp}
\DeclarePairedDelimiter\abs{\lvert}{\rvert}%
%\DeclarePairedDelimiter\norm{\lVert}{\rVert}%
\newcommand{\tr}{\operatorname{tr}}
\newcommand{\R}{\mathbb{R}}
\newcommand{\Q}{\mathbb{Q}}
\newcommand{\N}{\mathbb{N}}
\newcommand{\E}{\mathbb{E}}
\newcommand{\Z}{\mathbb{Z}}
\newcommand{\B}{\mathscr{B}}
\newcommand{\C}{\mathcal{C}}
\newcommand{\T}{\mathscr{T}}
\newcommand{\F}{\mathcal{F}}
\newcommand{\G}{\mathcal{G}}
%\newcommand{\ba}{\begin{align*}}
%\newcommand{\ea}{\end{align*}}

% Debug
\newcommand{\todo}[1]{\begin{color}{blue}{{\bf~[TODO:~#1]}}\end{color}}


\makeatletter
\def\th@plain{%
  \thm@notefont{}% same as heading font
  \itshape % body font
}
\def\th@definition{%
  \thm@notefont{}% same as heading font
  \normalfont % body font
}
\makeatother
\date{}
\author{}


\begin{document}
%\maketitle
\setcounter{chapter}{1}
\chapter{Renewal Processes}

\setcounter{section}{1}
\section*{\centering{Lecture 4 \linebreak }}
\subsection{Renewal Process}
We know that the interarrival times for the Poisson process are independent and identically distributed exponential random variables.
If we consider a counting process for which the interarrival times are independent and identically distributed with an arbitrary distribution function, then the counting process is called a \textit{renewal process}.

Let $X_n$ be the time between the $(n-1)$th and $n$th event and $\{X_n,n=1,2,\dots\}$ be a sequence of nonnegative independent random varibales with common distribution $F$ and $X_n \ge 0$.

The mean time $\mu$ between successive events is given by
\eq{
\mu = \E{X_n} = \int_0^\infty x dF(x).
}
We take $\mu > 0$. Let $S_0=0$ and $S_n = \sum_{i=1}^{n} X_i$ for $n \ge 1$, $S_n$ indicating the time of $n^{th}$ event.
The number of events by time $t$, is given by
\eq{
N(t) = \sup \{n:S_n \le t\}.
}

%In general finding $\E[{N(t)}]$ may not be easy.
\begin{defn}
The counting process $\{ N(t),t \ge 0\}$ is called a \textit{renewal process}.
\end{defn}

Note that the number of renewals by time $t$ is greater than or equal to $n$ if, and only if, the $n$th renewal occurs before or at time $t$.
That is,
\eq{
N(t) \ge n \Leftrightarrow S_n \le t.
}

The distribution of $N(t)$ can be written as $\P\{N(t) \ge n\} = \P\{S_n \le t\}$ and from this we can write $\P\{N(t)=n\}$ as follows,
\eq{
\P\{N(t)=n\} &= \P\{N(t) \ge n\}-\P\{N(t) \ge n+1\}\\
&= \P\{S_n \le t\}-\P\{S_{n+1} \le t\}.
}


\begin{prop}
By strong law of large numbers $\frac{S_n}{n} \to \mu$ as $n \to \infty$ with probability 1. Hence $S_n \rightarrow \infty$ a.s.
\end{prop}

\textit{Question}: Can an infinite number of renewals can occur in a finite time?

\begin{prop}
Infinite number of renewals cannot happen in a finite time.
\end{prop}
\begin{proof}
For all $t$, we can write the following limit as
\eq{
\lim_{n \to \infty} \P\{N(t)=n\} &= \lim_{n \to \infty} \P\{S_n \le t\}-\P\{S_{n+1} \le t\} \\
&= 0.
}
Also, 
\eq{
\P\{N(t) < \infty\} &= 1-\lim_{n \to \infty} \P\{N(t) \ge n\} \\
&=  1-\lim_{n \to \infty} \P\{S_n \le t\}\\
&= 1.\qedhere
}
\end{proof}
\begin{prop}
$\E[N^r(t)] < \infty$ for $r > 0, t \ge 0$.
\end{prop}
\begin{proof}
Construct a new process $X'_k$ from $X_k$ as follows,
\eq{
X'_k = \begin{cases}
0, 		& X_k < \alpha\\
\alpha,  & X_k \ge \alpha
\end{cases}
}
where $A_k$ are the interarrival times of the original process. Let $\beta = \P\{X_1\geq \alpha\}$.\\
\indent Suppose $\{N'(t)\}$ is constructed from $\{X'_k\}$, then it is clear that $X'_k \le X_k$ and $\{N'(t) \ge N(t)\}$.
Then, 
\eq{
\P\{ N'(\frac{\alpha}{2}) = n\} &= \P\{X_1 < \alpha\} \P\{X_2 < \alpha\} \dots \P\{X_{n-1} < \alpha\} \P\{X_n > \alpha\}\\
&\le (1-\beta)^n \beta.
}
Thus,
\eq{
\E[(N'(\frac{\alpha}{2}))^r] &= \sum_{n=0}^{\infty} n^r \P\{ N'_{\frac{\alpha}{2}} = n\}\\
&\le \sum_{n=0}^{\infty} n^r (1-\beta)^n \beta < \infty.
}
\eq{
\E[(N'(t))^r] &= \E \big{[}\sum_{k=1}^{\frac{t}{\alpha}+1} (\overline{N}_k)^r \big{]} \\
&\leq \left( \frac{t}{\alpha} +1\right)^r \E[N'(\frac{\alpha}{2})^r] < \infty. 
}
where $\overline{N}_{\alpha k} = N'_k-N'_{\alpha(k-1)}$. 
\end{proof}

\end{document}

