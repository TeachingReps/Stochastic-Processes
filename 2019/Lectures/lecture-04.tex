% !TEX spellcheck = en_US
% !TEX spellcheck = LaTeX
\documentclass[all-lectures.tex]{subfiles}
%\usepackage{%
	amsfonts,%
	amsmath,%	
	amssymb,%
	amsthm,%
%	babel,%
	bbm,%
	%biblatex,%
	caption,%
	centernot,%
	color,%
	enumerate,%
	epsfig,%
	epstopdf,%
	etex,%
	geometry,%
	graphicx,%
	hyperref,%
	latexsym,%
	mathtools,%
	multicol,%
	pgf,%
	pgfplots,%
	pgfplotstable,%
	pgfpages,%
	proof,%
	psfrag,%
	subfigure,%	
	tikz,%
	ulem,%
	url%
}	

\usepackage[mathscr]{eucal}
\usepgflibrary{shapes}
\usetikzlibrary{%
  arrows,%
  backgrounds,%
  chains,%
  decorations.pathmorphing,% /pgf/decoration/random steps | erste Graphik
  decorations.text,%
  matrix,%
  positioning,% wg. " of "
  fit,%
  patterns,%
  petri,%
  plotmarks,%
  scopes,%
  shadows,%
  shapes.misc,% wg. rounded rectangle
  shapes.arrows,%
  shapes.callouts,%
  shapes%
}

\theoremstyle{plain}
\newtheorem{thm}{Theorem}[section]
\newtheorem{lem}[thm]{Lemma}
\newtheorem{prop}[thm]{Proposition}
\newtheorem{cor}[thm]{Corollary}

\theoremstyle{definition}
\newtheorem{defn}[thm]{Definition}
\newtheorem{conj}[thm]{Conjecture}
\newtheorem{exmp}[thm]{Example}
\newtheorem{assum}[thm]{Assumptions}
\newtheorem{axiom}[thm]{Axiom}

\theoremstyle{remark}
\newtheorem{rem}[thm]{Remark}
\newtheorem{note}[thm]{Note}

\newcommand{\norm}[1]{\left\lVert#1\right\rVert}
\newcommand{\indep}{\!\perp\!\!\!\perp}
\DeclarePairedDelimiter\abs{\lvert}{\rvert}%
%\DeclarePairedDelimiter\norm{\lVert}{\rVert}%
\newcommand{\tr}{\operatorname{tr}}
\newcommand{\R}{\mathbb{R}}
\newcommand{\Q}{\mathbb{Q}}
\newcommand{\N}{\mathbb{N}}
\newcommand{\E}{\mathbb{E}}
\newcommand{\Z}{\mathbb{Z}}
\newcommand{\B}{\mathscr{B}}
\newcommand{\C}{\mathcal{C}}
\newcommand{\T}{\mathscr{T}}
\newcommand{\F}{\mathcal{F}}
\newcommand{\G}{\mathcal{G}}
%\newcommand{\ba}{\begin{align*}}
%\newcommand{\ea}{\end{align*}}

% Debug
\newcommand{\todo}[1]{\begin{color}{blue}{{\bf~[TODO:~#1]}}\end{color}}


\makeatletter
\def\th@plain{%
  \thm@notefont{}% same as heading font
  \itshape % body font
}
\def\th@definition{%
  \thm@notefont{}% same as heading font
  \normalfont % body font
}
\makeatother
\date{}
\author{}


\begin{document}
%\maketitle
\setcounter{chapter}{1}
\chapter{Renewal Processes}

\setcounter{section}{1}
\section*{\centering{Lecture 4 \linebreak }}
\subsection{Renewal Process}
We know that the interarrival times for the Poisson process are independent and identically distributed exponential random variables.
If we consider a counting process for which the interarrival times are independent and identically distributed with an arbitary function, then the counting process is called a \textit{renewal process}.

Let $X_n$ be the time between the $(n-1)$th and $n$th event and $\{X_n,n=1,2,\dots\}$ be a sequence of nonnegative independent random varibales with common distribution $F$ and $X_n \ge 0$.

The mean time $\mu$ between successive events is given by
\eq{
\mu = \E{X_n} = \int_0^\infty x dF(x).
}


Let $S_0=0$ and $S_n = \sum_{i=1}^{n} X_i$ for $n \ge 1$, $S_n$ indicating the time of $n$th event.
The number of events by time $t$, is given by
\eq{
N(t) = \sup \{n:S_n \le t\}.
}

%In general finding $\E[{N(t)}]$ may not be easy.
\begin{defn}
The counting process $\{ N(t),t \ge 0\}$ is called a \textit{renewal process}.
\end{defn}

Note that the number of renewals by time $t$ is greater than or equal to $n$ if, and only if, the $n$th renewal occurs before or at time $t$.
That is,
\eq{
N(t) \ge n \Leftrightarrow S_n \le t.
}

The distribution of $N(t)$ can be written as $\P\{N(t) \ge n\} = \P\{S_n \le t\}$ and from this we can write $\P\{N(t)=n\}$ as follows,
\eq{
\P\{N(t)=n\} &= \P\{N(t) \ge n\}-\P\{N(t) \ge n+1\}\\
&= \P\{S_n \le t\}-\P\{S_{n+1} \le t\}.
}


\begin{prop}
By strong law of large numbers $\frac{S_n}{n} \to \mu$ as $n \to \infty$ with probabiity 1.
\end{prop}


\textit{Question}: Can an infinite number of renewals can occur in a finite time?

\begin{prop}
Infinite number of renewals cannot happen in a finite time.
\end{prop}
\begin{proof}
For all $t$, we can write the follwing limit as
\eq{
\lim_{n \to \infty} \P\{N(t)=n\} &= \lim_{n \to \infty} \P\{S_n \le t\}-\P\{S_{n+1} \le t\} \\
&= 0.
}
Now let us answer the question whether an infinite number of renewals can occur in a finite time or not,
\eq{
\P\{N_t < \infty\} &= 1-\lim_{n \to \infty} \P\{N(t) \ge n\} \\
&=  1-\lim_{n \to \infty} \P\{S_n \le t\}\\
&= 1.
}
That is, infinite number of renewals cannot happen in a finite time, and the number of renewals is given by,
\eq{
N(t)=\max \{ n, S_n \le t\}.
}
\end{proof}

\begin{prop}
Not only $\E[N(t)]$ is finite but also all the moments $\E[N^r(t)]$ are finite for $r > 0, t \ge 0$.
\end{prop}

\begin{proof}
Construct a new process $A'_k$ form $A_k$ as follows,
\eq{
A'_k = \begin{cases}
0, 		& A_k < \alpha\\
\alpha,  & A_k \ge \alpha
\end{cases}
}
where $A_k$ are the interarrival times of the original process.

Suppose $\{N'(t)\}$ is constructed from $\{A'_k\}$, then it is clear that $A'_k \le A_k$ and $\{N'(t) \ge N(t)\}$.
The probability $\P\{ N'_{\frac{\alpha}{2}} = n\}$ is given by
\eq{
\P\{ N'_{\frac{\alpha}{2}} = n\} &= \P\{A_1 < \alpha\} \P\{A_2 < \alpha\} \dots \P\{A_{n-1} < \alpha\} \P\{A_n > \alpha\}\\
&\le (1-\beta)^n \beta.
}
From this we can write the expected value of $N'_{\frac{\alpha}{2}})^r$ as,
\eq{
\E[(N'_{\frac{\alpha}{2}})^r] &= \sum_{n=0}^{\infty} n^r \P\{ N'_{\frac{\alpha}{2}} = n\}\\
&\le \sum_{n=0}^{\infty} n^r (1-\beta)^n \beta
}
that is $\E[(N'_{\frac{\alpha}{2}})^r]$ is finite.
\eq{
\E[(N'_t)^r] = \E \big{[}\sum_{k=1}^{\frac{t}{\alpha}+1} (\bar{N_k})^r \big{]}
}
where the value of $\bar{N_k}$ is equal to $N'_k-N'_{k-1}$.Furhter it can be upper bounded as follows
\eq{
\E[(N'_t)^r] \le (\frac{t}{\alpha}+1)^r \E[(N'_1)^r]
}
Since both the terms in the right hand side are finite we can conclude that $\E[(N'_t)^r] < \infty$. Hence the proof.
\end{proof}

\end{document}

