% !TEX spellcheck = en_US
% !TEX spellcheck = LaTeX
%\documentclass[a4paper,10pt,english]{article}
\documentclass[all-lectures.tex]{subfiles}
%\usepackage{%
	amsfonts,%
	amsmath,%	
	amssymb,%
	amsthm,%
%	babel,%
	bbm,%
	%biblatex,%
	caption,%
	centernot,%
	color,%
	enumerate,%
	epsfig,%
	epstopdf,%
	etex,%
	geometry,%
	graphicx,%
	hyperref,%
	latexsym,%
	mathtools,%
	multicol,%
	pgf,%
	pgfplots,%
	pgfplotstable,%
	pgfpages,%
	proof,%
	psfrag,%
	subfigure,%	
	tikz,%
	ulem,%
	url%
}	

\usepackage[mathscr]{eucal}
\usepgflibrary{shapes}
\usetikzlibrary{%
  arrows,%
  backgrounds,%
  chains,%
  decorations.pathmorphing,% /pgf/decoration/random steps | erste Graphik
  decorations.text,%
  matrix,%
  positioning,% wg. " of "
  fit,%
  patterns,%
  petri,%
  plotmarks,%
  scopes,%
  shadows,%
  shapes.misc,% wg. rounded rectangle
  shapes.arrows,%
  shapes.callouts,%
  shapes%
}

\theoremstyle{plain}
\newtheorem{thm}{Theorem}[section]
\newtheorem{lem}[thm]{Lemma}
\newtheorem{prop}[thm]{Proposition}
\newtheorem{cor}[thm]{Corollary}

\theoremstyle{definition}
\newtheorem{defn}[thm]{Definition}
\newtheorem{conj}[thm]{Conjecture}
\newtheorem{exmp}[thm]{Example}
\newtheorem{assum}[thm]{Assumptions}
\newtheorem{axiom}[thm]{Axiom}

\theoremstyle{remark}
\newtheorem{rem}[thm]{Remark}
\newtheorem{note}[thm]{Note}

\newcommand{\norm}[1]{\left\lVert#1\right\rVert}
\newcommand{\indep}{\!\perp\!\!\!\perp}
\DeclarePairedDelimiter\abs{\lvert}{\rvert}%
%\DeclarePairedDelimiter\norm{\lVert}{\rVert}%
\newcommand{\tr}{\operatorname{tr}}
\newcommand{\R}{\mathbb{R}}
\newcommand{\Q}{\mathbb{Q}}
\newcommand{\N}{\mathbb{N}}
\newcommand{\E}{\mathbb{E}}
\newcommand{\Z}{\mathbb{Z}}
\newcommand{\B}{\mathscr{B}}
\newcommand{\C}{\mathcal{C}}
\newcommand{\T}{\mathscr{T}}
\newcommand{\F}{\mathcal{F}}
\newcommand{\G}{\mathcal{G}}
%\newcommand{\ba}{\begin{align*}}
%\newcommand{\ea}{\end{align*}}

% Debug
\newcommand{\todo}[1]{\begin{color}{blue}{{\bf~[TODO:~#1]}}\end{color}}


\makeatletter
\def\th@plain{%
  \thm@notefont{}% same as heading font
  \itshape % body font
}
\def\th@definition{%
  \thm@notefont{}% same as heading font
  \normalfont % body font
}
\makeatother
\date{}
%\title{Lecture 01: Poisson Processes}
%\author{}
\setcounter{chapter}{3}

\begin{document}
%\maketitle

%\chapter{Discrete Time Markov Chains}
\setcounter{section}{4}
\setcounter{subsection}{0}

\section*{\centering{Lecture 11 \linebreak }}
\subsection{Criterion for transience, null recurrence and positive recurrence}
\begin{thm} \label{prop:markov_chain_class_tests}
Let $S$ be irreducible and $f: S \rightarrow \R$. 
\begin{enumerate}
\item Let $f(i)\rightarrow \infty$ as $i\rightarrow \infty$. If $\E[f(X_1)|X_0 = i] \leq f(i)$ for all $i$ outside a finite set $S_0 \subset S$, then the Markov chain is recurrent. 
\item Let $f:S\rightarrow \R^+$ and $S_0 \subset S$ be finite. If 
	\begin{enumerate}
	\item  $\E[f(X_1)|X_0 = i] < \infty \ \forall i$,
	\item For $i \notin S_0$, $\E[f(X_1)|X_0 = i] - f(i) < -\epsilon$ for some $\epsilon > 0$
	\end{enumerate}
	 the Markov chain is positive recurrent. 
\item Let $f:S\rightarrow \R$ and $S_0 \subset S$ be finite.	The Markov chain is transient if
	\begin{enumerate}
	\item  $f$ is bounded and $\E[f(X_1)|X_0= i] \geq f(i)\ \forall i \in S_0$
	\item $f(i) > f(j)\ \forall j\in S_0$, for some $i\notin S_0$
	\end{enumerate}	
\end{enumerate}
\end{thm}
\indent The proof of this theorem requires Martingale methods. Thus we will prove it after we have studied Martingales. \\
\indent \textbf{Example:} Consider a slotted queuing system in which one service time is equal to one slot. Let $q_k$ be the queue length at the end of $k^{th}$ slot. Let $A_k$ denote the number of arrivals in the $k^{th}$ slot. Then, $q_{k+1} = (q_k -1)^+ + A_{k}$ and for $i>0$
\begin{align*}
\P\{q_{k+1} = j|q_k = i,q_{k-1} = i_{k-1}, \dots q_0 = i_0\} = \P\{A_k = j-i-1\}.
\end{align*}
Thus, $\{q_k\}$ is a Markov chain with state space $S = \{0,1,2,\dots\}$. If $\P\{A_1 \leq 1\} > 0$, every state is aperiodic. If further, $\P\{A_1 > 1\} >0$, the Markov chain is also irreducible. \\
\indent Consider $f(i) = i$. Let $S_0 = \{0\}$. For $i> 0$, $\E[f(q_1)|q_0 = i] - f(i) = \E[A_1] -1 $. Thus, we see from case $(1)$ in Theorem \ref{prop:markov_chain_class_tests}, if $\E[A_1] \leq 1 $, $\{q_k\}$ is recurrent. From $(2)$ in Theorem\ref{prop:markov_chain_class_tests}, we have positive recurrence if $\E[A_k] < 1$. \\
%\indent Choose $f(i) = 1\{i>0\}$ and $S_0 = \{0\}$. Then if $\E[A_1] > 1$, the conditions in case (3) in Prop \ref{prop:markov_chain_class_tests} is met and $\{q_k\}$ is transient Markov chain. \\
\subsection{Reversible Markov Chains}
We can easily check that 
\begin{align*}
\P\{X_{k-1} = j| X_k = i, X_{k+1} = i_{k+1}, X_{k+2} = i_{k+2}, \dots\} = \P\{X_{k-1} = j|X_k = i\}.
\end{align*}
Thus, the reversed Markov chain is also a Markov chain. For an irreducible  stationary Markov chain $\{X_k\}$ with stationary distribution $\pi$,
\begin{align*}
\P\{X_{k-1} = j| X_k = i,X_{k+1} = i_{k+1},X_{k+2} = i_{k+2},\dots\} &= \P\{X_{k-1} = j| X_k = i\} \\
&=  \frac{\P\{X_{k-1} = j, X_k = i\}}{\P\{ X_k = i\}} \\
&=  \frac{\P\{ X_k = i|X_{k-1} = j\}\P\{ X_{k-1} = j\}}{\pi(i)} \\
&=  \frac{P(j,i)\pi(j)}{\pi(i)}
\end{align*}
Let us define 
\begin{align*}
P^*(i,j)=\frac{P(j,i)\pi(j)}{\pi(i)}
\end{align*}
$P^*$ is the transition probability of the reversed Markov chain. \\
\indent \textbf{Reversible Markov Chain:} The stationary irreducible Markov chain $X_k$ is called \textit{reversible }if $P^*(i,j) = P(i,j)$. In the other words, for a reversible  Markov chain, we have, $P(i,j)\pi(i) = P(j,i)\pi(j)$.
\begin{prop}[Test for reversibility]
For an irreducible Markov chain with stationary distribution $\pi$,  for all paths $i\rightarrow i_1 \rightarrow i_2 \dots \rightarrow i_k \rightarrow i$,
\begin{align*}
\P_\pi\{i\rightarrow i_1 \rightarrow i_2 \dots \rightarrow i_k \rightarrow i\} = \P_\pi\{i \rightarrow i_k \rightarrow i_{k-1} \dots \rightarrow i_1 \rightarrow i\}
\end{align*}
under stationarity is a necessary and sufficient condition for reversibility of the Markov chain. 
\begin{proof}
Necessity: Assume $P^* = P$. Then, 
\begin{align*}
&\P_\pi\{i\rightarrow i_1 \rightarrow i_2 \dots \rightarrow i_k \rightarrow i\} \\
&= \pi(i)P(i,i_1)P(i_1,i_2)\dots P(i_{k-1},i_{k})P(i_k,i)\\
& = P(i_1,i)\pi(i_i) P(i_1,i_2)\dots P(i_{k-1},i_{k})P(i_k,i) \\
& = P(i_1,i)P(i_2,i_1)\pi(i_2) \dots P(i_{k-1},i_{k})P(i_k,i) \\
\dots \\
& = P(i_1,i)P(i_2,i_1) \dots P(i_{k},i_{k-1})\pi(i_k)P(i_k,i) \\
& = P(i_1,i)P(i_2,i_1) \dots P(i_{k},i_{k-1})P(i,i_k)\pi(i) \\
&= \P_\pi\{i \rightarrow i_k \rightarrow i_{k-1} \dots \rightarrow i_1 \rightarrow i\}.
\end{align*}
Sufficiency: Consider the path $i\rightarrow i_1 \rightarrow i_2 \dots \rightarrow i_k \rightarrow j\rightarrow i$ and its reverse path. Then, 
\begin{align*}
\sum_{i_1,i_2\dots i_k}\P_\pi\{i\rightarrow i_1 \rightarrow i_2 \dots \rightarrow i_k \rightarrow j \rightarrow i\} &= \sum_{i_1,i_2\dots i_k}\P_\pi\{i \rightarrow j \rightarrow i_k \rightarrow i_{k-1} \dots \rightarrow i_1 \rightarrow i\} 
\end{align*}
Thus, 
\begin{align*}
P^{k}(i,j)P(j,i) &= P(i,j)P^{k}(j,i)
\end{align*}
Now, taking the limit as $k\rightarrow \infty$, we get $\pi(j)P(j,i) = P(i,j)\pi(i)$, which is $P^* = P$.
\end{proof}
\end{prop}
\end{document}

