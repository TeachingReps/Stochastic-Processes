% !TEX spellcheck = en_US
% !TEX spellcheck = LaTeX
%\documentclass[a4paper,10pt,english]{article}
\documentclass[all-lectures.tex]{subfiles}
%\usepackage{%
	amsfonts,%
	amsmath,%	
	amssymb,%
	amsthm,%
%	babel,%
	bbm,%
	%biblatex,%
	caption,%
	centernot,%
	color,%
	enumerate,%
	epsfig,%
	epstopdf,%
	etex,%
	geometry,%
	graphicx,%
	hyperref,%
	latexsym,%
	mathtools,%
	multicol,%
	pgf,%
	pgfplots,%
	pgfplotstable,%
	pgfpages,%
	proof,%
	psfrag,%
	subfigure,%	
	tikz,%
	ulem,%
	url%
}	

\usepackage[mathscr]{eucal}
\usepgflibrary{shapes}
\usetikzlibrary{%
  arrows,%
  backgrounds,%
  chains,%
  decorations.pathmorphing,% /pgf/decoration/random steps | erste Graphik
  decorations.text,%
  matrix,%
  positioning,% wg. " of "
  fit,%
  patterns,%
  petri,%
  plotmarks,%
  scopes,%
  shadows,%
  shapes.misc,% wg. rounded rectangle
  shapes.arrows,%
  shapes.callouts,%
  shapes%
}

\theoremstyle{plain}
\newtheorem{thm}{Theorem}[section]
\newtheorem{lem}[thm]{Lemma}
\newtheorem{prop}[thm]{Proposition}
\newtheorem{cor}[thm]{Corollary}

\theoremstyle{definition}
\newtheorem{defn}[thm]{Definition}
\newtheorem{conj}[thm]{Conjecture}
\newtheorem{exmp}[thm]{Example}
\newtheorem{assum}[thm]{Assumptions}
\newtheorem{axiom}[thm]{Axiom}

\theoremstyle{remark}
\newtheorem{rem}[thm]{Remark}
\newtheorem{note}[thm]{Note}

\newcommand{\norm}[1]{\left\lVert#1\right\rVert}
\newcommand{\indep}{\!\perp\!\!\!\perp}
\DeclarePairedDelimiter\abs{\lvert}{\rvert}%
%\DeclarePairedDelimiter\norm{\lVert}{\rVert}%
\newcommand{\tr}{\operatorname{tr}}
\newcommand{\R}{\mathbb{R}}
\newcommand{\Q}{\mathbb{Q}}
\newcommand{\N}{\mathbb{N}}
\newcommand{\E}{\mathbb{E}}
\newcommand{\Z}{\mathbb{Z}}
\newcommand{\B}{\mathscr{B}}
\newcommand{\C}{\mathcal{C}}
\newcommand{\T}{\mathscr{T}}
\newcommand{\F}{\mathcal{F}}
\newcommand{\G}{\mathcal{G}}
%\newcommand{\ba}{\begin{align*}}
%\newcommand{\ea}{\end{align*}}

% Debug
\newcommand{\todo}[1]{\begin{color}{blue}{{\bf~[TODO:~#1]}}\end{color}}


\makeatletter
\def\th@plain{%
  \thm@notefont{}% same as heading font
  \itshape % body font
}
\def\th@definition{%
  \thm@notefont{}% same as heading font
  \normalfont % body font
}
\makeatother
\date{}
%\title{Lecture 01: Poisson Processes}
%\author{}
\setcounter{chapter}{5}

\begin{document}
%\maketitle

%\chapter{Martingales}
\setcounter{section}{1}
\setcounter{subsection}{1}

\section*{\centering{Lecture 18 \linebreak }}

\section{(Contd.) Optional Sampling Theorem:Example}
\begin{exmp}
Let $Y_0 = 0, Y_1, Y_2, \dots$ $i.i.d.$, $\E[|Y_1|] < \infty$ , $\E[Y_i]= 0$,
$S_0 = 0$, $S_n = \sum_{i=1}^{n} Y_i$, $T$ stopping time w.r.t. $\{Y_0, Y_1, Y_2. \dots\}$, and  $\E[T]< \infty$. Then

\eq{
\E[|S_{n+1}-S_n|\F_n] = \E[|Y_{n+1}|\F_n] = \E[|Y_{n+1}|] < \infty.
}
hence from previous optional sampling theorem
\eq{
\E[S_T] = \E[S_0].
}
\end{exmp}

\section{Martingale inequalities}
\begin{thm}(Doob's inequality for submatingales)
If $\{X_n\}$ is a submartingale, $M_n = sup_{0 \le k \le n} X_k$, then for $\alpha>0$
\eq{
\P[M_n \ge \alpha] \le \frac{\E[X_n]}{\alpha}.
}
\end{thm}
\begin{proof}
Let $T$ be a stopping time defined as $T= n \wedge inf\{k:M_k \ge \alpha\}$.
\eq{
\E[X_n] &= \sum_{k=0}^{n} \E[X_n 1_{\{T=k\}}]\\
&= \sum_{k=0}^{n} \E[\E[X_n 1_{\{T=k\}}|\F_k]]\\
&= \sum_{k=0}^{n} \E[1_{\{T=k\}}\E[X_n|\F_k]]\\
&\ge \sum_{k=0}^{n} \E[1_{\{T=k\}} X_k]\\
&= \E[X_T].
}
Thus,
\eq{
P\{M_n \ge \alpha\} = P\{X_T \ge \alpha\} \le \frac{\E[X_T]}{\alpha} \le  \frac{\E[X_n]}{\alpha}.
}
\end{proof}
If  $\{X_k\}$ is a martingale, $\E[|X_k|^\beta] < \infty$, $\forall k$, for some $\beta \ge 1$, then $\{|X_k|^\beta\}$ is also a submartingale. Thus from the above theorem
\eq{
	P\left[\sup_{1 \le k \le n} |X_k| \ge \alpha \right] = P\left[\sup_{1 \le k \le n} |X_k|^\beta \ge \alpha^\beta \right] \le \frac{\E[|X_n|^\beta]}{\alpha^\beta}
}

\begin{exmp}
$Y_0, Y_1, Y_2, \dots$, $\E[Y_1] = 0$, $\E[|Y_1| < \infty]$. Then $S_n$ is a martingale. Thus, from above theorem
\eq{
P[\sup_{0 \le k \le n} |S_k| \ge \alpha] \le \frac{\E[|S_n|]}{\alpha}.
}
If $\E[|Y_1|^\beta] < \infty$, $\beta \ge 1$  , then  $P[\sup_{0 \le k \le n} |S_k| \ge \alpha]  \le \frac{\E[|S_n|^\beta]}{\alpha^\beta}$.\\
For $\beta = 2$, 
\eq{
\frac{\E[|S_n|^2]}{\alpha^2} = \frac{n \sigma^2}{\alpha^2} \quad \text{where} \quad  \sigma^2 = var(Y_1).
}
This is called \textit{Kolmogorov's inequality}.
\end{exmp}

\begin{lem}
Let $\E[X]=0$, and $P[|X-a| \le b] = 1$ for some constants $a$ and $b$. Then
\eq{
\E[e^{\theta X}] \le \exp{\left( \frac{\theta^2 b^2}{2} \right)}.
}
\end{lem}

\begin{thm}[Azuma inequality]
Let $Y_n$ be a martingale, $\P[|Y_n - Y_{n-1}| \le d_n] = 1$, then
\eq{
\P[|Y_n-Y_0| \ge \alpha] \le 2 exp{\left(\frac{-\alpha^2}{2 \sum_{i=1}^{n} d_i^2}\right)}.
}
\end{thm}
\begin{proof}
	Since $\E[e^{(\theta (Y_n - Y_0))}] < \infty$ for all $\theta$,
\begin{align}
\label{azuma-1}
\P[Y_n-Y_0 \ge \alpha] \le\frac{ \E[e^{(\theta (Y_n -Y_0))}]}{e^{(\theta \alpha)}}.
\end{align}
Also,
\eq{
\E[exp(\theta (Y_n -Y_0))] &= \E[\E[e^{(\theta (Y_n-Y_{n-1})+ \theta (Y_{n-1}-Y_0))}]|\F_{n-1}] \nonumber\\ 
&= \E[e^{\theta (Y_{n-1}-Y_0)} \E[e^{\theta (Y_n-Y_{n-1})}|\F_{n-1}]]\nonumber \\
&\le \E[\theta (Y_{n-1}-Y_0)] e^{\left(\frac{\theta^2 d^2_n}{2}\right)} \nonumber\\.
}
by the above lemma, iterating, 
\begin{align}
\label{azuma-2}
\E[exp(\theta (Y_n - Y_0))] \le e^{\left(\frac{\theta^2}{2} \sum_{i=1}^{n} d^2_i\right)}
\end{align}
From \ref{azuma-1}, \ref{azuma-2},
\eq{
\P[|Y_n-Y_0| \ge \alpha] \le \frac{e^{\left(\frac{\theta^2}{2} \sum_{i=1}^{n} d^2_i\right)}}{e^{(\theta \alpha)}} \quad \text{holds for any $\theta > 0$}.
}
Choosing $\theta = \frac{\alpha}{\sum_{i=1}^{n} d^2_i}$, we get the tighest upper bound,
\eq{
\P[|Y_n-Y_0|\ge  \alpha] \le 2 exp{\left(\frac{-\alpha^2}{2 \sum_{i=0}^{n} d_i^2}\right)}.
}
\end{proof}


Consider independent random variables $X_1, X_2, \dots, X_n$, $\E[X_i] = \mu$, $S_n = \sum_{i=1}^{n} X_i$. 
\begin{defn}
$F: \R^m \to \R$ is called \textit{Lipschitz-c} if	
\eq{
|F(x_1, x_2, \dots, x_{i-1}, x_i, x_{i+1},\dots,  x_m) - F(x_1, x_2, \dots, x_{i-1}, y_i, x_{i+1},\dots,  x_m)| \le c, \enskip \forall i, \forall x_1, x_2, \dots, x_m, y_i.
}
\end{defn}

\begin{prop}[\textit{McDiarmid's Inequality}]
Under above conditions, if $F$ is Lipschitz-c,
\eq{
	\P[|F(X_1, X_2, \dots, X_m)-\E[F]| \ge \alpha] \le 2 e^{(\frac{- 2 \alpha^2}{m c^2})}.
}
\end{prop}

\begin{proof}
Let $Z = f(X_1, X_2, \dots, X_m)$, $Z_i = \E[Z|X_1, X_2, \dots, X_i]$. Then $\{Z_i\}$ is a martingale.

Also,
\eq{
|Z_{i+1} - Z_i| = &\lvert \E\left[f(x_1, x_2, \dots, x_{i+1}, X_{i+2}, X_{i+3}, \dots, X_n)|X_1=x_1,  \dots, X_i=x_i,X_{i+1}=x_{i+1}\right]\\
&-\E\left[f(x_1, x_2, \dots, x_{i}, X_{i+1}, \dots, X_n)|X_1=x_1, \dots, X_i=x_i\right]\rvert\\
&\le \E\left[|f(x_1, x_2, \dots, x_{i+1}, X_{i+2}, X_{i+3}, \dots, X_n)- f(x_1, x_2, \dots, x_i, X_{i+1}, X_{i+2}, X_{i+3}, \dots, X_n)|\right]\\
& \le c.
}
Thus, 
\eq{
\P[|Z_{i+1}-Z_i| \le c |\F_i] = 1.
}
and the result follows by Azuma's inequality,because $Z_0 = \E[F(X_1, X_2, \dots, X_m)]$.
\end{proof}
\end{document}