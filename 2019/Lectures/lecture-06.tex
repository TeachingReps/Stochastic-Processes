% !TEX spellcheck = en_US
% !TEX spellcheck = LaTeX
%\documentclass[a4paper,10pt,english]{article}
\documentclass[all-lectures.tex]{subfiles}
%\usepackage{%
	amsfonts,%
	amsmath,%	
	amssymb,%
	amsthm,%
%	babel,%
	bbm,%
	%biblatex,%
	caption,%
	centernot,%
	color,%
	enumerate,%
	epsfig,%
	epstopdf,%
	etex,%
	geometry,%
	graphicx,%
	hyperref,%
	latexsym,%
	mathtools,%
	multicol,%
	pgf,%
	pgfplots,%
	pgfplotstable,%
	pgfpages,%
	proof,%
	psfrag,%
	subfigure,%	
	tikz,%
	ulem,%
	url%
}	

\usepackage[mathscr]{eucal}
\usepgflibrary{shapes}
\usetikzlibrary{%
  arrows,%
  backgrounds,%
  chains,%
  decorations.pathmorphing,% /pgf/decoration/random steps | erste Graphik
  decorations.text,%
  matrix,%
  positioning,% wg. " of "
  fit,%
  patterns,%
  petri,%
  plotmarks,%
  scopes,%
  shadows,%
  shapes.misc,% wg. rounded rectangle
  shapes.arrows,%
  shapes.callouts,%
  shapes%
}

\theoremstyle{plain}
\newtheorem{thm}{Theorem}[section]
\newtheorem{lem}[thm]{Lemma}
\newtheorem{prop}[thm]{Proposition}
\newtheorem{cor}[thm]{Corollary}

\theoremstyle{definition}
\newtheorem{defn}[thm]{Definition}
\newtheorem{conj}[thm]{Conjecture}
\newtheorem{exmp}[thm]{Example}
\newtheorem{assum}[thm]{Assumptions}
\newtheorem{axiom}[thm]{Axiom}

\theoremstyle{remark}
\newtheorem{rem}[thm]{Remark}
\newtheorem{note}[thm]{Note}

\newcommand{\norm}[1]{\left\lVert#1\right\rVert}
\newcommand{\indep}{\!\perp\!\!\!\perp}
\DeclarePairedDelimiter\abs{\lvert}{\rvert}%
%\DeclarePairedDelimiter\norm{\lVert}{\rVert}%
\newcommand{\tr}{\operatorname{tr}}
\newcommand{\R}{\mathbb{R}}
\newcommand{\Q}{\mathbb{Q}}
\newcommand{\N}{\mathbb{N}}
\newcommand{\E}{\mathbb{E}}
\newcommand{\Z}{\mathbb{Z}}
\newcommand{\B}{\mathscr{B}}
\newcommand{\C}{\mathcal{C}}
\newcommand{\T}{\mathscr{T}}
\newcommand{\F}{\mathcal{F}}
\newcommand{\G}{\mathcal{G}}
%\newcommand{\ba}{\begin{align*}}
%\newcommand{\ea}{\end{align*}}

% Debug
\newcommand{\todo}[1]{\begin{color}{blue}{{\bf~[TODO:~#1]}}\end{color}}


\makeatletter
\def\th@plain{%
  \thm@notefont{}% same as heading font
  \itshape % body font
}
\def\th@definition{%
  \thm@notefont{}% same as heading font
  \normalfont % body font
}
\makeatother
\date{}
%\title{Lecture 01: Poisson Processes}
%\author{}
\setcounter{chapter}{2}

\begin{document}
%\maketitle

%\chapter{Renewal Theory}
\setcounter{section}{2}
\setcounter{subsection}{1}

\section*{\centering{Lecture 6 \linebreak }}
\subsection{Blackwell's Theorem}
\begin{thm}[Blackwell's theorem]
Let $F$ be the distribution of interarrival times. \\
\begin{enumerate} 
\item If $F$ is non-lattice
\begin{align*}
\lim_{t\to \infty}\E[N(t+a) - N(t)] = \frac{a}{\mu}
\end{align*}
for all $a \geq 0$.
\item If $F$ is lattice with period $d$
\begin{align*}
\lim_{n \to \infty}\E[N((n+1)d) - N(nd)] = \frac{d}{\mu}.
\end{align*}
\end{enumerate}
\begin{prop}
Blackwell's theorem implies elementary renewal theorem.
\end{prop}
\begin{proof}
\begin{align*}
\E\left[\frac{N_n}{n}\right] = \frac{1}{n}\sum_{k = 1}^{n} \left(\E[N_{k}] - \E[N_{k-1}]\right)
\end{align*}
From Blackwell's theorem and definition of limits, for every $\epsilon>0$, there exists $n_0$ such that for $n>n_0$
\begin{align}\label{eq:blackwell_implication}
\abs{\E[N_k] - \E[n_{k-1}] - \frac{1}{\mu}} < \epsilon. 
\end{align}
Therefore for $n> n_0$, 
\begin{align*}
\E\left[\frac{N_n}{n}\right] &= \frac{1}{n}\left(\sum_{k=1}^{n_0} \left(\E[N_{k}] - \E[N_{k-1}]\right) + \sum_{k=n_0+1}^{n}\left(\E[N_{k}] - \E[N_{k-1}]\right) \right). \\
&\leq \frac{1}{n}\left(\sum_{k=1}^{n_0} \left(\E[N_{k}] - \E[N_{k-1}]\right) + (n-n_0-1) \left(\epsilon + \frac{1}{\mu}\right)\right).
\end{align*}
Thus, taking $n \to \infty$,
\begin{align*}
\limsup_{n\to \infty} \frac{\E[N_n]}{n} \leq \frac{1}{\mu} + \epsilon.  
\end{align*}
Now take $\epsilon \to 0$. \\
\indent Similarly, by taking the opposite sign of the modulus in Eq \ref{eq:blackwell_implication}, we get
\begin{align*}
\liminf_{n\to \infty} \frac{\E[N_n]}{n} \geq \frac{1}{\mu}.  
\end{align*}
Thus, 
\begin{align*}
\lim_{n\to \infty} \frac{\E[N_n]}{n} = \frac{1}{\mu},
\end{align*}
which is elementary renewal theorem.
\end{proof}
\end{thm}
\subsection{Renewal Equation}
\begin{defn}[Renewal Equation] A functional equation of the form 
\begin{align*}
Z(t) = z(t) + F*Z(t)
\end{align*}
where $z(t)$ is a function on $[0,\infty)$ and $*$ denotes convolution is called a renewal equation. $F$ and $z$ are known and $Z$ is the unknown function.
\end{defn}
The renewal equation arises in several situations. We need to know the conditions for existence and uniqueness of the solution for renewal equation. The following theorem provides the answer.
\begin{prop}
If $F(\infty) = 1$, $F(0) < 1$ and $z(t):[0,\infty) \rightarrow [0,\infty)$ is bounded on bounded intervals the renewal equation has a unique solution given by 
\begin{align*}
Z(t) = U(t) * z(t)
\end{align*}
where $U(t) = \E[N(t)] = \sum_{k=0}^\infty F^{*n}(t)$. Here, $F^{*n}$ denotes $n$ fold convolution of $F$ with itself.
\begin{proof}
Define $U^n = \sum_{k=0}^n F^{*k}$. Now, $U^n \to U$ monotonically. Let $Z^n(t) = U^n*z(t) = \sum_{k=0}^n F^{*k}*z(t)$. We have
\begin{align*}
Z^{n+1} &= z + \sum_{k=1}^n F^{*k}*z(t) \\
&= z + F * Z^n.
\end{align*}
Let  $Z(t) :=  U* z(t)$. Since, $U^n \to U$ monotonically, from monotone convergence theorem, we get $Z^n \to Z$. Fixing $t$, we get from above equation 
\begin{align*}
\lim_{n \to \infty} Z^{n+1}(t) &= z(t) + \lim_{n\to \infty} Z^n*z(t)\\
\implies Z(t) &= z(t) + Z*z(t)
\end{align*}
Thus, we see that $Z(t) = U*z(t)$ is a solution of the renewal equation.\\
\indent \textit{Uniqueness:} Let $Z_1$ and $Z_2$ be two solutions. We have $Z_1-Z_2 = F * (Z_1 - Z_2)$. Let $V = Z_1 - Z_2$. Since $U$ and $z$ are bounded on bounded intervals, so are $Z_1,Z_2$ and $V$. Iterating $n$ times, we get $V = F^n*V$. Therefore
\begin{align*}
\abs{V} &= \abs{ \int_0^t V(t-s) d F^n(s) } \\
&\leq M_1 \int_0^t d F^n(s) \\
&= M_1 \P\{S_n \leq t\} \\
&= 0 \text{ as }n \to \infty.
\end{align*}
where $\abs{V(s)} \leq M_1$ for all $0\leq s \leq t$.

\end{proof}
\end{prop}

\textbf{Example of renewal equation:} Consider residual time of a renewal process $\{B_t\}$.
\begin{align*}
\P\{B_t \leq x\} = \P\{B_t \leq x,X_1>t\} + \int_{0}^{t} \P\{B_t \leq x|X_1=s\}F(ds)
\end{align*}
Letting  $Z(t) = \P\{B_t \leq x\}$, $z(t) = \P\{B_t \leq x,X_1>t\}$, and noting that $\P\{B_t \leq x|X_1=s\} = \P\{B_{t-s} \leq x\}$, we have 
\begin{align*}
Z(t) = z(t) + F*Z(t),
\end{align*}
which is a renewal equation whose solution is $\P\{B_t\leq x\}$. \\
\indent Let $A(t)$ be the age of the process. Then, we have $\{B_t \leq x\} = \{A_{t+x} \leq x\}$. If $\{B_t\}$ is stationary we have $\P\{B_{t+s} \leq x\} = \P\{B_t \leq x\}\ \forall x,t,s$. Thus, we get $\P\{A_{t+x+s} \leq x\} = \P\{A_{t+x} \leq x\}$. This means that $\{A_t\}$ is also stationary. We can similarly show that $\{A_t\}$ is stationary implies that $\{B_t\}$ is also stationary.

\end{document}

