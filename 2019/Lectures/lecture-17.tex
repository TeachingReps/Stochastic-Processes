% !TEX spellcheck = en_US
% !TEX spellcheck = LaTeX
%\documentclass[a4paper,10pt,english]{article}
\documentclass[all-lectures.tex]{subfiles}
%\usepackage{%
	amsfonts,%
	amsmath,%	
	amssymb,%
	amsthm,%
	algorithm,%
	babel,%
	bbm,%
	etex,%
	caption,%
	centernot,%
	color,%
	dsfont,%
	enumerate,%
	epsfig,%
	geometry,%
	graphicx,%
	hyperref,%
	latexsym,%
	mathtools,%
	multicol,%
	pgf,%
	pgfplots,%
	pgfplotstable,%
	pgfpages,%
	proof,%
	psfrag,%
	subfigure,%	
	tikz,%
	ulem,%
	url%
}	
\usepackage[noend]{algpseudocode}
\usepackage[mathscr]{eucal}
\usepgflibrary{shapes}
\usetikzlibrary{%
  arrows,%
  backgrounds,%
  chains,%
  decorations.pathmorphing,% /pgf/decoration/random steps | erste Graphik
  decorations.text,%
  fit,%
  matrix,%
  patterns,%
  petri,%
  positioning,% wg. " of "
  plotmarks,%
  scopes,%
  shadows,%
  shapes,%
  shapes.arrows,%
  shapes.callouts,%
  shapes.misc% wg. rounded rectangle
}

\theoremstyle{plain}
\newtheorem{thm}{Theorem}[section]
\newtheorem{lem}[thm]{Lemma}
\newtheorem{prop}[thm]{Proposition}
\newtheorem{cor}[thm]{Corollary}

\theoremstyle{definition}
\newtheorem{defn}[thm]{Definition}
\newtheorem{conj}[thm]{Conjecture}
\newtheorem{exmp}[thm]{Example}
\newtheorem{assum}[thm]{Assumptions}
\newtheorem{axiom}[thm]{Axiom}

\theoremstyle{remark}
\newtheorem{rem}{Remark}
\newtheorem{note}{Note}
\newtheorem{fact}{Fact}

\definecolor{lightgray}{gray}{0.9}

%\DeclarePairedDelimiter{\ceil}{\left\lceil}{\right\rceil}%
\newcommand{\eq}[1]{\begin{align*}#1\end{align*}}
\newcommand{\ceil}[1]{\left\lceil#1\right\rceil}%
\newcommand{\norm}[1]{\left\lVert#1\right\rVert}%
\newcommand{\indep}{\!\perp\!\!\!\perp}%
\DeclarePairedDelimiter\abs{\lvert}{\rvert}%
\newcommand\numberthis{\addtocounter{equation}{1}\tag{\theequation}}
\newcommand{\tr}{\operatorname{tr}}
\newcommand{\R}{\mathbb{R}}
\newcommand{\N}{\mathbb{N}}
\newcommand{\E}{\mathbb{E}}
\newcommand{\Z}{\mathbb{Z}}
\newcommand{\B}{\mathscr{B}}
\newcommand{\C}{\mathcal{C}}
\newcommand{\T}{\mathscr{T}}
\newcommand{\F}{\mathcal{F}}
\newcommand{\G}{\mathcal{G}}
\newcommand{\X}{\mathcal{X}}
%\newcommand{\ba}{\begin{align*}}
%\newcommand{\ea}{\end{align*}}
\DeclareMathOperator*{\argmax}{arg\,max}
\renewcommand{\qedsymbol}{$\blacksquare$}
\makeatletter
\def\BState{\State\hskip-\ALG@thistlm}
\makeatother

\makeatletter
\def\th@plain{%
  \thm@notefont{}% same as heading font
  \itshape % body font
}
\def\th@definition{%
  \thm@notefont{}% same as heading font
  \normalfont % body font
}
\makeatother
\date{}
%\title{Lecture 01: Poisson Processes}
%\author{}
\setcounter{chapter}{4}

\begin{document}
%\maketitle

%\chapter{Continuous-Time Markov Chains}
\setcounter{section}{5}
\setcounter{subsection}{0}

\section*{\centering{Lecture 17 \linebreak }}

\section{Birth-Death process}
Consider a system whose state $X_t$ at any time is represented by the number of people
in the system at that time. 
Suppose that whenever there are $n$ people in the system,
then $(i)$ new arrivals enter the system at an exponential rate $\beta_n$, and $(ii)$ people
leave the system at an exponential rate $\delta_n$. 
Such a system $\{X_t\}$ is called a \textit{birth-death process} (B-D). The parameters $\{\beta_n\}^{\infty}_{n=0}$ and $\{\delta_n\}^{\infty}_{n=0}$ are called, respectively, the
arrival (or birth) and departure (or death) rates.

The state space of the birth-death process is $\{0,1, \dots\}$.
The transitions from state $n$ may go only to either state $n-1$ (if $n>0$) or
state $n + 1$. 
Thus, it is a Markov chain with its jump chain $\{Y_n\}$ having the transition matrix,
\eq{
P_{01} = 1
}
\begin{alignat*}{2}
P_{i,i+1} = \frac{\beta_i}{\beta_i+\delta_i}, \quad i>0, && \quad  \quad  P_{i,i-1} = \frac{\delta_i}{\beta_i+\delta_i},  \quad i>0.
\end{alignat*}
\\

An example of a birth-death process is the $\{q_t\}$ process of an $M/M/1$ queue. \\
We use conditions for transience of DTMC given earlier to give conditions for recurrence of a B-D process.

Recurrence of $\{X_t\}$ $\leftrightarrow $ Recurrence of $\{Y_k\}$.
Thus, we look for a bounded solution $h:S\setminus\{0\} \to R$ with
\eq{
	h(j) = \sum_{k \neq 0} P_{jk} h(k), \quad j \neq 0,
}\\
Then,
\eq{
h(1) = \frac{\beta_2}{\beta_2+\delta_2} h(2).
}
Writing $p_i = \frac{\beta_i}{\beta_i + \delta_i}$ and $q_i=1-p_i = \frac{\delta_i}{\beta_i + \delta_i}$,
\eq{
(p_j+q_j)h(j) = q_j h(j-1) + p_j h(j+1)
}
Solving this iteratively , we get
\eq{
h(2)-h(1) = \frac{q_1}{p_1},
}
\eq{
h(j+1)-h(j) = h(1) \frac{q_j q_{j-1} \dots q_1}{p_j p_{j-1} \dots p_1}
}
For this to be a bounded function, we need
\eq{
\sum_{j=1}^{\infty}  \frac{q_j q_{j-1} \dots q_1}{p_j p_{j-1} \dots p_1} < \infty.
}
This is necessary and sufficient condition for $\{Y_n\}$ to be transient.

Therefore,
\eq{
\sum_{j=1}^{\infty}  \frac{q_j q_{j-1} \dots q_1}{p_j p_{j-1} \dots p_1} = \infty \quad \iff \quad \{Y_n\} \text{ is recurrent} \quad \iff \quad \{X_t\} \text{ is positive recurrent}.
}
Now we give conditions for positive recurrence of a B-D process.
Solving the equation $\pi Q = 0$, we get
\eq{
\pi(n) = \frac{\beta_n \beta_{n-1} \dots \beta_1}{\delta_{n+1} \delta_n \dots \delta_2}  \pi(0)
}
From this we can conclude that
\eq{
\sum \pi(i) <\infty \iff \sum_{k=1}^{\infty} \frac{\beta_k \beta_{n-1} \dots \beta_1}{\delta_{k+1} \delta_n \dots \delta_2} < \infty.
}
This is a necessary and sufficient condition for positive recurrence of the birth-death process.

\subsection{Reversibility of Birth-Death process}
\begin{prop}
A stationary birth-death process is reversible.
\end{prop}

\begin{proof}
We need to show that 
\eq{
	\pi(i) Q_{ij} = \pi(j) Q_{ji}
}
But
\eq{
\frac{\beta_i \beta_{n-1} \dots \beta_1}{\delta_{i+1} \delta_n \dots \delta_2} Q_{i,i+1} = \frac{\beta_{i+1} \beta_{n-1} \dots \beta_1}{\delta_{i+2} \delta_n \dots \delta_2} Q_{i+1,i} 
}
because, 
\eq{
Q_{i,i+1} = \beta_{i+1}, \quad  \quad Q_{i+1,i} = \delta_{i+2}.\\
}
\end{proof}
\begin{exmp}[The $M/M/1$ queue]
In the $M/M/1$ queue $\beta_n = \lambda, \delta_n = \mu$.

For recurrence,
\eq{
\sum_{j=1}^{\infty} \frac{\left(\frac{\mu}{\mu+\lambda}\right)^j}{\left(\frac{\lambda}{\mu+\lambda}\right)^j} &= \sum_{j=1}^{\infty} \left(\frac{\mu}{\lambda}\right)^j = \infty.
}
That is $\frac{\lambda}{\mu}  \le 1$ is the necessary and sufficient condition for an $M/M/1$ queue to be recurrent.\\\\
For positive recurrence,
\eq{
\sum_{k=1}^{\infty} \frac{\beta_k \beta_{n-1} \dots \beta_1}{\delta_{k+1} \delta_n \dots \delta_2} < \infty \quad \Rightarrow  \quad \sum_{k=1}^{\infty} \left(\frac{\lambda}{\mu}\right)^k < \infty,  \frac{\lambda}{\mu} <1
}
Then,
\eq{
\pi(n) = \left(\frac{\lambda}{\mu}\right)^n \pi(0) \quad \text{and} \quad \sum_{n=0}^{\infty} \pi(n) =1
}
Therefore,
\eq{
\pi(n) = \rho^n (1-\rho)	\quad \text{where} \quad \rho = \frac{\lambda}{\mu}.
}
\end{exmp}

%\begin{exmp}[The $M/M/1$ queue]
%In the $M/M/1$ queue the arrival rate of customers is $\lambda$ and the service rate is $\mu$.
%Then the queue length process $q_t$ is a birth-death process. 
%Thus by using the limiting probabilities for a birth and death process, we can write the distribution of queue length as
%\eq{
%	\pi_n &= \frac{(\frac{\lambda}{\mu})^n}{1+\sum_{n=1}^{\infty} \le \left(\frac{\lambda}{\mu}\right)^n}\\
%	&=\left(\frac{\lambda}{\mu}\right)^n \left(1-\frac{\lambda}{\mu}\right),
%}
%with
%\begin{align*}
%Q_{ii} = -\lambda - \mu
%\end{align*}
%\begin{align*}
%Q_{i,i+1} = \lambda
%\end{align*}
%\begin{align*}
%Q_{i,i-1} = \mu  \quad \text{for} \quad i >0,
%\end{align*}
%
%for $\lambda < \mu$, it has a stationary distribution $\pi$.
%\end{exmp}
\begin{exmp}[The $M/M/\infty$ queue]
	The queue has an $\infty$ number of servers. Whenever a customer arrives it joins an idle server and gets service with an exponential distribution $\exp(\mu)$.After completion of service it leaves the system. 
Let $q_t$ be the number of customers in the system at time $t$.
Its $Q$ matrix is given by
\eq{
q_{i, i+1} = \lambda, \quad q_{i,i-1}=i \mu \quad \text{for} \quad i>0.
}

For recurrence, we need
\eq{
\sum_{j} \frac{q_j q_{j-1} \dots  q_1}{p_j p_{j-1} \dots p_1} = \infty.
}
\eq{
\sum_{j} \frac{j \mu (j-1) \mu \dots \mu}{\lambda^j} &= \sum_j j! \left(\frac{q}{\lambda}\right)^j\\
&= \infty.
}
This holds if $\frac{q}{\lambda} \neq 0$. Positive recurrence also holds because
\eq{
\sum_{k=1}^{\infty} \frac{\beta_k \beta_{k-1} \dots \beta_1}{\delta_{k+1} \delta_n \dots \delta_2} &= \sum_k \frac{\lambda^k}{(k+1) \mu k \mu \dots \mu}\\
&= \sum_{k=0}^{\infty} \frac{\lambda^k}{\mu^k (k+1)!}\\
&< \infty.
}
\end{exmp}
\end{document}