% !TEX spellcheck = en_US
% !TEX spellcheck = LaTeX
%\documentclass[a4paper,10pt,english]{article}
\documentclass[all-lectures.tex]{subfiles}
%\usepackage{%
	amsfonts,%
	amsmath,%	
	amssymb,%
	amsthm,%
%	babel,%
	bbm,%
	%biblatex,%
	caption,%
	centernot,%
	color,%
	enumerate,%
	epsfig,%
	epstopdf,%
	etex,%
	geometry,%
	graphicx,%
	hyperref,%
	latexsym,%
	mathtools,%
	multicol,%
	pgf,%
	pgfplots,%
	pgfplotstable,%
	pgfpages,%
	proof,%
	psfrag,%
	subfigure,%	
	tikz,%
	ulem,%
	url%
}	

\usepackage[mathscr]{eucal}
\usepgflibrary{shapes}
\usetikzlibrary{%
  arrows,%
  backgrounds,%
  chains,%
  decorations.pathmorphing,% /pgf/decoration/random steps | erste Graphik
  decorations.text,%
  matrix,%
  positioning,% wg. " of "
  fit,%
  patterns,%
  petri,%
  plotmarks,%
  scopes,%
  shadows,%
  shapes.misc,% wg. rounded rectangle
  shapes.arrows,%
  shapes.callouts,%
  shapes%
}

\theoremstyle{plain}
\newtheorem{thm}{Theorem}[section]
\newtheorem{lem}[thm]{Lemma}
\newtheorem{prop}[thm]{Proposition}
\newtheorem{cor}[thm]{Corollary}

\theoremstyle{definition}
\newtheorem{defn}[thm]{Definition}
\newtheorem{conj}[thm]{Conjecture}
\newtheorem{exmp}[thm]{Example}
\newtheorem{assum}[thm]{Assumptions}
\newtheorem{axiom}[thm]{Axiom}

\theoremstyle{remark}
\newtheorem{rem}[thm]{Remark}
\newtheorem{note}[thm]{Note}

\newcommand{\norm}[1]{\left\lVert#1\right\rVert}
\newcommand{\indep}{\!\perp\!\!\!\perp}
\DeclarePairedDelimiter\abs{\lvert}{\rvert}%
%\DeclarePairedDelimiter\norm{\lVert}{\rVert}%
\newcommand{\tr}{\operatorname{tr}}
\newcommand{\R}{\mathbb{R}}
\newcommand{\Q}{\mathbb{Q}}
\newcommand{\N}{\mathbb{N}}
\newcommand{\E}{\mathbb{E}}
\newcommand{\Z}{\mathbb{Z}}
\newcommand{\B}{\mathscr{B}}
\newcommand{\C}{\mathcal{C}}
\newcommand{\T}{\mathscr{T}}
\newcommand{\F}{\mathcal{F}}
\newcommand{\G}{\mathcal{G}}
%\newcommand{\ba}{\begin{align*}}
%\newcommand{\ea}{\end{align*}}

% Debug
\newcommand{\todo}[1]{\begin{color}{blue}{{\bf~[TODO:~#1]}}\end{color}}


\makeatletter
\def\th@plain{%
  \thm@notefont{}% same as heading font
  \itshape % body font
}
\def\th@definition{%
  \thm@notefont{}% same as heading font
  \normalfont % body font
}
\makeatother
\date{}
%\title{Lecture 01: Poisson Processes}
%\author{}
\setcounter{chapter}{2}
\usepackage{enumitem}
\newcommand*{\QEDB}{\hfill\ensuremath{\square}}%
\begin{document}
%\maketitle

%\chapter{Renewal Theory}
\setcounter{section}{4}
\setcounter{subsection}{0}
 
\section*{\centering{Lecture 8 \linebreak (05-Feb-2019) }} %05-Feb-2019
{\color{blue}
\section{Renewal Theorems (Contd)}
In the previous lecture, we have seen that 
\begin{itemize}
\item $\P\{A_t \leq x\} \to F_0(x) \text{ as } t\to\infty.$
\item $\P\{B_t \leq x\} \to F_0(x) \text{ as } t\to\infty.$
\end{itemize}
where 
\begin{align*}
F_0(x) = \frac{1}{\mu} \int_0^x (1-F(s) ds.
\end{align*}
If the distribution of $A_0$ is $F_0$, then the processes $A(t)$ and $B(t)$ are stationary. The process $A(t)$ is said to be stationary if $(A(t+t_1),A(t+t_2),\dots,A(t+t_n)) \stackrel{d}{=} (A(t_1),A(t_2),\dots,A(t_n))$  for all $t\geq 0$ and $0 < t_1 <t_2 < \dots < t_n$. \\
Now, consider the renewal equation, $Z(t) = z(t) + F*z(t)$. Let $z(t) = 1\{t \in [0,a]\}$. It can be easily verified that $z(t)$ is d.r.i. The solution of this renewal equation is 
\begin{align*}
Z(t) &= z * \E[N_t] \\
&= \int_0^\infty z(t-s) d \E[N_s] \\
&= \int_{t-a}^t d\E[N_s]  \\
&= \E[N_t] - \E[N_{t-a}].
\end{align*}
From  key renewal theorem, we have 
\begin{align*}
\lim_{t \to \infty} Z(t) &= \frac{1}{\mu} \int_0^\infty z(s) ds \\
\lim_{t \to \infty} \E[N_t] - \E[N_{t-a}] &= \frac{1}{\mu} \int_0^a ds \\
&= \frac{a}{\mu}.
\end{align*}
This is the statement of Blackwell's theorem. 
}
\section{Regenerative Processes}
Let $\{X_t\}$ be a stochastic process and $Y_0,Y_1,Y_2,\dots$ be i.i.d with distribution $F$. Let $S_n = \sum_{k=0}^n Y_k$.
\begin{defn}
The process $\{X_t\}$ is \textit{regenerative }if there exists $Y_0,Y_1,\dots$ i.i.d such that the process $Z_{n+1} = \{X_{S_n +t}, t\geq 0\}$ is independent of $Y_0,Y_1,Y_2,\dots,Y_n$ and the distribution of $Z_{n+1}$ does not depend on $n$. $\{X\}$ is a delayed regenerative process if distribution of $Y_0$ is different from $Y_1$.
\end{defn}
\indent \textbf{Examples:} 
\begin{enumerate}
\item The process $\{B_t\}$ corresponding to residual life in a renewal process is regenerative if we take $Y_k = X_k$ where $X_k$ is the $k^{th}$ inter-arrival time.
\item In a Markov chain the time instants when the Markov chain visits a particular state, say $i_0$, the process regenerates itself. 
\item Consider a GI/GI/1 queue. The process $\{q_t\}$, the queue length at time $t$ is a continuous time regenerative process which regenerates when an arrival sees empty queue. The process $\{W_n\}$, the waiting time of the $n^{th}$ customer is a discrete time regenerative process with the above arrival epochs. 
\end{enumerate} 
\begin{thm}
Let $\{X_t,t\geq 0\}$ be a delayed regenerative process with $\mu = \E[Y_1] < \infty$ and $F$ is non-lattice. Let $f$ be a bounded,  continuous function a.s. Then,
\begin{align*}
\lim_{t\to \infty} \E[f(X_t)] = \E_e[f(X_t)] = \frac{1}{\mu} \E_0\left[ \int_0^{Y_1} f(X_s) ds\right]
\end{align*}
where $\E_e$ is the expectation w.r.t. equilibrium or stationary distribution and $\E_0$ is the expectation w.r.t the process when $S_0 = 0$.
\begin{proof} We show for $S_0 = 0$. Then, 
\begin{align*}
\E[f(X_t)] &= \E[f(X_t), Y_1 > t] + \int_0^t \E[f(X_t)| Y_1 = s] dF(s) \\
&= \E[f(X_t), Y_1 > t] + \int_0^t \E[f(X_{t-s})] dF(s)\\
&= \E[f(X_t), Y_1 > t] + \E[f(X)]*F(t)
\end{align*}
This is a renewal equation. Since $f$ is bounded and $X_t$ is right continuous, it follows that $\E[f(X_t), Y_1 > t]$ is d.r.i. Therefore,
\begin{align*}
\lim_{t\to \infty} \E[f(X_t)] &= \frac{1}{\mu} \int_0^\infty \E_0[f(X_s), Y_1 >  s] dt\\
&= \frac{1}{\mu} \E_0\left[ \int_0^\infty f(X_s)1\{ Y_1 >  s\} ds\right]\\
&= \frac{1}{\mu} \E_0\left[ \int_0^{Y_1} f(X_s) ds\right] \qedhere
\end{align*}
The following theorem for lattice $F$ can be proved in the same way.
\newpage
\end{proof}
\end{thm}
\begin{thm}\ 
\begin{enumerate}
\item For discrete time regenerative processes $\{X_n\}$ and $F$ is aperiodic,
\begin{align*}
\lim_{n\to \infty} \E[f(X_n)] = \E_e[f(X_n)] = \frac{1}{\mu} \E_0\left[ \sum_{k=1}^{Y_1} f(X_k) \right].
\end{align*}
\item For discrete time regenerative processes $\{X_n\}$ and $F$ has period $d$,
\begin{align*}
\lim_{n\to \infty} \frac{1}{d} \sum_{k=0}^{d-1} \E[f(X_{n+k})] = \E_e[f(X_n)] = \frac{1}{\mu} \E_0\left[ \sum_{k=1}^{Y_1} f(X_k) \right]. 
\end{align*}
\end{enumerate}
\QEDB
\end{thm}
Taking $f(X_t) = 1\{X_t \leq x\}$, for the non-lattice case we have 
\begin{align*}
\lim_{t\to \infty} \P\{X_t \leq x\} = \frac{\E\left[ \int_0^{Y_1} 1\{X_t \leq x\}\right]}{\E[Y_1]}.
\end{align*}
Similar results hold for the lattice case. \\
\indent \textbf{Example} ($GI/GI/1$ queue): If regenerative lengths $\tau$ of $\{W_n\}$ in $GI/GI/1$ queue satisfies $E[\tau] < \infty$ and it is aperiodic, then $W_n$ has unique stationary distribution and $W_n \to W_\infty$ where $W_\infty$ is a r.v. with the stationary distribution. We can show that if $\E[A] < \E[s]$, then the above conditions are satisfied. Also, if the queue starts empty with an arrival, then, if $\overline{\tau}$ is the regeneration length of $\{q_t\}$, then $\overline{\tau} = \sum_0^{\tau} A_k$. Since, $\tau$ is a stopping time w.r.t $\{ A_n, s_n\}$ sequence, by Wald's lemma, $\E[\overline{\tau}]=\E[A_1]\E[\tau] < \infty$ whenever $\E[\tau] < \infty$. Thus when $\E[A_1]<\E[s_1]$, $\{q_t\}$ also has a unique stationary distribution and starting from any initial distribution,  it converges in distribution to the limiting distribution.

\end{document}

