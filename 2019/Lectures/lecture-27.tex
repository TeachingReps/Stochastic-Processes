% !TEX spellcheck = en_US
% !TEX spellcheck = LaTeX
%\documentclass[a4paper,10pt,english]{article}
\documentclass[all-lectures.tex]{subfiles}
%\usepackage{%
	amsfonts,%
	amsmath,%	
	amssymb,%
	amsthm,%
%	babel,%
	bbm,%
	%biblatex,%
	caption,%
	centernot,%
	color,%
	enumerate,%
	epsfig,%
	epstopdf,%
	etex,%
	geometry,%
	graphicx,%
	hyperref,%
	latexsym,%
	mathtools,%
	multicol,%
	pgf,%
	pgfplots,%
	pgfplotstable,%
	pgfpages,%
	proof,%
	psfrag,%
	subfigure,%	
	tikz,%
	ulem,%
	url%
}	

\usepackage[mathscr]{eucal}
\usepgflibrary{shapes}
\usetikzlibrary{%
  arrows,%
  backgrounds,%
  chains,%
  decorations.pathmorphing,% /pgf/decoration/random steps | erste Graphik
  decorations.text,%
  matrix,%
  positioning,% wg. " of "
  fit,%
  patterns,%
  petri,%
  plotmarks,%
  scopes,%
  shadows,%
  shapes.misc,% wg. rounded rectangle
  shapes.arrows,%
  shapes.callouts,%
  shapes%
}

\theoremstyle{plain}
\newtheorem{thm}{Theorem}[section]
\newtheorem{lem}[thm]{Lemma}
\newtheorem{prop}[thm]{Proposition}
\newtheorem{cor}[thm]{Corollary}

\theoremstyle{definition}
\newtheorem{defn}[thm]{Definition}
\newtheorem{conj}[thm]{Conjecture}
\newtheorem{exmp}[thm]{Example}
\newtheorem{assum}[thm]{Assumptions}
\newtheorem{axiom}[thm]{Axiom}

\theoremstyle{remark}
\newtheorem{rem}[thm]{Remark}
\newtheorem{note}[thm]{Note}

\newcommand{\norm}[1]{\left\lVert#1\right\rVert}
\newcommand{\indep}{\!\perp\!\!\!\perp}
\DeclarePairedDelimiter\abs{\lvert}{\rvert}%
%\DeclarePairedDelimiter\norm{\lVert}{\rVert}%
\newcommand{\tr}{\operatorname{tr}}
\newcommand{\R}{\mathbb{R}}
\newcommand{\Q}{\mathbb{Q}}
\newcommand{\N}{\mathbb{N}}
\newcommand{\E}{\mathbb{E}}
\newcommand{\Z}{\mathbb{Z}}
\newcommand{\B}{\mathscr{B}}
\newcommand{\C}{\mathcal{C}}
\newcommand{\T}{\mathscr{T}}
\newcommand{\F}{\mathcal{F}}
\newcommand{\G}{\mathcal{G}}
%\newcommand{\ba}{\begin{align*}}
%\newcommand{\ea}{\end{align*}}

% Debug
\newcommand{\todo}[1]{\begin{color}{blue}{{\bf~[TODO:~#1]}}\end{color}}


\makeatletter
\def\th@plain{%
  \thm@notefont{}% same as heading font
  \itshape % body font
}
\def\th@definition{%
  \thm@notefont{}% same as heading font
  \normalfont % body font
}
\makeatother
\date{}
%\title{Lecture 01: Poisson Processes}
%\author{}
\newcommand*{\QEDB}{\hfill\ensuremath{\square}}%
\setcounter{chapter}{7}

\begin{document}
%\maketitle

%\chapter{Martingales}
\setcounter{section}{4}
\setcounter{subsection}{1}

\section*{\centering{Lecture 27 \linebreak }}
\chr
\subsection{Networks of quasireversible queues (contd.)}
\indent Let $X_t = (X_t(1),X_t(2),\dots,X_t(N))$ denote the state of the network with $N$ quasireversible queues and $X_t(i)$ denotes the state of the $i^{th}$ queue at time $t$. The queues can be of different type as long as they are quasireversible in isolation. 

In the last lecture, we showed the following. 

If $\rho_i < 1\ \forall i \in {1,2,\dots,N}$ (if the queue needs it for stability, e.g., $M/G/\infty$ does not), then $X_t$ has the stationary distribution 
\begin{align*}
\pi(x(1),x(2),\dots,x(N)) = \prod_{i}^N \pi_i(x(i))
\end{align*}
where $\pi_i$ is the stationary distribution of queue $i$.

Now, we claim that $X_t$ itself is a quasireversible process by showing that future arrivals, $X_t$ and past departures are independent. Consider the reversed process $\tilde{X}_t = X_{T-t}$ for some fixed time $T$ with generator matrix $\tilde{Q}$. We have 
\begin{align*}
\pi(i) \tilde{Q}(i,j) = \pi(j) Q(j,i).
\end{align*}
Here, $\tilde{Q}$ corresponds to the another quasireversible system with parameters 
\begin{align*}
\tilde{p}_{ij}^{cd} &= \frac{\overline{\lambda}_j^d}{\overline{\lambda}_i^c} p_{ji}^{dc}. \qedhere
\end{align*} 
This shows that future arrivals, $X_t$ and the past departures are independent. Also, the departures of all classes from the network form independent Poisson processes.\\

\noindent \textbf{Sojourn times:} \\
\indent We can compute the mean queue length or the mean number of customers in the system under stationarity from $\pi$. The mean sojourn time $\E[S]$ can be deduced by applying Little's law to the whole system: $\E[S] = \lambda \E[q]$, where $\E[q]$ is the mean number of customers in the system and $\lambda = \sum_i \lambda_i$ is the total external arrival rate into the system. Further, Little's law can be applied to each class of customers individually. The mean sojourn time of a class $c$ customer $\E[S_c] = \lambda_c \E[q_c]$ where $\E[q_c]$ is the mean number of customers of class $c$ in the system. 

\indent Next, consider a tandem of $N$ $M/M/1$ queues with service rate $\mu_i$ at queue $i$ and arrival rate $\lambda$ under stationarity. Let the random variable $S^i$ denote the sojourn time of a customer in queue $i$. Under stationarity, we know that the departure process of queue $1$ is Poisson with rate $\lambda$. This is also the arrival process of queue $2$. Thus, the second queue is also an $M/M/1$ queue. This way we can show that the arrival and departure processes for all queues are Poisson with rate $\lambda$. If an arriving customer at queue $i$ sees $n$ customers already in the queue, then $S^i = \sum_{k=1}^{n+1} s_k$ where $s_k$ are the service times, which are i.i.d. with exponential distribution with mean $1/\mu_i$. By PASTA, the probability that an arriving customer sees $n$ customers already in the queue is equal to the stationary probability of $q_t(i) = n$. Therefore,
\begin{align*}
\P\{S^i \leq x\} &= \sum_{n=0}^{n=\infty} \P\{S^i \leq x | q_i = n\} \P_{\pi}\{ q_i = n\} \\
&= \sum_{n=0}^{n=\infty} \P \left\{ \sum_{k=1}^{n+1}s^k \leq x \right\} \rho_i^n (1-\rho_i).
\end{align*}
We can show (e.g., by taking moment generating function) that the above quantity is an exponential distribution with mean $1/(\mu_i - \lambda)$. This is applicable to all the queues. The total sojourn time is $S = \sum_{i=1}^N S^i$. Furthermore, it can also be shown that $S^1,S^2,\dots,S^N$ are independent random variables. \\
\indent The above results hold for a general network of quasireversible queues. This is summarized below. Let $\mu_i$ be the service rate and $\overline{\lambda}_i$ be the total arrival rate at queue $i$.
\begin{defn}
An \textit{overtake free path }for a class $c$ is a path $1\to 2 \to ... \to N_1$ of queues it can pass through sequentially with a positive probability such that if \underline{any} customer can go through its two queues with positive probability, then it must pass through the all intermediate nodes as well in the same order.
\end{defn}
\begin{thm}
For a network of quasireversible queues, the sojourn times $\{S^1,S^2,\dots,S^{N_1}\}$ under stationarity of customers of a class $c$ on an overtake free path $1\to 2 \to ... \to N_1$ are independent and $S^i$ is exponentially distributed with mean $1/(\mu_i - \overline{\lambda}_i)$. \QEDB
\end{thm}

\noindent \textbf{Total arrival process at a queue:}  \\
\indent Consider an $M/M/1$ queue with feedback. A customer after service re-enters the queue with probability $p$ and exits the system with probability $1-p$. Let the external arrival rate be $\lambda$ and the aggregate arrival rate (including from feedback) be $\overline{\lambda}$. We have the relation $\overline{\lambda} = \lambda + p \overline{\lambda}$ from which we find
\begin{align*}
\overline{\lambda} = \frac{\lambda}{1-p}.
\end{align*}

We show that the aggregate arrival process is not a Poisson process. Let $N_t$ denote the aggregate arrival process. For small enough $\epsilon >0$,
\begin{align*}
\P\{N_{t+\epsilon} \geq N_t + 1\} = \lambda \epsilon + o(\epsilon) + p \mu \epsilon \P_{\pi}\{q_t > 0\}
\end{align*}
where $\mu \epsilon \P_{\pi}\{q_t > 0\}$ is the probability of a customer finishing service at time $t$. We also have \begin{align*}
\P\{N_{t+\epsilon} \geq N_t + 1| N_{t} \geq N_{t-\epsilon} + 1\} &= \lambda \epsilon + o(\epsilon) + p \mu \epsilon.
\end{align*}
The above equation follows from the fact that the event $\{N_{t} \geq N_{t-\epsilon} + 1\}$ implies $\{q_t > 0\}$. From these two equations, we see that $\P\{N_{t+\epsilon} \geq N_t + 1| N_{t} \geq N_{t-\epsilon} + 1\} \neq \P\{N_{t+\epsilon} \geq N_t + 1\}$. This shows that $N_t$ is not an independent increment process and hence cannot be Poisson. 

\indent As a generalization of this result, we get 
\begin{thm}
In an open quasireversible network, the aggregate arrival process at a node is not Poisson if there is non-zero probability of a customer entering that node.
\end{thm}

\noindent \textbf{Queue lengths seen by an arriving customer:} \\
\indent Consider again an $M/M/1$ queue. Define 
\begin{align*}
q_t &= \text{ queue length at time $t$} \\
q_n^A &= \text{queue length just before $n^{th}$ arrival} \\
q_n^D &= \text{queue length just after $n^{th}$ departure} \\
\tilde{q}_n^A &= \text{queue length just after $n^{th}$ arrival} \\
\end{align*}
Under stationarity, we have shown that for an $M/GI/1$ queue $q^A_n \stackrel{d}{=} q_t$ (PASTA) and $q^D_n \stackrel{d}{=} q^A_n$ (rate conservation law) for $GI/GI/1$ queue. But, $\P\{\tilde{q}^A_n = 0\} =0$ as there is always atleast one customer soon after a new arrival. Also, $\P\{q_t = 0\} = 1-\rho > 0$. Thus, $\P\{\tilde{q}^A_n = 0\} \neq \P\{q_t = 0\}$.

Now, consider a tandem of two $M/M/1$ queues. Let $S_n(i)$ be the time instant of an arrival into queue $i$. $q_{S_n(2)+}^D(1)$ denotes the length of queue $1$ just after a departure from queue $1$. $q_{S_n(2)-}^A(2)$ denotes the length of queue $2$ just before $n^{th}$ arrival into queue $2$. An argument as in preceding paragraph shows that the stationary distribution of $q^A_{S_n(2)}(2)$ is not the same as that of $q_t(2)$. But, we can show that the distribution of $\left(q_{S_n(2)+}^D(1),q_{S_n(2)-}^A(2)\right)$ under stationarity is 
\begin{align*}
\pi(n_1,n_2) = \rho_1^{n_1} (1-\rho_1) \rho_2^{n_2} (1-\rho_2).
\end{align*}
The above discussion is true in an open network of quasireversible queues. 
\begin{thm}
Let $X_t = (X_t(1),X_t(2),\dots,X_t(N))$ be the state of an open network of $N$ quasireversible queues. Let $\pi$ be the stationary distribution of $X_t$. Under stationarity, for a customer moving from queue $i$ to queue $j$ at time $S_n$,
\begin{align*}
(X_{S_n}(1),\dots,X_{S_n}(i),\dots,X_{S_n-}(j),\dots,X_{S_n}(N)) \stackrel{d}{=}&  \pi. && \QEDB 
\end{align*}
%and
%\begin{align*}
%\P\{X_{S_n}(1)=x_1,\dots,X_{S_n}(i)=x_i,\dots,X_{S_n}(j) =x_j,\dots,X_{S_n}(N)=x_N\} 
%= \pi(x_1,\dots,x_i,\dots,x_j -1,\dots,x_N). \QEDB
%\end{align*}
\end{thm}
\begin{figure}
\centering
\includegraphics[scale=0.28]{Figures/deterministic_routing.jpg}
\caption{A network of quasirevesible queues with deterministic routing for class 1}
\label{fig:det_routing}
\end{figure}
\noindent \textbf{Non-Markovian routing: } \\
\indent We give an example to show how non-Markovian routing can be taken care of in this framework. Consider a network of $4$ quasireversible queues in Figure \ref{fig:det_routing}. Class 1 customers enter queue 1 and class 2 customers enter queue 2. After service in queue 4, the class 2 customers exit the system with probability $p$ and with probability $1-p$ re-enter queue $2$. The class $1$ customers follow deterministic routing - they enter the network in queue $1$ and after service in queue $4$, they re-enter queue $1$ exactly $3$ times before exiting the system. This kind of routing cannot directly be modeled in a way we have been doing so far. However, it is possible to bring the network into our framework by introducing additional classes. We introduce new classes $3$ and $4$ as follows: after service in queue $4$ for the first time, a class $1$ customer changes to class $3$ and the second time, the same customer changes to class $4$ (from class 3). With this, $p_{41}^{13} = 1$, $p_{41}^{34} = 1$ and $p_{40}^{44} = 1$ satisfies the constraints of class 1 customers. We can now use the same theory to obtain the stationary distribution by finding total arrival rate at each node and computing $\rho_i$ s and multiplying distributions for individual queues.  \\

\end{document}