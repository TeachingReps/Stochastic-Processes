% !TEX spellcheck = en_US
% !TEX spellcheck = LaTeX
%\documentclass[a4paper,10pt,english]{article}
\documentclass[all-lectures.tex]{subfiles}
%\usepackage{%
	amsfonts,%
	amsmath,%	
	amssymb,%
	amsthm,%
%	babel,%
	bbm,%
	%biblatex,%
	caption,%
	centernot,%
	color,%
	enumerate,%
	epsfig,%
	epstopdf,%
	etex,%
	geometry,%
	graphicx,%
	hyperref,%
	latexsym,%
	mathtools,%
	multicol,%
	pgf,%
	pgfplots,%
	pgfplotstable,%
	pgfpages,%
	proof,%
	psfrag,%
	subfigure,%	
	tikz,%
	ulem,%
	url%
}	

\usepackage[mathscr]{eucal}
\usepgflibrary{shapes}
\usetikzlibrary{%
  arrows,%
  backgrounds,%
  chains,%
  decorations.pathmorphing,% /pgf/decoration/random steps | erste Graphik
  decorations.text,%
  matrix,%
  positioning,% wg. " of "
  fit,%
  patterns,%
  petri,%
  plotmarks,%
  scopes,%
  shadows,%
  shapes.misc,% wg. rounded rectangle
  shapes.arrows,%
  shapes.callouts,%
  shapes%
}

\theoremstyle{plain}
\newtheorem{thm}{Theorem}[section]
\newtheorem{lem}[thm]{Lemma}
\newtheorem{prop}[thm]{Proposition}
\newtheorem{cor}[thm]{Corollary}

\theoremstyle{definition}
\newtheorem{defn}[thm]{Definition}
\newtheorem{conj}[thm]{Conjecture}
\newtheorem{exmp}[thm]{Example}
\newtheorem{assum}[thm]{Assumptions}
\newtheorem{axiom}[thm]{Axiom}

\theoremstyle{remark}
\newtheorem{rem}[thm]{Remark}
\newtheorem{note}[thm]{Note}

\newcommand{\norm}[1]{\left\lVert#1\right\rVert}
\newcommand{\indep}{\!\perp\!\!\!\perp}
\DeclarePairedDelimiter\abs{\lvert}{\rvert}%
%\DeclarePairedDelimiter\norm{\lVert}{\rVert}%
\newcommand{\tr}{\operatorname{tr}}
\newcommand{\R}{\mathbb{R}}
\newcommand{\Q}{\mathbb{Q}}
\newcommand{\N}{\mathbb{N}}
\newcommand{\E}{\mathbb{E}}
\newcommand{\Z}{\mathbb{Z}}
\newcommand{\B}{\mathscr{B}}
\newcommand{\C}{\mathcal{C}}
\newcommand{\T}{\mathscr{T}}
\newcommand{\F}{\mathcal{F}}
\newcommand{\G}{\mathcal{G}}
%\newcommand{\ba}{\begin{align*}}
%\newcommand{\ea}{\end{align*}}

% Debug
\newcommand{\todo}[1]{\begin{color}{blue}{{\bf~[TODO:~#1]}}\end{color}}


\makeatletter
\def\th@plain{%
  \thm@notefont{}% same as heading font
  \itshape % body font
}
\def\th@definition{%
  \thm@notefont{}% same as heading font
  \normalfont % body font
}
\makeatother
\date{}
%\title{Lecture 01: Poisson Processes}
%\author{}
\newcommand*{\QEDB}{\hfill\ensuremath{\square}}%
\setcounter{chapter}{6}
\usepackage{enumitem}
\begin{document}
%\maketitle

%\chapter{Martingales}
\setcounter{section}{2}
\setcounter{subsection}{0}

\section*{\centering{Lecture 27 \linebreak }}
\section{Product Form Networks: Quasireversible networks}
%\subsection{Quasirevesible Queueing Systems:}
Till now we studied queuing networks with Markovian routing and exponential service times.  Now, both of these assumptions will be generalized. 
\subsection{Quasireversible Queues}
\indent Consider an $M/M/1$ queue with multiple classes of customers. It has a unique stationary distribution. Let $C$ denote the set of classes, $\lambda_c$ be the Poisson arrival rate for class $c$, $\mu_c$ be the service rate of class $c$. The arrival process for different classes are independent. The traffic intensity of class-$c$ is $\rho_c = \frac{\lambda_c}{\mu_c}$. Its total traffic intensity is $\rho = \sum_{i \in C} \rho_i$. \\
\indent Let $q_t$ be the number of customers in the queue. If $\rho < 1$ then $\pi(n) = (1-\rho) \rho^n$. The probability that a customer is of class-$c$ is ${\rho_c}/{\rho}$ independent of others. Let $\{X_t\}$ be the process which gives the class of each cutomer in the queue at time $t$. $\{X_t\}$ is a Markov  chain. The state space $S$ is countable. Its stationary distribution is 
\begin{align}\label{eq:prod_form_nets_multiclass}
P[X_t = (c_1, c_2, \dots, c_n)] = (1-\rho) \rho^n \prod_{i=1}^{n} \frac{\rho_{c_i}}{\rho}.
\end{align}
Although it is not reversible, its reversed process $\tilde{X}(t)$ also represents a multiclass $M/M/1$ queue, where the last customer $c_n$ leaves the queue first. Its service distributions and arrival processes are same as in the original process. Thus, the departure process of $X_t$ is again Poisson with rate $\lambda_c$ for class $c$ and the Poisson departure processes of different classes are independent. Also, arrivals from $t$ onward, $X_t$ and departures till time $t$ are independent of each other. \\
\indent It so turns out that the above properties of a multiclass $M/M/1$ queue are they key features needed for product-form stationary distributions of $X_t$.  Thus, we abstract these out to study more general classes of queuing systems.
\begin{defn}
A system is called \textit{quasireversible }if the future arrival processes from time $t$ onward, $X_t$ and past departure processes till time $t$ are independent. These are also independent for different classes.
\end{defn}
\indent The reversed process $\tilde{X}_t$ is quasireversible if $\{X_t\}$ is quasireversible. Let $N_+^c (t)$ be arrival process of class-$c$ and $N_-^c (t)$ be the departure process of class-$c$ from the system.
\begin{prop}
For a quasireversible system, 
\begin{enumerate}[label=(\arabic*)]
\item  the arrival processes of different classes are independent Poisson process and 
\item the departure  processes of different classes are also independent Poisson processes.
\end{enumerate}
\end{prop}
\begin{proof}
For a point process to be Poisson, it should have independent stationary increments. Let $(t_0,t_1]$, $(t_1,t_2], \dots, (t_{n-1},t_n]$ be time intervals with $t_0 < t_1 < t_2<\dots < t_n$. Let $z_i$ be number of arrivals of class-$c$ during interval $(t_i,t_{i+1}]$. We want to show, for any contiuous and bounded $f$,
\eq{
\E[f_0(z_0), f_1(z_1), \dots, f_{n-1}(z_{n-1})] = \prod_{i=0}^{n-1} \E[f_i(z_i)].
}
We have 
\begin{align*}
\E[f_0(z_0), f_1(z_1), \dots, f_{n-1}(z_{n-1})] &= \E[\E[f_0(z_0), f_1(z_1), \dots, f_{n-1}(z_{n-1})|\F_{t_1}]]\\
&= \E[f_0(z_0) \E[f_1(z_1), \dots, f_{n-1}(z_{n-1})|\F_{t_1}]] \\
&= \E[f_0(z_0) \E[f_1(z_1), \dots, f_{n-1}(z_{n-1})|X_{t_1}]] && \text{ ($X_t$ is a Markov chain)} \\
&=  \E[f_0(z_0)] \E[f_1(z_1), \dots, f_{n-1}(z_{n-1})] && \left(\parbox{3cm}{$X_t$ is quasireversible, $z_1, z_2, \dots, z_{n-1}$ are independent of $x_{t_1}$.} \right)
\end{align*}
Continuing this way by conditioning on $\mathcal{F}_{t_2},\mathcal{F}_{t_2},\dots,\mathcal{F}_{t_n}$, we obtain the result. We can show the other claim similarly.
\end{proof}

\textbf{Examples (Single queue):}
\begin{enumerate}[label=(\arabic*)]
\item $M/M/1/FCFS$ is quasireversible.
\item $M/GI/1/FCFS$ is not quasireversible unless service times are exponential.
\item $M/GI/\infty$ is quasireversible.
\item $M/GI/1/PS$ is quasireversible.
\item $M/GI/1/LCFS$ is quasireversible.
\end{enumerate}
In $(2)-(4)$, $q_t$ is not a Markov chain. However, $X_t = (q_t,r_t)$ is a Markov chain, where $r_t$ is residual service time of the customers in service, a real number. But, this is not a countable state Markov chain which we have been assuming so far. To overcome this problem, we use phase type distributions. \\
\textbf{Phase type distribution:}\\
Let $R_t$ be a finite state Markov chain with state space $\{1,2,3, \dots, m+1\}$ and generator matrix
\begin{align*}
	Q = 
\begin{bmatrix}
	Q_m      & q_o \\
	q_1      & q_2 \\	
\end{bmatrix},
\end{align*}
where $Q_m$ is an $m \times m$ matrix. \\
\indent Define $\tau = \inf\{t: R_t = m+1\}$. This will be a service time of a customer. Now, $X_t = (q_t,R_t)$ is a finite state Markov chain. We have,
\eq{
P[\tau > t] = \alpha \exp^{Q_m t} 1,
}
where $\alpha$ is the distribution of $R_0$ and $1 = [1,1,\dots,1]^T$. This is called \textit{phase type distribution }with parameters $(\alpha,Q_m)$.\\
\indent  Any distribution on $\R^+$ can be arbitrarily closely approximated by a phase type distribution. In the following, we will take the general distribution of service times as a phase type distribution. Then, $(q_t,R_t)$ will be a countable state Markov chain.\\
\indent Consider $X_t$ in $M/GI/1/LCFS$.  The arrival process is a Poisson process. Thus, $X_t$ and future arrivals are independent of each other. Let $\tilde{X}_t$ be the reversed process. 
\begin{prop}
In $M/GI/1/LCFS$ system, the reversed process $\tilde{X}_t$ also represents a $M/GI/1/LCFS$ system. \QEDB
\end{prop}
Thus, $X_t$ and past departures are independent. Hence, $M/GI/1/LCFS$ is a quasireversible system. Similarly, we can show that other queues in the above example are quasireversible. \\
\indent For $M/M/1$, $M/GI/1/LCFS$ and $M/GI/PS$, if $\rho < 1$, $\{q_t\}$ has a unique stationary distribution 
\begin{align*}
\pi(n) = (1-\rho)\rho^n,\ \forall n \geq 0.
\end{align*}
For a multiclass queue, we have Eq (\ref{eq:prod_form_nets_multiclass}) as stationary distribution.
For $M/GI/\infty$, for any $0<\rho<\infty$,
\begin{align*}
\pi(n) = \frac{\rho^n}{n!} e^{-\rho}, \ \forall  n \geq 0.
\end{align*}
For all these cases, we observe that the stationary distribution depends on service distribution only through its \textit{mean}. This property is called \textit{insensitivity}.\\
\indent All the examples given above for quasireversible queues are special cases of a quasireversible queue called the symmetric queue. \\
\indent The above results are shown for phase type service types. Using continuity arguments, these results can be extended to general service times. 
\subsection{Networks of Quasireversible Queues}
We now consider a multiclass queueing network of queues where each queue is quasireversible in isolation with Poisson input. Let $C$ be a countable set of classes of customers. Arrivals of different classes are independent Poisson processes. Let $\lambda^c_i$ be the external arrival rate of class $c$ customers at node $i$. Let $p^{cd}_{ij}$ be the probability that a class $c$ customer after service from node $i$ goes to node $j$ as a customer of class $d$. Denote by $\bar{\lambda}^c_i$ the total arrival rate of customers of class $c$ at node $i$. Then,
\begin{align}
\bar{\lambda}^c_i = \lambda^c_i + \sum_{j} p^{dc}_{ji} \bar{\lambda}^d_j.
\end{align}
Let $1/\mu^c_i$ be the mean service time for a class $c$ customer at node $i$. Let $\rho^c_i = \bar{\lambda}^c_i/\mu^c_i$. The total traffic intensity at node $i$ is $\rho_i = \sum_c \rho^c_i$. Then, if $\rho_i < 1, \ \forall i$, the system has product form distribution as in Eq (\ref{eq:prod_form_nets_multiclass}) with corresponding $\bar{\lambda}^c_i$ and $\mu^c_i$. The proof for this can be obtained by verifying that the ditribution in Eq (\ref{eq:prod_form_nets_multiclass}) solves $\pi Q =0$. We can also show that the system is quasireversible by verifying Eq (\ref{eq:serv_rate_class_c}) and (\ref{eq:serv_rate_class_c_minus}) below.\\
\indent Now, we provide conditions to study quasireversibility of general Markovian queueing systems. The state of the system $X_t$ is a Markov chain. Let $S$ denote the state space and $Q$ the generator matrix of $\{X_t\}$. Let $N_c^+ (t)$  be the arrival process of class $c$ and $N_c^-(t)$ be the departure process of class $c$ from the network. Let  $\mu_c^+$ and $\mu_c^-$ be the arrival rate and the departure rate of class $c$ customers respectively. \\ 
\indent Define for $i,j \in S$
\begin{align*}
A^c &= \{(i,j) \quad s.t \quad i \to j \text{ represents an arrival for class c}\} \text{, and} \\
D^c &= \{(i,j) \quad s.t \quad i \to j \text{ represents a departure of class c}\}
\end{align*}
For example, in an $M/M/1$ queue with $C$ classes, $X_t = (c_1, c_2, \dots, c_n)$, an arrival of class $c$ is 
\eq{
(c_1, c_2, \dots, c_n) \to (c_1, c_2, \dots, c_n, c) \in A^c
}
and a departure of class $c_1$ is 
\eq{
(c_1, c_2, \dots, c_n) \to (c_2, \dots, c_n) \in D^{c_1}.
}
Then, if $X_t = i$, $i \in S$, arrival rate of class $c$ = $\sum_{j:(i,j) \in A^c} q_{ij}$. Arrival rate of class $c$ under stationarity ($\pi$ is the stationary distribution) is 
\begin{align}
\label{eq:arrival_rate_class_c}
\sum_i \pi(i) \sum_{j:(i,j) \in A^c} q_{ij}.
\end{align}
But, $\{X_t\}$ is a quasi reversible process. Therefore, $X_t$ is independent of future arrivals. Thus, $\sum_{j:(i,j) \in A^c} q_{ij}$ does not depend on $i$. Thus Eq (\ref{eq:arrival_rate_class_c}) equals
\begin{align}\label{eq:serv_rate_class_c}
\sum_{j:(i,j) \in A^c} q_{ij} \sum_i \pi(i) = \sum_{j:(i,j) \in A^c} q_{ij} = \mu^+_c.
\end{align}
This is independent of $i$.\\
\indent Now, consider the reversed process $\tilde{X}_t$. This is also quasireversible. Its stationarity distribution is also $\pi$ with $\tilde{Q}$ given by 
%\eq{
%\pi(i) \tilde{Q}_{ij} = \pi(j) Q_{ji}
%}
%Therefore,
\eq{
\tilde{q}_{ij} = \frac{\pi(j) q_{ji}}{\pi(j)}.
}
The arrival rate of the class $c$ customers in $\tilde{X}_t$ is 
\begin{align} \label{eq:serv_rate_class_c_minus}
\sum_{j:(i,j) \in \tilde{A}^c}  \tilde{q}_{ij}  = \sum_{j:(i,j) \in D^c} \frac{\pi(j) q_{ji}}{\pi(j)} = \mu^-_c.
\end{align}
This also does not depend on $i$. \\
\indent We can show that if Eq (\ref{eq:serv_rate_class_c}) and (\ref{eq:serv_rate_class_c_minus}) hold for a Markovian queueing system, then it is quasireversible.
\end{document}