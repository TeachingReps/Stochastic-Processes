% !TEX spellcheck = en_US
% !TEX spellcheck = LaTeX
%\documentclass[a4paper,10pt,english]{article}
\documentclass[all-lectures.tex]{subfiles}
%\usepackage{%
	amsfonts,%
	amsmath,%	
	amssymb,%
	amsthm,%
	algorithm,%
	babel,%
	bbm,%
	etex,%
	caption,%
	centernot,%
	color,%
	dsfont,%
	enumerate,%
	epsfig,%
	geometry,%
	graphicx,%
	hyperref,%
	latexsym,%
	mathtools,%
	multicol,%
	pgf,%
	pgfplots,%
	pgfplotstable,%
	pgfpages,%
	proof,%
	psfrag,%
	subfigure,%	
	tikz,%
	ulem,%
	url%
}	
\usepackage[noend]{algpseudocode}
\usepackage[mathscr]{eucal}
\usepgflibrary{shapes}
\usetikzlibrary{%
  arrows,%
  backgrounds,%
  chains,%
  decorations.pathmorphing,% /pgf/decoration/random steps | erste Graphik
  decorations.text,%
  fit,%
  matrix,%
  patterns,%
  petri,%
  positioning,% wg. " of "
  plotmarks,%
  scopes,%
  shadows,%
  shapes,%
  shapes.arrows,%
  shapes.callouts,%
  shapes.misc% wg. rounded rectangle
}

\theoremstyle{plain}
\newtheorem{thm}{Theorem}[section]
\newtheorem{lem}[thm]{Lemma}
\newtheorem{prop}[thm]{Proposition}
\newtheorem{cor}[thm]{Corollary}

\theoremstyle{definition}
\newtheorem{defn}[thm]{Definition}
\newtheorem{conj}[thm]{Conjecture}
\newtheorem{exmp}[thm]{Example}
\newtheorem{assum}[thm]{Assumptions}
\newtheorem{axiom}[thm]{Axiom}

\theoremstyle{remark}
\newtheorem{rem}{Remark}
\newtheorem{note}{Note}
\newtheorem{fact}{Fact}

\definecolor{lightgray}{gray}{0.9}

%\DeclarePairedDelimiter{\ceil}{\left\lceil}{\right\rceil}%
\newcommand{\eq}[1]{\begin{align*}#1\end{align*}}
\newcommand{\ceil}[1]{\left\lceil#1\right\rceil}%
\newcommand{\norm}[1]{\left\lVert#1\right\rVert}%
\newcommand{\indep}{\!\perp\!\!\!\perp}%
\DeclarePairedDelimiter\abs{\lvert}{\rvert}%
\newcommand\numberthis{\addtocounter{equation}{1}\tag{\theequation}}
\newcommand{\tr}{\operatorname{tr}}
\newcommand{\R}{\mathbb{R}}
\newcommand{\N}{\mathbb{N}}
\newcommand{\E}{\mathbb{E}}
\newcommand{\Z}{\mathbb{Z}}
\newcommand{\B}{\mathscr{B}}
\newcommand{\C}{\mathcal{C}}
\newcommand{\T}{\mathscr{T}}
\newcommand{\F}{\mathcal{F}}
\newcommand{\G}{\mathcal{G}}
\newcommand{\X}{\mathcal{X}}
%\newcommand{\ba}{\begin{align*}}
%\newcommand{\ea}{\end{align*}}
\DeclareMathOperator*{\argmax}{arg\,max}
\renewcommand{\qedsymbol}{$\blacksquare$}
\makeatletter
\def\BState{\State\hskip-\ALG@thistlm}
\makeatother

\makeatletter
\def\th@plain{%
  \thm@notefont{}% same as heading font
  \itshape % body font
}
\def\th@definition{%
  \thm@notefont{}% same as heading font
  \normalfont % body font
}
\makeatother
\date{}
%\title{Lecture 01: Poisson Processes}
%\author{}
\setcounter{chapter}{4}

\begin{document}
%\maketitle

%\chapter{Continuous-Time Markov Chains}
\setcounter{section}{2}
\setcounter{subsection}{0}

\section*{\centering{Lecture 14 \linebreak }}
\chr
\section{Chapman Kolmogorov equations}
We define the \textit{generator matrix} $Q$ for MC $\{X_t\}$ as $Q_{ii} = - \lambda_i$ and $Q_{ij} = \lambda_i P_{ij}$ for $j \neq i$. 
Also, the fact that MC stays in state $i$ with $\exp(\lambda_i)$, implies that
\eq{
\frac{1-P_{ii}(t)}{t} \to \lambda_i \quad \text{as}  \quad t \to 0,
}
and then,
\eq{
\lim_{t \downarrow 0} \frac{P_{ij}(t)}{t} = \lim_{t \downarrow 0}  \frac{1-P_{ii}(t)}{t} P_{ij} = Q_{i,j} \quad \text{for all} \quad i \neq j.
}
\begin{thm}[Backward equation] For a homogeneous CTMC with transition matrix $P(t)$ and generator matrix $Q$, for the minimal construction,
	\begin{align*}
		%P'_{ij}(t)= \sum_{k \neq i}q_{ik}P_{kj}(t)-\nu_iP_{ij}(t) ,  i,j \in I
		\frac{dP(t)}{dt}=QP(t), ~~t \geqslant 0.
	\end{align*}
\end{thm}
\begin{proof} 
	Using semigroup property of transition probability matrix $P(t)$ for a homogeneous CTMC,  
	%we have
	%\begin{align*}
	%P_{ij}(t+h)=\sum_{k \in I}P_{ik}(h)P_{kj}(t),~~i,j \in I.
	%\end{align*}
	%Re-arranging terms, 
	we can write
	\begin{align*}
		%P_{ij}(t+h)-P_{ij}(t)=\sum_{k \neq i}P_{ik}(h)P_{kj}(t)-(1-P_{ii}(h))P_{ij}(t).
		\frac{P(t+h) - P(t)}{h} &= \frac{(P(h)- I)}{h}P(t). % = P(t)\frac{(P(h) - I)}{h}.
	\end{align*}
	%Dividing above equation by $h$,  and 
	Taking limits $h \downarrow 0$ and exchanging limits and summation, justified below, on the RHS we get 
	\begin{align*}
		\frac{dP_{ij}(t)}{dt} = \sum_{k \neq i}Q_{ik}P_{kj}(t)-\lambda_iP_{ij}(t). 
	\end{align*}
	
	
	Now we justify the exchange of limit and summation. For any finite subset $F \subset S$, we have
	\begin{align*}
		\liminf_{h \downarrow 0} \sum_{k \neq i}\frac{P_{ik}(h)}{h}P_{kj}(t) \geq \sum_{k \in F\setminus \{i\} }\liminf_{h \downarrow 0}\frac{P_{ik}(h)}{h}P_{kj}(t) = \sum_{k \in F\setminus \{i\}}Q_{ik}P_{kj}(t).
	\end{align*}
	Since, above is true for any finite set $F \subset E$, taking supremum over increasing sets $F$, we get the lower bound.
	%\begin{align*}
	%\liminf_{h \downarrow 0} \sum_{k \neq 1}\frac{P_{ik}(h)}{h}P_{kj}(t) \geq \sum_{k \neq i}Q_{jk}P_{kj}(t).
	%\end{align*}
	%%Suffices to show that $\limsup_{h \rightarrow 0} \sum_{k \neq 1}\frac{P_{ik}(h)}{h}P_{kj}(t) \leq \sum_{k \neq 1}q_{jk}P_{kj}(t)$. 
	For the upper bound, we observe for any finite subset $F \subseteq E$
	\begin{align*}
		\limsup_{h \downarrow 0}\sum_{k \neq i}\frac{P_{ik}(h)}{h}P_{kj}(t) &\leq \limsup_{h \downarrow 0}\left(\sum_{k \in F\setminus \{i\}}\frac{P_{ik}(h)}{h}P_{kj}(t)+\sum_{k \notin F\setminus\{i\} }\frac{P_{ik}(h)}{h}\right)\\
		& = \limsup_{h \downarrow 0}\left(\sum_{k \in F\setminus \{i\}}\frac{P_{ik}(h)}{h}P_{kj}(t)+\frac{1-P_{ii}(h)}{h} -\sum_{k \in F\setminus \{i\}}\frac{P_{ik}(h)}{h} \right)\\
		&= \sum_{k \in F\setminus \{i\}}Q_{ik}P_{kj}(t)+ \left(\lambda_i- \sum_{k \in F\setminus \{i\}}Q_{ik}\right). %&=\sum_{k \neq 1}q_{ik}P_{kj}(t). 
	\end{align*}
Now take $F \nearrow S$, then the term $\left(\lambda_i- \sum_{k \in F\setminus \{i\}}Q_{ik}\right)$ goes to zero.
	
\end{proof}

\begin{thm}[Forward equation] 
	%Under suitable regularity conditions, 
	For a homogeneous CTMC with transition matrix $P(t)$ and generator matrix $Q$, we have for the minimal construction,
	\eq{
		\frac{dP(t)}{dt} &=P(t)Q.
	}
	%$P'_{ij}(t)=\sum_{k \neq i}P_{ik}(t)q_{kj}-P_{ij}(t)\nu_i$, i.e. $\frac{dP(t)}{dt}=P(t)Q$.
\end{thm}
\begin{proof}
	Using semigroup property of transition probability matrix $P(t)$ for a homogeneous CTMC,  
	we can write
	\begin{align*}
		\frac{P(t+h) - P(t)}{h} &= P(t)\frac{(P(h) - I)}{h}.
	\end{align*}
	Taking limits $h \downarrow 0$, if we can justify the interchange of limit and summation on RHS, 
	\begin{align*}
		\frac{dP_{ij}(t)}{dt} = \sum_{k \neq j}P_{ik}(t)Q_{kj}-\lambda_jP_{ij}(t). 
	\end{align*}
 
\end{proof}

\begin{cor}
	For a homogeneous CTMC with finite state space $E$, the transition matrix $P(t)$ and generator matrix $Q$, we have
	\eq{
		P(t) &= e^{tQ} = I + \sum_{n \in \N}\frac{t^nQ^n}{n!},~t\geqslant 0.
	}
\end{cor}

%\begin{cor}
%	%For an irreducible and aperiodic 
%	For a homogeneous CTMC with finite state space $E$, the transition matrix $P(t)$ and generator matrix $Q$, the stationary distribution $\pi$ satisfies
%	\begin{xalignat*}{3}
%		&\pi Q = 0, &&\pi_i > 0, &&\sum_{i \in E}\pi_i = 1.
%	\end{xalignat*}
%\end{cor}
%\begin{proof}
%	We can define probability of being in state $j \in E$ at time $t$ by
%	\eq{
%		\pi_j(t) &= \Pr\{X(t) = j\}.% = \sum_{i \in E}\pi_i(0)P_{ij}(t).
%	}
%%	By Markov property, we have $\pi(t) = \{\pi_j(t): j \in E\}$ by
%Then
%	\eq{
%		\pi(t) &= \pi(0)P(t).
%	}
%	For a stationary distribution, we have $\pi = \pi P(t)$ and taking derivatives on both sides, we get
%\eq{
%0 = \pi \frac{dP_t}{dt} = \pi Q P_t.
%}
%
%Take $t = 0$, then $P_0 = I$ and we get the result.
%\end{proof}
\end{document}