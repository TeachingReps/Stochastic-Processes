
% !TEX spellcheck = en_US
% !TEX spellcheck = LaTeX
%\documentclass[a4paper,10pt,english]{article}
\documentclass[all-lectures.tex]{subfiles}
%\title{Lecture 00: Introduction}
%\author{Instructor: Prof. Vinod Sharma \\ Teaching assistants: Panju and Ajay}

\begin{document}
\maketitle
\chapter*{Introduction}
\section*{Topics:} 
The topics (a-f below) covered in the course are useful in the study of communication networks, computer networks and performance evaluation of algorithms among many other things. \\

\textit{Theory:} (a) Poisson processes, (b) renewal/regenerative processes, (c) discrete and continuous time Markov chains, (d) martingales and random walks. \\
\indent \textit{Applications:} (e) Single server  queues and (f) queuing networks. \\
\section*{Course structure:}
\begin{itemize}
\item Two sessions per week. Tuesdays and Thursdays between 5:15 p.m. to 6:45 p.m.
\item One assignment per topic. There will be a quiz based on each assignment. Tutorial and quiz sessions will be held by TAs on Saturdays 10-11 a.m. after each topic is finished.
\item One mid-term exam.
\item One final exam.
\end{itemize}

\begin{thebibliography}{1}
\bibitem{} S.M. Ross, "Stochastic Processes",2nd ed., 1996, Wiley.
\bibitem{} S. karlin and H.M. Taylor, "A First Course in Stochastic Processes", 2nd ed., 1975, Al. 
\bibitem{} B. Hajek, "Random Processes for Engineers", Cambridge Univesity press, 2015.
\bibitem{} S. Assmussen, "Applied Probability and Queues", 2nd ed., 2003, Springer. 
\bibitem{} J. Walrand, "An introduction to Queueing Netwoorks", Printice Hall, 1988.
\end{thebibliography}
\end{document}

