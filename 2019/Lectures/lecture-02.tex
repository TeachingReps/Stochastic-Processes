% !TEX spellcheck = en_US
% !TEX spellcheck = LaTeX
\documentclass[a4paper,10pt,english]{article}
\usepackage{%
	amsfonts,%
	amsmath,%	
	amssymb,%
	amsthm,%
%	babel,%
	bbm,%
	%biblatex,%
	caption,%
	centernot,%
	color,%
	enumerate,%
	epsfig,%
	epstopdf,%
	etex,%
	geometry,%
	graphicx,%
	hyperref,%
	latexsym,%
	mathtools,%
	multicol,%
	pgf,%
	pgfplots,%
	pgfplotstable,%
	pgfpages,%
	proof,%
	psfrag,%
	subfigure,%	
	tikz,%
	ulem,%
	url%
}	

\usepackage[mathscr]{eucal}
\usepgflibrary{shapes}
\usetikzlibrary{%
  arrows,%
  backgrounds,%
  chains,%
  decorations.pathmorphing,% /pgf/decoration/random steps | erste Graphik
  decorations.text,%
  matrix,%
  positioning,% wg. " of "
  fit,%
  patterns,%
  petri,%
  plotmarks,%
  scopes,%
  shadows,%
  shapes.misc,% wg. rounded rectangle
  shapes.arrows,%
  shapes.callouts,%
  shapes%
}

\theoremstyle{plain}
\newtheorem{thm}{Theorem}[section]
\newtheorem{lem}[thm]{Lemma}
\newtheorem{prop}[thm]{Proposition}
\newtheorem{cor}[thm]{Corollary}

\theoremstyle{definition}
\newtheorem{defn}[thm]{Definition}
\newtheorem{conj}[thm]{Conjecture}
\newtheorem{exmp}[thm]{Example}
\newtheorem{assum}[thm]{Assumptions}
\newtheorem{axiom}[thm]{Axiom}

\theoremstyle{remark}
\newtheorem{rem}[thm]{Remark}
\newtheorem{note}[thm]{Note}

\newcommand{\norm}[1]{\left\lVert#1\right\rVert}
\newcommand{\indep}{\!\perp\!\!\!\perp}
\DeclarePairedDelimiter\abs{\lvert}{\rvert}%
%\DeclarePairedDelimiter\norm{\lVert}{\rVert}%
\newcommand{\tr}{\operatorname{tr}}
\newcommand{\R}{\mathbb{R}}
\newcommand{\Q}{\mathbb{Q}}
\newcommand{\N}{\mathbb{N}}
\newcommand{\E}{\mathbb{E}}
\newcommand{\Z}{\mathbb{Z}}
\newcommand{\B}{\mathscr{B}}
\newcommand{\C}{\mathcal{C}}
\newcommand{\T}{\mathscr{T}}
\newcommand{\F}{\mathcal{F}}
\newcommand{\G}{\mathcal{G}}
%\newcommand{\ba}{\begin{align*}}
%\newcommand{\ea}{\end{align*}}

% Debug
\newcommand{\todo}[1]{\begin{color}{blue}{{\bf~[TODO:~#1]}}\end{color}}


\makeatletter
\def\th@plain{%
  \thm@notefont{}% same as heading font
  \itshape % body font
}
\def\th@definition{%
  \thm@notefont{}% same as heading font
  \normalfont % body font
}
\makeatother
\date{}
\title{Lecture 02: Poisson Processes (contd.)}
\author{}

\begin{document}
\maketitle
\section{Poisson Processes (Definition)}

\begin{thm*}[]
Definition $1$ of Poisson processes (see lecture-01) $\implies$ Definition $2$.
\begin{proof} 
\textbf{Step 1:} Density function of $n^{th}$ arrival $p_{S_n} = \frac{\lambda^n t^{n-1}}{(n-1)!} e^{-\lambda t}$. This can be shown using induction on $n$. $S_n = \sum_{k=1}^n A_k$, where $A_k \sim \mathrm{exp}(\lambda)$.\\
\textbf{Step 2:} Use the fact that $\{N_t = n\} = \{S_n \leq t < S_{n+1}\}$
\begin{align*}
\P\{N_t = n\} &= \P\{S_n \leq t < S_{n+1} \} \\
 &= \int_0^\infty \P\{S_n \leq t < S_{n+1} |S_n = s\} p_{S_n}(s) \\
 &= \int_0^t \P\{A_n > t-s\} p_{S_n}(s) &&\text{($\{S_n =  s<t, S_{n+1} > t\} \equiv A_n > t-s$)}\\
 &= 	\frac{(\lambda t)^n}{n!} e^{-\lambda t} &&\text{(Use the fact  that $A_n$ is $\mathrm{exp}(\lambda$))} \qedhere
\end{align*}
\end{proof}
\end{thm*}
\begin{thm*}[]
Definition $2$ of Poisson processes (see lecture-01) $\implies$ Definition $1$.
\begin{proof} 
\textbf{Step 1:} Show that $A_1$  must be exponential.
\begin{align*}
\P\{A_1 > t+s|A_1>t\} &= \P\{N_{t+s}-N_s = 0|N_t = 0\} \\
 &= \P\{N_{t+s}-N_s = 0\} &&\text{(increments are independent)}\\
 &= \P\{N_s = 0\} &&\text{(stationary increments)}\\
 &= \P\{A_1 > s\} \\  
 \implies \P\{A_1 > t+s\} &=  \P\{A_1 > s\} \P\{A_1 > t\}
\end{align*}
This leads to the functional equation whose only solution is that $A_1$ is an exponential r.v. (this was proved in lecture-01).\\
\textbf{Step 2:} Show that $A_2$ is independent of $A_1$ and has the same distribution. 
\begin{align*}
\P\{A_2>t|A_1 = s\} &= \P\{N_{t+s} -N_s = 0|N_s = 1\} \\
&= \P\{N_{t+s} -N_s = 0\} && \text{(independent increments)}\\
&= \P\{N_{t} = 0\} && \text{(stationary increments)}\\
&= \P\{A_1>t\}
\end{align*}
This shows that $A_2$ is independent of and also identically distributed as $A_1$. Similar argument holds for all other $A_n$s.
\end{proof}
\end{thm*}

\section{Properties of Poisson processes}
\subsection*{Conditional distribution of points in an interval:} Given that there are $n$ points in the interval $I = (a,b]$, these $n$ points are distributed uniformly in the interval $I$. Their joint distribution is given by oerder statistics of $n$ uniformly distributed point in the interval $I$.
\begin{proof}[Outline:]
For $h \downarrow 0$ and $I = (0,t]$:
\begin{align*}
&\P\{S_1\in (s_1,s_1+h],S_2\in (s_2,s_2+h],\cdots,S_n\in (s_n,s_n+h] |N_t = n\}  \\
&= \frac{\P\{S_1\in (s_1,s_1+h],S_2\in (s_2,s_2+h],\cdots,S_n\in (s_n,s_n+h],N_t=n\}}{\P\{N_t=n\}}  \\
 &= \frac{ (\lambda h)e^{-\lambda s_1} \times (\lambda h)e^{-\lambda (s_2-(s_1+h))} \times \ldots \times (\lambda h)e^{-\lambda (s_n-(s_{n-1}+h))} \times e^{-\lambda (t-(s_n+h))} }{ \frac{(\lambda t)^n }{n!} e^{-\lambda t} }\\
 &= \frac{ (\lambda h)^n e^{-\lambda(t-nh))}}{ \frac{(\lambda t)^n }{n!} e^{-\lambda t} } = \frac{n!}{t^n} h^n
\end{align*}
Now, we have 
\begin{align*}
&\P\{S_1\in (s_1,s_1+h],S_2\in (s_2,s_2+h],\cdots,S_n\in (s_n,s_n+h] |N_t = n\} \\
&= p_{S_1,S_2,\cdots S_n | N_t=n}(s_1,s_2,\cdots,s_n) \times h^n + o(h^n),
\end{align*}
where $p_{S_1,S_2,\cdots S_n | N_t=n}$ is the joint density of $S_1,S_2,\cdots,S_n$ conditioned on $N_t=n$. Therefore, 
\begin{align*}
&p_{S_1,S_2,\cdots S_n | N_t=n}(s_1,s_2,\cdots,s_n) = \frac{n!}{t}
\end{align*}
This is the density function of $n$ ordered random variables uniformly distributed in the interval $(0,t]$. By stationarity, this property holds for any interval. \qedhere
\end{proof}
\subsection*{Superposition of independent Poisson processes:}
If $n$ Poisson processes $N_t^{(1)},N_t^{(2)},\cdots,N_t^{(n)}$ of rates $\lambda_1,\lambda_2,\cdots,\lambda_n$ respectively are independent, then their supersposition is a Poisson process of the rate $\sum_{k=1}^n \lambda_k$. This follows from the fact that sum of independent Poisson random variables with mean $\lambda_1 t, \lambda_2 t, \cdots , \lambda_n t$ is a Poisson random variable with mean $\sum_{k=1}^n \lambda_k t$. We can then use Definition $3$ to conclude.
\subsection*{Splitting of Poisson processes:}
let $N_t$ be a Poisson process of the rate $\lambda$. Suppose each point of the process $N_t$ is marked \textit{independently} as type $i$ for $i \in \{1,2,\cdots,m\}$ with probaility $p_i$ such that $\sum_{i=1}^m p_i = 1$. Let the processes $\{N_t^{i}\}$ for $i \in \{1,2,\cdots,m\}$ be comprised of only those points marked as type $i$ respectively. Then,  $\{N_t^{i}\}$ are independent with respective rates $p_i \lambda$.
\end{document}

