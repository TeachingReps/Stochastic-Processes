% !TEX spellcheck = en_US
% !TEX spellcheck = LaTeX
%\documentclass[a4paper,10pt,english]{article}
\documentclass[all-lectures.tex]{subfiles}
%\usepackage{%
	amsfonts,%
	amsmath,%	
	amssymb,%
	amsthm,%
	algorithm,%
	babel,%
	bbm,%
	etex,%
	caption,%
	centernot,%
	color,%
	dsfont,%
	enumerate,%
	epsfig,%
	geometry,%
	graphicx,%
	hyperref,%
	latexsym,%
	mathtools,%
	multicol,%
	pgf,%
	pgfplots,%
	pgfplotstable,%
	pgfpages,%
	proof,%
	psfrag,%
	subfigure,%	
	tikz,%
	ulem,%
	url%
}	
\usepackage[noend]{algpseudocode}
\usepackage[mathscr]{eucal}
\usepgflibrary{shapes}
\usetikzlibrary{%
  arrows,%
  backgrounds,%
  chains,%
  decorations.pathmorphing,% /pgf/decoration/random steps | erste Graphik
  decorations.text,%
  fit,%
  matrix,%
  patterns,%
  petri,%
  positioning,% wg. " of "
  plotmarks,%
  scopes,%
  shadows,%
  shapes,%
  shapes.arrows,%
  shapes.callouts,%
  shapes.misc% wg. rounded rectangle
}

\theoremstyle{plain}
\newtheorem{thm}{Theorem}[section]
\newtheorem{lem}[thm]{Lemma}
\newtheorem{prop}[thm]{Proposition}
\newtheorem{cor}[thm]{Corollary}

\theoremstyle{definition}
\newtheorem{defn}[thm]{Definition}
\newtheorem{conj}[thm]{Conjecture}
\newtheorem{exmp}[thm]{Example}
\newtheorem{assum}[thm]{Assumptions}
\newtheorem{axiom}[thm]{Axiom}

\theoremstyle{remark}
\newtheorem{rem}{Remark}
\newtheorem{note}{Note}
\newtheorem{fact}{Fact}

\definecolor{lightgray}{gray}{0.9}

%\DeclarePairedDelimiter{\ceil}{\left\lceil}{\right\rceil}%
\newcommand{\eq}[1]{\begin{align*}#1\end{align*}}
\newcommand{\ceil}[1]{\left\lceil#1\right\rceil}%
\newcommand{\norm}[1]{\left\lVert#1\right\rVert}%
\newcommand{\indep}{\!\perp\!\!\!\perp}%
\DeclarePairedDelimiter\abs{\lvert}{\rvert}%
\newcommand\numberthis{\addtocounter{equation}{1}\tag{\theequation}}
\newcommand{\tr}{\operatorname{tr}}
\newcommand{\R}{\mathbb{R}}
\newcommand{\N}{\mathbb{N}}
\newcommand{\E}{\mathbb{E}}
\newcommand{\Z}{\mathbb{Z}}
\newcommand{\B}{\mathscr{B}}
\newcommand{\C}{\mathcal{C}}
\newcommand{\T}{\mathscr{T}}
\newcommand{\F}{\mathcal{F}}
\newcommand{\G}{\mathcal{G}}
\newcommand{\X}{\mathcal{X}}
%\newcommand{\ba}{\begin{align*}}
%\newcommand{\ea}{\end{align*}}
\DeclareMathOperator*{\argmax}{arg\,max}
\renewcommand{\qedsymbol}{$\blacksquare$}
\makeatletter
\def\BState{\State\hskip-\ALG@thistlm}
\makeatother

\makeatletter
\def\th@plain{%
  \thm@notefont{}% same as heading font
  \itshape % body font
}
\def\th@definition{%
  \thm@notefont{}% same as heading font
  \normalfont % body font
}
\makeatother
\date{}
%\title{Lecture 01: Poisson Processes}
%\author{}
\setcounter{chapter}{3}

\begin{document}
%\maketitle

%\chapter{Discrete Time Markov Chains}
\setcounter{section}{5}
\setcounter{subsection}{0}

\section*{\centering{Lecture 12 \linebreak }}
\section{Example: M/GI/1 queue}
Consider an $M/GI/1$ queue. Let $\lambda$  be the Poisson arrival rate, $S_k$ the service time of the $k^{th}$ customer and $\E[S]$ be the mean service time. Let 
\begin{itemize}
\item $q_k = $ queue length just after the $k^{th}$ departure. 
\item $\hat{q}_k = $ queue length just before the $k^{th}$ arrival. 
\item $q_t =$ queue length at arbitrary time $t$.
\item $W_k = $ waiting time of the $k^{th}$ customer.
\end{itemize}
The process $\{q_k\}$ satisfies $q_{k+1} = (q_k -1)^+ + A_{k+1}$ where $A_{k+1}$ is the number of arrivals during the service of $(k+1)^{th}$ customer. Since $\{A_k\}$ is i.i.d., $\{q_k\}$ is a Markov chain. The state space $S = \{0,1,2\dots\}$ and it is easy to check that it is aperiodic and irreducible. By choosing $f(i) = i$, we can deduce using the test for positive recurrence that $\{q_k\}$ is positive recurrent when $\E[A_1] = \lambda \E[S] < 1$. Thus, we conclude that when $\lambda \E[S] < 1$, the process $\{q_k\}$ has a stationary distribution. We will  see later that when $\lambda\E[S] = 1$, $\{q_k\}$ is recurrent and when $\lambda \E[S] > 1$, it is transient. \\
\indent  The process $\{\hat{q}_k\}$ however, is not a Markov chain if $S_k$ is not exponentially distributed. But $\{\hat{q}_k\}$ is a regenerative process with regeneration epochs occurring when $k^{th}$ arrival sees an empty queue. That is, regenerative epochs for $\{\hat{q}_k\}$ occur when $ \hat{q}_k = 0$. Let $\hat{\tau}$ be the regeneration length of $\{\hat{q}_k\}$. \\
\indent To obtain the conditions for existence for stationary distribution for $\{\hat{q}_k\}$, we can relate it to the process $\{q_k\}$. The process $\{q_k\}$ is also a regenerative process with regeneration epochs occurring when a departure leaves behind an empty queue. That is, the regeneration epochs correspond to $q_k = 0$. Let $\tau$ be the regeneration time of $\{q_k\}$. Now, we can see that $\tau = \hat{\tau}$. Since, $\{q_k\}$ has stationary distribution when $\lambda \E[S] < 1$, $\E[\tau] < \infty$. So, $\E[\hat{\tau}] < \infty$. Therefore, the process $\{\hat{q}_k\}$ is stationary iff $\{q_k\}$ is stationary. \\
\indent The process $\{W_k\}$ also has the same regeneration epochs as $\{q_k\}$ and hence has unique stationary distribution when $\lambda\E[S] < 1$.\\
\indent The process $\{q_t\}$ is also a regenerative process with regeneration epochs the arrival times that see an empty queue. Let $T$ be a regeneration length of $\{q_t\}$. Then, considering one regenerative cycle, we can write $T = \sum_{k = 1}^{\hat{\tau}} a_k$ where $a_k$ is the inter-arrival time between $(k-1)^{th}$ and $k^{th}$ arrival. Now, $\hat{\tau}$, which is equal to the number of services in one regeneration cycle is a stopping time for $\{a_k,S_k\}$ where $S_k$ is the service time of the $k^{th}$ arrival. Thus, we can use Wald's lemma to conclude that $\E[T] = \E[\hat{\tau}] \E[a_1]$. Thus, $\{q_t\}$ also has stationary distribution whenever $\lambda\E[S] < 1$.
\section{Rate of convergence to the stationary distribution}
The following results are stated without proofs.
\begin{defn}[Total Variation distance]
The total variation distance between two probability distributions $\mu$ and $\pi$ on $S$, 
\begin{align*}
||\mu - \pi||_{TV} = \frac{1}{2} \sum_{x\in S} (\mu(x) - \pi(x)).
\end{align*}
\end{defn}
The following results have been classically known.  We consider an irreducible Markov chain.
\begin{itemize}
\item If $\E[\tau^\alpha] < \infty$ for some $\alpha > 1$, then $|| X_k - \pi ||_{TV} < c_1 k^{-\alpha +1}$.
\item If $\E[\beta^{\tau}] < \infty$ for some $\beta > 1$, then $|| X_k - \pi||_{TV} \leq \exp(-\lambda k)$ for some $0 < \lambda < \beta$.
\end{itemize}
These results have been extensively used in the literature to obtain rates f convergence to stationary distributions for different queueing systems. \\
\indent Now, We consider a finite state space $S$. Let 
\begin{align*}
K_n^{x}(y) = \frac{P^n(x,y)}{\pi(y)}
\end{align*}
where $P$ is the transition probability matrix of the Markov chain. Then, $K_n^{x}(y) \rightarrow 1$ as $n \rightarrow \infty \ \forall x,y \in S$. 
\begin{defn}
$\mathcal{L}^p$ distance between distributions $P^n(x,*)$ and $\pi$, 
\begin{align*}
||K^x_n - 1||^p_{p,\pi} = \sum_{y\in S} |K^x_n(y)-1|^p \pi(y) \text{ for } 1 \leq p < \infty.
\end{align*}
Also, $\mathcal{L}^{\infty}(\nu,\mu) = \sup_{x\in S} |\mu(s) - \nu(s)|$.
\end{defn}

The following are known.
\begin{align}\label{eq:tau_bounds}
||\nu -\mu||_{TV} = \frac{1}{2} ||\frac{\nu}{\mu} -1||_{1,\mu} \leq \frac{1}{2} ||\frac{\nu}{\mu} -1 ||_{2,\mu}.
\end{align}
\begin{defn}[Mixing times]
\begin{align*}
\tau_1(\epsilon) &= \min\{n: \sup_x ||P^n(x,.)-\pi||_{TV} \leq \epsilon\}. \\
\tau_2(\epsilon) &= \min\{n: \sup_x ||K_n^x - 1||_{2,\pi} \leq \epsilon\}. \\
\tau_\infty(\epsilon) = &= \min\{n: \sup _x |P^n(x,*) - \pi|_\infty \leq \epsilon\}.
\end{align*}
\end{defn}

Let $\pi_* = \min_x \pi(x)$ and
\begin{align*}
||P^*|| = \sup_{f:S\to \R: \E[f] =0} \frac{||P^*f||_2}{||f||_2},
\end{align*}
where $P^*$ is the complex conjugate of $P$. 
\begin{prop}
\begin{align*}
\tau_2(\epsilon) \leq \frac{1}{1-||P^*||} \log\left(\frac{1}{\epsilon \sqrt{\pi_*}}\right)
\end{align*}
\end{prop}
\begin{prop}
\begin{align*}
\tau_2(\epsilon) \leq \frac{2}{\lambda_{PP^*}} \log\left( \frac{1}{\epsilon \sqrt{\pi_*}}\right) 
\end{align*}
where $\lambda_{PP^*} = 1-\lambda$ amd $\lambda$ is the largest eigenvalue of $PP^*$ less that 1.
\end{prop}
These results provide an exponential rate of convergence to the stationary ditribution for a finite state Markov chain. From $\tau_2(\epsilon)$ we get an upper bound on $\tau_1(\epsilon)$ through Eq (\ref{eq:tau_bounds}). We also have
\begin{align*}
\tau_2(\epsilon) \leq \tau_\infty (\epsilon) \leq \tau_2\left(\epsilon \sqrt{\frac{\pi_*}{1-\pi_*}}\,\right).
\end{align*}
If the Markov chain is reversible, the upper bound can be tightened to $2\tau_2(\sqrt{\epsilon})$. Thus, for most applications getting $\tau_2(\epsilon)$ is sufficient.  \\
\indent Mixing times have recently been used in Markov chain Monte Carlo (MCMC) algorithms, random graphs and many other applications.
\end{document}

