% !TEX spellcheck = en_US
% !TEX spellcheck = LaTeX
\documentclass[a4paper,10pt,english]{article}
\usepackage{%
	amsfonts,%
	amsmath,%	
	amssymb,%
	amsthm,%
%	babel,%
	bbm,%
	%biblatex,%
	caption,%
	centernot,%
	color,%
	enumerate,%
	epsfig,%
	epstopdf,%
	etex,%
	geometry,%
	graphicx,%
	hyperref,%
	latexsym,%
	mathtools,%
	multicol,%
	pgf,%
	pgfplots,%
	pgfplotstable,%
	pgfpages,%
	proof,%
	psfrag,%
	subfigure,%	
	tikz,%
	ulem,%
	url%
}	

\usepackage[mathscr]{eucal}
\usepgflibrary{shapes}
\usetikzlibrary{%
  arrows,%
  backgrounds,%
  chains,%
  decorations.pathmorphing,% /pgf/decoration/random steps | erste Graphik
  decorations.text,%
  matrix,%
  positioning,% wg. " of "
  fit,%
  patterns,%
  petri,%
  plotmarks,%
  scopes,%
  shadows,%
  shapes.misc,% wg. rounded rectangle
  shapes.arrows,%
  shapes.callouts,%
  shapes%
}

\theoremstyle{plain}
\newtheorem{thm}{Theorem}[section]
\newtheorem{lem}[thm]{Lemma}
\newtheorem{prop}[thm]{Proposition}
\newtheorem{cor}[thm]{Corollary}

\theoremstyle{definition}
\newtheorem{defn}[thm]{Definition}
\newtheorem{conj}[thm]{Conjecture}
\newtheorem{exmp}[thm]{Example}
\newtheorem{assum}[thm]{Assumptions}
\newtheorem{axiom}[thm]{Axiom}

\theoremstyle{remark}
\newtheorem{rem}[thm]{Remark}
\newtheorem{note}[thm]{Note}

\newcommand{\norm}[1]{\left\lVert#1\right\rVert}
\newcommand{\indep}{\!\perp\!\!\!\perp}
\DeclarePairedDelimiter\abs{\lvert}{\rvert}%
%\DeclarePairedDelimiter\norm{\lVert}{\rVert}%
\newcommand{\tr}{\operatorname{tr}}
\newcommand{\R}{\mathbb{R}}
\newcommand{\Q}{\mathbb{Q}}
\newcommand{\N}{\mathbb{N}}
\newcommand{\E}{\mathbb{E}}
\newcommand{\Z}{\mathbb{Z}}
\newcommand{\B}{\mathscr{B}}
\newcommand{\C}{\mathcal{C}}
\newcommand{\T}{\mathscr{T}}
\newcommand{\F}{\mathcal{F}}
\newcommand{\G}{\mathcal{G}}
%\newcommand{\ba}{\begin{align*}}
%\newcommand{\ea}{\end{align*}}

% Debug
\newcommand{\todo}[1]{\begin{color}{blue}{{\bf~[TODO:~#1]}}\end{color}}


\makeatletter
\def\th@plain{%
  \thm@notefont{}% same as heading font
  \itshape % body font
}
\def\th@definition{%
  \thm@notefont{}% same as heading font
  \normalfont % body font
}
\makeatother
\date{}
\title{Lecture 01: Poisson Processes}
\author{}

\begin{document}
\maketitle
\section{Introduction to stochastic processes}

\indent \textbf{Review:} Let $(\Omega,\sigma,\mathbb{P})$ be a probability space. A measurable mapping $X:\Omega \to \mathbb{R}$ is called a random variable (r.v.). $X(\omega$ for $\omega \in \Omega$ is called a realization of $X$. $F_X (x) = \mathbb{P}[X\leq x]$ is called the distribution function of r.v. $X$. The distribution function of a random variable always exists. $f_X(x) = dF(x)/dx$ is called the probability density function of $X$. The probability density function may not always exist. $\mathbb{E}[X] = \int x dF_X(x) $ is the expectation of $X$. When probability denisty of $X$ exists for r.v. $X$, $\mathbb{E}[X] = \int xf(x)dx$.\\


\textbf{Stochastic processes:} $\{X_t : t \in \R\}$, where $X_t$ is a r.v. is called a continous time stochastic process. $\{X_n : n \in \N\}$, where $X_n$ is a r.v. is called a discrete time stochastic process. \\
\indent The function $t \mapsto X_t(\omega)$ is called a sample path of the stochastic process. For each $\omega \in \Omega$, $X_t(\omega)$ is a function of $t$. $F_t$ is the distribution of $X_t$. An analogous definition holds for discrete time stocastic processes. A stochastic process is described by the joint distribution of $(X_{t_1},X_{t_2},\ldots,X_{t_n})$ for any  $-\infty < t_1 < t_2 < \ldots < t_n.$ and $n \in \N^+$.\\
\indent A stochastic process $\{X_t\}$ is said to be \textit{stationary} if the joint distribution of $(X_{t_1},X_{t_2},\ldots,X_{t_n})$ is identical to the joint distribution of $(X_{t_1+\tau},X_{t_2+\tau},\ldots,X_{t_n+\tau})$ for any $\tau \in \R$ and any $t_1,t_2,\ldots,t_n$. A stochastic process $\{X_t\}$ is said to have \textit{independent increments} if $(X_{t_1} - X_{s_1})$ is independent of $(X_{t_2} -X_{s_2})$ whenever $(s_1,t_1] \cap (s_2,t_2] = \phi$. If $X_{t+\tau} - X_t$ has the same distribution as $X_\tau - X_0\ \forall t$, then $X_t$ is said to have \textit{stationary increments}. If further, $X_{t+\tau} - X_t$ is independent of $X_t - X_0$, $X_t$ called process with \textit{stationary independent increment} process. \\

\textbf{Point process:} A stchastic process $\{N_t, t > 0\}$ with $N_0 = 0$, $N_t$ a non-negative integer, non-descreasing piecewise constant sample paths is called a point process. $N_t$ counts the number of points or 'arrivals' in the interval $(0,t]$. \\
\indent Let $A_n$ denote the interarrival time between $n^{th}$ and $(n-1)^{th}$ arrival. Let $S_0 = 0$ and $S_n = \sum^n_{k=1} A_k \forall n \geq 1$. Then $S_n$ denotes the time instant of the $n^{th}$ arrival. $N_t = \max\{n: S_n \leq t\}$. A point process with atmost one arrival at any time is called a \textit{simple point process}.\\
\indent Mathematically, a simple point process $\{N_t\}$ is described by following constraints for all $t$:
\begin{align*}
\P\{N_{t+h} - N_t=  0 \} &= 1-\lambda h + o(h),\\
\P\{N_{t+h} - N_t=  1 \} &= \lambda h +o(h) \text{ and}\\
\P\{N_{t+h} - N_t \geq  2 \} &= o(h).\\
\end{align*} 
Here, the notation $o(g(x))$ means a class of functions such that if $f(x) \in o(g(x))$, then $\lim_{x\rightarrow 0} \frac{f(x)}{g(x)} =0$.
\section{Poisson process}
In the following we customaritly take $N-0 = 0$. A point process $N_t$ is Poisson if 
\paragraph{Definition 1:}{
\begin{itemize}
\item $\{A_k, k \geq 1\}$ are independent and exponentially distributed with parameter $\lambda$.
\end{itemize}
$\P\{A_k \leq x\} = 1 - e^{-\lambda x}$. If $\lambda = 0$, $A_1 = \infty$ w.p.1. and $N_t = 0 \ \forall t$. If $\lambda = \infty$, $A_1 = 0$ w.p.1 and $N_t = \infty \ \forall t$. We restrict to $0<\lambda<\infty$. In this range for $\lambda$, $N_t$ is guaranteed to be simple because $\P\{A_k = 0\} = 0 \ \forall k$.
}
\paragraph{Definition 2:}{
\begin{itemize}
\item $N_t$ is simple.
\item $N_t$ has stationary independent increments.
\end{itemize}
}
\paragraph{Definition 3:}{
\begin{itemize}
\item $N_t$ has independent increment.
\item $N_t$ is a Poisson distributed random variable. For $s < t$, 
	\begin{align*}
	\P\{N_t - N_s = n\} = \frac{(\lambda (t-s))^n}{n!} e^{-\lambda (t-s)}
	\end{align*}
\end{itemize}
}
\ \\
\textbf{Exponential r.v. is memoryless:} Let $X$ be an exponential r.v.
\begin{align*}
\P\{X > t+s|X>t\} &= \frac{\P\{X\ > t+s,X>t\}}{\P\{X>t\}} \\
&= \frac{\P\{X\ > t+s\}}{\P\{X>t\}} \\
&= \frac{e^{-\lambda (t+s)}}{e^{-\lambda t}} \\
&= e^{-\lambda s} \\
&= \P\{X > s\} 
\end{align*}
If $X$ is interarrival time, this property of an exponential r.v. indicates that remaining time till next arrival does not depend time since last arrival. Thus, the term memoryless is used.

\begin{thm}[] 
Exponential distribution is the unique continous distribution with memoryless property.
\begin{proof}
If a continous r.v. $X$ is memoryless, we show that $X$ must be exponential. If $X$ is memoryless, we have for all $t,s\geq 0\ \P\{X>t+s\} = \P\{X>t\} \P\{X>s\}$. Let $f(t) = \P\{X>t\}$. We have the functional equation
\begin{align}
\label{eq:memeryless_func_eq}
f(t+s) = f(t) f(s)
\end{align}
Taking $t=s$, we get $f(2t) = f^2(t)$. By repeated application of Eq \ref{eq:memeryless_func_eq} $m$ times, we get $f(mt) = f^m(t)$ for positive integer $m$. Equivalently, we have $f(t/m) =f^{\frac{1}{m}}(t)$. Again by repeated application of Eq \ref{eq:memeryless_func_eq} $n$ times, $f(\frac{n}{m}t) =f^{\frac{n}{m}}(t)$ for any positive integers $m$ and $n$. So, we have $f(rt) = f^r(t)$ for any positive rational number $r$. We know that $0\leq f(t) \leq 1$ since $1-f$ is probability distribution. So, we can write $f(1) = e^{-\lambda}$ for some $\lambda > 0$. Therefore we have, $f(r) = f(r \times 1) = f^r(1) = e^{-\lambda r}$ for any positive rational number $r$. \\
\indent For any $x \in \R$, there is a sequence of rationals $r_n \rightarrow x$. By assumption $X$ is a continous r.v. This implies $f$ is continous. Therefore, $f(r_n) \rightarrow f(x)$. In the other words, for any $x \in \R$,
\begin{align*}
f(x) &= \lim_{r_n \rightarrow x} f(r_n) \\
 &= \lim_{r_n \rightarrow x} e^{-\lambda r_n} \\
 &= e^{-\lambda x} \\ 
\end{align*}
Thus, $\P\{X>x\} = e^{-\lambda x}$ and $X$ is an exponential random variable.
\end{proof}
\end{thm}

Definitions $[1-3]$ for Poisson process given above are equivalent. 

\begin{thm}[] 
Definitions $3$ $\implies$ Definition $2$.
\begin{proof}
We need to show that $N_t$ has stationary increments if $N_t$  has independent increments and $N_t-N_s$ is Poisson distributed with mean $\lambda(t-s)$. Stationarity follows directly from the definition since the distribution of number of points in an interval depends only in the length of interval. The conditions for a simple process is also met which can be easily verified from the definition:
\begin{align*}
\P\{N_{t+h} - N_t=  0 \} &= \frac{(\lambda h)^0}{0!} e^{-\lambda h} = 1 - \lambda h + o(h),\\
\P\{N_{t+h} - N_t=  1 \} &= \frac{(\lambda h)^1}{1!} e^{-\lambda h} = \lambda h +o(h) \text{ and}\\
\P\{N_{t+h} - N_t \geq  2 \} &= 1- \left(\frac{(\lambda h)^0}{0!} e^{-\lambda h} + \frac{(\lambda h)^1}{1!} e^{-\lambda h}\right)\\
&= o(h)
\end{align*} 
\end{proof}
\end{thm}


\end{document}

