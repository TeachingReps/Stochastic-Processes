% !TEX spellcheck = en_US
% !TEX spellcheck = LaTeX
\documentclass[a4paper,10pt, english]{article}
\usepackage{%
	amsfonts,%
	amsmath,%	
	amssymb,%
	amsthm,%
%	babel,%
	bbm,%
	%biblatex,%
	caption,%
	centernot,%
	color,%
	enumerate,%
	epsfig,%
	epstopdf,%
	etex,%
	geometry,%
	graphicx,%
	hyperref,%
	latexsym,%
	mathtools,%
	multicol,%
	pgf,%
	pgfplots,%
	pgfplotstable,%
	pgfpages,%
	proof,%
	psfrag,%
	subfigure,%	
	tikz,%
	ulem,%
	url%
}	

\usepackage[mathscr]{eucal}
\usepgflibrary{shapes}
\usetikzlibrary{%
  arrows,%
  backgrounds,%
  chains,%
  decorations.pathmorphing,% /pgf/decoration/random steps | erste Graphik
  decorations.text,%
  matrix,%
  positioning,% wg. " of "
  fit,%
  patterns,%
  petri,%
  plotmarks,%
  scopes,%
  shadows,%
  shapes.misc,% wg. rounded rectangle
  shapes.arrows,%
  shapes.callouts,%
  shapes%
}

\theoremstyle{plain}
\newtheorem{thm}{Theorem}[section]
\newtheorem{lem}[thm]{Lemma}
\newtheorem{prop}[thm]{Proposition}
\newtheorem{cor}[thm]{Corollary}

\theoremstyle{definition}
\newtheorem{defn}[thm]{Definition}
\newtheorem{conj}[thm]{Conjecture}
\newtheorem{exmp}[thm]{Example}
\newtheorem{assum}[thm]{Assumptions}
\newtheorem{axiom}[thm]{Axiom}

\theoremstyle{remark}
\newtheorem{rem}[thm]{Remark}
\newtheorem{note}[thm]{Note}

\newcommand{\norm}[1]{\left\lVert#1\right\rVert}
\newcommand{\indep}{\!\perp\!\!\!\perp}
\DeclarePairedDelimiter\abs{\lvert}{\rvert}%
%\DeclarePairedDelimiter\norm{\lVert}{\rVert}%
\newcommand{\tr}{\operatorname{tr}}
\newcommand{\R}{\mathbb{R}}
\newcommand{\Q}{\mathbb{Q}}
\newcommand{\N}{\mathbb{N}}
\newcommand{\E}{\mathbb{E}}
\newcommand{\Z}{\mathbb{Z}}
\newcommand{\B}{\mathscr{B}}
\newcommand{\C}{\mathcal{C}}
\newcommand{\T}{\mathscr{T}}
\newcommand{\F}{\mathcal{F}}
\newcommand{\G}{\mathcal{G}}
%\newcommand{\ba}{\begin{align*}}
%\newcommand{\ea}{\end{align*}}

% Debug
\newcommand{\todo}[1]{\begin{color}{blue}{{\bf~[TODO:~#1]}}\end{color}}


\makeatletter
\def\th@plain{%
  \thm@notefont{}% same as heading font
  \itshape % body font
}
\def\th@definition{%
  \thm@notefont{}% same as heading font
  \normalfont % body font
}
\makeatother
\date{}
\title{Lecture-05}
\author{}

\begin{document}
\maketitle
\section{Limit Theorems}
\begin{prop}
The total number of renewals that occur are $N(\infty)=\infty$ with probability 1.
\end{prop}
\begin{proof}
\eq{
\lim_{n\to\infty} \P\{ N_t \ge n\} = \lim_{n\to\infty} \P\{S_n \le t\} \Rightarrow \P\{N_t < \infty\} =1.
}
\eq{
\lim_{t\to\infty} \P\{ N_t \ge n\} = \lim_{t\to\infty} \P\{S_n \le t\} =1.
}
Hence, $\lim_{t\to\infty} N(t) = \infty$ $a.s$.
This also tells that the only way in which $N(\infty)$-the total number of renewals that occurs can be finite, is for one of the interarrival times to be infinite.
\end{proof}

Let us denote the random variable $S_{N(t)}$, the time of the last renewal \textit{prior to or at} time $t$ and $S_{N(t)+1}$ represents the time of the first renewal \textit{after} time $t$.

\begin{prop}
With probability 1,
\eq{
\lim_{t\to\infty} \frac{N(t)}{t} \to \frac{1}{\mu}.
}
\end{prop}

\begin{proof}
Since, $S_{N(t)} < t < S_{N(t)+1}$, we see that
\eq{
\frac{S_{N(t)}}{N(t)} < \frac{t}{N(t)} < \frac{S_{N(t)+1}}{N(t)}
}

Since, $\frac{S_{N(t)}}{N(t)}$ is the average of the first $N(t)$ interarrival times, it follows that by the strong law of large numbers that 
$\frac{S_{N(t)}}{N(t)} \to \mu$ as $N(t) \to \infty$. But since $N(t)\to \infty$ when $t \to \infty$, we have
\eq{
\lim_{t\to\infty} \frac{S_{N(t)}}{N(t)} \to \mu.
}
Further, we can write $\frac{S_{N(t)+1}}{N(t)}$ as follows,
\eq{
\frac{S_{N(t)+1}}{N(t)} = \Big{[}\frac{S_{N(t)+1}}{N(t)+1}\Big{]} \Big{[}\frac{N(t)+1}{N(t)}\Big{]}.
}
By using the same argument as above,
\eq{
\lim_{t\to\infty} \frac{S_{N(t)+1}}{N(t)} \to \mu.
}
Hence the result follows.
\end{proof}

\section{Elementary Renewal Theorem}
\begin{thm}
\eq{
\lim_{t \to \infty} \frac{\E[N(t)]}{t} = \frac{1}{\mu}
}
\end{thm}

\begin{proof}
Let us denote $\E[N(t)]$ as $m(t)$ and suppose $\mu < \infty$.

We know that $S_{N(t)+1} > t$, taking expectations we get,
\eq{
\mu(m(t)+1) > t,
}
which implies 
\eq{
\liminf_{t \to \infty} \frac{m(t)}{t} \ge \frac{1}{\mu}.
}

To get the other way, define a new renewal process $\{ A'_k, k=1,2,\dots \}$ as $A'_k= \min(M,A_k)$ for a constant $M$.
Let $S'_n=\sum_1^n A'_k$ and $N'(t) = sup\{n,S'_n \le t\}$. The interarrival times for this truncated renewal process are bounded by $M$, therefore
\eq{
S'_{N(t)+1} \le t+M
}
Taking expectations on both sides we get,
\eq{
\mu_M(m'(t)+1) \le t+M
}
where $\mu_M = \E[A'_k]$, thus
\eq{
\limsup_{t \to \infty} \frac{m'(t)}{t} \le \frac{1}{\mu_M}.
}

Letting $M \to \infty$ yields
\eq{
\limsup_{t \to \infty} \frac{m(t)}{t} \le \frac{1}{\mu}.
}
Hence the result follows.
\end{proof}

\begin{defn}
(Stopping Time)

$N$ a nonnegative integer valued random variable is a stopping time w.r.t. sequence $\{X_k\}$ if $\{ N\le n\}$ is independent of $\{X_{n+1}, X_{n+2}, \dots\}$ .
\end{defn}

\begin{thm}
(Wald's Lemma)

If $N$ is a stopping time w.r.t. $\{X_k\}$ and $\E[X_1] = \mu < \infty$, $E[N] < \infty$ then
\eq{
\E[\sum_{k=1}^{N} X_k] = \E[N] \E[X_1].
}
\end{thm}

\begin{proof}
Let 
\eq{
I_k = \begin{cases}
1,  & N \ge k\\
0,  & N < k
\end{cases}
}

Then we can write the following
\eq{
\sum_{k=1}^{N} X_k = \sum_{k=1}^{\infty} X_k I_k.
}

Hence,
\eq{
\E[\sum_{k=1}^{N} X_k] = \E[\sum_{k=1}^{\infty} X_k I_k] = \sum_{k=1}^{\infty} \E[X_k I_k].
}
Since $I_k$ is deterined by $X_1, X_2, \dots, X_{k-1}$, therefore $I_k$ is independent of $X_k$. Thus we obtain
\eq{
\E[\sum_{k=1}^{N} X_k] &= \sum_{k=1}^{\infty} \E[X_k] \E[I_k] \\
&= \E[X_1] \sum_{k=1}^{\infty} \E[I_k]\\
&= \E[X_1] \sum_{k=1}^{\infty} \P\{ N \ge k\}\\
&= \E[X_1] \E[N].
}
Hence the proof.
\end{proof}


\end{document}