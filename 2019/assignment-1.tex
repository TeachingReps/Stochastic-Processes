% !TEX spellcheck = en_US
% !TEX spellcheck = LaTeX
\documentclass[a4paper,10pt,english]{article}
\usepackage{%
	amsfonts,%
	amsmath,%	
	amssymb,%
	amsthm,%
	algorithm,%
	babel,%
	bbm,%
	etex,%
	caption,%
	centernot,%
	color,%
	dsfont,%
	enumerate,%
	epsfig,%
	geometry,%
	graphicx,%
	hyperref,%
	latexsym,%
	mathtools,%
	multicol,%
	pgf,%
	pgfplots,%
	pgfplotstable,%
	pgfpages,%
	proof,%
	psfrag,%
	subfigure,%	
	tikz,%
	ulem,%
	url%
}	
\usepackage[noend]{algpseudocode}
\usepackage[mathscr]{eucal}
\usepgflibrary{shapes}
\usetikzlibrary{%
  arrows,%
  backgrounds,%
  chains,%
  decorations.pathmorphing,% /pgf/decoration/random steps | erste Graphik
  decorations.text,%
  fit,%
  matrix,%
  patterns,%
  petri,%
  positioning,% wg. " of "
  plotmarks,%
  scopes,%
  shadows,%
  shapes,%
  shapes.arrows,%
  shapes.callouts,%
  shapes.misc% wg. rounded rectangle
}

\theoremstyle{plain}
\newtheorem{thm}{Theorem}[section]
\newtheorem{lem}[thm]{Lemma}
\newtheorem{prop}[thm]{Proposition}
\newtheorem{cor}[thm]{Corollary}

\theoremstyle{definition}
\newtheorem{defn}[thm]{Definition}
\newtheorem{conj}[thm]{Conjecture}
\newtheorem{exmp}[thm]{Example}
\newtheorem{assum}[thm]{Assumptions}
\newtheorem{axiom}[thm]{Axiom}

\theoremstyle{remark}
\newtheorem{rem}{Remark}
\newtheorem{note}{Note}
\newtheorem{fact}{Fact}

\definecolor{lightgray}{gray}{0.9}

%\DeclarePairedDelimiter{\ceil}{\left\lceil}{\right\rceil}%
\newcommand{\eq}[1]{\begin{align*}#1\end{align*}}
\newcommand{\ceil}[1]{\left\lceil#1\right\rceil}%
\newcommand{\norm}[1]{\left\lVert#1\right\rVert}%
\newcommand{\indep}{\!\perp\!\!\!\perp}%
\DeclarePairedDelimiter\abs{\lvert}{\rvert}%
\newcommand\numberthis{\addtocounter{equation}{1}\tag{\theequation}}
\newcommand{\tr}{\operatorname{tr}}
\newcommand{\R}{\mathbb{R}}
\newcommand{\N}{\mathbb{N}}
\newcommand{\E}{\mathbb{E}}
\newcommand{\Z}{\mathbb{Z}}
\newcommand{\B}{\mathscr{B}}
\newcommand{\C}{\mathcal{C}}
\newcommand{\T}{\mathscr{T}}
\newcommand{\F}{\mathcal{F}}
\newcommand{\G}{\mathcal{G}}
\newcommand{\X}{\mathcal{X}}
%\newcommand{\ba}{\begin{align*}}
%\newcommand{\ea}{\end{align*}}
\DeclareMathOperator*{\argmax}{arg\,max}
\renewcommand{\qedsymbol}{$\blacksquare$}
\makeatletter
\def\BState{\State\hskip-\ALG@thistlm}
\makeatother

\makeatletter
\def\th@plain{%
  \thm@notefont{}% same as heading font
  \itshape % body font
}
\def\th@definition{%
  \thm@notefont{}% same as heading font
  \normalfont % body font
}
\makeatother
\date{}
\title{Assignment 1}
\author{SPQT - 2019}

\begin{document}
\maketitle
\textbf{Problem 1:} An item has a random lifetime with exponential distribution with parameter $\lambda$. Each time it fails it is immediately replaced by an identical item. Let $N_t$ be the number of failures till time $t$. Show that $\{N_t, t \geq 0\}$ is a Possion process. Find the mean and variance of the total time $T$ when the fifth item fails. \\
\indent \textbf{Problem 2:} Let $A-1, A-2, \ldots, A_n$ be disjoint intervals on $\R^+$ with union $B$. Let $a_1,a_2,\ldots,a_n$ be their respective lengths. Let $b = \sum_{k=1}^n a_k$. Then for $k = k_1+k_2,\ldots,k_n$, show 
\begin{align*}
\P\{N_{A_1}=k_1,N_{A_2}=k_2,\ldots,N_{A_n}=k_n\} = \frac{k!}{k_1!k_2!\ldots k_n!}\left(\frac{a_1}{b}\right)^{k_1}\left(\frac{a_2}{b}\right)^{k_2}\ldots\left(\frac{a_n}{b}\right)^{k_n}
\end{align*}
\indent \textbf{Problem 3:} A department has three doors. Arrivals at each door form Posiion process with rates $\lambda_1 = 110$, $\lambda_2 = 90$ and $\lambda = 160$ customers/sec. $30\%$ of the customers are male and $70\%$ are female. Probability that a male customer buys a item $0.6$ and probability that a female buys an item is $0.1$. An average purchase is worth $Rs 4.50$. Assume all the ransom variables are independent. What is the probability that the third female to purchase anything arrives during the first $15$ minutes? What is the expected time of her arrival?\\
\indent \textbf{Problem 4:} The customers arrive at a facilty as a Poisson process with rate $\lambda$. There is a waiting cost of $c$ per customer per unit time. The customers wait till they are dispacthed. The dispatching takes place at times $T,2T,\ldots$. At time $kT$ all customers in waiting will be dispatched. There is dispatching cost of $\beta$ per customer. 
\begin{enumerate}
\item What is the total dispatch cost till time $t$.
\item What is the mean total customer waiting time till time $t$.
\item What value of $T$ minimizes the mean total customer and dispatch cost per unit time. 
\end{enumerate}

\indent \textbf{Problem 5:} Let $N_t$ be a Poisson process with rate $\lambda$ and let the $n^{th}$ arrival epoch be $S_n$. Calculate $\E[S_5|N_t = 3]$. \\
\indent \textbf{Problem 6:} Let $N^{(1)}_t$ and $N^{(2)}_t$ be two  Poisson processes with rates $\lambda_1$ and $\lambda_2$. Let the $n^{th}$ arrival epoch be $S^{(1)}_n$ and $S^{(2)}_n$ respectively. Calculate
\begin{enumerate}
\item $\P\{ S^{(1)}_1 < S^{(2)}_1\}$
\item $\P\{ S^{(1)}_2 < S^{(2)}_2\}$
\end{enumerate} 

\indent \textbf{Problem 7:} Shocks occur to a system according to a Poisson process $N_t$ of intensity $\lambda$. Each shock causes some damage to the system and these damages accumalate. Let $Y_i$ be the damage caused by the $i^{th}$ shock. Assume $Y_i$s are independent of each other and $N_t$. $X_t = \sum_{k=1}^{N_t} Y_k$ is the total damage till time $t$. Suppose the system continues to operate till $X_t > \alpha$ where $\alpha > 0$. If $\P\{Y_i = k\} = (1-\gamma)^{k-1}\gamma, \ k = 1,2,\ldots$ Calculate the mean time till system failure.  \\
\indent \textbf{Problem 8 (M/M/1 queue):} A Poisson process $N_t$ with rate $\lambda$ form the arrivals to a queue. each customer requires and i.i.d. service time of exponential distribution with rate $\mu$. Let $q_t$ be the number of customers at time $t$, $D_t$ the number of customers departed till time $t$. Then $q_t = N_t - D_t$. 
\begin{enumerate}
\item Calculate $\P\{D_t=m|N_t=n\}$
\item Calculate $\P\{q_t=m|N_t=n\}$
\item Calculate $\P\{q_t=m\}$, $\E[q_t^k]$ for $k=1,2,\ldots$
\item Let $q_n$ be the queue length when the $n_{th}$ customer arrives excluding itself. Calculate $\P[q_n = n|q_{n-1} = m]$.
\end{enumerate}

\indent \textbf{Problem 9 [Ross, Exercise 2.12]:} event, occuring according to a Poisson process with rate $\lambda$, are registered by a counter. However, each time an event is registered the counter becomes inoperative for the next $b$ units of time and does not register any new events that might occur during this time. Let $R_t$ denote the number of events that occurby time $t$ and are registered. 
\begin{enumerate}
\item Find the probability that the first k events are all registered.
\item For $t \geq (n-1)b$, find $\P\{R(t) \geq n\}$.
\end{enumerate}
\end{document}

