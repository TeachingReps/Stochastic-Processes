% !TEX spellcheck = en_US
% !TEX spellcheck = LaTeX
%\documentclass[a4paper,10pt,english]{article}
\documentclass[all-lectures.tex]{subfiles}

%\title{Lecture 02: Poisson Processes (contd.)}
%\author{}

\begin{document}
%\maketitle
\setcounter{chapter}{1}
\setcounter{section}{2}

\section*{\centering{Lecture 2 \linebreak }}
\setcounter{subsection}{0}
\subsection{Poisson Processes (Definition)}
\begin{prop}[]
Definition $[1]$ of Poisson processes (see lecture-01) $\implies$ Definition $[3]$.
\begin{proof} 
\textbf{Step 1:} Density function of $n^{th}$ arrival $p_{S_n}(t) = \frac{\lambda^n t^{n-1}}{(n-1)!} e^{-\lambda t}$. This can be shown using induction on $n$. $S_n = \sum_{k=1}^n A_k$, where $A_k \sim \mathrm{exp}(\lambda)$. For $n=1$ and $S_1 = A_1$, the expresssion is true since it reduces to density function of an expoenential random variable with mean $\lambda$. Now, assuming it is true for $S_n$, we will show that it is true for $S_{n+1}$.
\begin{align*}
p_{S_{n+1}}(t) &= p_{S_n} * A_{n+1} && \text{(density of sum of ind. r.v.s is their convolution)} \\
&= \int_{\tau = 0}^{t} \left(\frac{\lambda^n \tau^{n-1}}{(n-1)!} e^{-\lambda \tau} \right) \left( \lambda e^{ - \lambda (t-\tau)} \right) d\tau \\
&=  e^{-\lambda t} \lambda^{n+1}  \int_{\tau = 0}^{t} \frac{\tau^{n-1}}{(n-1)!} d\tau \\
&=  e^{-\lambda t} \lambda^{n+1}  \frac{t^n}{n!}
\end{align*}

\textbf{Step 2:} 
\begin{align*}
\P\{N_t = n\} &= \P\{S_n \leq t < S_{n+1} \} \\
 &= \int_0^\infty \P\{S_n \leq t < S_{n+1} |S_n = s\} p_{S_n}(s) \\
 &= \int_0^t \P\{A_{n+1} > t-s\} p_{S_n}(s) &&\text{($\{S_n =  s<t, S_{n+1} > t\} \equiv \{A_{n+1} > t-s\}$)}\\
 &= 	\frac{(\lambda t)^n}{n!} e^{-\lambda t} &&\text{(Use the fact  that $A_n$ is $\mathrm{exp}(\lambda$))} \qedhere
\end{align*}
Independent increments and stationary increments property follows from $A_k$ begin i.i.d. with $\exp(\lambda)$.
\end{proof}
\end{prop}
\begin{prop}[]
Definition $[2]$ of Poisson processes (see lecture-01) $\implies$ Definition $[1]$.
\begin{proof} 
\textbf{Step 1:} Show that $A_1$  must be exponential. 
\begin{align*}
\P\{A_1 > t+s|A_1>t\} &= \P\{N_{t+s}-N_t = 0|N_t = 0\} \\
 &= \P\{N_{t+s}-N_t = 0\} &&\text{(increments are independent)}\\
 &= \P\{N_s = 0\} &&\text{(stationary increments)}\\
 &= \P\{A_1 > s\} \\  
 \implies \P\{A_1 > t+s\} &=  \P\{A_1 > s\} \P\{A_1 > t\}
\end{align*}
This leads to the functional equation whose only right continuous solution is that $A_1$ is an exponential r.v. (this was proved in lecture-01).\\
\textbf{Step 2:} Show that $A_2$ is independent of $A_1$ and has the same distribution. 
\begin{align*}
\P\{A_2>t|A_1 = s\} &= \P\{N_{t+s} -N_s = 0|N_s = 1\} \\
&= \P\{N_{t+s} -N_s = 0\} && \text{(independent increments)}\\
&= \P\{N_{t} = 0\} && \text{(stationary increments)}\\
&= \P\{A_1>t\}
\end{align*}
This shows that $A_2$ is independent of and also identically distributed as $A_1$. Similar argument holds for all other $A_n$s.
\end{proof}
\end{prop}


\begin{prop}
Definition $[3] \implies Definition [4]$ 
\begin{proof}
We have already shown that definition $[3]$ implies stationary increments. We now show that it implies $(2)$ in definition $[4]$.
\begin{align*}
\P\{N_{t+h} - N_{t}  &= 1 \} = \frac{\lambda h e^{-\lambda h}}{1!} \\
&= \lambda h (1 - \lambda h + o(h)) \\
&= \lambda h + o(h)
\end{align*}
which shows $[4]-(2)-(a)$. 
\begin{align*}
\P\{N_{t+h} - N_{t}  \geq 2 \} &= \sum_{k=2}^{\infty} \frac{(\lambda h)^2 e^{-\lambda h}}{k!} \\
&= o(h)
\end{align*}
which shows $[4]-(2)-(c)$. These together also proves $[4]-(2)-(b)$.
\end{proof}
\end{prop}

\begin{prop}
Definition $[4]$ implies definition $[3]$
\end{prop}
\begin{proof}
Let $f_n(t) = \P\{N_t = n\}$. We will first find $f_0(t)$ by developing and solving a differential equation.
\begin{align*}
f_0(t+h) &= \P\{N_t = 0,N_{t+h} - N_t = 0\} \\
&= \P\{N_t = 0\} \P\{\N_{t+h} - N_t = 0\} && \text{(independent increments)}\\
&= f_0(t) (1- \lambda h + o(h)) && \text{(using definition $[4]-(2)-(b)$)}\\
\implies f'_0(t) &= -\lambda f_0(t) &&  \text{(rearraging and taking $h \downarrow 0$)} \\
\implies f_0(t) &= e^{-\lambda t} && \text{(Solving diff equation using $N_0 = 0$)}  \\ 
\end{align*} 
For $n \geq 1$, we have
\begin{align*}
f_n(t+h) &= \P\{N_t = n,N_{t+h} - N_t = 0\} \\ &+\P\{N_t = n-1,N_{t+h} - N_t = 1\} + o(h) && \text{(definition $[4]-(2)-(c)$)}\\
&= f_n(t) (1-\lambda h) - f_{n-1}(t)\lambda h + o(h) && \text{(independent increments and} \\& &&\text{definition $[4]-(2)-(a,b)$)}\\
\implies f'_n(t) &= - \lambda f_n(t) - \lambda f_{n-1}(t) &&\text{(taking $h \downarrow 0$)} \\
\implies \frac{d}{dt}  \left( e^{\lambda t}f_n(t) \right)  &= - \lambda e^{\lambda t} f_{n-1}(t) && \text{(mulitplying by $e^{\lambda t}$ and rearranging)} \\
\end{align*} 
Solving the above equation for $n=1$ using the initial condition $f_1(0) = 0$, we obtain
\begin{align*}
f_1(t) = \lambda t e^{-\lambda t}.
\end{align*}
For general $n$, we can verify using induction on $n$ that
\begin{align*}
f_n(t) = \P\{N_t = n \} = \frac{(\lambda t)^n}{n!} e^{\lambda t}.
\end{align*}
This verifies definition $[3]$. \qedhere
\end{proof}
\end{document}

