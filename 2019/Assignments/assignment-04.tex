% !TEX spellcheck = en_US
% !TEX spellcheck = LaTeX
\documentclass[a4paper,11pt,english]{article}

\usepackage{%
	amsfonts,%
	amsmath,%	
	amssymb,%
	amsthm,%
%	babel,%
	bbm,%
	%biblatex,%
	caption,%
	centernot,%
	color,%
	enumerate,%
	epsfig,%
	epstopdf,%
	etex,%
	geometry,%
	graphicx,%
	hyperref,%
	latexsym,%
	mathtools,%
	multicol,%
	pgf,%
	pgfplots,%
	pgfplotstable,%
	pgfpages,%
	proof,%
	psfrag,%
	subfigure,%	
	tikz,%
	ulem,%
	url%
}	

\usepackage[mathscr]{eucal}
\usepgflibrary{shapes}
\usetikzlibrary{%
  arrows,%
  backgrounds,%
  chains,%
  decorations.pathmorphing,% /pgf/decoration/random steps | erste Graphik
  decorations.text,%
  matrix,%
  positioning,% wg. " of "
  fit,%
  patterns,%
  petri,%
  plotmarks,%
  scopes,%
  shadows,%
  shapes.misc,% wg. rounded rectangle
  shapes.arrows,%
  shapes.callouts,%
  shapes%
}

\theoremstyle{plain}
\newtheorem{thm}{Theorem}[section]
\newtheorem{lem}[thm]{Lemma}
\newtheorem{prop}[thm]{Proposition}
\newtheorem{cor}[thm]{Corollary}

\theoremstyle{definition}
\newtheorem{defn}[thm]{Definition}
\newtheorem{conj}[thm]{Conjecture}
\newtheorem{exmp}[thm]{Example}
\newtheorem{assum}[thm]{Assumptions}
\newtheorem{axiom}[thm]{Axiom}

\theoremstyle{remark}
\newtheorem{rem}[thm]{Remark}
\newtheorem{note}[thm]{Note}

\newcommand{\norm}[1]{\left\lVert#1\right\rVert}
\newcommand{\indep}{\!\perp\!\!\!\perp}
\DeclarePairedDelimiter\abs{\lvert}{\rvert}%
%\DeclarePairedDelimiter\norm{\lVert}{\rVert}%
\newcommand{\tr}{\operatorname{tr}}
\newcommand{\R}{\mathbb{R}}
\newcommand{\Q}{\mathbb{Q}}
\newcommand{\N}{\mathbb{N}}
\newcommand{\E}{\mathbb{E}}
\newcommand{\Z}{\mathbb{Z}}
\newcommand{\B}{\mathscr{B}}
\newcommand{\C}{\mathcal{C}}
\newcommand{\T}{\mathscr{T}}
\newcommand{\F}{\mathcal{F}}
\newcommand{\G}{\mathcal{G}}
%\newcommand{\ba}{\begin{align*}}
%\newcommand{\ea}{\end{align*}}

% Debug
\newcommand{\todo}[1]{\begin{color}{blue}{{\bf~[TODO:~#1]}}\end{color}}


\makeatletter
\def\th@plain{%
  \thm@notefont{}% same as heading font
  \itshape % body font
}
\def\th@definition{%
  \thm@notefont{}% same as heading font
  \normalfont % body font
}
\makeatother
\date{}

%\usepackage[]{geometry}
\newgeometry{left=1in,right=1in,top=1in,bottom=1in}

\title{Assignment 4: Continuous Time Markov Chains}
\author{SPQT - 2019}

\begin{document}
\maketitle
\textbf{Problem 1:} Consider a population in which individual gives birth at an exponential rate $\lambda$ and dies at an exponential rate $\mu$. In addition, new individuals enter according to a Poisson process with rate $\theta$. Let $\{X(t)\}$ denote the number of individuals in the population at time $t$. 
\begin{enumerate}
\item Show that $\{X(t)\}$ is a Markov chain. 
\item Find the generator matrix of $\{X(t)\}$. 
\item Find the conditions for stationary distribution to exist. Also, find the stationary distribution under these conditions. 
\item Find $\E[X(t)|X(0)]$.
\end{enumerate}

\indent \textbf{Problem 2:} Let $A$ be a subset of state space of Markov chain $\{X(t)\}$. Let $T_i(t)$ be the amount of time spent in $A$ in time $[0,t]$ given that $X(0)=i$. Let $Y_1,Y_2,\dots,Y_n$ be i.i.d. with $\exp(1/\lambda)$ independent of $\{X(t)\}$. Let $t_i(n) = \E[T_i(Y_1 Y_2 \dots Y_n)]$.
\begin{enumerate}
\item Derive a set of linear equations for $t_i(1), \forall i$.
\item Derive a set of linear equations for $t_i(n)$ in terms of $t_j(1)$ and $t_i(n-1)$.
\item When $n$ is large, for what values of $\lambda$ is $t_i(n)$ a good approximation of $\E[T_i(t)]$.
\end{enumerate}

\indent \textbf{Problem 3:} Consider a CTMC $\{X(t)\}$ with stationary distribution $\pi$ and generator matrix $Q$.
\begin{enumerate}
\item Compute the probability that its sojourn time in state $i$ is greater that $\alpha>0$.
\item Consider the jump chain $\{Y_n\}$. Compute its transition matrix $P$. Find the first time it comes back to state $i$ if $X(0)$ is $i$.
\item Use the above two to find $\E[T|X(0)=i]$, where $T$ is the first time $\{X(t)\}$ has its sojourn time in state $i$ greater than $\alpha>0$.
\end{enumerate}

\indent \textbf{Problem 4:} Consider an M/M/1/2 queue. Arrival rate $\lambda = 3$ per hour and service times are i.i.d. $\exp(4)$. Let $q(t)$ be the number of customers in the system at time $t$. 
\begin{enumerate}
\item Find the generator matrix for $\{q(t)\}$
\item Find the proportion of customers that enter the queue. 
\item If the service rate is increased to $8$, find (2) above. 
\item Find the conditions for stationary distribution for $\{q(t)\}$.
\item Compute the mean queue length and mean delay of a customer entering the system. 
\end{enumerate}

\indent \textbf{Problem 5:} If $\{X(t)\}$ and $\{Y(t)\}$ are independent, reversible Markov chains, show that $\{X(t),Y(t)\}$ is also a reversible Markov chain. \\
\indent \textbf{Problem 6:} Customers move among $r$ servers circularly (after completion of service at service $i$, the customer moves to the server  $(i+1) \mod r$). Service times at server $i$  is $\exp(\mu_i)$. Consider the process $\{q(0),q(1),\dots,q(r-1)\}$ where $q(i)$ denotes the number of customers in server $i$ for $i \in \{0,1,\dots,(r-1)\}$. Show this process is reversible. Find its stationary distribution.  If a customer joins server $j$ with probability $1/(r-1)$,  is the Markov chain reversible?\\
\indent \textbf{Problem 7:} Consider an $M/M/\infty$ system with arrival rate $\lambda$ and service rate $\mu$. 
\begin{enumerate}
\item Let $q(t)$ be the number of customers in the system at time $t$. Find the generator matrix. Find the conditions for stationary distribution. Find the stationery distribution under these conditions. 
\end{enumerate}
Now, consider this system as follows: whenever a customer arrives, it joins the lowest numbered server that is free. In other words, when a customer arrives, it enters server $1$ if it is free. Otherwise, it enters server $2$ if it is free and so on.
\begin{enumerate}
\item Find the fraction of time server $1$ is free under stationarity.
\item By considering the $M/M/2$ loss system, find the fraction of time server $2$ is busy.
\item Find the fraction of time server $c$  is busy for arbitrary $c$.
\item What is the overflow rate from server $c$ to $c+1$. Is it a renewal process? Is it a Poisson process? Show wherever applicable.
\end{enumerate}

\indent \textbf{Problem 8:} Consider an ergodic CTMC $\{X(t)\}$ with generator matrix $Q$ and stationary distribution $\pi$. Let $E$ be a subset of the state space. Let $G = E^c$. Under stationarity.
\begin{enumerate}
\item compute $\P\{X(t)=i|X(t)\in B\}, i \in B$,
\item compute $\P\{X(t)=i|X(t)\in B, X(t^-) \in G \}, i \in B$ and 
\item show that 
\begin{align*}
\sum_{i\in G} \sum_{j \in B} \pi_i q_{ij} = \sum_{i \in B} \sum_{j \in G} \pi_i q_{ij}.
\end{align*}
\end{enumerate}
\end{document}

