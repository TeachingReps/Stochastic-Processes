% !TEX spellcheck = en_US
% !TEX spellcheck = LaTeX
\documentclass[all-lectures.tex]{subfiles}


\begin{document}
\section*{Problems}

\noindent \textbf{Problem 1: (Random Walks)} $\{X_n\}$ iid, \quad $\E{X_1} = \mu$, \quad $0<\mu<\infty$, \quad $S_0 = 0$, \quad $S_n = \sum_{k=1}^{n} X_k$,\\
$v(t) = \min \{n:S_n>t\}$, \quad $M_n = \max_{1\le k \le n} \{S_k\}$, \quad $M_0 = 0$.
\begin{enumerate}
\item Show $\P[v(t) < \infty] = 1$  for all $0<t<\infty$.
\item Show $\{v(t) > n \} = \{M_n \le t\}$ and $v(t) \to \infty$ $a.s$ as $t \to \infty$
\item Using strongly ascending ladder heights, show $\frac{v(t)}{t} \to \frac{1}{\mu}$ $a.s$.
\item Show $\frac{S_{v(t)}}{t} \to 1 $ $a.s$.
\end{enumerate}

\noindent \textbf{Problem 2: ($GI/M/1$ queue)} Let $\{A_n\}$ $i.i.d.$ interarrival time to a queue with a general distribution. 
The service times are $i.i.d.$ $exp(\mu)$.
\begin{enumerate}
\item Consider $q_n$ = the queue length just before $n$th arrival. 
Study the stability conditions for it.
\item Using $(1)$ obtain the stability conditions for the waiting time process. 
Also for $q_t$ and $V_t$.
\end{enumerate}

\noindent \textbf{Problem 3: (Queue with priority)} Consider a queue with two classes of traffic. 
Class-1 gets Poisson arrivals with rate $\lambda_1$ and class-2 with rate $\lambda_2$. 
All the service times are $i.i.d.$ with a general distribution and mean $\frac{1}{\mu}$. 
Class-1 has preemptive resume priority over class-2.
\begin{enumerate}
\item Study the stability (Existence of stationary distribution etc.).
\item Find probability under stationarity that customer of class-$i$, $i=1,2$ experiences zero delay in the queue.
\item Find the mean delay of each class under stationarity.
\end{enumerate}

\noindent \textbf{Problem 4:} Consider two queues in tandem with Poisson arrivals with rate $\lambda$ to $Q_1$.  The service times are exponential $i.i.d.$ with rate $\mu_i$ in $Q_i$. Let $\lambda < \mu_i, i=1,2$.
\begin{enumerate} 
\item Compute the stationary distribution of queue length seen by arrivals to $Q_2$. Buffer length of each queue is infinite.
\item Assume $Q_1$ has infinite buffer but $Q_2$ has finite buffer of size $N$. 
Find the stationary probability of buffer overflow at $Q_2$. 
\item Let $Q_i$ has finite buffer $N_i<\infty, i=1,2$. 
Compute the stationary probability of buffer overflow in $Q_i, i=1,2$. 
Also compute the mean sojourn time $\E_{\pi}{S_i}$ in $Q_i,i=1,2$. 
Also compute stationary distribution of $S_i$.
\end{enumerate}


\noindent \textbf{Problem 5:}
Consider a single queue with Poisson arrivals with rate $\lambda$, Processor sharing, mean service time $\frac{1}{\mu}$.
After completion of service a customer is fed back with probability $p$ and leaves the system with probability 1-$p$.
Assume the system is under stationarity.
\begin{enumerate}
\item Find the distribution of queue length seen by an external arrival.
Also find distribution of its sojourn time.
\item Find the distribution of queue length seen by a customer fed back.
\item Which of the following flows are Poisson:
\begin{itemize}
\item Fed back customers.
\item Customers completing service at the server.
\item Customers leaving the network.
\end{itemize}
\item Solve $(1)$-$(3)$ if the queue length buffer is of length $N$.
In part-$3$ also check the flow of external customers entering the queue.
Also compute the probability of external arrivals getting lost at the queue and the probability of a fed back customer getting lost.
\end{enumerate}

\noindent \textbf{Problem 6:}
Consider an open Jackson network with three nodes with exogenous arrivals to each queue as Poisson with rate $\lambda_i, i=1,2,3$ and exponential $i.i.d.$ service rates $\mu_i, i=1,2,3$.
The Markovian routing probabilities are $p_{12}+p_{13}=1$, $p_{21}+p_{23}=1$, $p_{30}=1$.
\begin{enumerate}
\item Find conditions for $q(t)=(q_1(t), q_2(t), q_3(t))$ to be a stable Markov chain.
\item On which arcs in the network the flows are Poisson under stationarity.
\item Find the mean sojourn time in the network under stationarity.
\item Find distribution of sojourn time in $Q_3$ under stationarity.
\item Find the distribution of sojourn time on a visit of a customer from $Q_1$ to $Q_2$.
\item Answer all above if $Q_3$ has a general service time distribution with Process sharing.
\end{enumerate}

\noindent \textbf{Problem 7:(Window flow control)} A source transmits its packets through a queue (router) to the destination via a window flow control mechanism.
The window size is $N$.
Packets from the source enter $Q_1$ and are served in $i.i.d.$ $exp(\mu_1)$ in FCFS fashion.
Whenever the destination receives a packet, it immediately releases an acknowledgement in $Q_2$.
Service times in $Q_2$ are $i.i.d.$ $exp(\mu_2)$.
At any time the sum of packets and acks in the system is $N$. 
Whenever an ack reaches source, it releases the next packet to $Q_1$.
(This models a simplistic version of TCP and is a closed queueing network.)
\begin{enumerate}
\item Find conditions for stationary distribution of queue lengths in network.
\item Find rate at which packets are released by source.
\item Find the mean sojourn time of packets in $Q_1$.
\end{enumerate}

\noindent \textbf{Problem 8:} Consider a $GI/GI/1$ queue with last come first serve preemptive resume discipline.
Whenever a customer arrives, server leaves other customers and starts serving a new one.
Whenever a server completes a service it goes back to previous customer to complete its service. 
Find conditions for its stability.
Show its mean sojourn time equals busy period of $GI/GI/1$ with FCFS.
\end{document}