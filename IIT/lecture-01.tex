% !TEX spellcheck = en_US
% !TEX spellcheck = LaTeX
\documentclass[a4paper,10pt,english]{article}
\usepackage{%
	amsfonts,%
	amsmath,%	
	amssymb,%
	amsthm,%
	algorithm,%
	babel,%
	bbm,%
	etex,%
	caption,%
	centernot,%
	color,%
	dsfont,%
	enumerate,%
	epsfig,%
	geometry,%
	graphicx,%
	hyperref,%
	latexsym,%
	mathtools,%
	multicol,%
	pgf,%
	pgfplots,%
	pgfplotstable,%
	pgfpages,%
	proof,%
	psfrag,%
	subfigure,%	
	tikz,%
	ulem,%
	url%
}	
\usepackage[noend]{algpseudocode}
\usepackage[mathscr]{eucal}
\usepgflibrary{shapes}
\usetikzlibrary{%
  arrows,%
  backgrounds,%
  chains,%
  decorations.pathmorphing,% /pgf/decoration/random steps | erste Graphik
  decorations.text,%
  fit,%
  matrix,%
  patterns,%
  petri,%
  positioning,% wg. " of "
  plotmarks,%
  scopes,%
  shadows,%
  shapes,%
  shapes.arrows,%
  shapes.callouts,%
  shapes.misc% wg. rounded rectangle
}

\theoremstyle{plain}
\newtheorem{thm}{Theorem}[section]
\newtheorem{lem}[thm]{Lemma}
\newtheorem{prop}[thm]{Proposition}
\newtheorem{cor}[thm]{Corollary}

\theoremstyle{definition}
\newtheorem{defn}[thm]{Definition}
\newtheorem{conj}[thm]{Conjecture}
\newtheorem{exmp}[thm]{Example}
\newtheorem{assum}[thm]{Assumptions}
\newtheorem{axiom}[thm]{Axiom}

\theoremstyle{remark}
\newtheorem{rem}{Remark}
\newtheorem{note}{Note}
\newtheorem{fact}{Fact}

\definecolor{lightgray}{gray}{0.9}

%\DeclarePairedDelimiter{\ceil}{\left\lceil}{\right\rceil}%
\newcommand{\eq}[1]{\begin{align*}#1\end{align*}}
\newcommand{\ceil}[1]{\left\lceil#1\right\rceil}%
\newcommand{\norm}[1]{\left\lVert#1\right\rVert}%
\newcommand{\indep}{\!\perp\!\!\!\perp}%
\DeclarePairedDelimiter\abs{\lvert}{\rvert}%
\newcommand\numberthis{\addtocounter{equation}{1}\tag{\theequation}}
\newcommand{\tr}{\operatorname{tr}}
\newcommand{\R}{\mathbb{R}}
\newcommand{\N}{\mathbb{N}}
\newcommand{\E}{\mathbb{E}}
\newcommand{\Z}{\mathbb{Z}}
\newcommand{\B}{\mathscr{B}}
\newcommand{\C}{\mathcal{C}}
\newcommand{\T}{\mathscr{T}}
\newcommand{\F}{\mathcal{F}}
\newcommand{\G}{\mathcal{G}}
\newcommand{\X}{\mathcal{X}}
%\newcommand{\ba}{\begin{align*}}
%\newcommand{\ea}{\end{align*}}
\DeclareMathOperator*{\argmax}{arg\,max}
\renewcommand{\qedsymbol}{$\blacksquare$}
\makeatletter
\def\BState{\State\hskip-\ALG@thistlm}
\makeatother

\makeatletter
\def\th@plain{%
  \thm@notefont{}% same as heading font
  \itshape % body font
}
\def\th@definition{%
  \thm@notefont{}% same as heading font
  \normalfont % body font
}
\makeatother
\date{}
\title{Lecture 01: Renewal Theory}
\author{}

\begin{document}
\maketitle

\section{Simple point processes}
\begin{defn} A stochastic process $\{N(t), t\geqslant 0\}$ is a \textbf{point process} if
\begin{enumerate}
  \item $N(0) = 0$, and 
  \item for each $\omega \in \Omega$, the map $t\mapsto N(t)$ is non-decreasing, integer valued, and right continuous.% and at points of discontinuity $(N(t)- N(t^{-}))\leqslant 1,~~\forall \omega \in \Omega$. 
\end{enumerate}
\end{defn} 
\begin{defn} A \textbf{simple point process} is a point process of jump size 1.
\end{defn}

\begin{figure}[hhhh]
\center
	\begin{tikzpicture}
[node distance=1cm, draw=black, thick, >=stealth',
axes/.style=,
codeword/.style={rectangle, draw=black, inner sep=0pt, minimum height=0.8cm, minimum width=0.5cm}]

\begin{scope}[axes]
\draw[->] (0,0) -- (8,0) node[right] {$t$} coordinate (time);
\draw[->] (0,0) -- (0,5) node[above] {$N(t)$} coordinate (number of  events);
\foreach \x/\xtext in {1/S_1, 2/S_2, 5/S_3,7/S_4}
	\draw[xshift=\x cm] (0pt,1pt) -- (0pt,-1pt) node[below,fill=white] {$\xtext$};
\foreach \y/\ytext in {1,...,4}
	\draw[yshift=\y cm] (1pt,0pt) -- (-1pt,0pt) node[left,fill=white] {$\ytext$};
\end{scope}
\draw[] (0,0)--(1,0)--(1,1)--(2,1)--(2,2)--(5,2)--(5,3)--(7,3)--(7,4)--(8,4);
\end{tikzpicture}

	\caption{Sample path of a simple point process.}
	\label{Fig:Poisson}
\end{figure}
\begin{defn} We can define a random variable $S_n$ as the time of $n^{\text{th}}$ discontinuity, written
\begin{xalignat*}{3}
&S_0 = 0, \text{ and}&&S_n = \inf\{t \geq 0: N(t) = n\}, n \in \N.
\end{xalignat*}
The points of discontinuity corresponds to the arrival instants of the point process. 
\end{defn}
\begin{lem} Simple point process $\{N(t), t \geqslant 0\}$ and arrival process $\{S_n: n \in \N\}$ are inverse processes. That is,
\begin{align*}
\{S_n \leqslant t\} = \{N(t) \geqslant n\}.
\end{align*}
\end{lem}
\begin{proof} Let $\omega \in \{S_n \leqslant t\}$, then $N(S_n) = n$ by definition. 
%from right continuity of the counting process. 
Since $N$ is a non-decreasing process, we have $N(t) \geq N(S_n) = n$. 
%It follows that $\{S_n \leqslant t\} \subseteq \{N(t) \geqslant n\}$.
Conversely, let $\omega \in \{N(t) \geqslant n\}$, then it follows from definition that $S_n \leq t$.
\end{proof}
\begin{cor} The following identity is true.
\begin{align*}
\{S_n \leqslant t, S_{n+1} > t\} = \{N(t) = n\}.
\end{align*}
\end{cor}
\begin{proof}
It is easy to see that 
\begin{align*}
\{S_{n+1} > t \} &= \{S_{n+1} \leqslant t\}^c = \{N(t) \geqslant n+1\}^c = \{N(t) < n+1\}.
\end{align*}
Hence, the result follows by writing 
\begin{align*}
\{N(t) = n\} &= \{N(t) \geqslant n, N(t) < n+1\} = \{S_n \leqslant t, S_{n+1} > t\}.
\end{align*}
\end{proof}
\begin{lem}
Let $F_n(x)$ be the distribution function for $S_n$, then 
\begin{align*}
P_n(t) \triangleq \Pr\{N(t) = n\} = F_{n}(t)-F_{n+1}(t).
\end{align*}
\end{lem}
\begin{proof} It suffices to observe that following is a union of disjoint events,
\begin{align*}
\{S_n \leqslant t, S_{n+1} > t\} \cup \{S_{n} \leqslant t, S_{n+1} \leqslant t\} = \{S_n \leqslant t \}.
\end{align*}
\end{proof}
\begin{defn} The inter arrival time between $(n-1)^{th}$ and $n^{th}$ arrival is denoted by $X_n$ and written as
\begin{align*}
X_n = S_n - S_{n-1}.
\end{align*}
\end{defn}
\begin{rem} For a simple point process, we have
\begin{align*}
\Pr\{X_{n} = 0\} &= \Pr\{X_n\leqslant 0\} = 0.
\end{align*}
%This is equivalent to a point process not having an atom at any point on the line. 
Notice that, generalized point processes in higher dimension don't have any inter-arrival time interpretation. 
\end{rem}
\begin{defn}
For an interval $I = (s,t]$, the number of arrivals in the interval $I$ is defined as $N(I) = N(t) - N(s)$. 
\end{defn}
\begin{defn} 
A point process $\{N(t), t\geqslant 0\}$ is called \textbf{stationary increment point process}, 
if for any collection of mutually exclusive intervals $\{I_j: j \in [n]\}$ and any $t \geqslant 0$ 
\begin{align*}
\Pr\cap_{j\in[n]}\{N(I_j) = k_j\} &= \Pr\cap_{j \in [n]}\{N(t + I_j)\}.
\end{align*}
%, the joint distribution of $(N(t_{n})-N(t_{n-1}),N(t_{n-1})-N(t_{n-2}),...,N(s))$ is identical to the joint distribution of $(N(t_{n}+t)-N(t_{n-1}+t),...,N(s+t)), ~ \forall t \geqslant 0$.
\end{defn}

\begin{defn} 
A point process $\{N(t), t\geqslant 0\}$ is called \textbf{stationary independent increment process}, if it has stationary increments and the increments are independent random variables.
\end{defn}

\begin{rem}
For any collection of points  $t_0 = 0 < s<t_{2} < \ldots < t_{n}$, 
we can define mutually exclusive intervals $I_j = (t_{j-1}, t_j]$ for $j \in [n]$. 
Then, a stationary independent increment point process is the one that has the joint probability distribution, a product of marginals and each marginal depends solely on the interval length $|I_j| = t_j - t_{j-1}$. 
That is, 
\begin{align*}
\Pr\cap_{j \in [n]}\{N(I_j) = n_j\} &= \prod_{j \in [n]}\Pr\{N(|I_j|) = n_j\}.
\end{align*} 
\end{rem}

\begin{lem} 
An arrival process $\{S_n, n \in \N_0\}$ has stationary and independent increments iff 
the sequence of inter-arrival times $\{X_n: n \in \N\}$ are \emph{iid} random variables.
\end{lem}
\begin{proof} 
We first suppose that $\{X_n: n \in \N\}$ is a sequence of \emph{iid} random variables. 
Then $S_{n+m} - S_m$ has the same distribution as $S_n$ and is independent of $(X_1, \ldots, X_m)$. 
Conversely, we suppose that $\{S_n: n \in \N_0\}$ has stationary, independent increments. 
Then, $\{X_n: n \in \N\}$ is a sequence of \emph{iid} random variables by looking at $X_n = S_n - S_{n-1}$.   
\end{proof}

\begin{lem} If a simple point process $\{N(t), t \geqslant 0\}$ has stationary and independent increments then 
the sequence of inter-arrival times $\{X_n: n \in \N\}$ are \emph{iid} random variables.
\end{lem}
\begin{proof} 
%Suppose first that $(X_1,X_2,\ldots)$ is a sequence of \emph{iid} random variables, 
%and then $S_{n+m} - S_m$ has same distribution as $S_n$ and is independent of $(X_1, \ldots, X_m)$. 
%Then, for the ordered disjoint intervals $\{I_j = (a_j ,b_j]: j \in [n]\}$ such that $b_{j-1} \leq a_j$ for each $j$, and $K_j \triangleq \sum_{i = 1}^{j}k_i$, we have
%\begin{align*}
%\{N(I_j) = k_j\} &= \{N(b_j) - N(a_j) = k_j\} = \{S_{N(a_j)+k_j+1} - S_{N(a_j)} > |I_j|, S_{N(a_j)+k_j} - S_{N(a_j)} \leqslant |I_j|\}
%\end{align*}
%Then, for the ordered disjoint intervals $\{I_j = (t_{j-1} ,t_j]: j \in [n]\}$ and $K_j \triangleq \sum_{i = 1}^{j}k_i$, we have 
%\begin{align*}
%\Pr\cap_{j \in [n]}\{N(I_j) = k_j\} &= \Pr\cap_{j \in [n]}\{ N(t_j) =  K_j\}\\
%&= \Pr\cap_{j \in [n]}\{X_{K_j+1}+ S_{K_j} - S_{K_{j-1}} > |I_j|, S_{K_j} - S_{K_{j-1}} \leqslant |I_j|\}. 
%\end{align*}
First, we notice that 
\begin{align*}
\{X_{n} > y\} = \{ N(S_{n-1}) \leqslant N(S_{n-1} + y) < N(S_{n})\} = \{N(S_{n-1}+y)-N(S_{n-1}) = 0\}. 
\end{align*}
To show that each inter-arrival time is identically distributed, we utilize the stationary increment property of the counting process $N(t)$, to observe 
\begin{align*}
%\{S_n - S_{n-1} > x\} &= \{N(S_{n-1}+x) - N(S_{n-1}) = 0\}\\
\Pr\{S_n - S_{n-1} > x\} &%= \Pr\{N(S_{n-1}+x) - N(S_{n-1}) = 0\}
= \int_{0}^{\infty}\Pr\{N(x) = 0\}dF_{n-1}(t) = \Pr\{N(x) = 0\} = \Pr\{X_1 > x\}.
\end{align*}
%If $\{X_n\}$ are \emph{iid}, then $S_n$ are independent. 
%Hence, for disjoint intervals $I_j = (a_j, b_j]$, we have 
%\begin{align*}
%\Pr\cap_{j=1}^{n}\{N(I_j) = n_j\} &= \Pr\cap_{j=1}^{n}\{ S_{n_j} - S_{}\leq a_j, S_{n_j+1} > b_{j}\} = \int_{}.
%\end{align*}
To show that inter-arrival times are independent, it suffices to show that $X_{n}$ is independent of $S_{n-1}$. 
% to show that all inter-arrival times are independent. 
Using the independent increments property, we see that 
%\begin{align*}
%\{S_n \leqslant x, S_{n+1} - S_n > y\} &= \{S_n \leqslant x, N(y+S_n) - N(S_n) = 0\},\\
%&=\{n = N(S_n) \leqslant N(x), N(y+S_n)-N(S_n)\}. 
%\end{align*}
%Further, we observe that
\begin{align*}
\Pr\{S_{n-1} \leqslant x, X_{n} > y\} &%= \Pr\{N(S_{n-1} + y) - N(S_{n-1}) = 0, S_{n-1} \leqslant x\} 
= \int_{0}^{x}\Pr\{N(y+t) - N(t) = 0|S_{n-1} = t\}dF_{n-1}(t)\\
&= \int_{0}^{x}\Pr\{N(y+t) - N(t) = 0|N(t) = n-1, N(s) < n-1, s < t \}dF_{n-1}(t) \\
&= \Pr\{X_n > y\}F_{n-1}(x).
\end{align*}

\end{proof}

\section{Renewal Theory}
One of the characterization for the Poisson process is of it being a counting process with \emph{iid} exponential inter-arrival times. Now we shall relax the ``exponential" part. 
\begin{defn} A counting process $\{N(t),t \geq 0\}$ with \emph{iid} general inter-arrival times is called a \textbf{renewal process}.
\end{defn}
As a result, we no longer have the nice properties such as Independent and stationary increments that Poisson processes had. However, we can still get some great results which also apply to Poisson Processes. 

\begin{defn}[Inter-arrival Times] Let $\{X_i: i \in \N\}$ be a sequence of non-negative \emph{iid} random variables with a common distribution $F$, with 
	\begin{enumerate}
		%\item Positive inter-arrival time,% i.e. $X_n \geq 0$,
		\item finite mean $\mu$, %i.e. $(0 \leq \mu = \E[X_1] < \infty)$, and
		\item $F(0) < 1$.%= \Pr\{X_n \leq 0\} = \Pr\{X_n = 0\} < 1$.
	\end{enumerate}
\end{defn}
Second condition implies non-degenerate renewal process, if $F(0)$ is equal to $1$ then it is a trivial process. We interpret $X_n$ as the time between $(n - 1)^{\text{st}}$ and the $n^{\text{th}}$ renewal event. 

\begin{defn}[Renewal Instants] Let $S_n$ denote the time of $n^{\text{th}}$ renewal, and assume $S_0 = 0$. Then, we have
\begin{align*} 
S_n = \sum_{i=1}^n X_i, \quad n\in \N. 
\end{align*}
\end{defn}
\begin{defn}[Renewal process] Let $\{N(t), t \geq 0\}$ be the counting process that counts number of events by time $t$. Then,
	\begin{align*} 
	N(t) = \sup\{n \in \N_0 : S_n \leq t\} = \sum_{n \in \N}1_{\{S_n \leq t\}}.
	\end{align*} 
	This counting process $\{N(t), t \geq 0\}$ is called a renewal process.
\end{defn}
\begin{lem}[Inverse Relationship]
	There is an inverse relationship between time of $n^{\text{th}}$ event $S_n$, and the counting process $N(t)$. That is
	\begin{align}
	\label{eq:InverseRelationship}
	\{S_n \leq t\} \iff \{N(t) \geq n\}.
	\end{align}
	%since $N(t) = \sum_{n \in \N}1_{\{S_n \leq t\}}$.
\end{lem}

\begin{lem}[Finiteness of $N(t)$]
	We are interested in knowing how many renewals occur per unit time. From SLLN, we have 
	\begin{align*} 
	\frac{S_n}{n} \to \mu \quad \mbox{a.s.}
	\end{align*} 
	Since $\mu > 0$, we must have $S_n$ growing arbitrarily large as $n$ increases. Thus,
	$S_n$ can be finite for at most finitely many $n$. Therefore, $N(t)$ must be finite,
	and
	\begin{align*} 
	N(t) = \max\{n \in \N_0 : S_n \leq t\}.
	\end{align*} 
\end{lem}

\subsection{Distribution of N(t)}
We need to know the distribution of $N(t)$. 
\begin{lem} Counting process $N(t)$ assumes non-negative integer values with distribution
	\begin{align*}
	\Pr\{N(t) = n\} = \Pr\{S_n \leq t\} - \Pr\{S_{n+1} \leq t\} = F_n(t) - F_{n+1}(t).
	\end{align*}
\end{lem}
\begin{proof} It follows from~\eqref{eq:InverseRelationship}.
\end{proof}
\begin{defn} Let $F_n$ be the distribution of renewal instant $S_n$ i.e. $\Pr\{S_n \leq t\} = F_n(t)$.
\end{defn}
\begin{lem} Distribution $F_n$ of renewal instant $S_n$  is given inductively by
	\begin{xalignat*}{3}
		&F_1 = F,&&F_n = F_{n-1}\ast F,
	\end{xalignat*}
	where $\ast$ denotes convolution.
\end{lem}
\begin{proof} It follows from induction over sum of \emph{iid} random variables.
\end{proof}
%Denote $F_n = F^{*(n)}$ where $*$ denotes convolution. Essentially, $F^{*(n)}$ is the distribution of $S_n$.
\begin{defn} We define $m(t) = \E[N(t)]$ to be the \textbf{renewal function}.
\end{defn}
%We are interested in the following quantity.
%\begin{xalignat*}{3}
%	&m(t) = \E[N(t)].
%	%, &&M_{N(t)}(\theta) = \E[e^{\theta N(t)}].
%\end{xalignat*}

\begin{prop} Renewal function can be expressed in terms of distribution of renewal instants as
	\begin{align*} 
	m(t) = \sum_{n \in \N} F_n(t).
	\end{align*}
\end{prop}
\begin{proof} 
	\begin{align*}
	m(t) &= \E[N(t)] \\ 
	&= \sum_{n \in \N} \Pr\{N(t) \geq n\} \\ 
	&= \sum_{n \in \N} \Pr\{S_n \leq t\} = \sum_{n \in \N} F_n(t).
	\end{align*}
	where the second equality follows from the fact that the expectation of a random variable being represented in terms of the \emph{ccdf} of the corresponding random variable, the third equality follows from the inverse relationship as seen in \eqref{eq:InverseRelationship}.	\\
	Alternatively,
	\begin{align*}
	m(t) &= \E[N(t)]. \\
	&= \E \left[\sum_{n \in \N} \mathbb{I}_{\{S_n \leq t\}} \right] \\
	&= \sum_{n \in \N} \E \left[ \mathbb{I}_{\{S_n \leq t\}} \right] \\
	&= \sum_{n \in \N} \Pr\{S_n \leq t\} = \sum_{n \in \N} F_n(t).
	\end{align*}
	where the third equality follows from the Monotone Convergence Theorem.
\end{proof}

\begin{prop} Renewal function is bounded for all finite times.
	%\begin{align*} 
	%m(t) < \infty \quad \forall 0 \leq t < \infty
	%\end{align*}
\end{prop}
\begin{proof}
	Since we assumed that $\Pr\{X_n = 0\} < 1$, it follow from continuity of probabilities that there exists $\alpha > 0$, such that $\Pr\{X_n \geq \alpha\} >0$. Define
	\begin{align*}
	\bar{X}_n = \alpha 1_{\{X_n \geq \alpha\}}.
	\end{align*}
	Let $\bar{N}(t)$ denote the renewal process with inter-arrival times $\bar{X}_n$. Note that since $X_i$'s are \emph{iid}, so are $\bar{X}_i$ (which will be evident from the proof of the distribution function of the number of arrivals till time t). In fact, the arrivals now happen at multiples of $\alpha$. And yes, they stack. Moreover, $X_n \geq \bar{X}_n$.
	
	\begin{align*}
	\Pr\{Number ~of ~arrivals ~at ~time ~0 ~= n\} &= \Pr\{\bar{X_1}=\bar{X_2}=\ldots=\bar{X_n}=0,\bar{X_{n+1}}=\alpha\} \\
	&= \Pr\{X_1 < \alpha,X_2 < \alpha,\ldots,X_n < \alpha,X_{n+1} \geq \alpha\} \\
	&= \prod_{i=1}^{n} \Pr\{X_i < \alpha\} . \Pr\{X_{n+1} \geq \alpha \}  \\
	&= \left(1- \Pr\{X_1 \geq \alpha \}\right) ^{n} . \Pr\{X_1 \geq \alpha\}.
	\end{align*}
	where the third equality follows from the fact that $ X_i, i \in \N$ are mutually independent, fourth equality follows from the fact that $ X_i, i \in \N$ are identical.
	\\
	The number of arrivals till time $t$ therefore is Geometric with mean $\frac{1}{\Pr\{X_n \geq \alpha\}}$. Thus 
	\begin{align*}
	\E[\bar{N}(t)] = \frac{\lfloor\frac{t}{\alpha} \rfloor + 1}{\Pr\{X_n \geq \alpha\}} \leq \frac{\frac{t}{\alpha} + 1}{\Pr\{X_n \geq \alpha\}} < \infty.
	\end{align*}
	Since $\E[N(t)] \leq \E[\bar{N}(t)]$ which follows from $N(t) \leq \bar{N}(t)$, we are done.
\end{proof} 

\subsection{Elementary Renewal Theorem}
Basic renewal theorem implies $N(t)/t$ converges to $1/\mu$ almost surely. Now, we are interested in convergence of $\E[N(t)]/t$. Note that this is not obvious, since almost sure convergence doesn't imply convergence in mean. Consider the following example.
\begin{exmp}
	\begin{align*}
	Y_n = \begin{cases}
	n, & \mbox{ w.p.  } 1/n,\\
	0, & \mbox{ w.p.  } 1- 1/n.
	\end{cases}
	\end{align*}
	Then, $\Pr\{ Y_n = 0 \} = 1 - 1/n$. %Then $\sum \Pr\{Y_n > \epsilon\} < \infty$. Hence by Borel Cantelli Lemma, $Y_n \to 0$
	That is $Y_n \to 0$ a.s. However, $\E[Y_n] = 1$ for all $n \in \N$. So $\E[Y_n] \to 1$.
\end{exmp}
Even though, basic renewal theorem does \textbf{NOT} imply it, we still have $\E[N(t)]/t$ converging to $1/\mu$.

\begin{thm}[Elementary Renewal Theorem] Let $m(t)$ denote mean $\E[N(t)]$ of renewal process $N(t)$, then under the hypotheses of basic renewal theorem, we have 
	\begin{align*}
	\lim_{t \to \infty}\frac{m(t)}{t} = \frac{1}{\mu}.
	\end{align*}
\end{thm}
\begin{proof}
	Take $\mu < \infty$. We know that $S_{N(t)+1} > t$. Therefore, taking expectations on both sides and using Proposition~\ref{prop:WaldRenewal}, we have 
	\begin{align*}
	\mu (m(t) + 1) > t.
	\end{align*}
	Dividing both sides by $\mu t$ and taking $\liminf$ on both sides, we get
	\begin{align}
	\label{eq:LiminfMean}
	\liminf_{t \to \infty} \frac{m(t)}{t} \geq \frac{1}{\mu}.
	\end{align}
	
	%Thus we now have to show 
	%\[\limsup_{t \to \infty} \frac{m(t)}{t} \leq \frac{1}{\mu}\]
	We employ a truncated random variable argument to show the reverse inequality. We define truncated inter-arrival times $\{\bar{X}_n\}$ as 
	\begin{align*}
	\bar{X}_n = X_n 1_{\{X_n \leq M\}} + M1_{\{X_n > M\}}.
	\end{align*}
	We will call $\E[\bar{X}_n] = \mu_M$. Further, we can define arrival instants $\{\bar{S}_n\}$ and renewal process $\bar{N}(t)$ for this set of truncated inter-arrival times $\{\bar{X}_n\}$ as 
	\begin{align*}
	\bar{S}_n &= \sum_{k=1}^n \bar{X}_k, & \bar{N}(t) &= \sup\{n \in \N_0: \bar{S}_n \leq t\}.
	\end{align*}
	Note that since $S_n \geq \bar{S}_n$, number of arrivals would be higher for renewal process with truncated random variables, i.e. 
	\begin{align}
	\label{eq:TruncRenewalInequality}
	N(t) \leq \bar{N}(t).
	\end{align}
	Further, due to truncation of inter-arrival time, next renewal happens with-in $M$ units of time, i.e.
	\begin{align*}
	\bar{S}_{N(t)+1} \leq t+M.
	\end{align*}
	Taking expectations on both sides in the above align, using Proposition~\ref{prop:WaldRenewal}, dividing both sides by $t \mu_M$ and taking $\limsup$ on both sides, we obtain
	\begin{align*}
	\limsup_{t \to \infty}\frac{\bar{m(t)}}{t} \leq \frac{1}{\mu_M}.
	\end{align*}
	Taking expectations on both sides of~\eqref{eq:TruncRenewalInequality} and letting $M$ go arbitrary large on RHS, we get
	\begin{align}
	\label{eq:LimsupMean}
	\limsup_{t \to \infty}\frac{m(t)}{t} \leq \frac{1}{\mu}.
	\end{align}
	%Now observe that LHS is independent of $M$. Take limits $M \to \infty$, noting that $\mu_M \to \mu$ (Why?) to get
	%\[\limsup_{t \to \infty}\frac{m(t)}{t} \leq \frac{1}{\mu}\]
	%Putting it all together,
	%\[\lim_{t \to \infty}\frac{m(t)}{t} = \frac{1}{\mu}\]
	Result for finite $\mu$ follows from~\eqref{eq:LiminfMean} and~\eqref{eq:LiminfMean}. When $\mu$ grows arbitrary large, results follow from~\eqref{eq:LiminfMean}, where RHS is zero. 
\end{proof}


\section{Poisson process}
\begin{defn} A simple point process $\{N(t),~ t\geqslant 0\} $ is called a \textbf{Poisson process} with a finite positive rate $\lambda$, if inter-arrival times $\{X_{n}: n \in \N\}$ are \emph{iid} random variables with an exponential distribution of rate $\lambda$. That is, it has a distribution function $F$, such that 
 \begin{align*}
 F(x) = \Pr\{X_{1}\leqslant x\} = 
	\begin{cases}
		1-e^{-\lambda x}, & x\geqslant 0   \\
		0,  & \text{ else}.
	\end{cases}
\end{align*}
\end{defn}

\begin{figure}[hhhh]
\center
	\begin{tikzpicture}
[xscale=1.2, yscale=0.7,
node distance=1cm, draw=black, thick, >=stealth',
axes/.style=,
codeword/.style={rectangle, draw=black, inner sep=0pt, minimum height=0.8cm, minimum width=0.5cm}]

\begin{scope}[axes]
\draw[->] (0,0) -- (9,0) node[right] {$t$} coordinate (time);
\draw[->] (0,0) -- (0,8) node[above] {$N(t)$} coordinate (number of  events);
\foreach \x/\xtext in {0.5/S_{n-3}, 1.5/S_{n-2}, 2.2/S_{n-1},3.2/S_{n}, 4/s, 5.1/S_{n+1}, 6/S_{n+2}, 7/t}
	\draw[xshift=\x cm] (0pt,1pt) -- (0pt,-1pt) node[below,fill=white] {$\xtext$};
\foreach \y/\ytext in {1/n-3,2/n-2, 3/n-1, 4/n,5/n+1,6/n+2,7/n+3}
	\draw[yshift=\y cm] (1pt,0pt) -- (-1pt,0pt) node[left,fill=white] {$\ytext$};
\end{scope}
\draw[] (0,0)--(.5,0)--(.5,1)--(1.5,1)--(1.5,2)--(2.2,2)--(2.2,3)--(3.2,3)--(3.2,4)--(5.1,4)--(5.1,5)--(6,5)--(6,6)--(7.5,6)--(7.5,7)--(9,7);
\draw[<->] (4,0) -- node[midway,left]{$N(s)$}(4,4);
\draw[<->] (7,0) -- node[midway,left]{$N(t)$}(7,6);
\draw[<->] (3.2,4.1)--node[near start, above]{$X'_{n+1}$}(4,4.1);
\draw[<->] (4,4.1) --node[midway,above]{$X''_{n+1}$}(5.1,4.1);
\end{tikzpicture}

  	\caption{Stationary independent increment property of Poisson process.}
	\label{Fig:IndependentIncrements}
\end{figure}
\begin{prop} A Poisson process $\{N(t), t\geqslant 0\}$ is simple point process with stationary independent increments.
\end{prop}
\begin{proof}
It is clear that Poisson process is a simple point process. 
To show that $N(t)$ has stationary independent increment property, it suffices to show that $N_t-N(s) \perp N(s)$ and $N(t) - N(s) \sim N(t-s)$. 
This follows from the fact that we can use induction to show stationary independent increment property for for any finite disjoint time-intervals.

Let arrival time-instants $\{S_n: n \in \N_0\}$ and inter-arrival times $\{X_n: n \in \N\}$ be defined as before. 
Given any time $s$, we can define the following variables
\begin{xalignat*}{3}
&X'_{N(s)+1} = s - S_{N(s)},&&X''_{N(s)+1} = S_{N(s)+1} - s.
\end{xalignat*}
It is clear that $s$ partitions $X_{N(s)+1}$ in two parts such that $X_{N(s)+1} = X^{'}_{N(s)}+1 + X^{''}_{N(s)+1}$ as seen in Figure~\ref{Fig:IndependentIncrements} for the case when $N(s) = n$. 
We look at joint distribution of $X'_{N(s)+1}, X''_{N(s)+1}$ and notice that
\begin{align*}
\{X'_{N(s)+1}  > x, X''_{N(s)+1} > y\} %&= \bigcup_{n \in \N_0}\{X' \leq x, X'' > y, N(s) = n\},\\
&= \bigcup_{n \in \N_0}\{ S_n <  s -x , S_{n+1} > s + y, N(s) = n\}\\
&= \bigcup_{n \in \N_0}\{ S_n <  s -x , S_{n+1} > s + y\}.
\end{align*}
From the fact that inter-arrival times are \emph{iid} exponentially distributed with rate $\lambda$, we conclude that
\begin{align*}
\Pr\{X'_{N(s)} > x, X''_{N(s)+1} > y\} &= \sum_{n \in \N_0}\int_{u=0}^{s-x}\Pr\{X_{n+1} > s + y + u \}dF_n(u),\\
&= \int_{u=0}^{s-x}(1 - F_1(s+y+u))\sum_{n \in \N_0}dF_n(u) = \int_{u=0}^{s-x}e^{-\lambda(s+y+u)}\lambda du,\\
&= (1-F_1(y))(F_1(s) - F_1(2s-x)).
\end{align*}
Therefore,  $X_{N(s)+1}^{''}$ is independent of $X_{N(s)+1}^{'}$ and has same distribution as $X_{n+1}$. 
The memoryless property of exponential distribution is crucially used. 
Further, we see that independent increment holds only if inter-arrival time is exponential. 
Therefore, 
\begin{align*}
\{ N(s) = n \} &\iff  \{ S_n = s + X'_{n+1} \}, \\
\{ N(t) - N(s) \geqslant m \} &\iff \{ X''_{n+1} + \sum_{i=n+2}^{n+m} X_i \leqslant t - s \}.
\end{align*}
Since, $\{X_i: i \geqslant n+2\}\cup\{X_{n+1}^{''}\}$ are independent of $\{X_i: i \leqslant n\}\cup{X_{n+1}^{'}}$, we have $N(t)-N(s) \perp N(s)$. Further, since $X_{n+1}^{''}$ has same distribution as $X_{n+1}$, we get $N(t) - N(s) \sim N(t-s)$. By induction we can extend this result to $(N(t_{n})-N(t_{n-1}),...,N(s))$. 
\end{proof}


\end{document}