\documentclass[a4paper,english,10pt]{article}
\usepackage{%
	amsfonts,%
	amsmath,%	
	amssymb,%
	amsthm,%
	algorithm,%
	babel,%
	bbm,%
	etex,%
	caption,%
	centernot,%
	color,%
	dsfont,%
	enumerate,%
	epsfig,%
	geometry,%
	graphicx,%
	hyperref,%
	latexsym,%
	mathtools,%
	multicol,%
	pgf,%
	pgfplots,%
	pgfplotstable,%
	pgfpages,%
	proof,%
	psfrag,%
	subfigure,%	
	tikz,%
	ulem,%
	url%
}	
\usepackage[noend]{algpseudocode}
\usepackage[mathscr]{eucal}
\usepgflibrary{shapes}
\usetikzlibrary{%
  arrows,%
  backgrounds,%
  chains,%
  decorations.pathmorphing,% /pgf/decoration/random steps | erste Graphik
  decorations.text,%
  fit,%
  matrix,%
  patterns,%
  petri,%
  positioning,% wg. " of "
  plotmarks,%
  scopes,%
  shadows,%
  shapes,%
  shapes.arrows,%
  shapes.callouts,%
  shapes.misc% wg. rounded rectangle
}

\theoremstyle{plain}
\newtheorem{thm}{Theorem}[section]
\newtheorem{lem}[thm]{Lemma}
\newtheorem{prop}[thm]{Proposition}
\newtheorem{cor}[thm]{Corollary}

\theoremstyle{definition}
\newtheorem{defn}[thm]{Definition}
\newtheorem{conj}[thm]{Conjecture}
\newtheorem{exmp}[thm]{Example}
\newtheorem{assum}[thm]{Assumptions}
\newtheorem{axiom}[thm]{Axiom}

\theoremstyle{remark}
\newtheorem{rem}{Remark}
\newtheorem{note}{Note}
\newtheorem{fact}{Fact}

\definecolor{lightgray}{gray}{0.9}

%\DeclarePairedDelimiter{\ceil}{\left\lceil}{\right\rceil}%
\newcommand{\eq}[1]{\begin{align*}#1\end{align*}}
\newcommand{\ceil}[1]{\left\lceil#1\right\rceil}%
\newcommand{\norm}[1]{\left\lVert#1\right\rVert}%
\newcommand{\indep}{\!\perp\!\!\!\perp}%
\DeclarePairedDelimiter\abs{\lvert}{\rvert}%
\newcommand\numberthis{\addtocounter{equation}{1}\tag{\theequation}}
\newcommand{\tr}{\operatorname{tr}}
\newcommand{\R}{\mathbb{R}}
\newcommand{\N}{\mathbb{N}}
\newcommand{\E}{\mathbb{E}}
\newcommand{\Z}{\mathbb{Z}}
\newcommand{\B}{\mathscr{B}}
\newcommand{\C}{\mathcal{C}}
\newcommand{\T}{\mathscr{T}}
\newcommand{\F}{\mathcal{F}}
\newcommand{\G}{\mathcal{G}}
\newcommand{\X}{\mathcal{X}}
%\newcommand{\ba}{\begin{align*}}
%\newcommand{\ea}{\end{align*}}
\DeclareMathOperator*{\argmax}{arg\,max}
\renewcommand{\qedsymbol}{$\blacksquare$}
\makeatletter
\def\BState{\State\hskip-\ALG@thistlm}
\makeatother

\makeatletter
\def\th@plain{%
  \thm@notefont{}% same as heading font
  \itshape % body font
}
\def\th@definition{%
  \thm@notefont{}% same as heading font
  \normalfont % body font
}
\makeatother
\date{}

%opening
\title{Lecture 25: Exchangeability}
\author{}

\begin{document}
\maketitle

\section{Exchangeability}
%\begin{defn} 
Let $X_i$ be a random variable on the probability space $(S_i, \sS_i, \mu_i)$. 
Consider the probability space $(\Omega, \F, P)$ for the process $X = \{X_i: i \in \N\}$ where 
\meq{3}{
&\Omega = \prod_{i \in \N} S_i, && \F = \prod_{i \in \N}\sS_i.
}
Let $p_i : \Omega \to S_i$ be a projection operator, such that $p_i(\omega) = \omega_i$, 
then for each $i \in \N$
\eq{
\mu_i &= P \circ p_i^{-1}.
}
%We define $X_n(\omega) = \omega_n$
%\end{defn}
%\begin{defn} 
A \textbf{finite permutation} of $\N$ is a bijective map $\pi: \N \to \N$ such that 
$\pi(i) \neq i$ for only finitely many $i$. 
That is, for a finite $I \subset \N$, we have 
\meq{3}{
&\pi(I) = I, &&\pi(i) = i,~~i \notin I.
} 
%\end{defn}
%\begin{defn} 
Let $\omega \in \Omega$, $p_i$ be projection operators, and $\pi$ be a finite permutation, 
then we can define a finitely permuted outcome $\pi(\omega)$ in terms of its projections, as  
\eq{
p_i\circ\pi(\omega) &= p_{\pi(i)} \circ \omega,~~ i \in \N.
}
%For a finite permutation $\pi$, we define $(\pi \omega)_i = \omega_{\pi(i)}$ for all $i \in \N$.
%\end{defn}
%\begin{defn} 
An event $A \subset \Omega$ is \textbf{permutable} if 
for any finite permutation $\pi$, 
\eq{
A = \pi^{-1}A = \{\omega \in \Omega: \pi(\omega) \in A\}. 
}
It is clear that a finite permutation $\pi$ can always be defined on an interval of form $[n]$, 
where $n = \max A$. 
%\end{defn}
%\begin{defn} 
The collection of permutable events is a $\sigma$-field called the \textbf{exchangeable} $\sigma$-field and denoted by $\sE$.
%\end{defn}
%\begin{defn} 
A sequence $X = \{X_n: n \in \N\}$ of random variables is called \textbf{exchangeable} if for each finite permutation $\pi$ on a finite set $[n]$, the joint distribution of $(X_1, X_2, \ldots, X_n)$ and $(X_{\pi(1)}, X_{\pi(2)}, \ldots, X_{\pi(n)})$ are identical. 
%\end{defn}
%\begin{defn}
%$X_1, \hdots ,X_n$ is exchangeable if $X_{i_1}, \hdots X_{i_n}$ has the same joint distribution for all permutations $(i_1,i_2 \hdots i_n)$ of $(1, \hdots ,n)$. The infinite sequence of random variables $X_1, X_2 \hdots$ is said to be exchangeable if every finite subsequence $X_1, \hdots ,X_n$ is exchangeable.
%\end{defn}
\begin{shaded*}
\begin{exmp}
Suppose balls are selected randomly, without replacement, from an urn consisting of $n$ balls of which $k$ are white. 
For $i \in [n]$, let $X_i$ be the indicator of the event that the $i$th selection is white. 
%\begin{align*}
   %X_i &= 1_{\{ i^{\text{th}}\text{ selection is white}\}},
%\end{align*}
Then the finite collection $(X_1, \ldots X_n)$ is exchangeable but not independent. 
In particular, let $A = \{ i \in [n]: X_i = 1\}$. 
Then, we know that $|A| = k$, 
and we can write 
\begin{align*}
\Pr\{X_i = 1, i \in A, X_j = 0, j \in A^c\} = \Pr\{A = (i_1, i_2, \ldots, i_k) \} = \frac{(n-k)!k!}{n!} = \frac{1}{\binom{n}{k}}.
\end{align*}
This joint distribution is independent of set of exact locations $A$, and hence exchangeable. 
%Further, we can show that all $X_i$ are identically distributed, since
%\begin{align*}
%\Pr\{X_1= 1, X_2, \ldots, X_n\} = \Pr\{X_i = 1, X_1, \ldots, X_{i-1}, X_i, \ldots, X_n\}. 
%\end{align*}
However, one can see the dependence from 
\begin{align*}
\Pr\{X_2 = 1|X_1= 1\} = \frac{k-1}{n-1} \neq \frac{k}{n-1} = \Pr\{X_2 = 1|X_1 =0\}.
\end{align*}
\end{exmp}
\begin{exmp}
Let $\Lambda$ denote a random variable having distribution $G$. 
Let $X$ be a sequence of dependent random variables, 
where each of these random variables are conditionally \textit{iid} with distribution $F_\lambda$ given $\Lambda= \lambda$. 
We can write the joint finite dimensional distribution of the sequence $X$, 
\eq{ 
\Pr\{X_1 \leq x_1 \ldots , X_n \leq x_n\} = \int_{\Lambda} \prod_{i=1}^nF_\lambda(x_i)dG(\lambda). 
}
Since any finite dimensional distribution of the sequence $X$ is symmetric in $(x_1, \ldots x_n)$, 
it follows that $X$ is exchangeable. %The are not independent.
\end{exmp}
\end{shaded*}

\begin{thm}[De Finetti's Theorem] 
If $X$ is an exchangeable sequence of random variables then conditioned on $\sE$, the sequence $X$ is \textit{iid}. 
\end{thm}
\begin{proof} 
To show the independence of exchangeable sequence $X$ of random variables, conditioned on exchangeable $\sigma$-field $\sE$, 
we need to show that for bounded functions $f_i: \R \to \R$
\eq{
\E[\prod_{i=1}^kf_i(X_i)|\sE] &= \prod_{i=1}^k\E[f_i(X_i)|\sE].
}
Using induction, it suffices to show that any two bounded functions $f: \R^{k-1} \to \R$ and $g: \R \to \R$  
are independent conditioned on the exchangeable $\sigma$-field. 
That is, 
\eq{
\E[f(X_1, \dots, X_{k-1})g(X_k)|\sE] &= E[f(X_1, \dots, X_{k-1})|\sE]\E[g(X_k)|\sE].
} 
Let $I_{n,k} = \{i \subseteq [n]^k: i_j \text{ distinct}\}$, 
then the cardinality of this set is denoted by 
\eq{
(n)_k &\triangleq |I_{n,k}| = n(n-1)\ldots(n-k+1). 
} 
For a function $\phi: \R^k \to \R$, we can define
\eq{
A_n(\phi) &= \frac{1}{|I_{n,k}|}\sum_{i \in I_{n,k}}\phi(X_{i_1}, X_{i_2}, \ldots, X_{i_k}). 
}
It is clear that $A_n(\phi) \in \sE_n$ measurable and hence $\E[A_n(\phi)|\sE_n] = A_n(\phi)$. 
For each $i \in I_{n,k}$, we can find a finite permutation on $[n]$, 
such that $\pi(i_j) = j$ for $j \in [k]$, and $\pi(j) = j \in [n] \setminus I_{n,k}$. 
Since $X$ is exchangeable, the distribution of $(X_{i_1}, \dots, X_{i_k})$ and $(X_1, \dots, X_k)$ are identical for each $i \in I_{n,k}$.  
Therefore, we have
\begin{align*}
A_n(\phi) &= \frac{1}{|I_{n,k}|}\sum_{i \in I_{n,k}}\E[\phi(X_{i_1}, X_{i_2}, \ldots, X_{i_k})|\sE_n] = \E[\phi(X_{1}, X_{2}, \ldots, X_{k})|\sE_n].
\end{align*}
Since $\sE_n \to \sE$, using bounded convergence theorem for conditional expectations, we have 
\begin{align*}
\lim_{n \in \N} A_n(\phi) &= \lim_{n \in \N} \E[\phi(X_{1}, X_{2}, \ldots, X_{k})|\sE_n] = \E[\phi(X_{1}, X_{2}, \ldots, X_{k})|\sE] .
\end{align*}
Let $f$ and $g$ be bounded functions on $\R^{k-1}$ and $\R$ respectively, such that $\phi(x_1,\ldots,x_k) = f(x_1,\ldots,x_{k-1})g(x_k)$. We also define $\phi_j(x_1,\ldots,x_{k-1}) = f(x_1,\ldots,x_{k-1})g(x_j)$, to write 
\begin{align*}
(n)_{k-1}A_n(f)nA_n(g) &= \sum_{i \in I_{n,k-1}}f(X_{i_1}, \ldots,X_{i_{k-1}})\sum_{m}g(X_{m})\\
&= \sum_{i \in I_{n,k}}f(X_{i_1}, \ldots,X_{i_{k-1}})g(X_{i_k}) + \sum_{i \in I_{n,k-1}}\sum_{j=1}^{k-1}f(X_{i_1},\ldots,X_{i_{k-1}})g(X_{i_j})\\
&= (n)_kA_n(\phi) + \sum_{j=1}^k(n)_{k-1}A_n(\phi_j).
\end{align*}
Dividing both sides by $(n)_k$ and rearranging terms, we get
\begin{align*}
A_n(\phi)& = \frac{n}{n-k+1}A_n(f)A_n(g) - \frac{1}{n-k+1}\sum_{j=1}^kA_n(\phi_j),\\
%\lim_{n\in \N}\E[f(X_1,\ldots, X_{k-1})g(X_k)| \sE_n] &= \lim_{n \in \N} \E[f(X_1,\ldots, X_{k-1})| \sE_n] \E[g(X_k)| \sE_n] 
\end{align*}
Taking limits on both sides, we obtain the result
\eq{
\E[f(X_1,\dots, X_{k-1})g(X_k)|\sE] &= \E[f(X_1, \dots, X_{k-1})|\sE]\E[g(X_k)|\sE].
}
%Theorem follows by induction.
\end{proof}
%\begin{thm}[de Finetti's Theorem] Every infinite sequence $X$ of random variables taking values either $0$ or $1$, there corresponds a probability distribution $G$ on $[0,1]$ such that, for all $0 \leq k \leq n$,
%\begin{equation*}
%\label{De Finetti}
%Pr(X_1=X_2= \hdots X_k =1, X_{k+1}= \hdots X_n = 0)= \int_{0}^{1}\lambda^k(1-\lambda)^{n-k}dG(\lambda).
%\end{equation*}  
%\end{thm}
%\begin{proof}
%Let $m \geq n $.
%\begin{eqnarray*}
%&Pr(X_1 = X_2 \hdots X_k =1, X_{k+1}= \hdots X_n 0 )\\
%&=\sum_{j=0}^{m}Pr(X_1=\hdots X_k=1, X_{k+1}= X_{n}=0|S_m=j)Pr(S_m=j)\\
%&=\sum_{j} \frac{j(j-1) \hdots (j-k+1)(m-j)(m-j-1) \hdots (m-j-(n-k)+1) }{m(m-1) \hdots (m-n+1)}Pr(S_m=j).
%\end{eqnarray*}
%The last equation follows by exchangeability as given $S_m=j$ each subset of size $j$ of $X_1 \hdots X_m$ is equally likely to be the one consisting of all $1'$s. Letting $S_m=mY_m$, the above equation for large $m$ is roughly equal to $E[Y_m^k(1-Y_m)^{n-k}]$, and the theorem follows letting $m \rightarrow \infty$. Indeed, from a result known as  Helly's theorem it can be shown that for some subsequence $m'$ converging to $\infty$, the distribution of $Y_m'$ will converge to a distribution $G$ and we get
%\begin{equation*}
%E[Y_{\infty}^k(1-Y_{\infty})^{n-k}] = \int_0^1 \lambda^k(1-\lambda)^{n-k}dG(\lambda).
%\end{equation*} 
%\end{proof}
\begin{cor}[De Finetti 1931]
A binary sequence $X = \{X_n: n \in \N\}$ is exchangeable iff there exists a distribution function $F(p)$ on $[0,1]$ such that for any $n \in \N$, and $S_n=\sum_i x_i$
\eq{
\Pr\{X_1=x_1,\dots,X_n=x_n\}=\int_0^1 p^{S_n}(1-p)^{n-S_n}dF(p). 
}
\end{cor}
\begin{proof}
Let $Y_n = \frac{S_n}{n}$ and $Y = \lim_{n \in \N}\frac{S_n}{n}$. 
It suffices to show that for binary sequences, 
the collection of finitely permutable events is $\sE_n = \sigma(Y_n)$. 
Hence, the exchangeable $\sigma$-field $\sE = \sigma(Y)$, 
and $F$ is the distribution function for the random variable $Y$. 
\end{proof}

\section{Polya's Urn Scheme}
We now discuss a non-trivial example of exchangeable random variables. 
%The number of black balls in the first $n$ draws would then have a $Bin(n,\frac{b_0}{b_0+w_0})$. 
Consider a discrete time stochastic process $\{(B_n,W_n): n \in \N\}$, 
where $B_n, W_n$ respectively denote the number of black and white balls in an urn after $n \in \N$ draws. 
At each draw $n$, balls are uniformly sampled from this urn. 
After each draw, one additional ball of the same color to the drawn ball, is returned to the urn. 
We are interested in characterizing evolution of this urn, given initial urn content $(B_0,W_0)$.  

Let $\xi_i$ be a random variable indicating the outcome of the $i$th draw being a black ball. 
For example, if the first drawn ball is a black, then $\xi_1 = 1$ and $(B_1,W_1) = (B_0+1, W_0)$. 
In general, 
\meq{3}{
&B_n = B_0 + \sum_{i=1}^{n}\xi_i = B_{n-1} + \xi_n,&&W_n = W_0 + \sum_{i=1}^{n}(1-\xi_i) = W_{n-1} + 1 - \xi_n.
}
It is clear that $B_n+W_n = B_0+W_0+n$. 
Let $\xi$ be any sequence of indicators $\xi \in \{0,1\}^\N$. 
We can find the indices of black balls being drawn in first $n$ draws, as 
\eq{
I_n(\xi) &= \{i \in [n]: \xi_i = 1\}.
}
With this, we can write the probability of $x \in \{0,1\}^n$
\eq{
\Pr\{\xi_1 = x_1, \dots, \xi_n = x_n\} &= \frac{\prod_{i = 1}^{|I_n(x)|}(B_0+i-1)\prod_{j = 1}^{n-|I_n(x)|}(W_0+i-1)}{\prod_{i=1}^{n}(B_0+W_0+i-1)}
}
Since this probability depends only on $|I_n(x)|$ and not $x$, 
it shows that any finite number of draws is finitely permutable event.  
That is, $\xi \in \sE_n$ for each $n \in \N$. 
Hence, any sequence of draws $\xi = \{\xi_i: i \in \N\}$ for Polya's Urn scheme is  exchangeable. 

%We note that $\xi_i$ represent the color of the $i$th drawn ball, then we know that Polya's Urn scheme generates exchangeable random sequence $\xi = \{\xi_i : i \in \N\}$.
%Let $\sE_n = \sigma(\xi_1, \dots, \xi_n)$ be the $\sigma$-field generated by finite $n$ permutation, and $\sE = \cap_{n \in \N}\sE_n$ be the exchangeable $\sigma$-field.
We are interested in limiting ratio of black balls. 
We represent the proportion of black balls after $n$ draws by 
\eq{
X_n &= \frac{B_n}{B_n+W_n}=\frac{B_n}{B_0+W_0+n}. %, 0\leq X_n \leq1.
}
Let $\F_n = \sigma(\xi_1, \dots, \xi_n)$ be the $\sigma$-field generated by the first $n$ indicators to black draws. 
It is clear that 
\eq{
\E[\xi_{n+1}|\F_n] = X_n.
} 
%Given the past realizations $\xi_1,\xi_2,\dots,\xi_n$, we can write 
%\eq{
%B_{n+1} &= B_n1\{\xi_n = w\} + (B_{n}+1)1\{\xi_n = b\} = B_n + 1\{\xi_n = b\}.
%}
Using this fact, we observe that $X = \{X_n: n \in \N\}$ is a martingale adapted to filtration $\F = \{\F_n: n \in \N\}$, since 
\eq{
\E[X_{n+1}|\F_n]&=\frac{1}{B_0+W_0+n+1}\E[B_{n+1}|\F_n] =\frac{B_n+X_n}{B_0+W_0+n+1} = X_n.
}
For each $n \in \N$, we have $\E X_n^+ = \E X_n \leq 1$. 
From Martingale convergence theorem, 
it follows almost surely that 
\eq{
\lim_{n \in \N}X_n = X_0 = \frac{B_0}{B_0+W_0}.
}
%Let $B_n$ be the number of black balls in urn after $n$ draws and $B_0 = b_0$. Now probability of getting a black ball in first draw is
% \eq{
%\mathbb{P}(B_1=b_0+1)= \frac{b_0}{b_0+w_0},
%}
%and probability of a getting white ball in first draw is
% \eq{
%\mathbb{P}(B_1=b_0)= \frac{w_0}{b_0+w_0}.
%}
%Similarly after two draws,
% \eq{
%\mathbb{P}(B_2=b_0) &= \frac{w_0}{b_0+w_0} \cdot \frac{w_0+1}{b_0+w_0+1}\\
%\mathbb{P}(B_2=b_0+1)&= \frac{b_0}{b_0+w_0} \cdot \frac{w_0}{b_0+w_0+1}+\frac{w_0}{b_0+w_0} \cdot \frac{b_0}{b_0+w_0+1}\\
%\mathbb{P}(B_2=b_0) &= \frac{b_0}{b_0+w_0} \cdot \frac{b_0+1}{b_0+w_0+1}.
%}
%For first three draws,
% \eq{
%\mathbb{P}(bwb)= \frac{b_0}{b_0+w_0} \cdot \frac{w_0}{b_0+w_0+1} \cdot \frac{b_0+1}{b_0+w_0+2}\\
%\mathbb{P}(bbw)= \frac{b_0}{b_0+w_0} \cdot \frac{b_0+1}{b_0+w_0+1} \cdot \frac{w_0}{b_0+w_0+2}\\
%\mathbb{P}(wbb)= \frac{w_0}{b_0+w_0} \cdot \frac{b_0}{b_0+w_0+1} \cdot \frac{b_0+1}{b_0+w_0+2},
%}
%where $b$ stands for black ball and $w$ stands for white ball. We observe above 3 equations, all of them are same. like wise
% \eq{
%\mathbb{P}(bbwww)= \frac{b_0(b_0+1)w_0(w_0+1)(w_0+2)}{\prod_{i=0}^4 (b_0+w_0+i)}\\
%\mathbb{P}(bwwwb)= \frac{b_0w_0(w_0+1)(w_0+2)(b_0+1)}{\prod_{i=0}^4 (b_0+w_0+i)}.
%}
%%Again above two probabilities are equal.
%\begin{defn}
%An infinite sequence $\{X_i\}_{i=1}^{\infty}$ of random variables is exchangeable if $\forall$ $n=1,2,\dots$
 %\eq{
%X_1,\dots,X_n=X_{\pi(1)},\dots ,X_{\pi(n)}, \forall \pi \in S(n),
%}
%where $S(n)$ is the symmetric group, the group of permutations.
%\end{defn}
%Polya's Urn Model is one of the examples for exchangeability. An Example would be following
 %\eq{
%\mathbb{P}(bbwww)=\mathbb{P}(bwwwb)
%}
%If $\xi_1,\xi_2,\dots,\xi_n$ denote the sigma algebra for the color of the drawn ball i.e., $\xi_i$ represents the color of the $i^{th}$ ball, from the definition of excangeability
%\eq{
%(\xi_1,\xi_2,\xi_3,\xi_4,\xi_5)=(\xi_2,\xi_1,\xi_5,\xi_4,\xi_3).
%}
%\begin{note}
%Polya's Urn scheme generate exchangeable sequences.
%\end{note}
%Let 
%\eq{
%X_n=\frac{B_n}{B_n+W_n}=\frac{B_n}{b_0+w_0+n},\ 0\leq X_n \leq1,
%}
%represents the proportion of black balls after $n$ draws, then given the past $\xi_1,\xi_2,\dots,\xi_n$
%\eq{
%B_{n+1} =
%\left\{
	%\begin{array}{ll}
		%B_n  & w.p \ (1-\frac{B_n}{b_0+w_0+n}) \\
		%B_{n+1} & if \ \xi_{n+1} \ w.p \ \frac{B_n}{b_0+w_0+n}.
	%\end{array}
%\right.
%}
%Now
%\eq{
%\E[X_{n+1}|\xi_1,\xi_2,\dots\xi_n]&=\frac{1}{b_0+w_0+n+1}\E[B_{n+1}|\xi_1,\xi_2,\dots\xi_n]\\
								%&=\frac{1}{b_0+w_0+n+1}\E[B_n(1-X_n)+(B_n+1)X_n]\\
								%&=\frac{B_n+X_n}{b_0+w_0+n+1}\\
								%&=X_n.
%}
%That means it is a martingale.
%\begin{note}
%$X_n$ is a martingale.
%\end{note}
\end{document}

\subsection{Analysis of the Polya urn model}
\begin{thm}(De Finetti 1931)
A binary sequence $\{X_n\}_{i=1}^{\infty}$ is exchageable iff there exixtes a distribution function $F(p)$ on $[0,1]$ such that for any $n\geq1$,
\eq{
\mathbb{P}(X_1=x_1,\dots,X_n=x_n)=\int_0^1 p^{S_n}(1-p)^{n-S_n}dF(p)
}
where $S_n=\sum_i x_i$.
\end{thm}
The distribution $F$ is a function of the limiting frequency
\begin{xalignat*}{3}
&Y =\bar{X}_{\infty} =\lim_{n \in \N}\frac{\sum_i X_i}{n}, &&\Pr\{Y\leq p\} =F(p),
\end{xalignat*}
and conditioning on $Y=p$ results in iid Bernoulli draws
\eq{
\mathbb{P}(X_1=x_1,\dots,X_N=x_n|Y=p)= p^{S_n}(1-p)^{n-S_n},
}
and for the Polya urn model
\eq{
\lim_{n\to \infty} \bar{X}_n=Y~Beta\left(\frac{B_0}{B_0+W_0},\frac{W_0}{B_0+W_0} \right)
}
The result can be interpreted from a statistical, probabilistic and function analytic perspective.

We will use De Finetti's theorem to compute the limiting distribution for the Polya urn model 
\eq{
\lim_{n\to \infty} \bar{X}_n=Y~Beta\left(\frac{B_0}{B_0+W_0},\frac{W_0}{B_0+W_0} \right).
}
We first define the Beta and Gamma functions
\begin{xalignat*}{5}
&\beta(x,y)=\int_0^1 p^{x-1}(1-p)^{y-1}dp, &&\Gamma(x+1)=x\Gamma(x), &&\beta(a,b)=\frac{\Gamma(a)\Gamma(b)}{\Gamma(a+b)}.
\end{xalignat*}
The probability of observing $k$ black balls given $n$ draws
\begin{align}
\mathbb{P}(k \text{ black balls given } n \text{ draws})&=\dbinom{n}{k}\frac{B_0(B_0+1)\dots(B_0+k-1)W_0(W_0+1)\dots(W_0+n-k-1)}{(B_0+W_0)(B_0+W_0+1)\dots(B_0+W_0+n-1)}\\
								&=\dbinom{n}{k}\frac{\beta(B_0+k,B_0+n-k)}{\beta(B_0,W_0)}.
\end{align}
Note that the proportion of black balls at any stage $n$ of the process is
\begin{xalignat*}{3}
&\rho_n=\frac{B_n}{B_n+W_n}, &&\rho_{\infty}=\lim_{n\to \infty}\frac{B_n}{B_n+W_n}.
\end{xalignat*}
We know that 
\eq{
\mathbb{P}(k \text{ black balls given } n \text{ draws}|\rho_{\infty}=p)=\dbinom{n}{k}p^k(1-p)^{n-k},
}
and if $\rho_{\infty}~F(p)$ then,
\eq{
\mathbb{P}(k \text{ black balls given } n \text{ draws})=\int_0^1\mathbb{P}(k \text{ black balls given } n \text{ draws}|\rho_{\infty}=p)dF(p),
}
\begin{equation}
\mathbb{P}(k \text{ black balls given } n \text{ draws})=\dbinom{n}{k} \int_0^1p^k(1-p)^{n-k}dF(p).
\end{equation}
By equating (2) and (3) we obtain,
\eq{
\int_0^1 p^k(1-p)^{n-k}dF(p) &= \frac{\beta(B_0+k,B_0+n-k)}{\beta(B_0,W_0)}\\
					   &=\frac{1}{\beta(B_0,W_0)}\int_0^1p^{B_0+k-1}(1-p)^{B_0+n-k-1}dp\\
					   &=\int_0^1 p^k (1-p)^{n-k}\frac{p^{B_0-1}(1-p)^{W_0-1}}{\beta(B_0,W_0)}dp,
}
which gives the limiting distribution for Polya's urn scheme as
\eq{
f(p)=\frac{1}{\beta(B_0,W_0)}p^{B_0-1}(1-p)^{W_0-1}.
}

