% !TEX spellcheck = en_US
% !TEX spellcheck = LaTeX
\documentclass[a4paper,10pt,english]{article}
%\uspackage{tfrupee}
\usepackage{%
	amsfonts,%
	amsmath,%	
	amssymb,%
	amsthm,%
%	babel,%
	bbm,%
	%biblatex,%
	caption,%
	centernot,%
	color,%
	enumerate,%
	epsfig,%
	epstopdf,%
	etex,%
	geometry,%
	graphicx,%
	hyperref,%
	latexsym,%
	mathtools,%
	multicol,%
	pgf,%
	pgfplots,%
	pgfplotstable,%
	pgfpages,%
	proof,%
	psfrag,%
	subfigure,%	
	tikz,%
	ulem,%
	url%
}	

\usepackage[mathscr]{eucal}
\usepgflibrary{shapes}
\usetikzlibrary{%
  arrows,%
  backgrounds,%
  chains,%
  decorations.pathmorphing,% /pgf/decoration/random steps | erste Graphik
  decorations.text,%
  matrix,%
  positioning,% wg. " of "
  fit,%
  patterns,%
  petri,%
  plotmarks,%
  scopes,%
  shadows,%
  shapes.misc,% wg. rounded rectangle
  shapes.arrows,%
  shapes.callouts,%
  shapes%
}

\theoremstyle{plain}
\newtheorem{thm}{Theorem}[section]
\newtheorem{lem}[thm]{Lemma}
\newtheorem{prop}[thm]{Proposition}
\newtheorem{cor}[thm]{Corollary}

\theoremstyle{definition}
\newtheorem{defn}[thm]{Definition}
\newtheorem{conj}[thm]{Conjecture}
\newtheorem{exmp}[thm]{Example}
\newtheorem{assum}[thm]{Assumptions}
\newtheorem{axiom}[thm]{Axiom}

\theoremstyle{remark}
\newtheorem{rem}[thm]{Remark}
\newtheorem{note}[thm]{Note}

\newcommand{\norm}[1]{\left\lVert#1\right\rVert}
\newcommand{\indep}{\!\perp\!\!\!\perp}
\DeclarePairedDelimiter\abs{\lvert}{\rvert}%
%\DeclarePairedDelimiter\norm{\lVert}{\rVert}%
\newcommand{\tr}{\operatorname{tr}}
\newcommand{\R}{\mathbb{R}}
\newcommand{\Q}{\mathbb{Q}}
\newcommand{\N}{\mathbb{N}}
\newcommand{\E}{\mathbb{E}}
\newcommand{\Z}{\mathbb{Z}}
\newcommand{\B}{\mathscr{B}}
\newcommand{\C}{\mathcal{C}}
\newcommand{\T}{\mathscr{T}}
\newcommand{\F}{\mathcal{F}}
\newcommand{\G}{\mathcal{G}}
%\newcommand{\ba}{\begin{align*}}
%\newcommand{\ea}{\end{align*}}

% Debug
\newcommand{\todo}[1]{\begin{color}{blue}{{\bf~[TODO:~#1]}}\end{color}}


\makeatletter
\def\th@plain{%
  \thm@notefont{}% same as heading font
  \itshape % body font
}
\def\th@definition{%
  \thm@notefont{}% same as heading font
  \normalfont % body font
}
\makeatother
\date{}
\title{Lecture 24: Exchangeability}
\author{}
\begin{document}
\maketitle

\section{Martingale Convergence Theorem}
Before we state and prove martingale convergence theorem, we state some results which will be used in the proof of the theorem.
\begin{lem}
\label{StoppingTimeBound}
If $\{X_i:  i \in \N \}$ is  a submartingale and $T$ is a stopping time such that $\Pr\{T \leq n\}=1$ then
\begin{align*}
 \E X_1 \leq \E X_T \leq \E X_n.
\end{align*}
\end{lem}
\begin{proof}
Since $T$ is bounded, it follows from Martingale stopping theorem, that $\E X_T  \geq \E X_1$. 
Now, since $T$ is a stopping time, we see that for $\{T = k\}$
\begin{align*}
\E[X_n1\{T = k\}|\F_T,T=k]&= %\E[Z_n1\{T = k\}|\F_k,T=k] = 
\E[X_n1\{T = k\}|\F_k] \geq X_k1\{T = k \} = X_T1\{T = k\}.
\end{align*}
%where $(a)$ follows from the fact that $N$ is  a stopping time. 
Result follows by taking expectation on both sides and summing over $k$. 
That is,
\eq{
\E X_n &= \E\sum_{k=1}^n X_n1\{T = k\} \geq \E\sum_{k=1}^nX_T1\{T=k\} = \E X_T.
}
\end{proof}

\begin{lem}
\label{ConvexFuncSubmart}
If $X = \{X_n: n \in \N\}$ is a martingale with respect to a filtration $\{\F_n: n \in \N\}$ and $f$ is a convex function, 
then $\{f(X_n): n \in \N\}$ is a sub martigale with respect to the same filtration.
\end{lem}
\begin{proof}
The result is a direct consequence of Jensen's inequality.
\begin{align*}
\E[f(X_{n+1})|\F_n] &\geq f(\E[X_{n+1}|\F_n])=f(X_n).
\end{align*}
\end{proof}

\begin{cor} 
Let $a \in \R$ be a constant. 
\begin{enumerate}[i\_]
\item If $\{X_n : n \in \N\}$ is a submartingale, then so is $\{(X_n - a)_+: n \in \N\}$.  
\item If $\{X_n : n \in \N\}$ is a supermartingale, then so is $\{X_n \wedge a: n \in \N\}$. 
\end{enumerate}
\end{cor}

%\begin{con}
Let $X = \{X_n : n\in \N_0\}$ be a submartingale. 
Let $a< b$ and $N_0  = -1$, and for $k \in \N$, we define 
\begin{xalignat*}{3}
&N_{2k-1}=\inf\{m>N_{2k-2}:X_m\leq a \}, && N_{2k}=\inf\{m>N_{2k-1}:X_m\geq b \}.
\end{xalignat*}
The above quantities $N_{2k-1}$,$N_{2k}$ are stopping times and the set containing values of $m$ in the transition from $a$ to $b$ can be defined as
\begin{align*}
H_m &\triangleq \{N_{2k-1}<m\leq N_{2k} \}= %\{N_{2k-1} \leq m-1\} \cap \{m>N_{2k}\}^{c} = 
\{m-1\geq N_{2k-1}\}\cap\{m-1\geq N_{2k}\}^{c} \in \F_{m-1}.		
\end{align*}
Clearly, the event of $X$ being in an up crossing at time $m$ is  predictable. 
The number of up crossings completed in time $n$ is
\eq{
U_n &= \sum_{m=1}^nH_m = \sup\{k: n \geq N_{2k}\}.
}
%Since the above set depends on $\{m-1\}$ values instead of $\{m\}$ values, So
%\begin{align*}
%H_m=&1_{\{N_{2k-1}<m\leq  N_{2k}\}}\\
%U_n=&\sup\{k:N_{2k}\leq n\}
%\end{align*}
%$H_m$ defines a predictable sequence and $U_n$ is the number of up crossings completed in time $n$.
%\end{con}
\begin{lem}[Upcrossing inequality]
If $X$ is a submartingale, 
then for $Y_n \triangleq a+ (X_n-a)^{+}$, we have 
\begin{align*}
(b-a)\E U_n\leq& \E Y_n-\E Y_0.
%\text{where }Y_n&:=a+ (X_n-a)^{+}
\end{align*}
\end{lem}
\begin{proof}
Since $X$ is a submartingale so is $Y$, as $Y_n$ is a convex function of $X_n$. 
Since each up crossing has a gain slightly more than $b-a$, the following inequality exists, 
\begin{align*}
(b-a)U_n\leq &(H\cdot Y)_n = \sum_{m=1}^{n}1_{\{N_{2k-1}<m\leq N_{2k}\}}(Y_{m+1}-Y_{m}) = \sum_{k=1}^{U_n}(Y_{N_{2k+1}}-Y_{N_{2k+1}}).
\end{align*}
Now let $K_m=1-H_m$, then $K$ is a predictable sequence, and
\eq{
Y_n-Y_0 &= (H\cdot Y)_n+(K\cdot Y)_n.
}
From the submartingale property of Y, it follows
\eq{
\E[(K\cdot Y)_n] &\geq \E[(K\cdot Y)_0]=0. 
} 
Therefore, it follows that 
\eq{
\E(Y_n - Y_0) &= \E(H\cdot Y)_n+\E(K\cdot Y)_n \geq \E(H\cdot Y)_n  \geq  (b-a)\E U_n.
}
\end{proof}

\begin{thm}[Martingale convergence theorem] 
\label{MartingaleConvergenceTheorem} 
If $\{X_n : n \in \N\}$ is a submartingale with ${\sup}_{n\in \N} \E X_n^{+} < \infty$ then ${\lim}_{n\in \N} X_n=X$ a.s with $\E|X|<\infty$.
\end{thm}
\begin{proof} 
Since $(X-a)^{+}\leq X^{+}+|a|$, it follows from upcrossing inequality that   		
\begin{align*}
\E U_n \leq& \frac{\E X_n^{+} +|a|}{b-a}.
\end{align*}
The number of upcrossings $U_n$ increases with $n$, however the mean $\E U_n$ is bounded above for each $n \in \N$. 
Hence, $\lim_{n \in \N}\E U_n$ exists and is finite. 
Let $U := \lim_{n \in \N} U_n$ and since $\E U \leq \E [X_n^{+}] < \infty$, we have $U<\infty$ almost surely. 
This conclusion leads to  
\eq{
\Pr\{_{a,b \in \mathbb{Q}}{\cup}\{{\lim \inf}_{n\in \N}  X_n<a<b<{\lim \sup}_{n\in \N}  X_n \}\} = 0.
}
From the above probability, we have almost sure equality 
\eq{
{\lim \sup}_{n\in \N}  X_n &={\lim \inf}_{n\in \N}  X_n.
}
That is, the limit $\lim_{n \in \N}X_n$ exists almost surely. 
Fatou's lemma guarantees
\eq{
\E X^{+} &\leq {\lim \inf}_{n\in \N}\E X_n^{+} <\infty,
}
which implies $X<\infty$  almost surely. 
From the submartingale property of $X_n$, it follows that 
\eq{
\E X_n^{-} &=\E X_n^{+} -\E X_n \leq \E X_n^{+} -\E X_0.
}
%The above inequality comes from the submartingale property of $X_n$. 
From Fatou's lemma, we get
\eq{
\E X^{-} \leq {\lim \inf} _{n\in \N}\E X_n^{-} \leq {\sup}_{n\in \N}\E X_n^{+} -\E X_0 < \infty.
}
This implies $X>-\infty$ almost surely, completing the proof. 
\end{proof}

%\begin{thm}[Martingale Convergence Theorem]
%\label{MartingaleConvergenceTheorem}
%If $\{Z_n,~n \geq 1\}$ is a martingale such that for some $M< \infty$
%\begin{align*}
%\E[|Z_n|] \leq M, ~ \text{for all}~ n
%\end{align*}
%then, with probability 1, $\lim_{n \rightarrow \infty}Z_n$ exists and is finite.
%\end{thm}
%\begin{proof}
%Assume $\E[Z_n^2]< \infty$ which is stronger than $\E[|Z_n|]< \infty$ (as a consequence of %Jensen's inequality). Observe that $\{Z_n^2\}$ is a submartingale (from Lemma %\ref{ConvexFuncSubmart}). Thus $\E[Z_n^2]<\infty$ and is non-decreasing in $n$. Thus, as $n %\rightarrow \infty$, $\E[Z_n^2]$ converges and let $\mu<\infty$ be given by $\mu=\lim_{n %\rightarrow \infty}\E[Z_n^2]$.
%\begin{equation}
%\label{KolmoBound}
%\Pr(\cup_{k \leq n} \{|Z_{m+k}-Z_m|> \epsilon\} )
%\end{equation}  
%\begin{align*}
%&\stackrel{(a)}{\leq }\E[(Z_{m+n}-Z_m)^2]/\epsilon^2
%&=\E[Z_{m+n}^2-2Z_mZ_{m+n}+Z_m^2]/\epsilon^2.
%\end{align*}
%Note that 
%\begin{align*}
%\E[Z_{m+n}Z_m]&=\E[\E[Z_mZ_{m+n}|Z_m]]\\
%&=\E[Z_m\E[Z_{m+n}|Z_m]]\\
%&=\E[Z_m^2].
%\end{align*}
%From \ref{KolmoBound}, 
%\begin{align*}
%\Pr(\cup_{k \leq n} \{|Z_{m+k}-Z_m|> \epsilon\}) \leq \frac{\E[Z_{m+n}^2]-\E[Z_m^2]}{\epsilon^2}.
%\end{align*}
%Letting $n \rightarrow \infty$
%\begin{align*}
%\Pr(\cup_{k \leq 1} \{|Z_{m+k}-Z_m|> \epsilon\}) \leq \frac{\mu-\E[Z_m^2]}{\epsilon^2}.
%\end{align*}
%Hence,
%\begin{align*}
%\Pr(\cup_{k \leq n} \{|Z_{m+k}-Z_m|> \epsilon\}) \rightarrow 0 ~\text{as}~ m \rightarrow \infty.
%\end{align*}
%Thus with probability 1, $\{Z_n\}$ will be  a Cauchy sequence, and thus $\lim_{n \rightarrow %\infty}Z_n$ will exist and be finite.`
%\end{proof}
%\begin{cor}
%If $\{Z_n,~m \geq 0\}$ is a non-negative martingale, then, with probability 1, $\lim_{n %\rightarrow \infty}Z_n$ exists and is finite.
%\end{cor}
%\begin{proof}
%Since $Z_n$ is non-negative,
%\begin{align*}
%\E[|Z_n|]=\E[Z_n]=\E[Z_1].
%\end{align*}

\section{Exchangeability}
%\begin{defn} 
Let $X_i$ belong to probability space $(S, \mathcal{S}, \mu)$. Consider the probability space $(\Omega, \F, P)$  for process $\{X_i: i \in \N\}$ where 
\begin{xalignat*}{5}
&\Omega = \prod_{i \in \N} S, &&\mathcal{F} = \prod_{i \in \N}\mathcal{S},&&P = \prod_{i \in \N}\mu.
\end{xalignat*}
%We define $X_n(\omega) = \omega_n$
%\end{defn}
%\begin{defn} 
A \textbf{finite permutation} of $\N$ is a bijective map $\pi: \N \to \N$ such that $\pi(i) \neq i$ for only finitely many $i$. 
That is for a finite $A \subset \N$, $\pi(A) = A$ and $\pi(i) = i$ when $i \notin A$. 
%\end{defn}
%\begin{defn} 
For a finite permutation $\pi$, we define $(\pi \omega)_i = \omega_{\pi(i)}$ for all $i \in \N$.
%\end{defn}

%\begin{defn} 
An event $A$ is \textbf{permutable} if $A = \pi^{-1}A = \{\omega \in \Omega: \pi \omega \in A\}$ for any finite permutation $\pi$.
%\end{defn}
%\begin{defn} 
The collection of permutable events is a $\sigma$-field called the \textbf{exchangeable} $\sigma$-field and denoted by $\sE$.
%\end{defn}
%\begin{defn} 
A sequence $X$ of random variables is called \textbf{exchangeable} if for each $n$ and permutation $\pi: [n] \to [n]$, joint distribution of $(X_1, X_2, \ldots, X_n)$ and $(X_{\pi(1)}, X_{\pi(2)}, \ldots, X_{\pi(n)})$ are same.
%\end{defn}
%\begin{defn}
%$X_1, \hdots ,X_n$ is exchangeable if $X_{i_1}, \hdots X_{i_n}$ has the same joint distribution for all permutations $(i_1,i_2 \hdots i_n)$ of $(1, \hdots ,n)$. The infinite sequence of random variables $X_1, X_2 \hdots$ is said to be exchangeable if every finite subsequence $X_1, \hdots ,X_n$ is exchangeable.
%\end{defn}
\begin{exmp}
Suppose balls are selected randomly, without replacement, from an urn consisting of $n$ balls of which $k$ are white. For $i \in [n]$, let
\begin{align*}
   X_i &= 1_{\{ i^{\text{th}}\text{ selection is white}\}},
\end{align*}
then $(X_1, \ldots X_n)$ will be exchangeable but not independent.  
In particular, let $A = \{ i \in [n]: X_i = 1\}$. Then, we know that $|A| = k$, and we can write 
\begin{align*}
\Pr\{X_i = 1, i \in A, X_j = 0, j \in A^c\} = \Pr\{A = (i_1, i_2, \ldots, i_k) \} = \frac{(n-k)!k!}{n!} = \frac{1}{\binom{n}{k}}.
\end{align*}
This joint distribution is independent of set of exact locations $A$, and hence exchangeable. 
Further, we can show that all $X_i$ are identically distributed, since
\begin{align*}
\Pr\{X_1= 1, X_2, \ldots, X_n\} = \Pr\{X_i = 1, X_1, \ldots, X_{i-1}, X_i, \ldots, X_n\}. 
\end{align*}
Further, it can be seen that
\begin{align*}
\Pr\{X_2 = 1|X_1= 1) = \frac{k-1}{n-1} \neq \frac{k}{n-1} = \Pr\{X_2 = 1|X_1 =0\}.
\end{align*}
\end{exmp}
\begin{exmp}
Let $\Lambda$ denote a random variable having distribution $G$. Let $X$ be a sequence of dependent random variables, where each of these random variables are conditionally \textit{iid} with distribution $F_\lambda$ given $\Lambda= \lambda$.Then, these random variables are exchangeable since
\begin{align*}
\Pr\{X_1 \leq x_1 \ldots , X_n \leq x_n\} = \int_{\lambda} \prod_{i=1}^nF_\lambda(x_i)dG(\lambda),
\end{align*}
which is symmetric in $(x_1, \ldots x_n)$. %The are not independent.
\end{exmp}

\begin{thm}[de Finetti's Theorem] 
If $X$ is an exchangeable sequence of random variables then conditioned on $\sE$, the sequence $X$ is \textit{iid}. 
\end{thm}
\begin{proof} Let $I_{n,k} = \{i \subseteq [n]^k: i_j \text{ distinct}\}$. 
Then, for a function $\phi: \R^k \to \R$, we can define
\begin{align*}
A_n(\phi) &= \frac{1}{(n)_k}\sum_{i \in I_{n,k}}\phi(X_{i_1}, X_{i_2}, \ldots, X_{i_k}),
\end{align*}
where $(n)_k = n(n-1)\ldots(n-k+1)$. Clearly, $A_n(\phi) \in \sE_n$ measurable and hence,
$\E[A_n(\phi)|\sE_n] = A_n(\phi)$. Since, $X$ is exchangeable, we have
\begin{align*}
A_n(\phi) &= \frac{1}{(n)_k}\sum_{i \in I_{n,k}}\E[\phi(X_{i_1}, X_{i_2}, \ldots, X_{i_k})|\sE_n] = \E[\phi(X_{1}, X_{2}, \ldots, X_{k})|\sE_n].
\end{align*}
Since $\sE_n \to \sE$, we have 
\begin{align*}
\lim_{n \in \N} A_n(\phi) &= \lim_{n \in \N} \E[\phi(X_{1}, X_{2}, \ldots, X_{k})|\sE_n] = \E[\phi(X_{1}, X_{2}, \ldots, X_{k})|\sE] .
\end{align*}
Let $f$ and $g$ be bounded functions on $\R^{k-1}$ and $\R$ respectively, such that $\phi(x_1,\ldots,x_k) = f(x_1,\ldots,x_{k-1})g(x_k)$. We also define $\phi_j(x_1,\ldots,x_{k-1}) = f(x_1,\ldots,x_{k-1})g(x_j)$, to write 
\begin{align*}
(n)_{k-1}A_n(f)nA_n(g) &= \sum_{i \in I_{n,k-1}}f(X_{i_1}, \ldots,X_{i_{k-1}})\sum_{m}g(X_{m})\\
&= \sum_{i \in I_{n,k}}f(X_{i_1}, \ldots,X_{i_{k-1}})g(X_{i_k}) + \sum_{i \in I_{n,k-1}}\sum_{j=1}^{k-1}f(X_{i_1},\ldots,X_{i_{k-1}})g(X_{i_j})\\
&= (n)_kA_n(\phi) + \sum_{j=1}^k(n)_{k-1}A_n(\phi_j).
\end{align*}
Dividing both sides by $(n)_k$ and rearranging terms, we get
\begin{align*}
A_n(\phi)& = \frac{n}{n-k+1}A_n(f)A_n(g) - \frac{1}{n-k+1}\sum_{j=1}^kA_n(\phi_j),\\
%\lim_{n\in \N}\E[f(X_1,\ldots, X_{k-1})g(X_k)| \sE_n] &= \lim_{n \in \N} \E[f(X_1,\ldots, X_{k-1})| \sE_n] \E[g(X_k)| \sE_n] 
\end{align*}
Taking limits on both sides, we obtain
\eq{
\E[f(X_1,\dots, X_{k-1})g(X_k)|\sE] &= \E[f(X_1, \dots, X_{k-1})|\sE]\E[g(X_k)|\sE].
}
Theorem follows by induction.
\end{proof}
%\begin{thm}[de Finetti's Theorem] Every infinite sequence $X$ of random variables taking values either $0$ or $1$, there corresponds a probability distribution $G$ on $[0,1]$ such that, for all $0 \leq k \leq n$,
%\begin{equation*}
%\label{De Finetti}
%Pr(X_1=X_2= \hdots X_k =1, X_{k+1}= \hdots X_n = 0)= \int_{0}^{1}\lambda^k(1-\lambda)^{n-k}dG(\lambda).
%\end{equation*}  
%\end{thm}
%\begin{proof}
%Let $m \geq n $.
%\begin{eqnarray*}
%&Pr(X_1 = X_2 \hdots X_k =1, X_{k+1}= \hdots X_n 0 )\\
%&=\sum_{j=0}^{m}Pr(X_1=\hdots X_k=1, X_{k+1}= X_{n}=0|S_m=j)Pr(S_m=j)\\
%&=\sum_{j} \frac{j(j-1) \hdots (j-k+1)(m-j)(m-j-1) \hdots (m-j-(n-k)+1) }{m(m-1) \hdots (m-n+1)}Pr(S_m=j).
%\end{eqnarray*}
%The last equation follows by exchangeability as given $S_m=j$ each subset of size $j$ of $X_1 \hdots X_m$ is equally likely to be the one consisting of all $1'$s. Letting $S_m=mY_m$, the above equation for large $m$ is roughly equal to $E[Y_m^k(1-Y_m)^{n-k}]$, and the theorem follows letting $m \rightarrow \infty$. Indeed, from a result known as  Helly's theorem it can be shown that for some subsequence $m'$ converging to $\infty$, the distribution of $Y_m'$ will converge to a distribution $G$ and we get
%\begin{equation*}
%E[Y_{\infty}^k(1-Y_{\infty})^{n-k}] = \int_0^1 \lambda^k(1-\lambda)^{n-k}dG(\lambda).
%\end{equation*} 
%\end{proof}



\end{document}

\begin{defn} Let $\{X_i: i \in \N\}$ be \textit{iid} random variables with finite $\E[|X_1|]$. Let
\begin{align*}
S_n &= \sum_{k=1}^n X_i, ~~n \in \N_0.
\end{align*}
Then the process $\{S_n: n \in \N_0\}$ is called a \textbf{random walk}. 
\end{defn}
\begin{defn} A random walk is called a \textbf{simple random walk} if
\begin{align*}
\Pr\{X_1 = 1\} &= 1- \Pr\{X_1 = -1\}.
\end{align*}
\end{defn}
\begin{rem} A simple random walk has the interpretation of the winnings of a gambler who plays a simple coin toss game and wins Rupee 1 if heads and loses Rupee 1 if tails. 
\end{rem}
\begin{rem} Random walks are useful in analyzing GI/GI/1 Queues, Ruin systems and even stock prices.
\end{rem}