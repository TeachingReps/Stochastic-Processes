% !TEX spellcheck = en_US
% !TEX spellcheck = LaTeX
\documentclass[a4paper,10pt,english]{article}
\usepackage{%
	amsfonts,%
	amsmath,%	
	amssymb,%
	amsthm,%
	algorithm,%
	babel,%
	bbm,%
	etex,%
	caption,%
	centernot,%
	color,%
	dsfont,%
	enumerate,%
	epsfig,%
	geometry,%
	graphicx,%
	hyperref,%
	latexsym,%
	mathtools,%
	multicol,%
	pgf,%
	pgfplots,%
	pgfplotstable,%
	pgfpages,%
	proof,%
	psfrag,%
	subfigure,%	
	tikz,%
	ulem,%
	url%
}	
\usepackage[noend]{algpseudocode}
\usepackage[mathscr]{eucal}
\usepgflibrary{shapes}
\usetikzlibrary{%
  arrows,%
  backgrounds,%
  chains,%
  decorations.pathmorphing,% /pgf/decoration/random steps | erste Graphik
  decorations.text,%
  fit,%
  matrix,%
  patterns,%
  petri,%
  positioning,% wg. " of "
  plotmarks,%
  scopes,%
  shadows,%
  shapes,%
  shapes.arrows,%
  shapes.callouts,%
  shapes.misc% wg. rounded rectangle
}

\theoremstyle{plain}
\newtheorem{thm}{Theorem}[section]
\newtheorem{lem}[thm]{Lemma}
\newtheorem{prop}[thm]{Proposition}
\newtheorem{cor}[thm]{Corollary}

\theoremstyle{definition}
\newtheorem{defn}[thm]{Definition}
\newtheorem{conj}[thm]{Conjecture}
\newtheorem{exmp}[thm]{Example}
\newtheorem{assum}[thm]{Assumptions}
\newtheorem{axiom}[thm]{Axiom}

\theoremstyle{remark}
\newtheorem{rem}{Remark}
\newtheorem{note}{Note}
\newtheorem{fact}{Fact}

\definecolor{lightgray}{gray}{0.9}

%\DeclarePairedDelimiter{\ceil}{\left\lceil}{\right\rceil}%
\newcommand{\eq}[1]{\begin{align*}#1\end{align*}}
\newcommand{\ceil}[1]{\left\lceil#1\right\rceil}%
\newcommand{\norm}[1]{\left\lVert#1\right\rVert}%
\newcommand{\indep}{\!\perp\!\!\!\perp}%
\DeclarePairedDelimiter\abs{\lvert}{\rvert}%
\newcommand\numberthis{\addtocounter{equation}{1}\tag{\theequation}}
\newcommand{\tr}{\operatorname{tr}}
\newcommand{\R}{\mathbb{R}}
\newcommand{\N}{\mathbb{N}}
\newcommand{\E}{\mathbb{E}}
\newcommand{\Z}{\mathbb{Z}}
\newcommand{\B}{\mathscr{B}}
\newcommand{\C}{\mathcal{C}}
\newcommand{\T}{\mathscr{T}}
\newcommand{\F}{\mathcal{F}}
\newcommand{\G}{\mathcal{G}}
\newcommand{\X}{\mathcal{X}}
%\newcommand{\ba}{\begin{align*}}
%\newcommand{\ea}{\end{align*}}
\DeclareMathOperator*{\argmax}{arg\,max}
\renewcommand{\qedsymbol}{$\blacksquare$}
\makeatletter
\def\BState{\State\hskip-\ALG@thistlm}
\makeatother

\makeatletter
\def\th@plain{%
  \thm@notefont{}% same as heading font
  \itshape % body font
}
\def\th@definition{%
  \thm@notefont{}% same as heading font
  \normalfont % body font
}
\makeatother
\date{}
\title{Lecture 22: Queueing Networks}
\author{}
\begin{document}
\maketitle

\section{Jackson Network}
Jackson network is a system of $k$ servers with independent service times, and two types of customer arrivals. 
For each server $i \in [k]$, we have
\begin{itemize}
	\item \textit{iid} service times distributed exponentially with rate $\mu_i$,
	\item external Poisson arrivals with rate $r_i$,
	\item internal arrivals from service completions at server $j \in [k]$ with probability $P_{ji}$,
	\item service completions leading to departure from the system with probability $1-\sum_{j \in [k]}P_{ij}$. 
\end{itemize}  
%Once a customer is served at server $i$, the customer joins server $j$ with probability $P_{ij}$, 
%such that $\sum_{i}P_{ij} \leq 1$. 
%The probability of a customer at server $i$ completely departing the system is $1-\sum_{j}P_{ij}$. 
\begin{lem} 
Let $X_i(t)$ be the number of customers in server $i$ at time $t$ in the Jackson network, 
and $X(t) = (X_1(t), \dots, X_k(t))$. 
Then, $\{X(t): t \in \R\}$ is a continuous time Markov chain with state space $\N_0^k$. 
\end{lem}
\begin{proof}

\end{proof}
\begin{lem} 
The generator matrix $Q$ for the Markov process $X(t)$ consists of the following transitions, 
and the associated transition rates. 
\end{lem}
\begin{proof}
\end{proof}
If we denote $\lambda_i$ as the total rate at which the customers join server $i$, 
then $\lambda_i$s can be obtained as a solution to,
\eq{
\lambda_j=r_j+\sum_{i=1}^{k}\lambda_i P_{ij},~ j \in [k]. 
}
The model can be analysed by a stationary continuous-time Markov chain $\{X(t) \in \N_0^k: t \in \R\}$ with states $X(t) = (X_1(t),X_2(t), \dots X_k(k)) \in \N_0^k$, 
where $X_i(t)$ denotes the number of customers in server $i$ at time $t$. 
From the tandem queue results, we expect the customers at each server to be independent random variables. 
Let $n \in \N_0^k$, and we are interested in knowing the the joint probability,
\eq{
\pi(n) &\triangleq \Pr\{X(t) = n\} = \Pr\{X_1(t) = n_1, \dots, X_k(t) = n_k\}.
}
We write the marginal probabilities as 
\eq{
\pi(m_i) &\triangleq \Pr\{X_i(t) = m_i\} = \sum_{n \in \N_0^k: n_i = m_i }\pi(n).
}
%where $Pr(n_i)$ is the limiting probability that there are $n_i$ customers to serve at server $i$. 
Since, $X(t)$ is a CTMC, only a single transition takes place in any infinitesimal time interval. 
Let $X(t) = n$, then the possible transitions from this Markov process at time $t$ and the associated rates are the following.  
\begin{enumerate}[i\_]
\item An external arrival takes place in queue $i$, with rate 
\eq{
Q(n,n+e_i) &= r_i.
}
\item If $n_i > 0$, a service completes and a customer departs from queue $i$ exiting the system, with rate 
\eq{
Q(n,n-e_i) &= \mu_i(1 - \sum_{j=1}^kP_{ij}).
}
\item If $n_i > 0$, a service completes and a customer departs from queue $i$ and joins queue $j$, with rate 
\eq{
Q(n,n-e_i+e_j) &= \mu_iP_{ij}.
}
\end{enumerate}

\begin{thm}
If $\lambda_i < \mu_i$ for each $i \in [k]$, then the reversed stochastic process is a network process of the same type as original. 
That is, the reversed process $X(-t)$ is a CTMC with the generator matrix $Q^{\ast}$ given by the following.  
\begin{enumerate}[(i)]
\item The system has external Poisson arrivals to queue $i$ at rate $\lambda_i(1-\sum_jP_{ij})$. 
That is, 
\eq{
Q^{\ast}(n,n+e_i) &= \lambda_i(1-\sum_jP_{ij}).
}
\item The service times at queue $i$ are \textit{iid} exponential with rate $\mu_i$. 
\item  A departure from queue $j$ goes to queue $i$ with probability $\bar{P}_{ji}$ as given by
\eq{
\bar{P}_{ji}=\frac{\lambda_i P_{ij}}{\lambda_j}.
}
That is, if $n_j > 0$, then a customer joins queue $i$ after a service completion from queue $j$ with rate 
\eq{
Q^{\ast}(n,n-e_j+e_i) &= \mu_j\bar{P}_{ji} = \frac{\mu_j}{\lambda_j}\lambda_iP_{ij}.
}
The corresponding rate of customer departing the system after service completion from server $j$ is
\eq{
Q^{\ast}(n, n-e_j) &= \mu_j(1-\sum_{i=1}^k\bar{P}_{ji}) = \frac{\mu_j}{\lambda_j}(\lambda_j - \sum_{i=1}^k\lambda_iP_{ij}) = \frac{\mu_j}{\lambda_j}r_j.
}
\item For $\rho_i \triangleq \lambda_i/\mu_i$, the stationary distribution satisfies
\eq{
\pi(n)= \prod_{i=1}^k\pi_i(n_i) = \prod_{i=1}^k\rho_i^{n_i}(1-\rho_i).
}
\end{enumerate}
\end{thm}
\begin{proof} 
It suffices to check that the detailed balanced conditions hold with the candidate generator matrix $Q^{\ast}$ for the reversed process and the candidate distribution $\pi$. 
We first focus at the detailed balance equations associated with the external arrivals 
\eq{
\pi(n)Q(n,n+e_i) &= \pi(n+e_i)Q^{\ast}(n+e_i,n).
}
This equation holds since for each $n_i \in \N_0$, we have 
\eq{
\pi_i(n_i+e_i) &= \rho_i\pi_i(n_i).
}
Second for $n_i > 0$, we look at the detailed balanced equations corresponding the exits from the system from queue $i$,
\eq{
\pi(n)Q(n,n-e_i) &= \pi(n-e_i)Q^{\ast}(n-e_i,n). 
}
This equation holds since for each $n_i \in \N$, we have 
\eq{
\pi_i(n_i-e_i)\mu_i(1-\sum_{i=1}^kP_{ij}) &= \lambda_i(1-\sum_{i=1}^kP_{ij})\pi_i(n_i).
}
Finally for $n_i > 0$, we look at the detailed balanced equations corresponding the transitions where a customer joins queues $j$ after getting service from queue $i$,
\eq{
\pi(n)Q(n,n-e_i+e_j) &= \pi(n-e_i+e_j)Q^{\ast}(n-e_i+e_j,n). 
}
This equation holds since for each $n_i \in \N$, we have 
\eq{
\pi_i(n_i)\pi_j(n_j)\mu_iP_{ij} &= \pi_i(n_i-1)\pi_j(n_j+1)\mu_j\bar{P}_{ji}.
}
\end{proof}
% Consider transitions resulting from an outside arrival. Consider states $\underline{n}=(n_1,n_2 \hdots n_i, \hdots n_k)$ and $\underline{n}'=(n_1,\hdots n_i+1,\hdots n_k)$. Now
% \eq{
% q_{n,n'}=r_i,
% }
% and, if the conjecture is true,
% \eq{
% q_{n',n}^*&=\mu_i(1-\sum_{j}\bar{P}_{ij})\\
% &=\mu_i \frac{(\lambda_i-\sum_{j}\lambda_j {P}_{ji})}{\lambda_i}\\
% &=\frac{\mu_i r_i}{\lambda_i}
% }
% and 
% \eq{
% P(\underline{n})=\Pi_j P_j(n_j),~ P(\underline{n}')=P_i(n_i+1)\Pi_{j \neq i}P_j(n_j).
% }
% Hence, from the previous theorem, we need that
% \eq{
%  r_i \Pi_j P_j(n_j)= \frac{\mu_i r_i}{\lambda_i}\Pi_{j \neq i}P_j(n_j).
% } 
% That is,
% \eq{
% P_i(n+1)= \frac{\lambda_i}{\mu_i}P_i(n)={(\frac{\lambda_i}{\mu_i})}^{n+1}P_i(0),
% }
% and using the fact that 
% \eq{
%\sum_{n=0}^{\infty}P_i(n)=1
% }
% yields 
% \eq{
%P_i(n) ={(\frac{\lambda_i}{\mu_i})}^n(1-\frac{\lambda_i}{\mu_i}).
%}
%Thus $\frac{\lambda_i}{\mu_i}< 1$ and $P_i$ must be as given before for the conjecture to be true. Now consider transitions that result from a departure from server $j$ going to server $i$. That is, let $\underline{n}=(n_1,n_2 \hdots n_i, \hdots n_k)$ and $\underline{n}'=(n_1,\hdots n_i+1,\hdots n_j+1,\hdots n_k)$, where $n_j >0$. Since
%\eq{
%q_{n,n'}=\mu_j P_{ji}
%}
%and the conjecture yields
%\eq{
%q_{n,n'}^*=\mu_i \bar{P}_{ij}.
%}
%We need to show that 
%\eq{
%P(\underline{n})\mu_j P_{ji}= P(\underline{n}')\mu_i \bar{P}_{ij}
%}
%or, equivalently,
%\eq{
%\lambda_j P_{ji}=\lambda_i \bar{P}_{ij}.
%}
%which is the definition of $\bar{P}_{ij}$.
%The last lecture ended with a conjecture on the network of queues. First we state the following theorem.
%\begin{thm}
%Assuming that $\lambda_i < \mu_i$, for all $i$, in steady state, the number of customers at service $i$ are independent and the limiting probabilities are given by
%\begin{equation*}
%Pr(n_1,n_2, \hdots n_k) = \Pi_{i=1}^{k}{(\frac{\lambda_i}{\mu_i})}^n (1-\frac{\lambda_i}{\mu_i}).
%\end{equation*}
%\end{thm}
%Also, form the reversed chain, we have the following.
\begin{cor}
The process of customers departing the system from the server $i$, $i=1,2 \hdots k,$ are independent Poisson processes having respective rates $\lambda_i (1-\sum_j P_{ij} )$.
\end{cor}
\begin{proof}
We have already shown that in the reverse process, customers arrive to server $i$ from outside the system according to independent Poisson processes having rates $\lambda_i(1-\sum_{i}P_{ij}),~ i \geq 1$. Since an arrival from outside corresponds to a departure out of the system from server $i$ in the forward process, the result follows. 
\end{proof}
\end{document}
